% !TEX TS-program = pdflatex
\documentclass[11pt]{article}

% -------------------- Packages --------------------
\usepackage[a4paper,margin=1in]{geometry}
\usepackage{amsmath,amssymb}
\usepackage[T1]{fontenc}
\usepackage{lmodern}
\usepackage{xcolor}
\usepackage{tcolorbox}
\tcbuselibrary{skins,breakable}
\usepackage{enumitem}
\usepackage{hyperref}
\usepackage{tikz}

\pagestyle{empty}

% -------------------- Dark Theme Colors --------------------
\definecolor{bg}{HTML}{000000}
\definecolor{pairbg}{HTML}{121212}
\definecolor{solbg}{HTML}{0A0A0A}
\definecolor{border}{HTML}{2A2A2A}
\definecolor{text}{HTML}{FFFFFF}
\definecolor{muted}{HTML}{C9CDD3}
\definecolor{gold}{HTML}{FFD700}
\definecolor{green}{HTML}{4ADE80}
\definecolor{cyan}{HTML}{38BDF8}

\pagecolor{bg}
\color{text}

\hypersetup{
  colorlinks=true,
  linkcolor=cyan,
  urlcolor=cyan
}

\setlength{\parindent}{0pt}
\setlength{\parskip}{10pt}

\setlist[itemize]{left=1.4em,itemsep=6pt,topsep=6pt}
\setlist[enumerate]{left=1.6em,itemsep=4pt,topsep=4pt}

% -------------------- tcolorbox Base --------------------
\tcbset{
  enhanced,
  breakable,
  arc=12pt,
  boxrule=0.8pt,
  left=16pt,right=16pt,top=12pt,bottom=12pt
}

\newtcolorbox{QAPair}[1]{%
  colback=pairbg,
  colbacklower=solbg,
  colframe=border,
  coltext=text,
  title=\textcolor{gold}{\bfseries #1},
  fonttitle=\bfseries,
  coltitle=text,
  segmentation style={draw=border, dashed, line width=0.6pt},
}

% Visible text inside this box (fix)
\newtcolorbox{QuickBox}{%
  colback=pairbg,
  colframe=cyan,
  coltext=text,
  fontupper=\color{text},
  borderline north={4pt}{0pt}{cyan},
  arc=14pt,
  boxrule=0.8pt
}

% Helper for step headings
\newcommand{\Step}[1]{\textcolor{muted}{\textbf{Step #1:}}}

% ============================================================
\begin{document}

\begin{center}
{\LARGE\bfseries \textcolor{gold}{Exercise 7.1 --- Solutions}}\\[-2pt]
\end{center}

\begin{QuickBox}
{\color{cyan}\bfseries Quick formulas (Coordinate Geometry)}\par\medskip
\begin{itemize}
\item \textbf{Distance formula:}\quad
$d\bigl((x_1,y_1),(x_2,y_2)\bigr)=\sqrt{(x_2-x_1)^2+(y_2-y_1)^2}$.
\item \textbf{Collinear (by distances):}\quad
If for three points $P,Q,R$ we get $PQ+QR=PR$ (or any such ordering), then they are \textbf{collinear}.
\item \textbf{Slope:}\quad $m=\dfrac{y_2-y_1}{x_2-x_1}$ (when $x_2\neq x_1$).
\item \textbf{Parallel / Perpendicular:}\quad
Parallel lines have equal slope; perpendicular lines satisfy $m_1m_2=-1$ (when defined).
\item \textbf{Right triangle test (Pythagoras):}\quad
If $a^2+b^2=c^2$, the triangle is \textbf{right-angled}.
\item \textbf{Parallelogram test (vectors):}\quad
$\overrightarrow{AB}=\overrightarrow{DC}$ and $\overrightarrow{BC}=\overrightarrow{AD}$.
\item \textbf{Rectangle / Square:}\quad
Rectangle: adjacent sides perpendicular; Square: rectangle + all sides equal.
\end{itemize}
\end{QuickBox}

% ============================================================
% Q1
\begin{QAPair}{Question 1 (i)}
\textcolor{gold}{\bfseries Question:} Find the distance between $(-2,1)$ and $(1,5)$.\\
\tcblower
\textcolor{green}{\bfseries Answer:}
\[
\begin{aligned}
\Step{1}\;& d=\sqrt{(1-(-2))^2+(5-1)^2}=\sqrt{3^2+4^2}\\
\Step{2}\;&=\sqrt{9+16}=\sqrt{25}=5.
\end{aligned}
\]
\end{QAPair}

\begin{QAPair}{Question 1 (ii)}
\textcolor{gold}{\bfseries Question:} Find the distance between $(1,1)$ and $(7,9)$.\\
\tcblower
\textcolor{green}{\bfseries Answer:}
\[
\begin{aligned}
\Step{1}\;& d=\sqrt{(7-1)^2+(9-1)^2}=\sqrt{6^2+8^2}\\
\Step{2}\;&=\sqrt{36+64}=\sqrt{100}=10.
\end{aligned}
\]
\end{QAPair}

\begin{QAPair}{Question 1 (iii)}
\textcolor{gold}{\bfseries Question:} Find the distance between $(0,-5)$ and $(-2,1)$.\\
\tcblower
\textcolor{green}{\bfseries Answer:}
\[
\begin{aligned}
\Step{1}\;& d=\sqrt{(-2-0)^2+(1-(-5))^2}=\sqrt{(-2)^2+6^2}\\
\Step{2}\;&=\sqrt{4+36}=\sqrt{40}=2\sqrt{10}.
\end{aligned}
\]
\end{QAPair}

\begin{QAPair}{Question 1 (iv)}
\textcolor{gold}{\bfseries Question:} Find the distance between $(0,-5)$ and $(10,5)$.\\
\tcblower
\textcolor{green}{\bfseries Answer:}
\[
\begin{aligned}
\Step{1}\;& d=\sqrt{(10-0)^2+(5-(-5))^2}=\sqrt{10^2+10^2}\\
\Step{2}\;&=\sqrt{200}=10\sqrt{2}.
\end{aligned}
\]
\end{QAPair}

\begin{QAPair}{Question 1 (v)}
\textcolor{gold}{\bfseries Question:} Find the distance between $(4,-1)$ and $(2,0)$.\\
\tcblower
\textcolor{green}{\bfseries Answer:}
\[
\begin{aligned}
\Step{1}\;& d=\sqrt{(2-4)^2+(0-(-1))^2}=\sqrt{(-2)^2+1^2}\\
\Step{2}\;&=\sqrt{5}.
\end{aligned}
\]
\end{QAPair}

\begin{QAPair}{Question 1 (vi)}
\textcolor{gold}{\bfseries Question:} Find the distance between $(-3,1)$ and $(-2,-3)$.\\
\tcblower
\textcolor{green}{\bfseries Answer:}
\[
\begin{aligned}
\Step{1}\;& d=\sqrt{((-2)-(-3))^2+((-3)-1)^2}=\sqrt{1^2+(-4)^2}\\
\Step{2}\;&=\sqrt{17}.
\end{aligned}
\]
\end{QAPair}

\begin{QAPair}{Question 1 (vii)}
\textcolor{gold}{\bfseries Question:} Find the distance between $\left(\frac12,2\right)$ and $(-2,1)$.\\
\tcblower
\textcolor{green}{\bfseries Answer:}
\[
\begin{aligned}
\Step{1}\;& d=\sqrt{\left(-2-\frac12\right)^2+(1-2)^2}
=\sqrt{\left(-\frac{5}{2}\right)^2+(-1)^2}\\
\Step{2}\;&=\sqrt{\frac{25}{4}+1}
=\sqrt{\frac{29}{4}}
=\frac{\sqrt{29}}{2}.
\end{aligned}
\]
\end{QAPair}

\begin{QAPair}{Question 1 (viii)}
\textcolor{gold}{\bfseries Question:} Find the distance between $\left(\frac{7}{2},11\right)$ and $\left(13,\frac{17}{2}\right)$.\\
\tcblower
\textcolor{green}{\bfseries Answer:}
\[
\begin{aligned}
\Step{1}\;& d=\sqrt{\left(13-\frac{7}{2}\right)^2+\left(\frac{17}{2}-11\right)^2}
=\sqrt{\left(\frac{19}{2}\right)^2+\left(-\frac{5}{2}\right)^2}\\
\Step{2}\;&=\sqrt{\frac{361}{4}+\frac{25}{4}}
=\sqrt{\frac{386}{4}}
=\frac{\sqrt{386}}{2}.
\end{aligned}
\]
\end{QAPair}

\begin{QAPair}{Question 1 (ix)}
\textcolor{gold}{\bfseries Question:} Find the distance between $\left(-\frac{9}{4},-\frac{7}{3}\right)$ and $\left(-\frac{3}{2},-\frac{5}{4}\right)$.\\
\tcblower
\textcolor{green}{\bfseries Answer:}
\[
\begin{aligned}
\Step{1}\;& d=\sqrt{\left(-\frac{3}{2}-\left(-\frac{9}{4}\right)\right)^2+\left(-\frac{5}{4}-\left(-\frac{7}{3}\right)\right)^2}\\
\Step{2}\;&=\sqrt{\left(\frac{3}{4}\right)^2+\left(\frac{13}{12}\right)^2}
=\sqrt{\frac{9}{16}+\frac{169}{144}}\\
\Step{3}\;&=\sqrt{\frac{81}{144}+\frac{169}{144}}
=\sqrt{\frac{250}{144}}
=\frac{\sqrt{250}}{12}
=\frac{5\sqrt{10}}{12}.
\end{aligned}
\]
\end{QAPair}

\begin{QAPair}{Question 1 (x)}
\textcolor{gold}{\bfseries Question:} Find the distance between $\left(2\frac12,\,5\frac14\right)$ and $(5,-1)$.\\
\tcblower
\textcolor{green}{\bfseries Answer:}
\[
\begin{aligned}
\Step{1}\;& \left(2\frac12,5\frac14\right)=\left(\frac{5}{2},\frac{21}{4}\right).\\
\Step{2}\;& d=\sqrt{\left(5-\frac{5}{2}\right)^2+\left(-1-\frac{21}{4}\right)^2}
=\sqrt{\left(\frac{5}{2}\right)^2+\left(-\frac{25}{4}\right)^2}\\
\Step{3}\;&=\sqrt{\frac{25}{4}+\frac{625}{16}}
=\sqrt{\frac{100}{16}+\frac{625}{16}}
=\sqrt{\frac{725}{16}}
=\frac{\sqrt{725}}{4}
=\frac{5\sqrt{29}}{4}.
\end{aligned}
\]
\end{QAPair}

% ============================================================
% Q2
\begin{QAPair}{Question 2 (i)}
\textcolor{gold}{\bfseries Question:} Check whether $(0,1),(2,3),(3,4)$ are collinear (by distance formula).\\
\tcblower
\textcolor{green}{\bfseries Answer:}
Let $A(0,1),\,B(2,3),\,C(3,4)$.
\[
\begin{aligned}
\Step{1}\;& AB=\sqrt{(2-0)^2+(3-1)^2}=\sqrt{4+4}=2\sqrt2.\\
\Step{2}\;& BC=\sqrt{(3-2)^2+(4-3)^2}=\sqrt{1+1}=\sqrt2.\\
\Step{3}\;& AC=\sqrt{(3-0)^2+(4-1)^2}=\sqrt{9+9}=3\sqrt2.\\
\Step{4}\;& AB+BC=2\sqrt2+\sqrt2=3\sqrt2=AC.
\end{aligned}
\]
\textbf{Hence, the points are collinear} (and $B$ lies between $A$ and $C$).
\end{QAPair}

\begin{QAPair}{Question 2 (ii)}
\textcolor{gold}{\bfseries Question:} Check whether $(-5,-4),(1,0),(6,7)$ are collinear (by distance formula).\\
\tcblower
\textcolor{green}{\bfseries Answer:}
Let $A(-5,-4),\,B(1,0),\,C(6,7)$.
\[
\begin{aligned}
\Step{1}\;& AB=\sqrt{(1+5)^2+(0+4)^2}=\sqrt{36+16}=2\sqrt{13}.\\
\Step{2}\;& BC=\sqrt{(6-1)^2+(7-0)^2}=\sqrt{25+49}=\sqrt{74}.\\
\Step{3}\;& AC=\sqrt{(6+5)^2+(7+4)^2}=\sqrt{121+121}=11\sqrt2.\\
\Step{4}\;& AB+BC=2\sqrt{13}+\sqrt{74}\neq 11\sqrt2=AC.
\end{aligned}
\]
\textbf{Hence, the points are not collinear.}
\end{QAPair}

\begin{QAPair}{Question 2 (iii)}
\textcolor{gold}{\bfseries Question:} Check whether $(0,-5),(3,7),(5,15)$ are collinear (by distance formula).\\
\tcblower
\textcolor{green}{\bfseries Answer:}
Let $A(0,-5),\,B(3,7),\,C(5,15)$.
\[
\begin{aligned}
\Step{1}\;& AB=\sqrt{(3-0)^2+(7+5)^2}=\sqrt{9+144}=3\sqrt{17}.\\
\Step{2}\;& BC=\sqrt{(5-3)^2+(15-7)^2}=\sqrt{4+64}=2\sqrt{17}.\\
\Step{3}\;& AC=\sqrt{(5-0)^2+(15+5)^2}=\sqrt{25+400}=5\sqrt{17}.\\
\Step{4}\;& AB+BC=3\sqrt{17}+2\sqrt{17}=5\sqrt{17}=AC.
\end{aligned}
\]
\textbf{Hence, the points are collinear} (and $B$ lies between $A$ and $C$).
\end{QAPair}

\begin{QAPair}{Question 2 (iv)}
\textcolor{gold}{\bfseries Question:} Check whether $(6,3),(14,7),(-6,-3)$ are collinear (by distance formula).\\
\tcblower
\textcolor{green}{\bfseries Answer:}
Let $A(6,3),\,B(14,7),\,C(-6,-3)$.
\[
\begin{aligned}
\Step{1}\;& BA=\sqrt{(6-14)^2+(3-7)^2}=\sqrt{64+16}=4\sqrt5.\\
\Step{2}\;& AC=\sqrt{(-6-6)^2+(-3-3)^2}=\sqrt{144+36}=6\sqrt5.\\
\Step{3}\;& BC=\sqrt{(-6-14)^2+(-3-7)^2}=\sqrt{400+100}=10\sqrt5.\\
\Step{4}\;& BA+AC=4\sqrt5+6\sqrt5=10\sqrt5=BC.
\end{aligned}
\]
\textbf{Hence, the points are collinear} (and $A$ lies between $B$ and $C$).
\end{QAPair}

\begin{QAPair}{Question 2 (v)}
\textcolor{gold}{\bfseries Question:} Check whether $(0,15),(2,9),(7,-6)$ are collinear (by distance formula).\\
\tcblower
\textcolor{green}{\bfseries Answer:}
Let $A(0,15),\,B(2,9),\,C(7,-6)$.
\[
\begin{aligned}
\Step{1}\;& AB=\sqrt{(2-0)^2+(9-15)^2}=\sqrt{4+36}=2\sqrt{10}.\\
\Step{2}\;& BC=\sqrt{(7-2)^2+(-6-9)^2}=\sqrt{25+225}=5\sqrt{10}.\\
\Step{3}\;& AC=\sqrt{(7-0)^2+(-6-15)^2}=\sqrt{49+441}=7\sqrt{10}.\\
\Step{4}\;& AB+BC=2\sqrt{10}+5\sqrt{10}=7\sqrt{10}=AC.
\end{aligned}
\]
\textbf{Hence, the points are collinear} (and $B$ lies between $A$ and $C$).
\end{QAPair}

\begin{QAPair}{Question 2 (vi)}
\textcolor{gold}{\bfseries Question:} Check whether $(1,-2),(7,8),(-2,-7)$ are collinear (by distance formula).\\
\tcblower
\textcolor{green}{\bfseries Answer:}
Let $A(1,-2),\,B(7,8),\,C(-2,-7)$.
\[
\begin{aligned}
\Step{1}\;& BA=\sqrt{(1-7)^2+(-2-8)^2}=\sqrt{36+100}=2\sqrt{34}.\\
\Step{2}\;& AC=\sqrt{(-2-1)^2+(-7+2)^2}=\sqrt{9+25}=\sqrt{34}.\\
\Step{3}\;& BC=\sqrt{(-2-7)^2+(-7-8)^2}=\sqrt{81+225}=3\sqrt{34}.\\
\Step{4}\;& BA+AC=2\sqrt{34}+\sqrt{34}=3\sqrt{34}=BC.
\end{aligned}
\]
\textbf{Hence, the points are collinear} (and $A$ lies between $B$ and $C$).
\end{QAPair}

% ============================================================
% Q3
\begin{QAPair}{Question 3 (i)}
\textcolor{gold}{\bfseries Question:} For $(2,3),(5,3),(2,1)$, identify the triangle type and find its perimeter.\\
\tcblower
\textcolor{green}{\bfseries Answer:}
Let $A(2,3),\,B(5,3),\,C(2,1)$.
\[
\begin{aligned}
\Step{1}\;& AB=\sqrt{(5-2)^2+(3-3)^2}=3,\quad
AC=\sqrt{(2-2)^2+(1-3)^2}=2,\\
& BC=\sqrt{(2-5)^2+(1-3)^2}=\sqrt{13}.\\
\Step{2}\;& AB^2+AC^2=3^2+2^2=9+4=13=BC^2 \Rightarrow \text{right-angled.}\\
\Step{3}\;& \text{All sides are different } \Rightarrow \text{scalene.}\\
\Step{4}\;& \text{Perimeter }=AB+BC+CA=3+\sqrt{13}+2=5+\sqrt{13}.
\end{aligned}
\]
\end{QAPair}

\begin{QAPair}{Question 3 (ii)}
\textcolor{gold}{\bfseries Question:} For $(0,0),(1,\sqrt3),(2,0)$, identify the triangle type and find its perimeter.\\
\tcblower
\textcolor{green}{\bfseries Answer:}
Let $A(0,0),\,B(1,\sqrt3),\,C(2,0)$.
\[
\begin{aligned}
\Step{1}\;& AB=\sqrt{(1-0)^2+(\sqrt3-0)^2}=\sqrt{1+3}=2.\\
\Step{2}\;& BC=\sqrt{(2-1)^2+(0-\sqrt3)^2}=\sqrt{1+3}=2.\\
\Step{3}\;& CA=\sqrt{(0-2)^2+(0-0)^2}=2.\\
\Step{4}\;& AB=BC=CA \Rightarrow \textbf{equilateral triangle}.\\
\Step{5}\;& \text{Perimeter }=2+2+2=6.
\end{aligned}
\]
\end{QAPair}

\begin{QAPair}{Question 3 (iii)}
\textcolor{gold}{\bfseries Question:} For $(0,1),(8,4),(0,8)$, identify the triangle type and find its perimeter.\\
\tcblower
\textcolor{green}{\bfseries Answer:}
Let $A(0,1),\,B(8,4),\,C(0,8)$.
\[
\begin{aligned}
\Step{1}\;& AB=\sqrt{(8-0)^2+(4-1)^2}=\sqrt{64+9}=\sqrt{73}.\\
\Step{2}\;& BC=\sqrt{(0-8)^2+(8-4)^2}=\sqrt{64+16}=4\sqrt5.\\
\Step{3}\;& CA=\sqrt{(0-0)^2+(1-8)^2}=7.\\
\Step{4}\;& \text{All three sides are different } \Rightarrow \textbf{scalene triangle}.\\
\Step{5}\;& \text{Perimeter }=\sqrt{73}+4\sqrt5+7.
\end{aligned}
\]
\end{QAPair}

\begin{QAPair}{Question 3 (iv)}
\textcolor{gold}{\bfseries Question:} For $(-3,-5),(3,-3),(0,6)$, identify the triangle type and find its perimeter.\\
\tcblower
\textcolor{green}{\bfseries Answer:}
Let $A(-3,-5),\,B(3,-3),\,C(0,6)$.
\[
\begin{aligned}
\Step{1}\;& AB=\sqrt{(3+3)^2+(-3+5)^2}=\sqrt{36+4}=2\sqrt{10}.\\
\Step{2}\;& BC=\sqrt{(0-3)^2+(6+3)^2}=\sqrt{9+81}=3\sqrt{10}.\\
\Step{3}\;& CA=\sqrt{(-3-0)^2+(-5-6)^2}=\sqrt{9+121}=\sqrt{130}.\\
\Step{4}\;& AB^2+BC^2=40+90=130=CA^2 \Rightarrow \textbf{right-angled}.\\
\Step{5}\;& \text{All sides different } \Rightarrow \textbf{scalene}.\\
\Step{6}\;& \text{Perimeter }=2\sqrt{10}+3\sqrt{10}+\sqrt{130}=5\sqrt{10}+\sqrt{130}.
\end{aligned}
\]
\end{QAPair}

\begin{QAPair}{Question 3 (v)}
\textcolor{gold}{\bfseries Question:} For $(-1,-1),(-1,5),(4,2)$, identify the triangle type and find its perimeter.\\
\tcblower
\textcolor{green}{\bfseries Answer:}
Let $A(-1,-1),\,B(-1,5),\,C(4,2)$.
\[
\begin{aligned}
\Step{1}\;& AB=\sqrt{( -1+1)^2+(5+1)^2}=6.\\
\Step{2}\;& BC=\sqrt{(4+1)^2+(2-5)^2}=\sqrt{25+9}=\sqrt{34}.\\
\Step{3}\;& CA=\sqrt{(-1-4)^2+(-1-2)^2}=\sqrt{25+9}=\sqrt{34}.\\
\Step{4}\;& BC=CA \Rightarrow \textbf{isosceles triangle}.\\
\Step{5}\;& \text{Perimeter }=6+\sqrt{34}+\sqrt{34}=6+2\sqrt{34}.
\end{aligned}
\]
\end{QAPair}

\begin{QAPair}{Question 3 (vi)}
\textcolor{gold}{\bfseries Question:} For $(1,2),(7,2),(7,8)$, identify the triangle type and find its perimeter.\\
\tcblower
\textcolor{green}{\bfseries Answer:}
Let $A(1,2),\,B(7,2),\,C(7,8)$.
\[
\begin{aligned}
\Step{1}\;& AB=\sqrt{(7-1)^2+(2-2)^2}=6,\quad
BC=\sqrt{(7-7)^2+(8-2)^2}=6.\\
\Step{2}\;& CA=\sqrt{(1-7)^2+(2-8)^2}=\sqrt{36+36}=6\sqrt2.\\
\Step{3}\;& AB^2+BC^2=36+36=72=(6\sqrt2)^2 \Rightarrow \textbf{right-angled}.\\
\Step{4}\;& AB=BC \Rightarrow \textbf{isosceles} as well.\\
\Step{5}\;& \text{Perimeter }=6+6+6\sqrt2=12+6\sqrt2.
\end{aligned}
\]
\end{QAPair}

% ============================================================
% Q4
\begin{QAPair}{Question 4}
\textcolor{gold}{\bfseries Question:} Show that $A(-4,2), B(1,4), C(3,-1), D(-2,-3)$ are the vertices of a square.\\
\tcblower
\textcolor{green}{\bfseries Answer:}
\[
\begin{aligned}
\Step{1}\;& AB^2=(1+4)^2+(4-2)^2=5^2+2^2=29.\\
\Step{2}\;& BC^2=(3-1)^2+(-1-4)^2=2^2+(-5)^2=29.\\
\Step{3}\;& CD^2=(-2-3)^2+(-3+1)^2=(-5)^2+(-2)^2=29.\\
\Step{4}\;& DA^2=(-4+2)^2+(2+3)^2=(-2)^2+5^2=29.\\
\Step{5}\;& \Rightarrow AB=BC=CD=DA \quad (\text{all sides equal}).\\[4pt]
\Step{6}\;& \overrightarrow{AB}=(5,2),\ \overrightarrow{BC}=(2,-5)\Rightarrow
\overrightarrow{AB}\cdot\overrightarrow{BC}=5\cdot 2+2\cdot(-5)=0\\
&\Rightarrow AB\perp BC.\\
\Step{7}\;& \text{A rhombus with one right angle is a square. Hence }ABCD\text{ is a square.}\\
&(\text{Also } AC^2=58,\ BD^2=58 \Rightarrow AC=BD.)
\end{aligned}
\]
\end{QAPair}

% ============================================================
% Q5
\begin{QAPair}{Question 5}
\textcolor{gold}{\bfseries Question:} Show that $A(-4,-1), B(0,-2), C(6,1), D(2,2)$ are the vertices of a parallelogram.\\
\tcblower
\textcolor{green}{\bfseries Answer:}
\[
\begin{aligned}
\Step{1}\;& \overrightarrow{AB}=(0-(-4),-2-(-1))=(4,-1).\\
\Step{2}\;& \overrightarrow{DC}=(6-2,1-2)=(4,-1).\\
\Step{3}\;& \overrightarrow{BC}=(6-0,1-(-2))=(6,3).\\
\Step{4}\;& \overrightarrow{AD}=(2-(-4),2-(-1))=(6,3).\\
\Step{5}\;& \overrightarrow{AB}=\overrightarrow{DC}\ \text{and}\ \overrightarrow{BC}=\overrightarrow{AD}
\Rightarrow \textbf{$ABCD$ is a parallelogram.}
\end{aligned}
\]
\end{QAPair}

% ============================================================
% Q6
\begin{QAPair}{Question 6}
\textcolor{gold}{\bfseries Question:} Show that $A(1,2), B(7,2), C(7,8), D(1,8)$ are the vertices of a rectangle, and prove its diagonals are equal.\\
\tcblower
\textcolor{green}{\bfseries Answer:}
\[
\begin{aligned}
\Step{1}\;& \overrightarrow{AB}=(7-1,2-2)=(6,0),\quad \overrightarrow{BC}=(7-7,8-2)=(0,6).\\
\Step{2}\;& \overrightarrow{AB}\cdot\overrightarrow{BC}=6\cdot 0+0\cdot 6=0 \Rightarrow AB\perp BC.\\
\Step{3}\;& AB=6,\ BC=6,\ \text{and opposite sides are parallel (horizontal/vertical).}\\
&\Rightarrow ABCD \text{ is a rectangle.}\\[4pt]
\Step{4}\;& AC=\sqrt{(7-1)^2+(8-2)^2}=\sqrt{36+36}=6\sqrt2.\\
\Step{5}\;& BD=\sqrt{(1-7)^2+(8-2)^2}=\sqrt{36+36}=6\sqrt2.\\
\Step{6}\;& \Rightarrow AC=BD \quad \textbf{(diagonals are equal).}
\end{aligned}
\]
\end{QAPair}

% ============================================================
% Q7
\begin{QAPair}{Question 7}
\textcolor{gold}{\bfseries Question:} Show that $A(4,2), B(7,2), C(7,5), D(4,5)$ are the vertices of a square, and show diagonals are equal.\\
\tcblower
\textcolor{green}{\bfseries Answer:}
\[
\begin{aligned}
\Step{1}\;& AB=\sqrt{(7-4)^2+(2-2)^2}=3,\quad
BC=\sqrt{(7-7)^2+(5-2)^2}=3.\\
\Step{2}\;& \overrightarrow{AB}=(3,0),\ \overrightarrow{BC}=(0,3)\Rightarrow \overrightarrow{AB}\cdot\overrightarrow{BC}=0
\Rightarrow AB\perp BC.\\
\Step{3}\;& \text{Adjacent sides are equal and perpendicular } \Rightarrow ABCD \text{ is a square.}\\[4pt]
\Step{4}\;& AC=\sqrt{(7-4)^2+(5-2)^2}=\sqrt{9+9}=3\sqrt2.\\
\Step{5}\;& BD=\sqrt{(4-7)^2+(5-2)^2}=\sqrt{9+9}=3\sqrt2.\\
\Step{6}\;& \Rightarrow AC=BD \quad \textbf{(diagonals are equal).}
\end{aligned}
\]
\end{QAPair}

% ============================================================
% Q8
\begin{QAPair}{Question 8}
\textcolor{gold}{\bfseries Question:} Are $A(-6,2), B(1,2), C(1,5), D(-6,5)$ the vertices of a rectangle? Also plot them.\\
\tcblower
\textcolor{green}{\bfseries Answer:}
\[
\begin{aligned}
\Step{1}\;& \overrightarrow{AB}=(1-(-6),2-2)=(7,0),\quad \overrightarrow{BC}=(1-1,5-2)=(0,3).\\
\Step{2}\;& \overrightarrow{AB}\cdot\overrightarrow{BC}=7\cdot 0+0\cdot 3=0 \Rightarrow AB\perp BC.\\
\Step{3}\;& AB=7,\ BC=3,\ \text{and opposite sides are parallel (horizontal/vertical).}\\
&\Rightarrow \textbf{$ABCD$ is a rectangle.}
\end{aligned}
\]
\textcolor{muted}{(Plot)}:
\begin{center}
\begin{tikzpicture}[scale=0.65, every node/.style={text=white}]
  \draw[->,gray] (-8,0) -- (3,0) node[right] {$x$};
  \draw[->,gray] (0,-1) -- (0,7) node[above] {$y$};

  \foreach \x in {-7,-6,-5,-4,-3,-2,-1,0,1,2}
    \draw[gray!45] (\x,0.12)--(\x,-0.12) node[below,scale=0.65]{\x};
  \foreach \y in {0,1,2,3,4,5,6}
    \draw[gray!45] (0.12,\y)--(-0.12,\y) node[left,scale=0.65]{\y};

  \coordinate (A) at (-6,2);
  \coordinate (B) at (1,2);
  \coordinate (C) at (1,5);
  \coordinate (D) at (-6,5);

  \draw[cyan,thick] (A)--(B)--(C)--(D)--cycle;

  \fill[cyan] (A) circle (2.2pt) node[below left] {$A(-6,2)$};
  \fill[cyan] (B) circle (2.2pt) node[below right] {$B(1,2)$};
  \fill[cyan] (C) circle (2.2pt) node[above right] {$C(1,5)$};
  \fill[cyan] (D) circle (2.2pt) node[above left] {$D(-6,5)$};
\end{tikzpicture}
\end{center}
\end{QAPair}

% ============================================================
% Q9
\begin{QAPair}{Question 9}
\textcolor{gold}{\bfseries Question:} Prove that $A(-3,0), B(3,0), C(6,4), D(0,4)$ are the vertices of a parallelogram.\\
\tcblower
\textcolor{green}{\bfseries Answer:}
\[
\begin{aligned}
\Step{1}\;& \overrightarrow{AB}=(3-(-3),0-0)=(6,0).\\
\Step{2}\;& \overrightarrow{DC}=(6-0,4-4)=(6,0).\\
\Step{3}\;& \overrightarrow{BC}=(6-3,4-0)=(3,4).\\
\Step{4}\;& \overrightarrow{AD}=(0-(-3),4-0)=(3,4).\\
\Step{5}\;& \overrightarrow{AB}=\overrightarrow{DC}\ \text{and}\ \overrightarrow{BC}=\overrightarrow{AD}
\Rightarrow \textbf{$ABCD$ is a parallelogram.}
\end{aligned}
\]
\end{QAPair}

% ============================================================
% Q10
\begin{QAPair}{Question 10}
\textcolor{gold}{\bfseries Question:} Show that $A(2,-1), B(8,-1), C(8,3), D(2,3)$ are the vertices of a rectangle and prove that $\triangle ABC$ and $\triangle ABD$ are right triangles.\\
\tcblower
\textcolor{green}{\bfseries Answer:}
\[
\begin{aligned}
\Step{1}\;& \overrightarrow{AB}=(8-2,-1-(-1))=(6,0),\quad \overrightarrow{BC}=(8-8,3-(-1))=(0,4).\\
\Step{2}\;& \overrightarrow{AB}\cdot\overrightarrow{BC}=6\cdot 0+0\cdot 4=0 \Rightarrow AB\perp BC.\\
\Step{3}\;& AB\parallel CD\ (\text{both horizontal}),\ BC\parallel AD\ (\text{both vertical})
\Rightarrow \textbf{$ABCD$ is a rectangle.}\\[4pt]
\Step{4}\;& \triangle ABC:\ AB\perp BC \Rightarrow \angle ABC=90^\circ \Rightarrow \triangle ABC \text{ is right-angled.}\\
\Step{5}\;& \triangle ABD:\ \overrightarrow{AD}=(2-2,3-(-1))=(0,4),\ \overrightarrow{AB}=(6,0),\\
&\overrightarrow{AB}\cdot\overrightarrow{AD}=6\cdot 0+0\cdot 4=0 \Rightarrow AB\perp AD\\
&\Rightarrow \angle BAD=90^\circ \Rightarrow \triangle ABD \text{ is right-angled.}
\end{aligned}
\]
\end{QAPair}

\end{document}
