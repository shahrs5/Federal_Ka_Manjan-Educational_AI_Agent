% !TEX TS-program = pdflatex
\documentclass[11pt]{article}

% -------------------- Packages --------------------
\usepackage[a4paper,margin=1in]{geometry}
\usepackage{amsmath,amssymb}
\usepackage[T1]{fontenc}
\usepackage{lmodern}
\usepackage{xcolor}
\usepackage{tcolorbox}
\tcbuselibrary{skins,breakable}
\usepackage{enumitem}
\usepackage{hyperref}

\pagestyle{empty}

% -------------------- Dark Theme Colors --------------------
\definecolor{bg}{HTML}{000000}
\definecolor{pairbg}{HTML}{121212}
\definecolor{solbg}{HTML}{0A0A0A}
\definecolor{border}{HTML}{2A2A2A}
\definecolor{text}{HTML}{FFFFFF}
\definecolor{muted}{HTML}{C9CDD3}
\definecolor{gold}{HTML}{FFD700}
\definecolor{green}{HTML}{4ADE80}
\definecolor{cyan}{HTML}{38BDF8}

\pagecolor{bg}
\color{text}

\hypersetup{
  colorlinks=true,
  linkcolor=cyan,
  urlcolor=cyan
}

\setlength{\parindent}{0pt}
\setlength{\parskip}{10pt}

\setlist[itemize]{left=1.4em,itemsep=6pt,topsep=6pt}
\setlist[enumerate]{left=1.6em,itemsep=4pt,topsep=4pt}

% -------------------- tcolorbox Base --------------------
\tcbset{
  enhanced,
  breakable,
  arc=12pt,
  boxrule=0.8pt,
  left=16pt,right=16pt,top=12pt,bottom=12pt
}

\newtcolorbox{QAPair}[1]{%
  colback=pairbg,
  colbacklower=solbg,
  colframe=border,
  coltext=text,
  title=\textcolor{gold}{\bfseries #1},
  fonttitle=\bfseries,
  coltitle=text,
  segmentation style={draw=border, dashed, line width=0.6pt},
}

% Visible text inside this box (fix)
\newtcolorbox{QuickBox}{%
  colback=pairbg,
  colframe=cyan,
  coltext=text,
  fontupper=\color{text},
  borderline north={4pt}{0pt}{cyan},
  arc=14pt,
  boxrule=0.8pt
}

% Helper for step headings
\newcommand{\Step}[1]{\textcolor{muted}{\textbf{Step #1:}}}

% ============================================================
\begin{document}

\begin{center}
{\LARGE\bfseries \textcolor{gold}{Exercise 5.1 --- Solutions}}\\[-2pt]
\end{center}

\begin{QuickBox}
{\color{cyan}\bfseries Quick tips for linear equations}\par\medskip
\begin{itemize}
\item \textbf{Distribute:} $k(a\pm b)=ka\pm kb$.
\item \textbf{Collect like terms:} move all variables to one side, constants to the other.
\item \textbf{Clear fractions:} multiply both sides by the LCM of denominators.
\item \textbf{Decimals:} convert to fractions (e.g., $0.3=\tfrac{3}{10}$) or multiply by $10,100,\dots$
\item \textbf{Final check:} substitute the value back (optional).
\end{itemize}
\end{QuickBox}

% ============================================================
\begin{QAPair}{Question 1}
\textcolor{gold}{\bfseries Question:} $5x-2-x=4-3x-27$\\
\tcblower
\textcolor{green}{\bfseries Answer:}
\[
\begin{aligned}
\Step{1}\;& 5x-2-x=4x-2,\qquad 4-3x-27=-3x-23.\\
\Step{2}\;& 4x-2=-3x-23.\\
\Step{3}\;& 4x+3x=-23+2 \;\Rightarrow\; 7x=-21.\\
\Step{4}\;& x=\boxed{-3}.
\end{aligned}
\]
\end{QAPair}

\begin{QAPair}{Question 2}
\textcolor{gold}{\bfseries Question:} $4a-3(5a-14)=5(7+a)-9$\\
\tcblower
\textcolor{green}{\bfseries Answer:}
\[
\begin{aligned}
\Step{1}\;& 4a-3(5a-14)=4a-15a+42=-11a+42.\\
\Step{2}\;& 5(7+a)-9=35+5a-9=26+5a.\\
\Step{3}\;& -11a+42=5a+26 \;\Rightarrow\; -16a=-16.\\
\Step{4}\;& a=\boxed{1}.
\end{aligned}
\]
\end{QAPair}

\begin{QAPair}{Question 3}
\textcolor{gold}{\bfseries Question:} $7(2-5x)+27=18x-3(8-4x)$\\
\tcblower
\textcolor{green}{\bfseries Answer:}
\[
\begin{aligned}
\Step{1}\;& 7(2-5x)+27=14-35x+27=41-35x.\\
\Step{2}\;& 18x-3(8-4x)=18x-24+12x=30x-24.\\
\Step{3}\;& 41-35x=30x-24 \;\Rightarrow\; 41=65x-24.\\
\Step{4}\;& 65=65x \;\Rightarrow\; x=\boxed{1}.
\end{aligned}
\]
\end{QAPair}

\begin{QAPair}{Question 4}
\textcolor{gold}{\bfseries Question:} $\dfrac{5x}{4}+\dfrac12=0$\\
\tcblower
\textcolor{green}{\bfseries Answer:}
\[
\begin{aligned}
\Step{1}\;& \text{Multiply both sides by }4:\\
& 5x+2=0.\\
\Step{2}\;& 5x=-2 \;\Rightarrow\; x=\boxed{-\frac{2}{5}}.
\end{aligned}
\]
\end{QAPair}

\begin{QAPair}{Question 5}
\textcolor{gold}{\bfseries Question:} $\dfrac{x-2}{2}+\dfrac{x+10}{9}=5$\\
\tcblower
\textcolor{green}{\bfseries Answer:}
\[
\begin{aligned}
\Step{1}\;& \text{LCM}(2,9)=18.\ \text{Multiply both sides by }18:\\
& 9(x-2)+2(x+10)=90.\\
\Step{2}\;& 9x-18+2x+20=90 \;\Rightarrow\; 11x+2=90.\\
\Step{3}\;& 11x=88 \;\Rightarrow\; x=\boxed{8}.
\end{aligned}
\]
\end{QAPair}

\begin{QAPair}{Question 6}
\textcolor{gold}{\bfseries Question:} $\dfrac{4(x+2)}{3}-\dfrac{6(x-7)}{7}=12$\\
\tcblower
\textcolor{green}{\bfseries Answer:}
\[
\begin{aligned}
\Step{1}\;& \text{LCM}(3,7)=21.\ \text{Multiply both sides by }21:\\
& 28(x+2)-18(x-7)=252.\\
\Step{2}\;& 28x+56-18x+126=252 \;\Rightarrow\; 10x+182=252.\\
\Step{3}\;& 10x=70 \;\Rightarrow\; x=\boxed{7}.
\end{aligned}
\]
\end{QAPair}

\begin{QAPair}{Question 7}
\textcolor{gold}{\bfseries Question:} $\dfrac{x}{2}+\dfrac{x}{3}-\dfrac{x}{4}+\dfrac{x}{5}=7\dfrac{5}{6}$\\
\tcblower
\textcolor{green}{\bfseries Answer:}
\[
\begin{aligned}
\Step{1}\;& 7\dfrac{5}{6}=\frac{47}{6}.\\
\Step{2}\;& \frac{x}{2}+\frac{x}{3}-\frac{x}{4}+\frac{x}{5}
=\frac{30x+20x-15x+12x}{60}=\frac{47x}{60}.\\
\Step{3}\;& \frac{47x}{60}=\frac{47}{6}.\\
\Step{4}\;& \text{Multiply both sides by }60:\ 47x=470 \;\Rightarrow\; x=\boxed{10}.
\end{aligned}
\]
\end{QAPair}

\begin{QAPair}{Question 8}
\textcolor{gold}{\bfseries Question:} $\dfrac{y+1}{3}+\dfrac{y+1}{2}=2-\dfrac{y+3}{2}$\\
\tcblower
\textcolor{green}{\bfseries Answer:}
\[
\begin{aligned}
\Step{1}\;& \text{LCM}(3,2)=6.\ \text{Multiply both sides by }6:\\
& 2(y+1)+3(y+1)=12-3(y+3).\\
\Step{2}\;& 5(y+1)=12-3y-9=3-3y.\\
\Step{3}\;& 5y+5=3-3y \;\Rightarrow\; 8y=-2.\\
\Step{4}\;& y=\boxed{-\frac{1}{4}}.
\end{aligned}
\]
\end{QAPair}

\begin{QAPair}{Question 9}
\textcolor{gold}{\bfseries Question:} $\dfrac{1}{5}(x-8)+\dfrac{4+x}{7}=7-\dfrac{23-x}{5}$\\
\tcblower
\textcolor{green}{\bfseries Answer:}
\[
\begin{aligned}
\Step{1}\;& \frac{1}{5}(x-8)=\frac{x-8}{5},\qquad
7-\frac{23-x}{5}=7-\frac{23}{5}+\frac{x}{5}=\frac{x+12}{5}.\\
\Step{2}\;& \frac{x-8}{5}+\frac{x+4}{7}=\frac{x+12}{5}.\\
\Step{3}\;& \frac{x+4}{7}=\frac{x+12}{5}-\frac{x-8}{5}=\frac{20}{5}=4.\\
\Step{4}\;& x+4=28 \;\Rightarrow\; x=\boxed{24}.
\end{aligned}
\]
\end{QAPair}

\begin{QAPair}{Question 10}
\textcolor{gold}{\bfseries Question:} $\dfrac{1}{2y}-\dfrac{1}{6}=\dfrac{1}{4y}-1-\dfrac{1}{y}\,;\; y\neq 0$\\
\tcblower
\textcolor{green}{\bfseries Answer:}
\[
\begin{aligned}
\Step{1}\;& \text{Multiply both sides by }12y\ (y\neq 0):\\
& 12y\left(\frac{1}{2y}\right)-12y\left(\frac{1}{6}\right)
=12y\left(\frac{1}{4y}\right)-12y(1)-12y\left(\frac{1}{y}\right).\\
\Step{2}\;& 6-2y=3-12y-12=-9-12y.\\
\Step{3}\;& 6+10y=-9 \;\Rightarrow\; 10y=-15.\\
\Step{4}\;& y=\boxed{-\frac{3}{2}} \quad (\text{satisfies }y\neq 0).
\end{aligned}
\]
\end{QAPair}

\begin{QAPair}{Question 11}
\textcolor{gold}{\bfseries Question:} $4-0.3(1-x)=7$\\
\tcblower
\textcolor{green}{\bfseries Answer:}
\[
\begin{aligned}
\Step{1}\;& 0.3=\frac{3}{10}. \ \Rightarrow\ 4-\frac{3}{10}(1-x)=7.\\
\Step{2}\;& 4-\frac{3}{10}+\frac{3}{10}x=7
\;\Rightarrow\; \frac{37}{10}+\frac{3}{10}x=7.\\
\Step{3}\;& \frac{3}{10}x=7-\frac{37}{10}=\frac{33}{10}.\\
\Step{4}\;& x=\frac{\frac{33}{10}}{\frac{3}{10}}=11 \;\Rightarrow\; x=\boxed{11}.
\end{aligned}
\]
\end{QAPair}

\begin{QAPair}{Question 12}
\textcolor{gold}{\bfseries Question:} $0.5x=6.3-0.2x$\\
\tcblower
\textcolor{green}{\bfseries Answer:}
\[
\begin{aligned}
\Step{1}\;& 0.5=\frac{1}{2},\ 6.3=\frac{63}{10},\ 0.2=\frac{1}{5}.\\
\Step{2}\;& \frac{1}{2}x=\frac{63}{10}-\frac{1}{5}x
\;\Rightarrow\; \left(\frac{1}{2}+\frac{1}{5}\right)x=\frac{63}{10}.\\
\Step{3}\;& \frac{7}{10}x=\frac{63}{10}\;\Rightarrow\; x=9.\\
\Step{4}\;& x=\boxed{9}.
\end{aligned}
\]
\end{QAPair}

\begin{QAPair}{Question 13}
\textcolor{gold}{\bfseries Question:} $1.3x-0.2=0.3x-1.5$\\
\tcblower
\textcolor{green}{\bfseries Answer:}
\[
\begin{aligned}
\Step{1}\;& 1.3=\frac{13}{10},\ 0.2=\frac{1}{5},\ 0.3=\frac{3}{10},\ 1.5=\frac{3}{2}.\\
\Step{2}\;& \frac{13}{10}x-\frac{1}{5}=\frac{3}{10}x-\frac{3}{2}.\\
\Step{3}\;& \left(\frac{13}{10}-\frac{3}{10}\right)x=-\frac{3}{2}+\frac{1}{5}
\;\Rightarrow\; x=-\frac{15}{10}+\frac{2}{10}=-\frac{13}{10}.\\
\Step{4}\;& x=\boxed{-\frac{13}{10}}.
\end{aligned}
\]
\end{QAPair}

\end{document}
