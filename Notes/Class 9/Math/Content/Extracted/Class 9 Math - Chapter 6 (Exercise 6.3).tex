% !TEX TS-program = pdflatex
\documentclass[11pt]{article}

% -------------------- Packages --------------------
\usepackage[a4paper,margin=1in]{geometry}
\usepackage{amsmath,amssymb}
\usepackage[T1]{fontenc}
\usepackage{lmodern}
\usepackage{xcolor}
\usepackage{tcolorbox}
\tcbuselibrary{skins,breakable}
\usepackage{enumitem}
\usepackage{hyperref}

\pagestyle{empty}

% -------------------- Dark Theme Colors --------------------
\definecolor{bg}{HTML}{000000}
\definecolor{pairbg}{HTML}{121212}
\definecolor{solbg}{HTML}{0A0A0A}
\definecolor{border}{HTML}{2A2A2A}
\definecolor{text}{HTML}{FFFFFF}
\definecolor{muted}{HTML}{C9CDD3}
\definecolor{gold}{HTML}{FFD700}
\definecolor{green}{HTML}{4ADE80}
\definecolor{cyan}{HTML}{38BDF8}

\pagecolor{bg}
\color{text}

\hypersetup{
  colorlinks=true,
  linkcolor=cyan,
  urlcolor=cyan
}

\setlength{\parindent}{0pt}
\setlength{\parskip}{10pt}

\setlist[itemize]{left=1.4em,itemsep=6pt,topsep=6pt}
\setlist[enumerate]{left=1.6em,itemsep=4pt,topsep=4pt}

% -------------------- tcolorbox Base --------------------
\tcbset{
  enhanced,
  breakable,
  arc=12pt,
  boxrule=0.8pt,
  left=16pt,right=16pt,top=12pt,bottom=12pt
}

\newtcolorbox{QAPair}[1]{%
  colback=pairbg,
  colbacklower=solbg,
  colframe=border,
  coltext=text,
  title=\textcolor{gold}{\bfseries #1},
  fonttitle=\bfseries,
  coltitle=text,
  segmentation style={draw=border, dashed, line width=0.6pt},
}

% Visible text inside this box (fix)
\newtcolorbox{QuickBox}{%
  colback=pairbg,
  colframe=cyan,
  coltext=text,
  fontupper=\color{text},
  borderline north={4pt}{0pt}{cyan},
  arc=14pt,
  boxrule=0.8pt
}

% Helper for step headings
\newcommand{\Step}[1]{\textcolor{muted}{\textbf{Step #1:}}}

% ============================================================
\begin{document}

\begin{center}
{\LARGE\bfseries \textcolor{gold}{Exercise 6.3 --- Solutions}}\\[-2pt]
\end{center}

\begin{QuickBox}
{\color{cyan}\bfseries Quick formulas (useful)}\par\medskip
\begin{itemize}
\item \textbf{Reciprocal identities:} $\csc\theta=\dfrac1{\sin\theta}$,\; $\sec\theta=\dfrac1{\cos\theta}$,\; $\cot\theta=\dfrac1{\tan\theta}$.
\item \textbf{Quotient identities:} $\tan\theta=\dfrac{\sin\theta}{\cos\theta}$,\; $\cot\theta=\dfrac{\cos\theta}{\sin\theta}$.
\item \textbf{Pythagorean:} $\sin^2\theta+\cos^2\theta=1$,\; $\sec^2\theta-\tan^2\theta=1$,\; $\csc^2\theta=1+\cot^2\theta$.
\item \textbf{Compound angle:} $\tan(A-B)=\dfrac{\tan A-\tan B}{1+\tan A\,\tan B}$,\quad $\sin(A-B)=\sin A\cos B-\cos A\sin B$.
\end{itemize}
\end{QuickBox}

% ============================================================
% Q1
\begin{QAPair}{Question 1}
\textcolor{gold}{\bfseries Question:} Complete the following table.\\
\tcblower
\textcolor{green}{\bfseries Answer:}

\renewcommand{\arraystretch}{1.5}
\[
\begin{array}{c|c|c|c|c|c|c|c|c}
\theta & 0^\circ & 30^\circ & 45^\circ & 60^\circ & 90^\circ & 180^\circ & 270^\circ & 360^\circ\\
\hline
\sin\theta & 0 & \dfrac12 & \dfrac{\sqrt2}{2} & \dfrac{\sqrt3}{2} & 1 & 0 & -1 & 0\\
\hline
\cos\theta & 1 & \dfrac{\sqrt3}{2} & \dfrac{\sqrt2}{2} & \dfrac12 & 0 & -1 & 0 & 1\\
\hline
\tan\theta & 0 & \dfrac{\sqrt3}{3} & 1 & \sqrt3 & \infty & 0 & \infty & 0\\
\hline
\csc\theta & \infty & 2 & \sqrt2 & \dfrac{2\sqrt3}{3} & 1 & \infty & -1 & \infty\\
\hline
\sec\theta & 1 & \dfrac{2\sqrt3}{3} & \sqrt2 & 2 & \infty & -1 & \infty & 1\\
\hline
\cot\theta & \infty & \sqrt3 & 1 & \dfrac{\sqrt3}{3} & 0 & \infty & 0 & \infty
\end{array}
\]
\end{QAPair}

% ============================================================
% Q2
\begin{QAPair}{Question 2 (i)}
\textcolor{gold}{\bfseries Question:} $\sin 60^\circ-\cos 30^\circ$\\
\tcblower
\textcolor{green}{\bfseries Answer:}
\[
\begin{aligned}
\Step{1}\;& \sin 60^\circ=\frac{\sqrt3}{2},\quad \cos 30^\circ=\frac{\sqrt3}{2}.\\
\Step{2}\;& \sin 60^\circ-\cos 30^\circ=\frac{\sqrt3}{2}-\frac{\sqrt3}{2}=0.
\end{aligned}
\]
\end{QAPair}

\begin{QAPair}{Question 2 (ii)}
\textcolor{gold}{\bfseries Question:} $\sec 45^\circ\,\csc 45^\circ-\sin 30^\circ\,\cos 60^\circ$\\
\tcblower
\textcolor{green}{\bfseries Answer:}
\[
\begin{aligned}
\Step{1}\;& \sec 45^\circ=\sqrt2,\quad \csc 45^\circ=\sqrt2 \;\Rightarrow\; \sec 45^\circ\,\csc 45^\circ=2.\\
\Step{2}\;& \sin 30^\circ=\frac12,\quad \cos 60^\circ=\frac12 \;\Rightarrow\; \sin 30^\circ\cos 60^\circ=\frac14.\\
\Step{3}\;& 2-\frac14=\frac{7}{4}.
\end{aligned}
\]
\end{QAPair}

\begin{QAPair}{Question 2 (iii)}
\textcolor{gold}{\bfseries Question:} $\displaystyle \frac{\tan\frac{\pi}{3}-\tan\frac{\pi}{6}}{1+\tan\frac{\pi}{3}\tan\frac{\pi}{6}}$\\
\tcblower
\textcolor{green}{\bfseries Answer:}
\[
\begin{aligned}
\Step{1}\;& \tan\frac{\pi}{3}=\sqrt3,\quad \tan\frac{\pi}{6}=\frac{1}{\sqrt3}.\\
\Step{2}\;& \frac{\sqrt3-\frac{1}{\sqrt3}}{1+\sqrt3\cdot \frac{1}{\sqrt3}}
=\frac{\frac{3-1}{\sqrt3}}{2}
=\frac{\frac{2}{\sqrt3}}{2}
=\frac{1}{\sqrt3}
=\frac{\sqrt3}{3}.
\end{aligned}
\]
\end{QAPair}

\begin{QAPair}{Question 2 (iv)}
\textcolor{gold}{\bfseries Question:} $\cos^2\!\left(\frac{\pi}{3}\right)-\sin^2\!\left(\frac{\pi}{3}\right)$\\
\tcblower
\textcolor{green}{\bfseries Answer:}
\[
\begin{aligned}
\Step{1}\;& \cos\frac{\pi}{3}=\frac12 \Rightarrow \cos^2\frac{\pi}{3}=\frac14,\qquad
\sin\frac{\pi}{3}=\frac{\sqrt3}{2} \Rightarrow \sin^2\frac{\pi}{3}=\frac34.\\
\Step{2}\;& \frac14-\frac34=-\frac12.
\end{aligned}
\]
\end{QAPair}

\begin{QAPair}{Question 2 (v)}
\textcolor{gold}{\bfseries Question:} $\sin\frac{\pi}{2}\cos\frac{\pi}{3}-\cos\frac{\pi}{2}\sin\frac{\pi}{3}$\\
\tcblower
\textcolor{green}{\bfseries Answer:}
\[
\begin{aligned}
\Step{1}\;& \sin\frac{\pi}{2}=1,\quad \cos\frac{\pi}{3}=\frac12 \;\Rightarrow\; \sin\frac{\pi}{2}\cos\frac{\pi}{3}=\frac12.\\
\Step{2}\;& \cos\frac{\pi}{2}=0 \;\Rightarrow\; \cos\frac{\pi}{2}\sin\frac{\pi}{3}=0.\\
\Step{3}\;& \frac12-0=\frac12.
\end{aligned}
\]
\end{QAPair}

% ============================================================
% Q3
\begin{QAPair}{Question 3 (i)}
\textcolor{gold}{\bfseries Question:} For $\theta=\dfrac{\pi}{6}$, verify $\displaystyle \tan 2\theta=\frac{2\tan\theta}{1-\tan^2\theta}$.\\
\tcblower
\textcolor{green}{\bfseries Answer:}
\[
\begin{aligned}
\Step{1}\;& \theta=\frac{\pi}{6}\Rightarrow 2\theta=\frac{\pi}{3},\quad \tan 2\theta=\tan\frac{\pi}{3}=\sqrt3.\\
\Step{2}\;& \tan\theta=\tan\frac{\pi}{6}=\frac{1}{\sqrt3}\Rightarrow \tan^2\theta=\frac{1}{3}.\\
\Step{3}\;& \frac{2\tan\theta}{1-\tan^2\theta}
=\frac{2\cdot \frac{1}{\sqrt3}}{1-\frac13}
=\frac{\frac{2}{\sqrt3}}{\frac{2}{3}}
=\frac{2}{\sqrt3}\cdot\frac{3}{2}
=\frac{3}{\sqrt3}
=\sqrt3.\\
\Step{4}\;& \text{LHS}=\text{RHS}=\sqrt3\;\; \checkmark
\end{aligned}
\]
\end{QAPair}

\begin{QAPair}{Question 3 (ii)}
\textcolor{gold}{\bfseries Question:} For $\theta=\dfrac{\pi}{6}$, verify $\sin 3\theta=3\sin\theta-4\sin^3\theta$.\\
\tcblower
\textcolor{green}{\bfseries Answer:}
\[
\begin{aligned}
\Step{1}\;& 3\theta=\frac{\pi}{2}\Rightarrow \sin 3\theta=\sin\frac{\pi}{2}=1.\\
\Step{2}\;& \sin\theta=\sin\frac{\pi}{6}=\frac12 \Rightarrow \sin^3\theta=\left(\frac12\right)^3=\frac18.\\
\Step{3}\;& 3\sin\theta-4\sin^3\theta
=3\cdot\frac12-4\cdot\frac18
=\frac{3}{2}-\frac{1}{2}
=1.\\
\Step{4}\;& \text{LHS}=\text{RHS}=1\;\; \checkmark
\end{aligned}
\]
\end{QAPair}

\begin{QAPair}{Question 3 (iii)}
\textcolor{gold}{\bfseries Question:} For $\theta=\dfrac{\pi}{6}$, verify $\cos 2\theta=1-2\sin^2\theta$.\\
\tcblower
\textcolor{green}{\bfseries Answer:}
\[
\begin{aligned}
\Step{1}\;& \cos 2\theta=\cos\frac{\pi}{3}=\frac12.\\
\Step{2}\;& \sin\theta=\frac12 \Rightarrow \sin^2\theta=\frac14.\\
\Step{3}\;& 1-2\sin^2\theta=1-2\cdot\frac14=1-\frac12=\frac12.\\
\Step{4}\;& \text{LHS}=\text{RHS}=\frac12\;\; \checkmark
\end{aligned}
\]
\end{QAPair}

\begin{QAPair}{Question 3 (iv)}
\textcolor{gold}{\bfseries Question:} For $\theta=\dfrac{\pi}{6}$, verify $\sec^2\theta-\tan^2\theta=1$.\\
\tcblower
\textcolor{green}{\bfseries Answer:}
\[
\begin{aligned}
\Step{1}\;& \cos\theta=\cos\frac{\pi}{6}=\frac{\sqrt3}{2}
\Rightarrow \sec\theta=\frac{1}{\cos\theta}=\frac{2}{\sqrt3}
\Rightarrow \sec^2\theta=\frac{4}{3}.\\
\Step{2}\;& \tan\theta=\tan\frac{\pi}{6}=\frac{1}{\sqrt3}
\Rightarrow \tan^2\theta=\frac{1}{3}.\\
\Step{3}\;& \sec^2\theta-\tan^2\theta=\frac{4}{3}-\frac{1}{3}=1\;\; \checkmark
\end{aligned}
\]
\end{QAPair}

\begin{QAPair}{Question 3 (v)}
\textcolor{gold}{\bfseries Question:} For $\theta=\dfrac{\pi}{6}$, verify $\csc^2\theta=1+\cot^2\theta$.\\
\tcblower
\textcolor{green}{\bfseries Answer:}
\[
\begin{aligned}
\Step{1}\;& \sin\theta=\sin\frac{\pi}{6}=\frac12
\Rightarrow \csc\theta=\frac{1}{\sin\theta}=2
\Rightarrow \csc^2\theta=4.\\
\Step{2}\;& \cot\theta=\frac{\cos\theta}{\sin\theta}
=\frac{\frac{\sqrt3}{2}}{\frac12}=\sqrt3
\Rightarrow \cot^2\theta=3.\\
\Step{3}\;& 1+\cot^2\theta=1+3=4.\\
\Step{4}\;& \text{LHS}=\text{RHS}=4\;\; \checkmark
\end{aligned}
\]
\end{QAPair}

% ============================================================
% Q4
\begin{QAPair}{Question 4 (i)}
\textcolor{gold}{\bfseries Question:} Find $\theta$ if $\sin\theta=\dfrac{\sqrt2}{2}$,\; $(0<\theta<2\pi)$.\\
\tcblower
\textcolor{green}{\bfseries Answer:}
\[
\sin\theta=\frac{\sqrt2}{2}\;\Rightarrow\;
\theta=\frac{\pi}{4},\;\frac{3\pi}{4}.
\]
\end{QAPair}

\begin{QAPair}{Question 4 (ii)}
\textcolor{gold}{\bfseries Question:} Find $\theta$ if $\csc\theta=-1$,\; $(0<\theta<2\pi)$.\\
\tcblower
\textcolor{green}{\bfseries Answer:}
\[
\csc\theta=-1 \Rightarrow \sin\theta=-1 \Rightarrow \theta=\frac{3\pi}{2}.
\]
\end{QAPair}

\begin{QAPair}{Question 4 (iii)}
\textcolor{gold}{\bfseries Question:} Find $\theta$ if $\tan\theta=1$,\; $(0<\theta<2\pi)$.\\
\tcblower
\textcolor{green}{\bfseries Answer:}
\[
\tan\theta=1\;\Rightarrow\;\theta=\frac{\pi}{4},\;\frac{5\pi}{4}.
\]
\end{QAPair}

\begin{QAPair}{Question 4 (iv)}
\textcolor{gold}{\bfseries Question:} Find $\theta$ if $\sec\theta=\dfrac{\sqrt{12}}{3}$,\; $(0<\theta<2\pi)$.\\
\tcblower
\textcolor{green}{\bfseries Answer:}
\[
\begin{aligned}
\Step{1}\;& \sec\theta=\frac{\sqrt{12}}{3}=\frac{2\sqrt3}{3}
\;\Rightarrow\; \cos\theta=\frac{1}{\sec\theta}=\frac{3}{2\sqrt3}=\frac{\sqrt3}{2}.\\
\Step{2}\;& \cos\theta=\frac{\sqrt3}{2}\;\Rightarrow\; \theta=\frac{\pi}{6},\;\frac{11\pi}{6}.
\end{aligned}
\]
\end{QAPair}

\begin{QAPair}{Question 4 (v)}
\textcolor{gold}{\bfseries Question:} Find $\theta$ if $\cos\theta=0$,\; $(0<\theta<2\pi)$.\\
\tcblower
\textcolor{green}{\bfseries Answer:}
\[
\cos\theta=0\;\Rightarrow\;\theta=\frac{\pi}{2},\;\frac{3\pi}{2}.
\]
\end{QAPair}

% ============================================================
% Q5
\begin{QAPair}{Question 5}
\textcolor{gold}{\bfseries Question:} If terminal ray of $\theta$ is in first quadrant and $\cos\theta=\dfrac12$, find remaining trigonometric ratios of $\theta$.\\
\tcblower
\textcolor{green}{\bfseries Answer:}
\[
\begin{aligned}
\Step{1}\;& \cos\theta=\frac{\text{adj}}{\text{hyp}}=\frac12
\;\Rightarrow\; \text{adj}=1,\;\text{hyp}=2.\\
\Step{2}\;& \text{opp}=\sqrt{\text{hyp}^2-\text{adj}^2}=\sqrt{4-1}=\sqrt3.\\
\Step{3}\;&
\sin\theta=\frac{\sqrt3}{2},\quad
\tan\theta=\sqrt3,\quad
\csc\theta=\frac{2}{\sqrt3}=\frac{2\sqrt3}{3},\\
&\sec\theta=2,\quad
\cot\theta=\frac{1}{\sqrt3}=\frac{\sqrt3}{3}.
\end{aligned}
\]
\end{QAPair}

% ============================================================
% Q6
\begin{QAPair}{Question 6}
\textcolor{gold}{\bfseries Question:} If terminal ray of $\theta$ is in second quadrant and $\csc\theta=2$, find the value of
\[
\cos\theta\left(\frac{\sin^2\theta-\cos^2\theta}{\sin^2\theta+\cos^2\theta}\right).
\]
\tcblower
\textcolor{green}{\bfseries Answer:}
\[
\begin{aligned}
\Step{1}\;& \csc\theta=2 \Rightarrow \sin\theta=\frac12 \Rightarrow \sin^2\theta=\frac14.\\
\Step{2}\;& \sin^2\theta+\cos^2\theta=1 \Rightarrow \cos^2\theta=1-\frac14=\frac34
\Rightarrow \cos\theta=-\frac{\sqrt3}{2}\quad(\text{QII}).\\
\Step{3}\;& \frac{\sin^2\theta-\cos^2\theta}{\sin^2\theta+\cos^2\theta}
=\frac{\frac14-\frac34}{1}=-\frac12.\\
\Step{4}\;& \cos\theta\left(-\frac12\right)=\left(-\frac{\sqrt3}{2}\right)\left(-\frac12\right)=\frac{\sqrt3}{4}.
\end{aligned}
\]
\end{QAPair}

% ============================================================
% Q7
\begin{QAPair}{Question 7}
\textcolor{gold}{\bfseries Question:} If $\tan\theta=-1$, and terminal ray of $\theta$ is not in second quadrant, find remaining trigonometric ratios of $\theta$.\\
\tcblower
\textcolor{green}{\bfseries Answer:}
\[
\begin{aligned}
\Step{1}\;& \tan\theta=-1 \text{ occurs in QII or QIV. Not in QII }\Rightarrow \theta \text{ is in QIV.}\\
\Step{2}\;& \text{Reference angle }=\frac{\pi}{4}\Rightarrow
\sin\theta=-\frac{\sqrt2}{2},\quad \cos\theta=\frac{\sqrt2}{2}.\\
\Step{3}\;&
\csc\theta=\frac{1}{\sin\theta}=-\sqrt2,\quad
\sec\theta=\frac{1}{\cos\theta}=\sqrt2,\quad
\cot\theta=\frac{1}{\tan\theta}=-1.
\end{aligned}
\]
\end{QAPair}

\end{document}
