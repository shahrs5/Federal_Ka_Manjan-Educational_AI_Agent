% !TEX TS-program = pdflatex
\documentclass[11pt]{article}

% -------------------- Packages --------------------
\usepackage[a4paper,margin=1in]{geometry}
\usepackage{amsmath,amssymb}
\usepackage[T1]{fontenc}
\usepackage{lmodern}
\usepackage{xcolor}
\usepackage{tcolorbox}
\tcbuselibrary{skins,breakable}
\usepackage{enumitem}
\usepackage{hyperref}

\pagestyle{empty}

% -------------------- Dark Theme Colors --------------------
\definecolor{bg}{HTML}{000000}
\definecolor{pairbg}{HTML}{121212}
\definecolor{solbg}{HTML}{0A0A0A}
\definecolor{border}{HTML}{2A2A2A}
\definecolor{text}{HTML}{FFFFFF}
\definecolor{muted}{HTML}{C9CDD3}
\definecolor{gold}{HTML}{FFD700}
\definecolor{green}{HTML}{4ADE80}
\definecolor{cyan}{HTML}{38BDF8}

\pagecolor{bg}
\color{text}

\hypersetup{
  colorlinks=true,
  linkcolor=cyan,
  urlcolor=cyan
}

\setlength{\parindent}{0pt}
\setlength{\parskip}{10pt}

\setlist[itemize]{left=1.4em,itemsep=6pt,topsep=6pt}
\setlist[enumerate]{left=1.6em,itemsep=4pt,topsep=4pt}

% -------------------- tcolorbox Base --------------------
\tcbset{
  enhanced,
  breakable,
  arc=12pt,
  boxrule=0.8pt,
  left=16pt,right=16pt,top=12pt,bottom=12pt
}

\newtcolorbox{QAPair}[1]{%
  colback=pairbg,
  colbacklower=solbg,
  colframe=border,
  coltext=text,
  title=\textcolor{gold}{\bfseries #1},
  fonttitle=\bfseries,
  coltitle=text,
  segmentation style={draw=border, dashed, line width=0.6pt},
}

% Visible text inside this box (fix)
\newtcolorbox{QuickBox}{%
  colback=pairbg,
  colframe=cyan,
  coltext=text,
  fontupper=\color{text},
  borderline north={4pt}{0pt}{cyan},
  arc=14pt,
  boxrule=0.8pt
}

% Helper for step headings
\newcommand{\Step}[1]{\textcolor{muted}{\textbf{Step #1:}}}

% ============================================================
\begin{document}

\begin{center}
{\LARGE\bfseries \textcolor{gold}{Exercise 2.1 --- Solutions}}\\[-2pt]
\end{center}

\begin{QuickBox}
{\color{cyan}\bfseries Quick formulas (useful)}\par\medskip
\begin{itemize}
\item \textbf{Scientific notation:} $a\times 10^n$ where $1\le a<10$ and $n\in\mathbb{Z}$.
\item \textbf{Shift rule:} multiplying by $10^n$ shifts the decimal $n$ places (right if $n>0$, left if $n<0$).
\item \textbf{Multiply:} $(a\times 10^m)(b\times 10^n)=(ab)\times 10^{m+n}$.
\item \textbf{Divide:} $\dfrac{a\times 10^m}{b\times 10^n}=\left(\dfrac{a}{b}\right)\times 10^{m-n}$.
\item \textbf{Time:} $60\text{ s}=1\text{ min}$,\; $3600\text{ s}=1\text{ h}$,\; $24\text{ h}=1\text{ day}$.
\end{itemize}
\end{QuickBox}

% ============================================================
% Q1
\begin{QAPair}{Question 1 (i)}
\textcolor{gold}{\bfseries Question:} Write $0.00053407$ in scientific notation.\\
\tcblower
\textcolor{green}{\bfseries Answer:}
\[
\begin{aligned}
\Step{1}\;& 0.00053407 = 5.3407 \text{ (decimal moved 4 places right)}\\
\Step{2}\;& \Rightarrow\; 0.00053407 = 5.3407\times 10^{-4}.
\end{aligned}
\]
\end{QAPair}

\begin{QAPair}{Question 1 (ii)}
\textcolor{gold}{\bfseries Question:} Write $53400000$ in scientific notation.\\
\tcblower
\textcolor{green}{\bfseries Answer:}
\[
\begin{aligned}
\Step{1}\;& 53400000 = 5.34 \text{ (decimal moved 7 places left)}\\
\Step{2}\;& \Rightarrow\; 53400000 = 5.34\times 10^{7}.
\end{aligned}
\]
\end{QAPair}

\begin{QAPair}{Question 1 (iii)}
\textcolor{gold}{\bfseries Question:} Write $0.000000000012$ in scientific notation.\\
\tcblower
\textcolor{green}{\bfseries Answer:}
\[
\begin{aligned}
\Step{1}\;& 0.000000000012 = 1.2 \text{ (decimal moved 11 places right)}\\
\Step{2}\;& \Rightarrow\; 0.000000000012 = 1.2\times 10^{-11}.
\end{aligned}
\]
\end{QAPair}

\begin{QAPair}{Question 1 (iv)}
\textcolor{gold}{\bfseries Question:} Write $2.5326$ in scientific notation.\\
\tcblower
\textcolor{green}{\bfseries Answer:}
\[
\begin{aligned}
\Step{1}\;& 2.5326 \text{ already has } 1\le a<10.\\
\Step{2}\;& \Rightarrow\; 2.5326 = 2.5326\times 10^{0}.
\end{aligned}
\]
\end{QAPair}

% ============================================================
% Q2
\begin{QAPair}{Question 2 (i)}
\textcolor{gold}{\bfseries Question:} Write $9.067\times 10^{-5}$ in standard notation.\\
\tcblower
\textcolor{green}{\bfseries Answer:}
\[
\begin{aligned}
\Step{1}\;& 10^{-5} \Rightarrow \text{move decimal 5 places left.}\\
\Step{2}\;& 9.067\times 10^{-5}=0.00009067.
\end{aligned}
\]
\end{QAPair}

\begin{QAPair}{Question 2 (ii)}
\textcolor{gold}{\bfseries Question:} Write $5.64\times 10^{0}$ in standard notation.\\
\tcblower
\textcolor{green}{\bfseries Answer:}
\[
\Step{1}\; 5.64\times 10^{0}=5.64.
\]
\end{QAPair}

\begin{QAPair}{Question 2 (iii)}
\textcolor{gold}{\bfseries Question:} Write $6.53\times 10^{-6}$ in standard notation.\\
\tcblower
\textcolor{green}{\bfseries Answer:}
\[
\begin{aligned}
\Step{1}\;& 10^{-6} \Rightarrow \text{move decimal 6 places left.}\\
\Step{2}\;& 6.53\times 10^{-6}=0.00000653.
\end{aligned}
\]
\end{QAPair}

\begin{QAPair}{Question 2 (iv)}
\textcolor{gold}{\bfseries Question:} Write $3.1415\times 10^{9}$ in standard notation.\\
\tcblower
\textcolor{green}{\bfseries Answer:}
\[
\begin{aligned}
\Step{1}\;& 10^{9} \Rightarrow \text{move decimal 9 places right.}\\
\Step{2}\;& 3.1415\times 10^{9}=3{,}141{,}500{,}000.
\end{aligned}
\]
\end{QAPair}

% ============================================================
% Q3
\begin{QAPair}{Question 3 (i)}
\textcolor{gold}{\bfseries Question:} Simplify $563.71\times 10^{-3}\times 2.54\times 10^{4}$ and write in \textbf{scientific notation}.\\
\tcblower
\textcolor{green}{\bfseries Answer:}
\[
\begin{aligned}
\Step{1}\;& (563.71\times 10^{-3})(2.54\times 10^{4})
=(563.71\cdot 2.54)\times 10^{-3+4}\\
\Step{2}\;&=1431.8234\times 10^{1}=14318.234\\
\Step{3}\;&=1.4318234\times 10^{4}.
\end{aligned}
\]
\end{QAPair}

\begin{QAPair}{Question 3 (ii)}
\textcolor{gold}{\bfseries Question:} Simplify $\dfrac{0.023\times 10^{5}}{10^{-3}}$ and write in \textbf{standard notation}.\\
\tcblower
\textcolor{green}{\bfseries Answer:}
\[
\begin{aligned}
\Step{1}\;& \frac{0.023\times 10^{5}}{10^{-3}}
=0.023\times 10^{5-(-3)}=0.023\times 10^{8}\\
\Step{2}\;& 0.023\times 10^{8}=2.3\times 10^{6}\\
\Step{3}\;& \Rightarrow\; \text{standard notation }=2{,}300{,}000.
\end{aligned}
\]
\end{QAPair}

\begin{QAPair}{Question 3 (iii)}
\textcolor{gold}{\bfseries Question:} Simplify $\dfrac{2.549\times 5067\times 10^{-3}}{10^{3}}$ and write in \textbf{scientific notation}.\\
\tcblower
\textcolor{green}{\bfseries Answer:}
\[
\begin{aligned}
\Step{1}\;& \frac{2.549\times 5067\times 10^{-3}}{10^{3}}
=(2.549\times 5067)\times 10^{-3-3}\\
\Step{2}\;&=12915.783\times 10^{-6}=0.012915783\\
\Step{3}\;&=1.2915783\times 10^{-2}.
\end{aligned}
\]
\end{QAPair}

\begin{QAPair}{Question 3 (iv)}
\textcolor{gold}{\bfseries Question:} Simplify $0.0009988\times 10^{10}$ and write in \textbf{standard notation}.\\
\tcblower
\textcolor{green}{\bfseries Answer:}
\[
\begin{aligned}
\Step{1}\;& 0.0009988=9.988\times 10^{-4}.\\
\Step{2}\;& 0.0009988\times 10^{10}=(9.988\times 10^{-4})\times 10^{10}
=9.988\times 10^{6}.\\
\Step{3}\;& \Rightarrow\; \text{standard notation }=9{,}988{,}000.
\end{aligned}
\]
\end{QAPair}

% ============================================================
% Q4
\begin{QAPair}{Question 4}
\textcolor{gold}{\bfseries Question:} If it takes 5 seconds to recite \textit{Kalma Pak} once, how many hours will it take to recite it one million times? Convert hours into days and write the answer in standard form. Round off the answer by discarding the decimal part.\\
\tcblower
\textcolor{green}{\bfseries Answer:}
\[
\begin{aligned}
\Step{1}\;& \text{Total time in seconds} = 5\times 1{,}000{,}000 = 5{,}000{,}000\text{ s}.\\
\Step{2}\;& \text{Hours}=\frac{5{,}000{,}000}{3600}=1388.888\ldots\text{ h}.\\
\Step{3}\;& \text{Days}=\frac{1388.888\ldots}{24}=57.870\ldots\text{ days}.\\
\Step{4}\;& \text{Discard decimal part: } 57.870\ldots \mapsto 57\text{ days}.
\end{aligned}
\]
\end{QAPair}

% ============================================================
% Q5
\begin{QAPair}{Question 5}
\textcolor{gold}{\bfseries Question:} Distance between Earth and Sun is $9.3225600\times 10^{7}$ miles. If speed of light is approximately $1.86\times 10^{5}$ miles per second, how long does it take for light to reach the Earth? Convert the answer into minutes (standard form).\\
\tcblower
\textcolor{green}{\bfseries Answer:}
\[
\begin{aligned}
\Step{1}\;& t=\frac{\text{distance}}{\text{speed}}
=\frac{9.3225600\times 10^{7}}{1.86\times 10^{5}}
=\left(\frac{9.3225600}{1.86}\right)\times 10^{7-5}\\
\Step{2}\;&=5.012129\ldots\times 10^{2}\text{ s}=501.2129\ldots\text{ s}.\\
\Step{3}\;& \text{Minutes}=\frac{501.2129\ldots}{60}=8.353548\ldots\text{ min}.\\
\Step{4}\;& \Rightarrow\; \text{Light takes about } 8.35\text{ minutes (standard form).}
\end{aligned}
\]
\end{QAPair}

\end{document}
