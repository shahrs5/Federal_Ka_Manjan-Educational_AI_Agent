% !TEX TS-program = pdflatex
\documentclass[11pt]{article}

% -------------------- Packages --------------------
\usepackage[a4paper,margin=1in]{geometry}
\usepackage{amsmath,amssymb}
\usepackage[T1]{fontenc}
\usepackage{lmodern}
\usepackage{xcolor}
\usepackage{tcolorbox}
\tcbuselibrary{skins,breakable}
\usepackage{enumitem}
\usepackage{hyperref}
\usepackage{tikz}
\usetikzlibrary{calc,patterns,angles,quotes,arrows.meta,intersections}

\pagestyle{empty}

% -------------------- Dark Theme Colors --------------------
\definecolor{bg}{HTML}{000000}
\definecolor{pairbg}{HTML}{121212}
\definecolor{solbg}{HTML}{0A0A0A}
\definecolor{border}{HTML}{2A2A2A}
\definecolor{text}{HTML}{FFFFFF}
\definecolor{muted}{HTML}{C9CDD3}
\definecolor{gold}{HTML}{FFD700}
\definecolor{green}{HTML}{4ADE80}
\definecolor{cyan}{HTML}{38BDF8}

\pagecolor{bg}
\color{text}

\hypersetup{
  colorlinks=true,
  linkcolor=cyan,
  urlcolor=cyan
}

\setlength{\parindent}{0pt}
\setlength{\parskip}{10pt}

\setlist[itemize]{left=1.4em,itemsep=6pt,topsep=6pt}
\setlist[enumerate]{left=1.6em,itemsep=4pt,topsep=4pt}

% -------------------- tcolorbox Base --------------------
\tcbset{
  enhanced,
  breakable,
  arc=12pt,
  boxrule=0.8pt,
  left=16pt,right=16pt,top=12pt,bottom=12pt
}

\newtcolorbox{QAPair}[1]{%
  colback=pairbg,
  colbacklower=solbg,
  colframe=border,
  coltext=text,
  title=\textcolor{gold}{\bfseries #1},
  fonttitle=\bfseries,
  coltitle=text,
  segmentation style={draw=border, dashed, line width=0.6pt}
}

\newtcolorbox{QuickBox}{%
  colback=pairbg,
  colframe=cyan,
  coltext=text,
  fontupper=\color{text},
  borderline north={4pt}{0pt}{cyan},
  arc=14pt,
  boxrule=0.8pt
}

% Helper for step headings
\newcommand{\Step}[1]{\textcolor{muted}{\textbf{Step #1:}}}

% -------------------- TikZ Styles --------------------
\tikzset{
  geoLine/.style={draw=muted, line width=0.9pt},
  geoBold/.style={draw=text, line width=1.1pt},
  geoDash/.style={draw=muted, dashed, line width=0.9pt},
  geoFill/.style={fill=cyan, fill opacity=0.18, draw=none},
  geoPoint/.style={fill=text, draw=none}, % used with \fill[geoPoint] ... circle(...)
  geoLabel/.style={text=text, font=\small},
  geoSmall/.style={text=text, font=\scriptsize},
}

% ============================================================
\begin{document}

\begin{center}
{\LARGE\bfseries \textcolor{gold}{Miscellaneous Exercise 9 --- Solutions}}\\[-2pt]
\end{center}

\begin{QuickBox}
{\color{cyan}\bfseries Quick formulas (useful)}\par\medskip
\begin{itemize}
\item \textbf{Sum of interior angles (any $n$-gon):} $(n-2)\times 180^\circ$.
\item \textbf{Each interior angle (regular $n$-gon):} $\displaystyle \frac{(n-2)180^\circ}{n}$.
\item \textbf{Sum of exterior angles (any polygon):} $360^\circ$.
\item \textbf{Each exterior angle (regular $n$-gon):} $\displaystyle \frac{360^\circ}{n}$.
\item \textbf{Similar triangles:} corresponding sides are proportional.
\item \textbf{Scaling for similar solids:} linear ratio $k \Rightarrow$ surface area ratio $k^2$, volume ratio $k^3$.
\item \textbf{Angle bisector theorem:} if a line bisects a triangle angle, it divides the opposite side in the ratio of adjacent sides.
\end{itemize}
\end{QuickBox}

% ============================================================
% Q1 (MCQs)

\begin{QAPair}{Question 1 (i) --- MCQ}
\textcolor{gold}{\bfseries Question:} Which of the following is polygon?
\begin{itemize}
\item[(a)] circle \hfill (b) pyramid \hfill (c) quadrilateral \hfill (d) sphere
\end{itemize}
\tcblower
\textcolor{green}{\bfseries Answer:} \textbf{(c) quadrilateral}
\[
\begin{aligned}
\Step{1}\;& \text{A polygon is a \emph{closed plane figure} made of straight line segments.}\\
\Step{2}\;& \text{A quadrilateral is a 4-sided polygon.}
\end{aligned}
\]
\end{QAPair}

\begin{QAPair}{Question 1 (ii) --- MCQ}
\textcolor{gold}{\bfseries Question:} Which of the following is regular polygon?
\begin{itemize}
\item[(a)] kite \hfill (b) rhombus \hfill (c) rectangle \hfill (d) square
\end{itemize}
\tcblower
\textcolor{green}{\bfseries Answer:} \textbf{(d) square}
\[
\begin{aligned}
\Step{1}\;& \text{Regular polygon: all sides equal and all angles equal.}\\
\Step{2}\;& \text{Only a square has both equal sides and equal angles among the options.}
\end{aligned}
\]
\end{QAPair}

\begin{QAPair}{Question 1 (iii) --- MCQ}
\textcolor{gold}{\bfseries Question:} If two triangles are similar, their corresponding sides are:
\begin{itemize}
\item[(a)] proportional \hfill (b) equal \hfill (c) congruent \hfill (d) parallel
\end{itemize}
\tcblower
\textcolor{green}{\bfseries Answer:} \textbf{(a) proportional}
\[
\begin{aligned}
\Step{1}\;& \triangle_1 \sim \triangle_2 \Rightarrow \frac{a_1}{a_2}=\frac{b_1}{b_2}=\frac{c_1}{c_2}.\\
\Step{2}\;& \text{So corresponding sides are proportional.}
\end{aligned}
\]
\end{QAPair}

\begin{QAPair}{Question 1 (iv) --- MCQ}
\textcolor{gold}{\bfseries Question:} What is the sum of interior angles for an irregular hexagon?
\begin{itemize}
\item[(a)] $120^\circ$ \hfill (b) $720^\circ$ \hfill (c) $135^\circ$ \hfill (d) $360^\circ$
\end{itemize}
\tcblower
\textcolor{green}{\bfseries Answer:} \textbf{(b) $720^\circ$}
\[
\begin{aligned}
\Step{1}\;& \text{Sum of interior angles of an $n$-gon}=(n-2)\cdot 180^\circ.\\
\Step{2}\;& n=6 \Rightarrow (6-2)\cdot 180^\circ=4\cdot 180^\circ=720^\circ.
\end{aligned}
\]
\end{QAPair}

\begin{QAPair}{Question 1 (v) --- MCQ}
\textcolor{gold}{\bfseries Question:} What is the sum of interior angles for a regular 12 sided polygon?
\begin{itemize}
\item[(a)] $1800^\circ$ \hfill (b) $2160^\circ$ \hfill (c) $1980^\circ$ \hfill (d) $360^\circ$
\end{itemize}
\tcblower
\textcolor{green}{\bfseries Answer:} \textbf{(a) $1800^\circ$}
\[
\begin{aligned}
\Step{1}\;& (n-2)\cdot 180^\circ \text{ with } n=12.\\
\Step{2}\;& (12-2)\cdot 180^\circ=10\cdot 180^\circ=1800^\circ.
\end{aligned}
\]
\end{QAPair}

\begin{QAPair}{Question 1 (vi) --- MCQ}
\textcolor{gold}{\bfseries Question:} How many sides a regular polygon has if its exterior angle is $15^\circ$?
\begin{itemize}
\item[(a)] 20 \hfill (b) 21 \hfill (c) 24 \hfill (d) 25
\end{itemize}
\tcblower
\textcolor{green}{\bfseries Answer:} \textbf{(c) 24}
\[
\begin{aligned}
\Step{1}\;& \text{Each exterior angle of a regular $n$-gon}=\frac{360^\circ}{n}.\\
\Step{2}\;& \frac{360^\circ}{n}=15^\circ \Rightarrow n=\frac{360}{15}=24.
\end{aligned}
\]
\end{QAPair}

\begin{QAPair}{Question 1 (vii) --- MCQ}
\textcolor{gold}{\bfseries Question:} In the figure, $AB=21$ cm and $P$ divides $AB$ in the ratio $3:4$. What is length of $\overline{AP}$?
\begin{itemize}
\item[(a)] $8$ cm \hfill (b) $10$ cm \hfill (c) $12$ cm \hfill (d) $9$ cm
\end{itemize}

\begin{center}
\begin{tikzpicture}[scale=1]
  \draw[geoLine] (0,0) -- (8,0);
  \node[geoLabel] at (0,-0.35) {$A$};
  \node[geoLabel] at (8,-0.35) {$B$};
  \fill[geoPoint] (0,0) circle (0.8pt);
  \fill[geoPoint] (8,0) circle (0.8pt);
  \fill[geoPoint] (3.2,0) circle (1.2pt);
  \node[geoLabel] at (3.2,-0.35) {$P$};
\end{tikzpicture}
\end{center}

\tcblower
\textcolor{green}{\bfseries Answer:} \textbf{(d) $9$ cm}
\[
\begin{aligned}
\Step{1}\;& AP:PB = 3:4 \Rightarrow AB \text{ has } 3+4=7 \text{ equal parts.}\\
\Step{2}\;& AB=21 \Rightarrow 1\text{ part}=\frac{21}{7}=3.\\
\Step{3}\;& AP=3 \text{ parts}=3\times 3=9\text{ cm.}
\end{aligned}
\]
\end{QAPair}

\begin{QAPair}{Question 1 (viii) --- MCQ}
\textcolor{gold}{\bfseries Question:} In the figure, $\dfrac{a}{b}=\dfrac{c}{d}$. Which one is true?
\begin{itemize}
\item[(a)] $DE=BC$ \hfill (b) $DE>BC$ \hfill (c) $DE\cong BC$ \hfill (d) $DE \parallel BC$
\end{itemize}

\begin{center}
\begin{tikzpicture}[scale=1.0]
  \coordinate (B) at (0,0);
  \coordinate (C) at (6,0);
  \coordinate (A) at (2.8,4.2);
  \coordinate (D) at ($(A)!0.55!(B)$);
  \coordinate (E) at ($(A)!0.55!(C)$);

  \draw[geoLine] (B)--(A)--(C)--cycle;
  \draw[geoBold] (D)--(E);

  \node[geoLabel] at ($(A)+(0,0.35)$) {$A$};
  \node[geoLabel] at ($(B)+(-0.25,-0.35)$) {$B$};
  \node[geoLabel] at ($(C)+(0.25,-0.35)$) {$C$};
  \node[geoLabel] at ($(D)+(-0.3,0.1)$) {$D$};
  \node[geoLabel] at ($(E)+(0.3,0.1)$) {$E$};

  \node[geoSmall, text=muted] at ($(A)!0.25!(D)$) {$a$};
  \node[geoSmall, text=muted] at ($(D)!0.5!(B)$) {$b$};
  \node[geoSmall, text=muted] at ($(A)!0.25!(E)$) {$c$};
  \node[geoSmall, text=muted] at ($(E)!0.5!(C)$) {$d$};
\end{tikzpicture}
\end{center}

\tcblower
\textcolor{green}{\bfseries Answer:} \textbf{(d) $DE \parallel BC$}
\[
\begin{aligned}
\Step{1}\;& \frac{AD}{DB}=\frac{AE}{EC}\ (\text{same as } \frac{a}{b}=\frac{c}{d}).\\
\Step{2}\;& \text{By the converse of Basic Proportionality Theorem, } DE \parallel BC.
\end{aligned}
\]
\end{QAPair}

\begin{QAPair}{Question 1 (ix) --- MCQ}
\textcolor{gold}{\bfseries Question:} In the figure if $x=y$, then the value of $b$ is:
\begin{itemize}
\item[(a)] $\dfrac{ac}{d}$ \hfill (b) $\dfrac{ad}{c}$ \hfill (c) $\dfrac{cd}{a}$ \hfill (d) $\dfrac{c}{cd}$
\end{itemize}

\begin{center}
\begin{tikzpicture}[scale=1.0]
  \coordinate (L) at (0,0);
  \coordinate (R) at (6,0);
  \coordinate (T) at (2.8,4.2);
  \coordinate (A) at (2.8,0);

  \draw[geoLine] (L)--(T)--(R)--cycle;
  \draw[geoDash] (T)--(A);
  \draw[geoDash] (A)--(2.8,0.7);

  % right angle at A
  \draw[geoLine] (A) ++(0.25,0) -- ++(0,0.25) -- ++(-0.25,0);

  \node[geoLabel] at ($(A)+(0,-0.35)$) {$A$};

  \node[geoSmall, text=muted] at ($(T)!0.45!(L)$) {$a$};
  \node[geoSmall, text=muted] at ($(T)!0.45!(R)$) {$b$};
  \node[geoSmall, text=muted] at ($(L)!0.5!(A)$) {$c$};
  \node[geoSmall, text=muted] at ($(A)!0.5!(R)$) {$d$};

  \node[geoSmall, text=muted] at ($(T)!0.35!(A)$)+(-0.35,0.2) {$x$};
  \node[geoSmall, text=muted] at ($(T)!0.35!(A)$)+(0.35,0.2) {$y$};

  \fill[geoPoint] (L) circle (0.8pt);
  \fill[geoPoint] (R) circle (0.8pt);
  \fill[geoPoint] (T) circle (0.8pt);
  \fill[geoPoint] (A) circle (0.8pt);
\end{tikzpicture}
\end{center}

\tcblower
\textcolor{green}{\bfseries Answer:} \textbf{(b) $\dfrac{ad}{c}$}
\[
\begin{aligned}
\Step{1}\;& x=y \Rightarrow TA \text{ is an angle bisector at the top vertex.}\\
\Step{2}\;& \text{Angle bisector theorem: }\frac{c}{d}=\frac{a}{b}.\\
\Step{3}\;& \Rightarrow b=\frac{ad}{c}.
\end{aligned}
\]
\end{QAPair}

\begin{QAPair}{Question 1 (x) --- MCQ}
\textcolor{gold}{\bfseries Question:} In the figure, $\triangle ABC$ is equilateral and $DE\parallel BC$, then $\triangle ADE$ is:
\begin{itemize}
\item[(a)] right angled \hfill (b) scalene \hfill (c) isosceles \hfill (d) equilateral
\end{itemize}

\begin{center}
\begin{tikzpicture}[scale=1.0]
  \coordinate (B) at (0,0);
  \coordinate (C) at (6,0);
  \coordinate (A) at (3,4.2);
  \coordinate (D) at ($(A)!0.55!(B)$);
  \coordinate (E) at ($(A)!0.55!(C)$);

  \draw[geoLine] (B)--(A)--(C)--cycle;
  \draw[geoBold] (D)--(E);

  \node[geoLabel] at ($(A)+(0,0.35)$) {$A$};
  \node[geoLabel] at ($(B)+(-0.25,-0.35)$) {$B$};
  \node[geoLabel] at ($(C)+(0.25,-0.35)$) {$C$};
  \node[geoLabel] at ($(D)+(-0.25,0.1)$) {$D$};
  \node[geoLabel] at ($(E)+(0.25,0.1)$) {$E$};
\end{tikzpicture}
\end{center}

\tcblower
\textcolor{green}{\bfseries Answer:} \textbf{(d) equilateral}
\[
\begin{aligned}
\Step{1}\;& DE\parallel BC \Rightarrow \triangle ADE \sim \triangle ABC.\\
\Step{2}\;& \triangle ABC \text{ is equilateral} \Rightarrow \angle A=\angle B=\angle C=60^\circ.\\
\Step{3}\;& \text{So } \triangle ADE \text{ also has all angles } 60^\circ \Rightarrow \text{equilateral.}
\end{aligned}
\]
\end{QAPair}

\begin{QAPair}{Question 1 (xi) --- MCQ}
\textcolor{gold}{\bfseries Question:} In the figure, $GH\parallel EF$. Then $a:b=?$
\begin{itemize}
\item[(a)] $DG:DE$ \hfill (b) $DG:DH$ \hfill (c) $DG:GE$ \hfill (d) $DE:EF$
\end{itemize}

\begin{center}
\begin{tikzpicture}[scale=1.0]
  \coordinate (E) at (0,0);
  \coordinate (F) at (6,0);
  \coordinate (D) at (3,4.3);
  \coordinate (G) at ($(D)!0.55!(E)$);
  \coordinate (H) at ($(D)!0.55!(F)$);

  \draw[geoLine] (E)--(D)--(F)--cycle;
  \draw[geoBold] (G)--(H);

  \node[geoLabel] at ($(D)+(0,0.35)$) {$D$};
  \node[geoLabel] at ($(E)+(-0.25,-0.35)$) {$E$};
  \node[geoLabel] at ($(F)+(0.25,-0.35)$) {$F$};
  \node[geoLabel] at ($(G)+(-0.25,0.1)$) {$G$};
  \node[geoLabel] at ($(H)+(0.25,0.1)$) {$H$};

  \node[geoSmall, text=muted] at ($(G)!0.5!(H)$)+(0,0.25) {$a$};
  \node[geoSmall, text=muted] at ($(E)!0.5!(F)$)+(0,-0.35) {$b$};
\end{tikzpicture}
\end{center}

\tcblower
\textcolor{green}{\bfseries Answer:} \textbf{(a) $DG:DE$}
\[
\begin{aligned}
\Step{1}\;& GH\parallel EF \Rightarrow \triangle DGH \sim \triangle DEF.\\
\Step{2}\;& \frac{a}{b}=\frac{GH}{EF}=\frac{DG}{DE}=\frac{DH}{DF}.\\
\Step{3}\;& \Rightarrow a:b = DG:DE.
\end{aligned}
\]
\end{QAPair}

\begin{QAPair}{Question 1 (xii) --- MCQ}
\textcolor{gold}{\bfseries Question:} What is the sum of all the exterior angles of a 13-sided polygon whose one interior angle is equal to $x^\circ$?
\begin{itemize}
\item[(a)] $90^\circ+x$ \hfill (b) $360^\circ$ \hfill (c) $360^\circ+x$ \hfill (d) $180^\circ+x$
\end{itemize}
\tcblower
\textcolor{green}{\bfseries Answer:} \textbf{(b) $360^\circ$}
\[
\begin{aligned}
\Step{1}\;& \text{Sum of exterior angles of \emph{any} polygon (one at each vertex) is }360^\circ.\\
\Step{2}\;& \text{It does not depend on }n\text{ or on }x.
\end{aligned}
\]
\end{QAPair}

\begin{QAPair}{Question 1 (xiii) --- MCQ}
\textcolor{gold}{\bfseries Question:} Which polygon has both its interior and exterior angles the same?
\begin{itemize}
\item[(a)] pentagon \hfill (b) triangle \hfill (c) square \hfill (d) hexagon
\end{itemize}
\tcblower
\textcolor{green}{\bfseries Answer:} \textbf{(c) square}
\[
\begin{aligned}
\Step{1}\;& \text{For a regular }n\text{-gon: exterior }=\frac{360^\circ}{n},\ \text{interior}=180^\circ-\frac{360^\circ}{n}.\\
\Step{2}\;& \text{Set them equal: } 180^\circ-\frac{360^\circ}{n}=\frac{360^\circ}{n}.\\
\Step{3}\;& 180^\circ=\frac{720^\circ}{n}\Rightarrow n=4 \Rightarrow \text{square.}
\end{aligned}
\]
\end{QAPair}

\begin{QAPair}{Question 1 (xiv) --- MCQ}
\textcolor{gold}{\bfseries Question:} The formation or expression of an opinion or theory without sufficient evidence for proof is known as:
\begin{itemize}
\item[(a)] axiom \hfill (b) conjecture \hfill (c) corollary \hfill (d) theorem
\end{itemize}
\tcblower
\textcolor{green}{\bfseries Answer:} \textbf{(b) conjecture}
\[
\begin{aligned}
\Step{1}\;& \text{A conjecture is a statement believed to be true but not yet proved.}
\end{aligned}
\]
\end{QAPair}

\begin{QAPair}{Question 1 (xv) --- MCQ}
\textcolor{gold}{\bfseries Question:} A mathematical statement that is proved true based on already accepted statements is called:
\begin{itemize}
\item[(a)] axiom \hfill (b) conjecture \hfill (c) postulate \hfill (d) theorem
\end{itemize}
\tcblower
\textcolor{green}{\bfseries Answer:} \textbf{(d) theorem}
\[
\begin{aligned}
\Step{1}\;& \text{A theorem is proved using axioms/postulates and earlier results.}
\end{aligned}
\]
\end{QAPair}

\begin{QAPair}{Question 1 (xvi) --- MCQ}
\textcolor{gold}{\bfseries Question:} A mathematical statement that is assumed to be true without proof is called:
\begin{itemize}
\item[(a)] axiom \hfill (b) conjecture \hfill (c) postulate \hfill (d) theorem
\end{itemize}
\tcblower
\textcolor{green}{\bfseries Answer:} \textbf{(c) postulate}
\[
\begin{aligned}
\Step{1}\;& \text{A postulate (axiom) is accepted as true without proof.}\\
\Step{2}\;& \text{Here the intended option is \textbf{postulate}.}
\end{aligned}
\]
\end{QAPair}

\begin{QAPair}{Question 1 (xvii) --- MCQ}
\textcolor{gold}{\bfseries Question:} Two solids with equal ratios of corresponding linear measures:
\begin{itemize}
\item[(a)] are similar \hfill (b) are congruent \hfill (c) have different area \hfill (d) have same volume
\end{itemize}
\tcblower
\textcolor{green}{\bfseries Answer:} \textbf{(a) are similar}
\[
\begin{aligned}
\Step{1}\;& \text{Equal ratios of corresponding linear measures is the definition of similar solids.}
\end{aligned}
\]
\end{QAPair}

% ============================================================
% Q2
\begin{QAPair}{Question 2}
\textcolor{gold}{\bfseries Question:} Find the exterior angle of a polygon with 6 sides.\\
\tcblower
\textcolor{green}{\bfseries Answer:}
\[
\begin{aligned}
\Step{1}\;& \text{For a regular }n\text{-gon, each exterior angle}=\frac{360^\circ}{n}.\\
\Step{2}\;& n=6 \Rightarrow \frac{360^\circ}{6}=60^\circ.
\end{aligned}
\]
\[
\boxed{60^\circ}
\]
\end{QAPair}

% ============================================================
% Q3
\begin{QAPair}{Question 3}
\textcolor{gold}{\bfseries Question:} Is it possible to have a polygon in which sum of interior angles is 9 right angles?\\
\tcblower
\textcolor{green}{\bfseries Answer:}
\[
9\text{ right angles}=9\times 90^\circ=810^\circ.
\]
\[
\begin{aligned}
\Step{1}\;& (n-2)\cdot 180^\circ = 810^\circ\\
\Step{2}\;& n-2 = \frac{810}{180}=4.5 \Rightarrow n=6.5
\end{aligned}
\]
Since $n$ must be a whole number, \textbf{it is not possible.}
\end{QAPair}

% ============================================================
% Q4
\begin{QAPair}{Question 4}
\textcolor{gold}{\bfseries Question:} Is it possible to have a polygon whose sum of interior angles is $7200^\circ$?\\
\tcblower
\textcolor{green}{\bfseries Answer:}
\[
\begin{aligned}
\Step{1}\;& (n-2)\cdot 180^\circ = 7200^\circ\\
\Step{2}\;& n-2 = \frac{7200}{180}=40 \Rightarrow n=42
\end{aligned}
\]
\[
\boxed{\text{Yes. It is a }42\text{-sided polygon (42-gon).}}
\]
\end{QAPair}

% ============================================================
% Q5
\begin{QAPair}{Question 5}
\textcolor{gold}{\bfseries Question:} Find the measure of each angle of a regular nonagon.\\
\tcblower
\textcolor{green}{\bfseries Answer:} A nonagon has $n=9$ sides.
\[
\begin{aligned}
\Step{1}\;& \text{Each interior angle}=\frac{(n-2)180^\circ}{n}=\frac{(9-2)180^\circ}{9}\\
\Step{2}\;& =\frac{7\cdot 180^\circ}{9}=\frac{1260^\circ}{9}=140^\circ
\end{aligned}
\]
\[
\boxed{140^\circ}
\]
\end{QAPair}

% ============================================================
% Q6
\begin{QAPair}{Question 6}
\textcolor{gold}{\bfseries Question:} In the figure $XY$ is a concave mirror, $OA$ is object and $IB$ is its image.\\
(i) Show that $\triangle OAP \sim \triangle IBP$.\\
(ii) Find height of object if $IB=2$ cm, $OP=10$ cm, $IP=4$ cm.\\

\begin{center}
\begin{tikzpicture}[scale=1.0]
  % axis
  \draw[geoLine] (0,0) -- (9,0);
  \node[geoSmall, text=muted] at (8.7,-0.35) {principal axis};

  % mirror arc at right, pole P
  \draw[geoBold] (8.4,-2.0) arc (-70:70:2.2);
  \coordinate (P) at (8.4,0);
  \fill[geoPoint] (P) circle (1.1pt);
  \node[geoLabel] at ($(P)+(0.25,-0.35)$) {$P$};

  % object OA
  \coordinate (O) at (1.3,0);
  \coordinate (A) at (1.3,2.2);
  \draw[geoBold] (O)--(A);
  \fill[geoPoint] (O) circle (0.9pt);
  \fill[geoPoint] (A) circle (0.9pt);
  \node[geoLabel] at ($(O)+(0,-0.35)$) {$O$};
  \node[geoLabel] at ($(A)+(0,0.35)$) {$A$};

  % image IB (inverted)
  \coordinate (I) at (6.2,0);
  \coordinate (B) at (6.2,-1.5);
  \draw[geoBold] (I)--(B);
  \fill[geoPoint] (I) circle (0.9pt);
  \fill[geoPoint] (B) circle (0.9pt);
  \node[geoLabel] at ($(I)+(0,-0.35)$) {$I$};
  \node[geoLabel] at ($(B)+(0,-0.35)$) {$B$};

  % rays to P (schematic)
  \draw[geoLine] (A) -- (P);
  \draw[geoLine] (B) -- (P);
\end{tikzpicture}
\end{center}

\tcblower
\textcolor{green}{\bfseries Answer:}

\textcolor{muted}{\bfseries (i) Similarity:}
\[
\begin{aligned}
\Step{1}\;& \angle OAP = 90^\circ \ \text{and}\ \angle IBP = 90^\circ.\\
\Step{2}\;& \angle OPA = \angle IPB \quad (\text{angles with principal axis are equal by reflection}).\\
\Step{3}\;& \Rightarrow \triangle OAP \sim \triangle IBP \quad (\text{AA similarity}).
\end{aligned}
\]

\textcolor{muted}{\bfseries (ii) Height of object:}
\[
\begin{aligned}
\Step{1}\;& \triangle OAP \sim \triangle IBP \Rightarrow \frac{OA}{IB}=\frac{OP}{IP}.\\
\Step{2}\;& OA = IB\cdot \frac{OP}{IP}=2\cdot \frac{10}{4}=5.
\end{aligned}
\]
\[
\boxed{OA=5\text{ cm}}
\]
\end{QAPair}

% ============================================================
% Q7
\begin{QAPair}{Question 7}
\textcolor{gold}{\bfseries Question:} In the figure, $AB=6$ cm, $BD=9$ cm. Find the diameter of smaller circle, if diameter of bigger circle is $80$ cm.\\

\begin{center}
\begin{tikzpicture}[scale=0.9]
  \coordinate (A) at (0,0);
  \coordinate (B) at (3,0);
  \coordinate (D) at (7.5,0);

  \def\r{1.2}
  \def\R{2.4}

  \coordinate (C) at ($(B)+(0,\r)$);
  \coordinate (E) at ($(D)+(0,\R)$);

  \draw[geoLine] (C) circle (\r);
  \draw[geoLine] (E) circle (\R);

  \draw[geoLine] (A)--(10,0);
  \draw[geoLine] (A)--(E);

  \draw[geoDash] (C)--(B);
  \draw[geoDash] (E)--(D);

  \fill[geoPoint] (A) circle (0.9pt);
  \fill[geoPoint] (B) circle (0.9pt);
  \fill[geoPoint] (D) circle (0.9pt);
  \fill[geoPoint] (C) circle (0.9pt);
  \fill[geoPoint] (E) circle (0.9pt);

  \node[geoLabel] at ($(A)+(-0.15,-0.35)$) {$A$};
  \node[geoLabel] at ($(B)+(0,-0.35)$) {$B$};
  \node[geoLabel] at ($(D)+(0,-0.35)$) {$D$};
  \node[geoLabel] at ($(C)+(0,0.3)$) {$C$};
  \node[geoLabel] at ($(E)+(0,0.3)$) {$E$};

  \node[geoSmall, text=muted] at ($(A)!0.5!(B)$)+(0,-0.55) {$AB=6$};
  \node[geoSmall, text=muted] at ($(B)!0.5!(D)$)+(0,-0.55) {$BD=9$};
\end{tikzpicture}
\end{center}

\tcblower
\textcolor{green}{\bfseries Answer:}

Let the radius of smaller circle be $r$ and radius of bigger circle be $R$.
Given diameter of bigger circle $=80$ cm $\Rightarrow R=40$ cm.

\[
\begin{aligned}
\Step{1}\;& AB=6,\ BD=9 \Rightarrow AD=AB+BD=15.\\
\Step{2}\;& \triangle ABC \sim \triangle ADE \quad (\text{both right-angled and share }\angle A).\\
\Step{3}\;& \frac{BC}{DE}=\frac{AB}{AD}\Rightarrow \frac{r}{R}=\frac{6}{15}=\frac{2}{5}.\\
\Step{4}\;& r=\frac{2}{5}R=\frac{2}{5}\cdot 40=16\text{ cm}.
\end{aligned}
\]
Diameter of smaller circle $=2r=32$ cm.
\[
\boxed{32\text{ cm}}
\]
\end{QAPair}

% ============================================================
% Q8
\begin{QAPair}{Question 8}
\textcolor{gold}{\bfseries Question:} Prove that if vertex angles of two isosceles triangles are equal, then the two triangles are similar.\\
\tcblower
\textcolor{green}{\bfseries Answer:}

Let $\triangle ABC$ and $\triangle PQR$ be isosceles triangles with vertex angles
\[
\angle A=\angle P.
\]
Since the triangles are isosceles:
\[
\angle B=\angle C \quad\text{and}\quad \angle Q=\angle R.
\]

\[
\begin{aligned}
\Step{1}\;& \angle A+\angle B+\angle C = 180^\circ \Rightarrow \angle B=\angle C=\frac{180^\circ-\angle A}{2}.\\
\Step{2}\;& \angle P+\angle Q+\angle R = 180^\circ \Rightarrow \angle Q=\angle R=\frac{180^\circ-\angle P}{2}.\\
\Step{3}\;& \angle A=\angle P \Rightarrow \angle B=\angle Q \text{ and } \angle C=\angle R.\\
\Step{4}\;& \Rightarrow \triangle ABC \sim \triangle PQR \quad (\text{AA similarity}).
\end{aligned}
\]
\end{QAPair}

% ============================================================
% Q9
\begin{QAPair}{Question 9}
\textcolor{gold}{\bfseries Question:} Find the length of $\overline{BC}$ and $\overline{DE}$ in the figure.\\

\begin{center}
\begin{tikzpicture}[scale=0.95]
  \coordinate (A) at (0,0);
  \coordinate (B) at (4,0);
  \coordinate (D) at (6,0);

  \coordinate (C) at (4,2.0);
  \coordinate (E) at (6,3.5);

  \draw[geoLine] (A)--(D);
  \draw[geoLine] (B)--(C);
  \draw[geoLine] (D)--(E);
  \draw[geoBold] (A)--(E);

  % right angles at B and D
  \draw[geoLine] (B) ++(0.25,0) -- ++(0,0.25) -- ++(-0.25,0);
  \draw[geoLine] (D) ++(0.25,0) -- ++(0,0.25) -- ++(-0.25,0);

  \fill[geoPoint] (A) circle (0.9pt);
  \fill[geoPoint] (B) circle (0.9pt);
  \fill[geoPoint] (D) circle (0.9pt);
  \fill[geoPoint] (C) circle (0.9pt);
  \fill[geoPoint] (E) circle (0.9pt);

  \node[geoLabel] at ($(A)+(-0.2,-0.35)$) {$A$};
  \node[geoLabel] at ($(B)+(0,-0.35)$) {$B$};
  \node[geoLabel] at ($(D)+(0,-0.35)$) {$D$};
  \node[geoLabel] at ($(C)+(0,0.35)$) {$C$};
  \node[geoLabel] at ($(E)+(0,0.35)$) {$E$};

  \node[geoSmall, text=muted] at ($(A)!0.5!(B)$)+(0,-0.55) {$8\text{ cm}$};
  \node[geoSmall, text=muted] at ($(B)!0.5!(D)$)+(0,-0.55) {$4\text{ cm}$};

  \node[geoSmall, text=muted] at ($(B)!0.5!(C)$)+(0.55,0) {$x+2$};
  \node[geoSmall, text=muted] at ($(D)!0.5!(E)$)+(0.55,0) {$2x$};
\end{tikzpicture}
\end{center}

\tcblower
\textcolor{green}{\bfseries Answer:}

\[
AB=8,\quad BD=4 \Rightarrow AD=12.
\]
Triangles $\triangle ABC$ and $\triangle ADE$ are similar (both right-angled and share $\angle A$).
\[
\begin{aligned}
\Step{1}\;& \frac{BC}{DE}=\frac{AB}{AD}\\
\Step{2}\;& \frac{x+2}{2x}=\frac{8}{12}=\frac{2}{3}\\
\Step{3}\;& 3(x+2)=4x \Rightarrow x=6
\end{aligned}
\]
So
\[
BC=x+2=8\text{ cm},\qquad DE=2x=12\text{ cm}.
\]
\[
\boxed{BC=8\text{ cm},\ \ DE=12\text{ cm}}
\]
\end{QAPair}

% ============================================================
% Q10
\begin{QAPair}{Question 10}
\textcolor{gold}{\bfseries Question:} Triangles $SQT$ and $PQR$ are similar. Find the ratio of area of triangle $SQT$ to that of triangle $PQR$.\\

\begin{center}
\begin{tikzpicture}[scale=0.95]
  \coordinate (Q) at (0,0);
  \coordinate (R) at (6,0);
  \coordinate (P) at (5,5);

  \coordinate (T) at (2,0);
  \coordinate (S) at ($(Q)!0.333!(P)$);

  \draw[geoLine] (Q)--(P)--(R)--cycle;
  \draw[geoBold] (S)--(T);

  \fill[geoPoint] (Q) circle (0.9pt);
  \fill[geoPoint] (R) circle (0.9pt);
  \fill[geoPoint] (P) circle (0.9pt);
  \fill[geoPoint] (T) circle (0.9pt);
  \fill[geoPoint] (S) circle (0.9pt);

  \node[geoLabel] at ($(Q)+(-0.2,-0.35)$) {$Q$};
  \node[geoLabel] at ($(R)+(0.2,-0.35)$) {$R$};
  \node[geoLabel] at ($(P)+(0.2,0.2)$) {$P$};
  \node[geoLabel] at ($(T)+(0,-0.35)$) {$T$};
  \node[geoLabel] at ($(S)+(-0.3,0.1)$) {$S$};

  \node[geoSmall, text=muted] at ($(Q)!0.5!(T)$)+(0,-0.55) {$2\text{ cm}$};
  \node[geoSmall, text=muted] at ($(T)!0.5!(R)$)+(0,-0.55) {$4\text{ cm}$};
  \node[geoSmall, text=muted] at ($(Q)!0.5!(S)$)+(-0.4,0.1) {$3\text{ cm}$};
  \node[geoSmall, text=muted] at ($(S)!0.5!(P)$)+(-0.4,0.1) {$6\text{ cm}$};
\end{tikzpicture}
\end{center}

\tcblower
\textcolor{green}{\bfseries Answer:}

\[
QT=2,\ TR=4 \Rightarrow QR=6.
\]
\[
QS=3,\ SP=6 \Rightarrow QP=9.
\]
Similarity ratio (small to big):
\[
\begin{aligned}
\Step{1}\;& \frac{QT}{QR}=\frac{2}{6}=\frac{1}{3}\quad \text{(also } \frac{QS}{QP}=\frac{3}{9}=\frac{1}{3}\text{).}\\
\Step{2}\;& \text{Area ratio}=\left(\frac{1}{3}\right)^2=\frac{1}{9}.
\end{aligned}
\]
\[
\boxed{\text{Area}(SQT):\text{Area}(PQR)=1:9}
\]
\end{QAPair}

% ============================================================
% Q11
\begin{QAPair}{Question 11}
\textcolor{gold}{\bfseries Question:} The following pyramids are similar and larger pyramid has a surface area of $392\ \text{cm}^2$. What is the surface area of smaller pyramid? (Corresponding lengths: $14$ cm and $10$ cm.)\\
\tcblower
\textcolor{green}{\bfseries Answer:}

Linear scale factor (small to large):
\[
k=\frac{10}{14}=\frac{5}{7}.
\]
Surface area scales as $k^2$:
\[
\begin{aligned}
\Step{1}\;& SA_{\text{small}}=SA_{\text{large}}\cdot k^2\\
\Step{2}\;& =392\cdot \left(\frac{5}{7}\right)^2
=392\cdot \frac{25}{49}
\end{aligned}
\]
Since $392/49=8$,
\[
SA_{\text{small}}=8\cdot 25=200.
\]
\[
\boxed{200\ \text{cm}^2}
\]
\end{QAPair}

% ============================================================
% Q12
\begin{QAPair}{Question 12}
\textcolor{gold}{\bfseries Question:} The two cylinders are similar. What is the volume of the larger cylinder if the volume of the smaller cylinder is $40\ \text{ft}^3$? (Given $R=3r$ and $r=3$ ft.)\\
\tcblower
\textcolor{green}{\bfseries Answer:}

\[
R=3r \Rightarrow k=\frac{R}{r}=3.
\]
For similar solids, volume scales as $k^3$:
\[
\begin{aligned}
\Step{1}\;& V_{\text{large}}=V_{\text{small}}\cdot k^3\\
\Step{2}\;& =40\cdot 3^3=40\cdot 27=1080
\end{aligned}
\]
\[
\boxed{1080\ \text{ft}^3}
\]
\end{QAPair}

% ============================================================
% Q13
\begin{QAPair}{Question 13}
\textcolor{gold}{\bfseries Question:} The following pairs of solids are similar. Find the surface area of red solid.\\
(i) Blue: length $4$ m, surface area $336\ \text{m}^2$; Red: length $6$ m.\\
(ii) Blue: $20$ in, surface area $1800\ \text{in}^2$; Red: $15$ in.\\
\tcblower
\textcolor{green}{\bfseries Answer:}

\textcolor{muted}{\bfseries (i)}\;
Linear factor (red to blue):
\[
k=\frac{6}{4}=\frac{3}{2}.
\]
Surface area scales as $k^2$:
\[
SA_{\text{red}}=336\left(\frac{3}{2}\right)^2
=336\cdot \frac{9}{4}
=84\cdot 9
=756.
\]
\[
\boxed{SA_{\text{red}}=756\ \text{m}^2}
\]

\textcolor{muted}{\bfseries (ii)}\;
Linear factor (red to blue):
\[
k=\frac{15}{20}=\frac{3}{4}.
\]
\[
SA_{\text{red}}=1800\left(\frac{3}{4}\right)^2
=1800\cdot \frac{9}{16}
=\frac{16200}{16}
=1012.5.
\]
\[
\boxed{SA_{\text{red}}=1012.5\ \text{in}^2 \; \left(=\frac{2025}{2}\ \text{in}^2\right)}
\]
\end{QAPair}

\end{document}
