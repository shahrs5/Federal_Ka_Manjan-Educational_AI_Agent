% !TEX TS-program = pdflatex
\documentclass[11pt]{article}

% -------------------- Packages --------------------
\usepackage[a4paper,margin=1in]{geometry}
\usepackage{amsmath,amssymb}
\usepackage[T1]{fontenc}
\usepackage{lmodern}
\usepackage{xcolor}
\usepackage{tcolorbox}
\tcbuselibrary{skins,breakable}
\usepackage{enumitem}
\usepackage{hyperref}

\pagestyle{empty}

% -------------------- Dark Theme Colors --------------------
\definecolor{bg}{HTML}{000000}
\definecolor{pairbg}{HTML}{121212}
\definecolor{solbg}{HTML}{0A0A0A}
\definecolor{border}{HTML}{2A2A2A}
\definecolor{text}{HTML}{FFFFFF}
\definecolor{muted}{HTML}{C9CDD3}
\definecolor{gold}{HTML}{FFD700}
\definecolor{green}{HTML}{4ADE80}
\definecolor{cyan}{HTML}{38BDF8}

\pagecolor{bg}
\color{text}

\hypersetup{
  colorlinks=true,
  linkcolor=cyan,
  urlcolor=cyan
}

\setlength{\parindent}{0pt}
\setlength{\parskip}{10pt}

\setlist[itemize]{left=1.4em,itemsep=6pt,topsep=6pt}
\setlist[enumerate]{left=1.6em,itemsep=4pt,topsep=4pt}

% -------------------- tcolorbox Base --------------------
\tcbset{
  enhanced,
  breakable,
  arc=12pt,
  boxrule=0.8pt,
  left=16pt,right=16pt,top=12pt,bottom=12pt
}

\newtcolorbox{QAPair}[1]{%
  colback=pairbg,
  colbacklower=solbg,
  colframe=border,
  coltext=text,
  title=\textcolor{gold}{\bfseries #1},
  fonttitle=\bfseries,
  coltitle=text,
  segmentation style={draw=border, dashed, line width=0.6pt},
}

% Visible text inside this box (fix)
\newtcolorbox{QuickBox}{%
  colback=pairbg,
  colframe=cyan,
  coltext=text,
  fontupper=\color{text},
  borderline north={4pt}{0pt}{cyan},
  arc=14pt,
  boxrule=0.8pt
}

% Helper for step headings
\newcommand{\Step}[1]{\textcolor{muted}{\textbf{Step #1:}}}

% ============================================================
\begin{document}

\begin{center}
{\LARGE\bfseries \textcolor{gold}{Chapter 1 --- Exercise 1.3 Solutions}}\\[-2pt]
\end{center}

\begin{QuickBox}
{\color{cyan}\bfseries Quick formulas (useful)}\par\medskip
\begin{itemize}
\item \textbf{Current balance:} New balance $=$ Old balance $-$ (sum of cheques/withdrawals).
\item \textbf{Average km/L:} $\text{km/L}=\dfrac{\text{distance (km)}}{\text{fuel (L)}}$.
\item \textbf{Unit price:} $\text{price per unit}=\dfrac{\text{total cost}}{\text{quantity}}$.
\item \textbf{Distance:} $\text{distance}=\text{speed}\times \text{time}$.
\item \textbf{Perimeter of rectangle:} $P=2(L+W)$.
\item \textbf{Temperature:} $F=\dfrac{9}{5}C+32,\qquad K=C+273$ (as given).
\item \textbf{Ratio sharing:} Total parts $=a+b+c$; each share $=\dfrac{\text{part}}{\text{total parts}}\times \text{total}$.
\end{itemize}
\end{QuickBox}

% ============================================================
% Q1
\begin{QAPair}{Question 1}
\textcolor{gold}{\bfseries Question:}
On his last bank statement, Qasim had a balance of Rs.\ 175,000 in his checking account.
He wrote one cheque for Rs.\ 45,790 and another for Rs.\ 112,921.
What is his current balance?\\
\tcblower
\textcolor{green}{\bfseries Answer:}
\[
\begin{aligned}
\Step{1}\;& \text{Total cheques} = 45{,}790 + 112{,}921 = 158{,}711.\\
\Step{2}\;& \text{Current balance} = 175{,}000 - 158{,}711 = 16{,}289.
\end{aligned}
\]
\textbf{Current balance = Rs.\ 16,289.}
\end{QAPair}

% ============================================================
% Q2
\begin{QAPair}{Question 2}
\textcolor{gold}{\bfseries Question:}
Last week Wajid drove 283.4 km on 16.2 litres of petrol.
He says that he averaged about 1.75 km/liter.
Is his answer reasonable? Explain.\\
\tcblower
\textcolor{green}{\bfseries Answer:}
\[
\begin{aligned}
\Step{1}\;& \text{Average km/L}=\frac{283.4}{16.2}\approx 17.49\ \text{km/L}.\\
\Step{2}\;& 17.49 \text{ is far from } 1.75,\ \text{so the statement is not reasonable.}
\end{aligned}
\]
\textbf{Correct average $\approx 17.49$ km/L, not $1.75$ km/L.}
\end{QAPair}

% ============================================================
% Q3
\begin{QAPair}{Question 3}
\textcolor{gold}{\bfseries Question:}
Salma bought 3.2 yard of fabric for a total price of Rs.\ 139.2.
How much did the fabric cost per yard?\\
\tcblower
\textcolor{green}{\bfseries Answer:}
\[
\begin{aligned}
\Step{1}\;& \text{Cost per yard}=\frac{139.2}{3.2}.\\
\Step{2}\;& \frac{139.2}{3.2}=43.5.
\end{aligned}
\]
\textbf{Fabric cost = Rs.\ 43.5 per yard.}
\end{QAPair}

% ============================================================
% Q4
\begin{QAPair}{Question 4}
\textcolor{gold}{\bfseries Question:}
Momina walks 3.5 km/h. She took a 12 h walk. How far did she walk?\\
\tcblower
\textcolor{green}{\bfseries Answer:}
\[
\begin{aligned}
\Step{1}\;& \text{Distance}=\text{speed}\times \text{time}.\\
\Step{2}\;& \text{Distance}=3.5\times 12=42.
\end{aligned}
\]
\textbf{She walked 42 km.}
\end{QAPair}

% ============================================================
% Q5
\begin{QAPair}{Question 5}
\textcolor{gold}{\bfseries Question:}
The hiking club went on a 7 day trip. Each day they hiked between 5.5 and 7.5 miles.
It is reasonable to assume that clubbing the days the club hiked.\\
(a) Less than 35 miles \quad
(b) Between 35 and 55 miles \quad
(c) Exactly 55 miles \quad
(d) More than 55 miles\\
\tcblower
\textcolor{green}{\bfseries Answer:}
\[
\begin{aligned}
\Step{1}\;& \text{Minimum total} = 7\times 5.5 = 38.5\ \text{miles}.\\
\Step{2}\;& \text{Maximum total} = 7\times 7.5 = 52.5\ \text{miles}.
\end{aligned}
\]
So the total distance is \textbf{between 35 and 55 miles}. \\
\textbf{Correct option: (b).}
\end{QAPair}

% ============================================================
% Q6
\begin{QAPair}{Question 6}
\textcolor{gold}{\bfseries Question:}
For a class party the students council purchased 42 balloons at Rs.\ 1.85 each.
What is the total amount the student council paid for the balloons?\\
\tcblower
\textcolor{green}{\bfseries Answer:}
\[
\begin{aligned}
\Step{1}\;& \text{Total cost}=42\times 1.85.\\
\Step{2}\;& 42\times 1.85=(40\times 1.85)+(2\times 1.85)=74.00+3.70=77.70.
\end{aligned}
\]
\textbf{Total amount = Rs.\ 77.70.}
\end{QAPair}

% ============================================================
% Q7
\begin{QAPair}{Question 7}
\textcolor{gold}{\bfseries Question:}
A group of friends made 4-yard long rectangular banner.
They paid Rs.\ 3.75 per yard for the fabric and Rs.\ 9 for the firm to go around the banner,
10-yard perimeter. What was the width of the banner?\\
\tcblower
\textcolor{green}{\bfseries Answer:}
Let the length be $L=4$ yards and width be $W$ yards. Perimeter $P=10$ yards.
\[
\begin{aligned}
\Step{1}\;& P=2(L+W).\\
\Step{2}\;& 10=2(4+W).\\
\Step{3}\;& 5=4+W.\\
\Step{4}\;& W=1.
\end{aligned}
\]
\textbf{Width of the banner = 1 yard.}
\end{QAPair}

% ============================================================
% Q8
\begin{QAPair}{Question 8}
\textcolor{gold}{\bfseries Question:}
A shoe factory has an asset for Rs.\ 2,000,000 of which $\dfrac{3}{5}$ is the capital and rest is the debt.
Find the amount of capital and debt. (Asset = capital + debt)\\
\tcblower
\textcolor{green}{\bfseries Answer:}
\[
\begin{aligned}
\Step{1}\;& \text{Capital}=\frac{3}{5}\times 2{,}000{,}000=1{,}200{,}000.\\
\Step{2}\;& \text{Debt}=2{,}000{,}000-1{,}200{,}000=800{,}000.
\end{aligned}
\]
\textbf{Capital = Rs.\ 1,200,000 \quad and \quad Debt = Rs.\ 800,000.}
\end{QAPair}

% ============================================================
% Q9
\begin{QAPair}{Question 9}
\textcolor{gold}{\bfseries Question:}
World lowest temperature in past 100 years was recorded to be $-89.2^\circ$C at Vostok, Antarctica
on July 21, 1983. Convert this temperature into Fahrenheit and Kelvin scales.\\
\hfill ( $F=\dfrac{9}{5}C+32,\ \ K=C+273$ )\\
\tcblower
\textcolor{green}{\bfseries Answer:}
For $C=-89.2$:
\[
\begin{aligned}
\Step{1}\;& F=\frac{9}{5}C+32=\frac{9}{5}(-89.2)+32.\\
\Step{2}\;& F=-160.56+32=-128.56^\circ\text{F}.\\[4pt]
\Step{3}\;& K=C+273=-89.2+273=183.8\ \text{K}.
\end{aligned}
\]
\textbf{$F=-128.56^\circ$F \quad and \quad $K=183.8$ K.}
\end{QAPair}

% ============================================================
% Q10
\begin{QAPair}{Question 10}
\textcolor{gold}{\bfseries Question:}
A company was penalized by the government act for low quality production.
If the company has 3 share holders, Farah, Maryam and Tehreem investing in the ratios of $1:2:3$
and the amount of penalty is Rs.\ 456,868.97.
Find the amount of penalty paid by each of 3 share holders.\\
\tcblower
\textcolor{green}{\bfseries Answer:}
Ratio $1:2:3$ gives total parts $=1+2+3=6$.
\[
\begin{aligned}
\Step{1}\;& \text{One part}=\frac{456{,}868.97}{6}\approx 76{,}144.8283.\\
\Step{2}\;& \text{Farah (1 part)}\approx 76{,}144.83.\\
\Step{3}\;& \text{Maryam (2 parts)}\approx 2\times 76{,}144.8283\approx 152{,}289.66.\\
\Step{4}\;& \text{Tehreem (3 parts)}\approx 3\times 76{,}144.8283\approx 228{,}434.48.
\end{aligned}
\]
\textcolor{muted}{(Rounded to 2 decimal places; last value adjusted by 0.01 to keep the total exactly Rs.\ 456,868.97.)}
\end{QAPair}

\end{document}
