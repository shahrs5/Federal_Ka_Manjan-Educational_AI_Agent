% !TEX TS-program = pdflatex
\documentclass[11pt]{article}

% -------------------- Packages --------------------
\usepackage[a4paper,margin=1in]{geometry}
\usepackage{amsmath,amssymb}
\usepackage[T1]{fontenc}
\usepackage{lmodern}
\usepackage{xcolor}
\usepackage{tcolorbox}
\tcbuselibrary{skins,breakable}
\usepackage{enumitem}
\usepackage{hyperref}

\pagestyle{empty}

% -------------------- Dark Theme Colors --------------------
\definecolor{bg}{HTML}{000000}
\definecolor{pairbg}{HTML}{121212}
\definecolor{solbg}{HTML}{0A0A0A}
\definecolor{border}{HTML}{2A2A2A}
\definecolor{text}{HTML}{FFFFFF}
\definecolor{muted}{HTML}{C9CDD3}
\definecolor{gold}{HTML}{FFD700}
\definecolor{green}{HTML}{4ADE80}
\definecolor{cyan}{HTML}{38BDF8}

\pagecolor{bg}
\color{text}

\hypersetup{
  colorlinks=true,
  linkcolor=cyan,
  urlcolor=cyan
}

\setlength{\parindent}{0pt}
\setlength{\parskip}{10pt}

\setlist[itemize]{left=1.4em,itemsep=6pt,topsep=6pt}
\setlist[enumerate]{left=1.6em,itemsep=4pt,topsep=4pt}

% -------------------- tcolorbox Base --------------------
\tcbset{
  enhanced,
  breakable,
  arc=12pt,
  boxrule=0.8pt,
  left=16pt,right=16pt,top=12pt,bottom=12pt
}

\newtcolorbox{QAPair}[1]{%
  colback=pairbg,
  colbacklower=solbg,
  colframe=border,
  coltext=text,
  title=\textcolor{gold}{\bfseries #1},
  fonttitle=\bfseries,
  coltitle=text,
  segmentation style={draw=border, dashed, line width=0.6pt},
}

\newtcolorbox{QuickBox}{%
  colback=pairbg,
  colframe=cyan,
  coltext=text,
  fontupper=\color{text},
  borderline north={4pt}{0pt}{cyan},
  arc=14pt,
  boxrule=0.8pt
}

% Helper for step headings
\newcommand{\Step}[1]{\textcolor{muted}{\textbf{Step #1:}}}

% ============================================================
\begin{document}

\begin{center}
{\LARGE\bfseries \textcolor{gold}{Exercise 2.2 --- Solutions}}\\[-2pt]
\end{center}

\begin{QuickBox}
{\color{cyan}\bfseries Quick formulas (useful)}\par\medskip
\begin{itemize}
\item \textbf{Definition:} $\log_a(b)=c \iff a^c=b$.
\item \textbf{Conditions (log defined):} $a>0$, $a\neq 1$, and $b>0$.
\item \textbf{Useful facts:} $\log_a(1)=0$,\; $\log_a(a)=1$.
\item \textbf{Powers:} If $m^p=m^q$ (same base $m>0$, $m\neq 1$), then $p=q$.
\end{itemize}
\end{QuickBox}
% ============================================================
% Q1 (SEPARATE PARTS)

\begin{QAPair}{Question 1 (i)}
\textcolor{gold}{\bfseries Question:} Check whether $\log_{x}(7-x)$ is defined for $x=0$.\\
\tcblower
\textcolor{green}{\bfseries Answer:}


\[
\begin{aligned}
\Step{1}\;& \text{Base condition: } x>0,\; x\neq 1.\\
\Step{2}\;& x=0 \text{ violates }x>0.\\
\Step{3}\;& \Rightarrow\; \log_{0}(7-0)\ \text{is \textbf{not defined}.}
\end{aligned}
\]


\end{QAPair}

\begin{QAPair}{Question 1 (ii)}
\textcolor{gold}{\bfseries Question:} Check whether $\log_{x}(7-x)$ is defined for $x=1$.\\
\tcblower
\textcolor{green}{\bfseries Answer:}


\[
\begin{aligned}
\Step{1}\;& \text{A logarithm cannot have base }1.\\
\Step{2}\;& \Rightarrow\; \log_{1}(7-1)\ \text{is \textbf{not defined}.}
\end{aligned}
\]


\end{QAPair}

\begin{QAPair}{Question 1 (iii)}
\textcolor{gold}{\bfseries Question:} Check whether $\log_{x}(7-x)$ is defined for $x=6$.\\
\tcblower
\textcolor{green}{\bfseries Answer:}


\[
\begin{aligned}
\Step{1}\;& 6>0 \text{ and }6\neq 1 \Rightarrow \text{base is valid}.\\
\Step{2}\;& 7-6=1>0 \Rightarrow \text{argument is valid}.\\
\Step{3}\;& \Rightarrow\; \log_{6}(1)\ \text{is \textbf{defined} (and equals }0\text{).}
\end{aligned}
\]


\end{QAPair}

\begin{QAPair}{Question 1 (iv)}
\textcolor{gold}{\bfseries Question:} Check whether $\log_{x}(7-x)$ is defined for $x\ge 7$.\\
\tcblower
\textcolor{green}{\bfseries Answer:}



\[
\begin{aligned}
\Step{1}\;& \text{We need }7-x>0 \Rightarrow x<7.\\
\Step{2}\;& \text{If }x\ge 7,\text{ then }7-x\le 0,\text{ which is invalid for a logarithm}.\\
\Step{3}\;& \Rightarrow\; \log_{x}(7-x)\ \text{is \textbf{not defined for any }$x\ge 7$}.
\end{aligned}
\]





\end{QAPair}

% ============================================================
% Q2
\begin{QAPair}{Question 2 (i)}
\textcolor{gold}{\bfseries Question:} Convert $\log_{6}216=3$ to exponential form.\\
\tcblower
\textcolor{green}{\bfseries Answer:}
\[
\Step{1}\; \log_{6}216=3 \iff 6^{3}=216.
\]
\end{QAPair}

\begin{QAPair}{Question 2 (ii)}
\textcolor{gold}{\bfseries Question:} Convert $7^{4}=2401$ to logarithmic form.\\
\tcblower
\textcolor{green}{\bfseries Answer:}
\[
\Step{1}\; 7^{4}=2401 \iff \log_{7}2401=4.
\]
\end{QAPair}

\begin{QAPair}{Question 2 (iii)}
\textcolor{gold}{\bfseries Question:} Convert $\log_{5}x=5$ to exponential form.\\
\tcblower
\textcolor{green}{\bfseries Answer:}
\[
\Step{1}\; \log_{5}x=5 \iff 5^{5}=x.
\]
\end{QAPair}

\begin{QAPair}{Question 2 (iv)}
\textcolor{gold}{\bfseries Question:} Convert $b^{-\frac{3}{4}}=\frac{1}{27}$ to logarithmic form.\\
\tcblower
\textcolor{green}{\bfseries Answer:}
\[
\Step{1}\; b^{-\frac{3}{4}}=\frac{1}{27} \iff \log_{b}\!\left(\frac{1}{27}\right)=-\frac{3}{4}.
\]
\end{QAPair}

\begin{QAPair}{Question 2 (v)}
\textcolor{gold}{\bfseries Question:} Convert $125^{\frac{x}{3}}=25$ to logarithmic form.\\
\tcblower
\textcolor{green}{\bfseries Answer:}
\[
\Step{1}\; 125^{\frac{x}{3}}=25 \iff \log_{125}25=\frac{x}{3}.
\]
\end{QAPair}

\begin{QAPair}{Question 2 (vi)}
\textcolor{gold}{\bfseries Question:} Convert $\log_{10}\left(10^{12}\right)=y$ to exponential form.\\
\tcblower
\textcolor{green}{\bfseries Answer:}
\[
\Step{1}\; \log_{10}\left(10^{12}\right)=y \iff 10^{y}=10^{12}.
\]
\end{QAPair}

\begin{QAPair}{Question 2 (vii)}
\textcolor{gold}{\bfseries Question:} Convert $(256)^{\frac{x}{4}}=\frac{1}{64}$ to logarithmic form.\\
\tcblower
\textcolor{green}{\bfseries Answer:}
\[
\Step{1}\; (256)^{\frac{x}{4}}=\frac{1}{64} \iff \log_{256}\!\left(\frac{1}{64}\right)=\frac{x}{4}.
\]
\end{QAPair}

\begin{QAPair}{Question 2 (viii)}
\textcolor{gold}{\bfseries Question:} Convert $\log_{3}(x^{3}+1)=2$ to exponential form.\\
\tcblower
\textcolor{green}{\bfseries Answer:}
\[
\Step{1}\; \log_{3}(x^{3}+1)=2 \iff 3^{2}=x^{3}+1.
\]
\end{QAPair}

\begin{QAPair}{Question 2 (ix)}
\textcolor{gold}{\bfseries Question:} Convert $\log_{5}(2x-3)=1$ to exponential form.\\
\tcblower
\textcolor{green}{\bfseries Answer:}
\[
\Step{1}\; \log_{5}(2x-3)=1 \iff 5^{1}=2x-3.
\]
\end{QAPair}

\begin{QAPair}{Question 2 (x)}
\textcolor{gold}{\bfseries Question:} Convert $2x+1=2^{3}$ to logarithmic form.\\
\tcblower
\textcolor{green}{\bfseries Answer:}
\[
\Step{1}\; 2x+1=2^{3} \iff \log_{2}(2x+1)=3.
\]
\end{QAPair}

% ============================================================
% Q3
\begin{QAPair}{Question 3 (i)}
\textcolor{gold}{\bfseries Question:} Find $x$ if $\log_{x}3=1$.\\
\tcblower
\textcolor{green}{\bfseries Answer:}
\[
\begin{aligned}
\Step{1}\;& \log_{x}3=1 \iff x^{1}=3\\
\Step{2}\;& \Rightarrow\; x=3.
\end{aligned}
\]
\end{QAPair}

\begin{QAPair}{Question 3 (ii)}
\textcolor{gold}{\bfseries Question:} Find $x$ if $\log_{x+1}9=2$.\\
\tcblower
\textcolor{green}{\bfseries Answer:}
\[
\begin{aligned}
\Step{1}\;& \log_{x+1}9=2 \iff (x+1)^2=9\\
\Step{2}\;& x+1=\pm 3. \ \text{But base }(x+1)>0\Rightarrow x+1=3\\
\Step{3}\;& \Rightarrow\; x=2.
\end{aligned}
\]
\end{QAPair}

\begin{QAPair}{Question 3 (iii)}
\textcolor{gold}{\bfseries Question:} Find $x$ if $\log_{3}81=x$.\\
\tcblower
\textcolor{green}{\bfseries Answer:}
\[
\begin{aligned}
\Step{1}\;& 81=3^{4}\\
\Step{2}\;& \Rightarrow\; \log_{3}81=4 \Rightarrow x=4.
\end{aligned}
\]
\end{QAPair}

\begin{QAPair}{Question 3 (iv)}
\textcolor{gold}{\bfseries Question:} Find $x$ if $\log_{2}64=x+1$.\\
\tcblower
\textcolor{green}{\bfseries Answer:}
\[
\begin{aligned}
\Step{1}\;& 64=2^{6}\Rightarrow \log_{2}64=6\\
\Step{2}\;& x+1=6 \Rightarrow x=5.
\end{aligned}
\]
\end{QAPair}

\begin{QAPair}{Question 3 (v)}
\textcolor{gold}{\bfseries Question:} Find $x$ if $\log_{2}x=4$.\\
\tcblower
\textcolor{green}{\bfseries Answer:}
\[
\begin{aligned}
\Step{1}\;& \log_{2}x=4 \iff 2^{4}=x\\
\Step{2}\;& \Rightarrow\; x=16.
\end{aligned}
\]
\end{QAPair}

\begin{QAPair}{Question 3 (vi)}
\textcolor{gold}{\bfseries Question:} Find $x$ if $\log_{2}(x^{2}-1)=3$.\\
\tcblower
\textcolor{green}{\bfseries Answer:}
\[
\begin{aligned}
\Step{1}\;& \log_{2}(x^{2}-1)=3 \iff x^{2}-1=2^{3}=8\\
\Step{2}\;& x^{2}=9 \Rightarrow x=\pm 3\\
\Step{3}\;& \text{Check: }x^{2}-1=8>0\ \text{(valid)}\Rightarrow x=3\ \text{or}\ x=-3.
\end{aligned}
\]
\end{QAPair}
% ============================================================
% Q4 (rewritten as requested)
\begin{QAPair}{Question 4}
\textcolor{gold}{\bfseries Question:} Using the statements from \textbf{Question 2}, find the unknowns:

\begin{enumerate}[label=\textbf{(\roman*)}]
\item $\log_{5}x=5$
\item $b^{-\frac{3}{4}}=\dfrac{1}{27}$
\item $125^{\frac{x}{3}}=25$
\item $\log_{10}\!\left(10^{12}\right)=y$
\item $(256)^{\frac{x}{4}}=\dfrac{1}{64}$
\item $\log_{3}(x^{3}+1)=2$
\item $\log_{5}(2x-3)=1$
\item $2x+1=2^{3}$
\end{enumerate}

\tcblower
\textcolor{green}{\bfseries Answer:}

% -------- (i)
\textcolor{gold}{\bfseries (i) }\,$\log_{5}x=5$
\[
\begin{aligned}
\Step{1}\;& \log_{5}x=5 \iff 5^{5}=x\\
\Step{2}\;& x=3125.
\end{aligned}
\]

% -------- (ii)
\textcolor{gold}{\bfseries (ii) }\,$b^{-\frac{3}{4}}=\dfrac{1}{27}$
\[
\begin{aligned}
\Step{1}\;& b^{-\frac{3}{4}}=\frac{1}{27}\ \Rightarrow\ b^{\frac{3}{4}}=27 \qquad (\text{take reciprocal})\\
\Step{2}\;& b=27^{\frac{4}{3}}=\left(\sqrt[3]{27}\right)^{4}=3^{4}=81.
\end{aligned}
\]

% -------- (iii)
\textcolor{gold}{\bfseries (iii) }\,$125^{\frac{x}{3}}=25$
\[
\begin{aligned}
\Step{1}\;& 125=5^{3},\ \ 25=5^{2}\\
\Step{2}\;& 125^{\frac{x}{3}}=25 \iff (5^{3})^{\frac{x}{3}}=5^{2}\\
\Step{3}\;& 5^{x}=5^{2}\Rightarrow x=2.
\end{aligned}
\]

% -------- (iv)
\textcolor{gold}{\bfseries (iv) }\,$\log_{10}\!\left(10^{12}\right)=y$
\[
\begin{aligned}
\Step{1}\;& \log_{10}\left(10^{12}\right)=12 \qquad (\text{since }10^{12}\text{ is a power of }10)\\
\Step{2}\;& y=12.
\end{aligned}
\]

% -------- (v)
\textcolor{gold}{\bfseries (v) }\,$(256)^{\frac{x}{4}}=\dfrac{1}{64}$
\[
\begin{aligned}
\Step{1}\;& 256=2^{8},\quad \frac{1}{64}=2^{-6}\\
\Step{2}\;& (256)^{\frac{x}{4}}=\frac{1}{64}
\iff (2^{8})^{\frac{x}{4}}=2^{-6}\\
\Step{3}\;& 2^{2x}=2^{-6}\Rightarrow 2x=-6\Rightarrow x=-3.
\end{aligned}
\]

% -------- (vi)
\textcolor{gold}{\bfseries (vi) }\,$\log_{3}(x^{3}+1)=2$
\[
\begin{aligned}
\Step{1}\;& \log_{3}(x^{3}+1)=2 \iff x^{3}+1=3^{2}=9\\
\Step{2}\;& x^{3}=8 \Rightarrow x=2.
\end{aligned}
\]

% -------- (vii)
\textcolor{gold}{\bfseries (vii) }\,$\log_{5}(2x-3)=1$
\[
\begin{aligned}
\Step{1}\;& \log_{5}(2x-3)=1 \iff 2x-3=5^{1}=5\\
\Step{2}\;& 2x=8 \Rightarrow x=4.
\end{aligned}
\]

% -------- (viii)
\textcolor{gold}{\bfseries (viii) }\,$2x+1=2^{3}$
\[
\begin{aligned}
\Step{1}\;& 2^{3}=8\\
\Step{2}\;& 2x+1=8 \Rightarrow 2x=7 \Rightarrow x=\frac{7}{2}.
\end{aligned}
\]

\end{QAPair}

\end{document}