% !TEX TS-program = pdflatex
\documentclass[11pt]{article}

% -------------------- Packages --------------------
\usepackage[a4paper,margin=1in]{geometry}
\usepackage{amsmath,amssymb}
\usepackage[T1]{fontenc}
\usepackage{lmodern}
\usepackage{xcolor}
\usepackage{tcolorbox}
\tcbuselibrary{skins,breakable}
\usepackage{enumitem}
\usepackage{hyperref}

\pagestyle{empty}

% -------------------- Dark Theme Colors --------------------
\definecolor{bg}{HTML}{000000}
\definecolor{pairbg}{HTML}{121212}
\definecolor{solbg}{HTML}{0A0A0A}
\definecolor{border}{HTML}{2A2A2A}
\definecolor{text}{HTML}{FFFFFF}
\definecolor{muted}{HTML}{C9CDD3}
\definecolor{gold}{HTML}{FFD700}
\definecolor{green}{HTML}{4ADE80}
\definecolor{cyan}{HTML}{38BDF8}

\pagecolor{bg}
\color{text}

\hypersetup{
  colorlinks=true,
  linkcolor=cyan,
  urlcolor=cyan
}

\setlength{\parindent}{0pt}
\setlength{\parskip}{10pt}

\setlist[itemize]{left=1.4em,itemsep=6pt,topsep=6pt}
\setlist[enumerate]{left=1.6em,itemsep=4pt,topsep=4pt}

% -------------------- tcolorbox Base --------------------
\tcbset{
  enhanced,
  breakable,
  arc=12pt,
  boxrule=0.8pt,
  left=16pt,right=16pt,top=12pt,bottom=12pt
}

\newtcolorbox{QAPair}[1]{%
  colback=pairbg,
  colbacklower=solbg,
  colframe=border,
  coltext=text,
  title=\textcolor{gold}{\bfseries #1},
  fonttitle=\bfseries,
  coltitle=text,
  segmentation style={draw=border, dashed, line width=0.6pt},
}

% Visible text inside this box (fix)
\newtcolorbox{QuickBox}{%
  colback=pairbg,
  colframe=cyan,
  coltext=text,
  fontupper=\color{text},
  borderline north={4pt}{0pt}{cyan},
  arc=14pt,
  boxrule=0.8pt
}

% Helper for step headings
\newcommand{\Step}[1]{\textcolor{muted}{\textbf{Step #1:}}}

% ============================================================
\begin{document}

\begin{center}
{\LARGE\bfseries \textcolor{gold}{Exercise 4.7 --- Solutions}}\\[-2pt]
\end{center}

\begin{QuickBox}
{\color{cyan}\bfseries Quick formulas (useful)}\par\medskip
\begin{itemize}
\item \textbf{Perfect-square identities:}
\[
(u+v)^2=u^2+2uv+v^2,\qquad (u-v)^2=u^2-2uv+v^2.
\]
\item \textbf{Trick for “by division” (matching coefficients):}\\
If $\sqrt{\text{(quartic)}}=x^2+mx+n$, then
\[
(x^2+mx+n)^2=x^4+2mx^3+(m^2+2n)x^2+2mn\,x+n^2.
\]
Match coefficients to solve $m,n$.
\item \textbf{With reciprocals:} Try substitutions like
$t=x^2-\dfrac{1}{x^2}$ or $t=a-\dfrac{1}{a}$ to reduce to a simple square.
\end{itemize}
\end{QuickBox}

% ============================================================
% Factorization (1--5)

\begin{QAPair}{Question 1 (Factorization)}
\textcolor{gold}{\bfseries Question:} Find $\sqrt{\,16y^2-56y+49\,}$\\
\tcblower
\textcolor{green}{\bfseries Answer:}
\[
\begin{aligned}
\Step{1}\;& 16y^2-56y+49=(4y)^2-2(4y)(7)+7^2 \\
\Step{2}\;& \Rightarrow\; 16y^2-56y+49=(4y-7)^2.\\
\Step{3}\;& \therefore\; \sqrt{\,16y^2-56y+49\,}=4y-7.
\end{aligned}
\]
\end{QAPair}

\begin{QAPair}{Question 2 (Factorization)}
\textcolor{gold}{\bfseries Question:} Find $\sqrt{\,25a^4-30a^3+9a^2\,}$\\
\tcblower
\textcolor{green}{\bfseries Answer:}
\[
\begin{aligned}
\Step{1}\;& 25a^4-30a^3+9a^2
= a^2(25a^2-30a+9).\\
\Step{2}\;& 25a^2-30a+9=(5a)^2-2(5a)(3)+3^2=(5a-3)^2.\\
\Step{3}\;& \Rightarrow\; 25a^4-30a^3+9a^2=\bigl(a(5a-3)\bigr)^2.\\
\Step{4}\;& \therefore\; \sqrt{\,25a^4-30a^3+9a^2\,}=a(5a-3).
\end{aligned}
\]
\end{QAPair}

\begin{QAPair}{Question 3 (Factorization)}
\textcolor{gold}{\bfseries Question:} Find $\sqrt{\,\left(x^2-\frac1{x^2}\right)^2+4\left(x^2-\frac1{x^2}\right)+4\,}$,\; $x\neq 0$\\
\tcblower
\textcolor{green}{\bfseries Answer:}
Let $t=x^2-\dfrac{1}{x^2}$.
\[
\begin{aligned}
\Step{1}\;& t^2+4t+4=(t+2)^2.\\
\Step{2}\;& \Rightarrow\; \left(x^2-\frac1{x^2}\right)^2+4\left(x^2-\frac1{x^2}\right)+4
=\left(x^2-\frac1{x^2}+2\right)^2.\\
\Step{3}\;& \therefore\; \sqrt{\;\cdots\;}=x^2-\frac1{x^2}+2.
\end{aligned}
\]
\end{QAPair}

\begin{QAPair}{Question 4 (Factorization)}
\textcolor{gold}{\bfseries Question:} Find $\sqrt{\,\left(a^2+\frac1{a^2}\right)-8\left(a-\frac1a\right)+14\,}$,\; $a\neq 0$\\
\tcblower
\textcolor{green}{\bfseries Answer:}
Let $t=a-\dfrac{1}{a}$. Note that $t^2=a^2+\dfrac{1}{a^2}-2$, so $a^2+\dfrac{1}{a^2}=t^2+2$.
\[
\begin{aligned}
\Step{1}\;& \left(a^2+\frac1{a^2}\right)-8\left(a-\frac1a\right)+14
=(t^2+2)-8t+14\\
\Step{2}\;&=t^2-8t+16=(t-4)^2.\\
\Step{3}\;& \Rightarrow\; \sqrt{\;\cdots\;}=t-4=a-\frac1a-4.
\end{aligned}
\]
\end{QAPair}

\begin{QAPair}{Question 5 (Factorization)}
\textcolor{gold}{\bfseries Question:} Find $\sqrt{\, (a+2)(a+4)(a+6)(a+8)+16\,}$\\
\tcblower
\textcolor{green}{\bfseries Answer:}
\[
\begin{aligned}
\Step{1}\;& (a+2)(a+8)=(a+5-3)(a+5+3)=(a+5)^2-9.\\
\Step{2}\;& (a+4)(a+6)=(a+5-1)(a+5+1)=(a+5)^2-1.\\
\Step{3}\;& \Rightarrow\; (a+2)(a+4)(a+6)(a+8)+16\\
&=\bigl((a+5)^2-9\bigr)\bigl((a+5)^2-1\bigr)+16.\\
\Step{4}\;& \text{Let } u=(a+5)^2. \text{ Then } (u-9)(u-1)+16=u^2-10u+25=(u-5)^2.\\
\Step{5}\;& u-5=(a+5)^2-5=a^2+10a+20.\\
\Step{6}\;& \therefore\; \sqrt{\;\cdots\;}=a^2+10a+20.
\end{aligned}
\]
\end{QAPair}

% ============================================================
% Division (6--14)

\begin{QAPair}{Question 6 (Division)}
\textcolor{gold}{\bfseries Question:} Find $\sqrt{\,x^4+8x^3+20x^2+16x+4\,}$\\
\tcblower
\textcolor{green}{\bfseries Answer:}
Assume $\sqrt{\cdots}=x^2+mx+n$.
\[
\begin{aligned}
\Step{1}\;& (x^2+mx+n)^2=x^4+2mx^3+(m^2+2n)x^2+2mn\,x+n^2.\\
\Step{2}\;& 2m=8 \Rightarrow m=4.\\
\Step{3}\;& m^2+2n=20 \Rightarrow 16+2n=20 \Rightarrow n=2.\\
\Step{4}\;& \text{Check: }2mn=2(4)(2)=16,\; n^2=4 \text{ (matches).}\\
\Step{5}\;& \therefore\; \sqrt{\,x^4+8x^3+20x^2+16x+4\,}=x^2+4x+2.
\end{aligned}
\]
\end{QAPair}

\begin{QAPair}{Question 7 (Division)}
\textcolor{gold}{\bfseries Question:} Find $\sqrt{\,x^4+10x^3+31x^2+30x+9\,}$\\
\tcblower
\textcolor{green}{\bfseries Answer:}
Assume $\sqrt{\cdots}=x^2+mx+n$.
\[
\begin{aligned}
\Step{1}\;& 2m=10 \Rightarrow m=5.\\
\Step{2}\;& m^2+2n=31 \Rightarrow 25+2n=31 \Rightarrow n=3.\\
\Step{3}\;& \text{Check: }2mn=2(5)(3)=30,\; n^2=9 \text{ (matches).}\\
\Step{4}\;& \therefore\; \sqrt{\,x^4+10x^3+31x^2+30x+9\,}=x^2+5x+3.
\end{aligned}
\]
\end{QAPair}

\begin{QAPair}{Question 8 (Division)}
\textcolor{gold}{\bfseries Question:} Find $\sqrt{\,49b^4+28ab^3+18a^2b^2+4a^3b+a^4\,}$\\
\tcblower
\textcolor{green}{\bfseries Answer:}
Assume $\sqrt{\cdots}=a^2+mab+nb^2$.
\[
\begin{aligned}
\Step{1}\;& (a^2+mab+nb^2)^2
=a^4+2m a^3b+(m^2+2n)a^2b^2+2mn\,ab^3+n^2b^4.\\
\Step{2}\;& 2m=4 \Rightarrow m=2.\\
\Step{3}\;& 2mn=28 \Rightarrow 2(2)n=28 \Rightarrow n=7.\\
\Step{4}\;& \text{Check: }(m^2+2n)=4+14=18,\; n^2=49 \text{ (matches).}\\
\Step{5}\;& \therefore\; \sqrt{\;\cdots\;}=a^2+2ab+7b^2.
\end{aligned}
\]
\end{QAPair}

\begin{QAPair}{Question 9 (Division)}
\textcolor{gold}{\bfseries Question:} Find $\sqrt{\,4x^4-12x^3+29x^2-30x+25\,}$\\
\tcblower
\textcolor{green}{\bfseries Answer:}
Assume $\sqrt{\cdots}=2x^2+mx+n$.
\[
\begin{aligned}
\Step{1}\;& (2x^2+mx+n)^2
=4x^4+4m x^3+(m^2+4n)x^2+2mn\,x+n^2.\\
\Step{2}\;& 4m=-12 \Rightarrow m=-3.\\
\Step{3}\;& m^2+4n=29 \Rightarrow 9+4n=29 \Rightarrow n=5.\\
\Step{4}\;& \text{Check: }2mn=2(-3)(5)=-30,\; n^2=25 \text{ (matches).}\\
\Step{5}\;& \therefore\; \sqrt{\;\cdots\;}=2x^2-3x+5.
\end{aligned}
\]
\end{QAPair}

\begin{QAPair}{Question 10 (Division)}
\textcolor{gold}{\bfseries Question:} Find $\sqrt{\,x^4-10x^3+27x^2-10x+1\,}$\\
\tcblower
\textcolor{green}{\bfseries Answer:}
Assume $\sqrt{\cdots}=x^2+mx+n$.
\[
\begin{aligned}
\Step{1}\;& 2m=-10 \Rightarrow m=-5.\\
\Step{2}\;& m^2+2n=27 \Rightarrow 25+2n=27 \Rightarrow n=1.\\
\Step{3}\;& \text{Check: }2mn=2(-5)(1)=-10,\; n^2=1 \text{ (matches).}\\
\Step{4}\;& \therefore\; \sqrt{\;\cdots\;}=x^2-5x+1.
\end{aligned}
\]
\end{QAPair}

\begin{QAPair}{Question 11 (Division)}
\textcolor{gold}{\bfseries Question:} Find $\sqrt{\,x^4+\frac1{x^4}+4x^2-\frac4{x^2}+2\,}$,\; $x\neq 0$\\
\tcblower
\textcolor{green}{\bfseries Answer:}
\[
\begin{aligned}
\Step{1}\;& x^4+\frac1{x^4}+4x^2-\frac4{x^2}+2
=\left(x^2-\frac1{x^2}\right)^2+4\left(x^2-\frac1{x^2}\right)+4.\\
\Step{2}\;& \Rightarrow\; \text{Let } t=x^2-\frac1{x^2}. \text{ Then } t^2+4t+4=(t+2)^2.\\
\Step{3}\;& \therefore\; \sqrt{\;\cdots\;}=t+2=x^2-\frac1{x^2}+2.
\end{aligned}
\]
\end{QAPair}

\begin{QAPair}{Question 12 (Division)}
\textcolor{gold}{\bfseries Question:} Find $\sqrt{\,a^2-8a+2+\frac{56}{a}+\frac{49}{a^2}\,}$,\; $a\neq 0$\\
\tcblower
\textcolor{green}{\bfseries Answer:}
Let $t=a-\dfrac{7}{a}$. Then
\[
t^2=a^2-14+\frac{49}{a^2}.
\]
\[
\begin{aligned}
\Step{1}\;& a^2-8a+2+\frac{56}{a}+\frac{49}{a^2}
=\left(a^2-14+\frac{49}{a^2}\right)-8\left(a-\frac{7}{a}\right)+16\\
\Step{2}\;&=t^2-8t+16=(t-4)^2.\\
\Step{3}\;& \therefore\; \sqrt{\;\cdots\;}=t-4=a-\frac{7}{a}-4.
\end{aligned}
\]
\end{QAPair}

\begin{QAPair}{Question 13 (Division)}
\textcolor{gold}{\bfseries Question:} Find $\sqrt{\,x^4-2x^3+\frac{3x^2}{2}-\frac{x}{2}+\frac1{16}\,}$\\
\tcblower
\textcolor{green}{\bfseries Answer:}
Assume $\sqrt{\cdots}=x^2+mx+n$.
\[
\begin{aligned}
\Step{1}\;& 2m=-2 \Rightarrow m=-1.\\
\Step{2}\;& m^2+2n=\frac{3}{2} \Rightarrow 1+2n=\frac{3}{2} \Rightarrow n=\frac{1}{4}.\\
\Step{3}\;& \text{Check: }2mn=2\left(-1\right)\left(\frac14\right)=-\frac12,\;
n^2=\left(\frac14\right)^2=\frac1{16}.\\
\Step{4}\;& \therefore\; \sqrt{\;\cdots\;}=x^2-x+\frac14.
\end{aligned}
\]
\end{QAPair}

\begin{QAPair}{Question 14 (Division)}
\textcolor{gold}{\bfseries Question:} Find $\sqrt{\,4x^4+32x^2+96+\frac{128}{x^2}+\frac{64}{x^4}\,}$,\; $x\neq 0$\\
\tcblower
\textcolor{green}{\bfseries Answer:}
\[
\begin{aligned}
\Step{1}\;& 4x^4+32x^2+96+\frac{128}{x^2}+\frac{64}{x^4}
=4\left(x^4+8x^2+24+\frac{32}{x^2}+\frac{16}{x^4}\right).\\
\Step{2}\;& x^4+8x^2+24+\frac{32}{x^2}+\frac{16}{x^4}
=\left(x^2+4+\frac{4}{x^2}\right)^2.\\
\Step{3}\;& \Rightarrow\; \text{Expression }=4\left(x^2+4+\frac{4}{x^2}\right)^2
=\left(2x^2+8+\frac{8}{x^2}\right)^2.\\
\Step{4}\;& \therefore\; \sqrt{\;\cdots\;}=2x^2+8+\frac{8}{x^2}.
\end{aligned}
\]
\end{QAPair}

% ============================================================
% 15--17 (Perfect square conditions)

\begin{QAPair}{Question 15}
\textcolor{gold}{\bfseries Question:} To make $a^4-10a^3+27a^2-9a+2$ a perfect square:\\
(i) What should be added in it? \quad
(ii) What should be subtracted from it? \quad
(iii) What will be the value of $a$?\\
\tcblower
\textcolor{green}{\bfseries Answer:}
Compare with $(a^2-5a+1)^2$.
\[
\begin{aligned}
\Step{1}\;& (a^2-5a+1)^2
= a^4-10a^3+27a^2-10a+1.\\
\Step{2}\;& \bigl(a^4-10a^3+27a^2-9a+2\bigr)-\bigl(a^2-5a+1\bigr)^2
= a+1.\\
\Step{3}\;& \Rightarrow\; a^4-10a^3+27a^2-9a+2=(a^2-5a+1)^2+(a+1).
\end{aligned}
\]
\[
\begin{aligned}
\Step{4}\;& \text{(i) Add }-(a+1)\text{ (i.e., add }-a-1\text{): }\;
(a^2-5a+1)^2.\\
\Step{5}\;& \text{(ii) Subtract }(a+1): \;
(a^2-5a+1)^2.\\
\Step{6}\;& \text{(iii) For the original expression to be a perfect square, need }a+1=0\\
&\Rightarrow a=-1,\ \text{and then the value becomes }(7)^2=49.
\end{aligned}
\]
\end{QAPair}

\begin{QAPair}{Question 16}
\textcolor{gold}{\bfseries Question:} Find $p$ and $q$ if $x^4-12x^3+px+q$ is a complete square.\\
\tcblower
\textcolor{green}{\bfseries Answer:}
Assume $x^4-12x^3+px+q=(x^2+mx+n)^2$.
\[
\begin{aligned}
\Step{1}\;& (x^2+mx+n)^2=x^4+2mx^3+(m^2+2n)x^2+2mn\,x+n^2.\\
\Step{2}\;& 2m=-12 \Rightarrow m=-6.\\
\Step{3}\;& m^2+2n=0 \Rightarrow 36+2n=0 \Rightarrow n=-18.\\
\Step{4}\;& p=2mn=2(-6)(-18)=216,\qquad q=n^2=(-18)^2=324.\\
\Step{5}\;& \therefore\; x^4-12x^3+px+q=(x^2-6x-18)^2.
\end{aligned}
\]
\end{QAPair}

\begin{QAPair}{Question 17}
\textcolor{gold}{\bfseries Question:} For what value of $k$ does
\[
y^4+4y^2+k+\frac{8}{y^2}+\frac{4}{y^4}
\]
become a perfect square, where $y\neq 0$?\\
\tcblower
\textcolor{green}{\bfseries Answer:}
Assume it is of the form $\left(y^2+A+\dfrac{B}{y^2}\right)^2$.
\[
\begin{aligned}
\Step{1}\;& \left(y^2+A+\frac{B}{y^2}\right)^2
= y^4+2A y^2+\left(A^2+2B\right)+\frac{2AB}{y^2}+\frac{B^2}{y^4}.\\
\Step{2}\;& \text{Match coefficients: }2A=4 \Rightarrow A=2.\\
\Step{3}\;& \frac{2AB}{y^2}=\frac{8}{y^2} \Rightarrow 2A B=8 \Rightarrow 4B=8 \Rightarrow B=2.\\
\Step{4}\;& \text{Then constant }k=A^2+2B=4+4=8.\\
\Step{5}\;& \therefore\; y^4+4y^2+8+\frac{8}{y^2}+\frac{4}{y^4}
=\left(y^2+2+\frac{2}{y^2}\right)^2,
\ \text{so }k=8.
\end{aligned}
\]
\end{QAPair}

\end{document}
