% !TEX TS-program = pdflatex
\documentclass[11pt]{article}

% -------------------- Packages --------------------
\usepackage[a4paper,margin=1in]{geometry}
\usepackage{amsmath,amssymb}
\usepackage[T1]{fontenc}
\usepackage{lmodern}
\usepackage{xcolor}
\usepackage{tcolorbox}
\tcbuselibrary{skins,breakable}
\usepackage{enumitem}
\usepackage{hyperref}
\usepackage{tikz}
\usetikzlibrary{calc,patterns,angles,quotes,intersections}

\pagestyle{empty}

% -------------------- Dark Theme Colors --------------------
\definecolor{bg}{HTML}{000000}
\definecolor{pairbg}{HTML}{121212}
\definecolor{solbg}{HTML}{0A0A0A}
\definecolor{border}{HTML}{2A2A2A}
\definecolor{text}{HTML}{FFFFFF}
\definecolor{muted}{HTML}{C9CDD3}
\definecolor{gold}{HTML}{FFD700}
\definecolor{green}{HTML}{4ADE80}
\definecolor{cyan}{HTML}{38BDF8}

\pagecolor{bg}
\color{text}

\hypersetup{
  colorlinks=true,
  linkcolor=cyan,
  urlcolor=cyan
}

\setlength{\parindent}{0pt}
\setlength{\parskip}{10pt}

\setlist[itemize]{left=1.4em,itemsep=6pt,topsep=6pt}
\setlist[enumerate]{left=1.6em,itemsep=4pt,topsep=4pt}

% -------------------- tcolorbox Base --------------------
\tcbset{
  enhanced,
  breakable,
  arc=12pt,
  boxrule=0.8pt,
  left=16pt,right=16pt,top=12pt,bottom=12pt
}

\newtcolorbox{QAPair}[1]{%
  colback=pairbg,
  colbacklower=solbg,
  colframe=border,
  coltext=text,
  title=\textcolor{gold}{\bfseries #1},
  fonttitle=\bfseries,
  coltitle=text,
  segmentation style={draw=border, dashed, line width=0.6pt},
}

\newtcolorbox{QuickBox}{%
  colback=pairbg,
  colframe=cyan,
  coltext=text,
  fontupper=\color{text},
  borderline north={4pt}{0pt}{cyan},
  arc=14pt,
  boxrule=0.8pt
}

% Helper for step headings
% IMPORTANT: This macro ALREADY prints a colon (:)
% So use \Step{1} not \Step{1:}
\newcommand{\Step}[1]{\textcolor{muted}{\textbf{Step #1:}}}

% -------------------- TikZ Styles --------------------
\tikzset{
  geom/.style={draw=muted, line width=0.95pt},
  strong/.style={draw=cyan, line width=1.05pt},
  helper/.style={draw=muted, dashed, line width=0.75pt},
  pt/.style={circle, fill=cyan, inner sep=1.2pt},
  lab/.style={text=text, font=\small},
  ang/.style={draw=cyan, line width=0.9pt},
  note/.style={text=muted, font=\small}
}

% Usage:
% \StepFig{1}{<text>}{<tikzpicture contents ONLY>}
\newcommand{\StepFig}[3]{%
  \Step{#1} #2\par\medskip
  \begin{center}
    \begin{tikzpicture}[scale=0.92]
      #3
    \end{tikzpicture}
  \end{center}
  \vspace{-2pt}
}

% ============================================================
\begin{document}

\begin{center}
{\LARGE\bfseries \textcolor{gold}{Miscellaneous Exercise 11 --- Solutions}}\\[-2pt]
\end{center}

\begin{QuickBox}
{\color{cyan}\bfseries Quick facts (very useful)}\par\medskip
\begin{itemize}
\item \textbf{Class mark (mid-point):} $\dfrac{\text{lower limit}+\text{upper limit}}{2}$.
\item \textbf{Class size/width (inclusive classes):} use boundaries. For $4$--$7$, boundaries are $3.5$ to $7.5$, so width $=7.5-3.5=4$.
\item \textbf{Mean (discrete frequency):} $\bar{x}=\dfrac{\sum fx}{\sum f}$.
\item \textbf{Median (grouped):} $\text{Median}=l+\left(\dfrac{\frac{N}{2}-c}{f}\right)h$.
\item \textbf{Mode (grouped):} $\text{Mode}=l+\left(\dfrac{f_1-f_0}{2f_1-f_0-f_2}\right)h$.
\item \textbf{Probability:} $0\le P(E)\le 1$. Certain event $\Rightarrow P=1$, impossible event $\Rightarrow P=0$.
\item \textbf{Expected number:} for $n$ trials with probability $p$, \; $E=np$.
\end{itemize}
\end{QuickBox}

% ============================================================
% Q1 (MCQs) — now with step-by-step explanations for ALL

\begin{QAPair}{Question 1 (i) --- MCQ}
\textcolor{gold}{\bfseries Question:} Which of the following is a class mark of the interval $(10-15)$?\par
(a) 10 \quad (b) 12.5 \quad (c) 15 \quad (d) 16
\tcblower
\textcolor{green}{\bfseries Answer:} \textbf{(b) 12.5}\par
\Step{1} Class mark means the \textbf{mid-point} of the class interval.\par
\Step{2} Mid-point $=\dfrac{10+15}{2}=\dfrac{25}{2}=12.5$.\par
\Step{3} Therefore, the correct option is \textbf{(b) 12.5}.\par
\StepFig{4}{(Visual) Midpoint lies exactly in the middle of 10 and 15.}{%
  \draw[geom] (0,0) -- (6,0);
  \foreach \x/\lab in {0/10,6/15}{
    \draw[geom] (\x,0.12) -- (\x,-0.12);
    \node[lab, below] at (\x,-0.12) {\lab};
  }
  \coordinate (M) at (3,0);
  \fill[pt] (M) circle(1.2pt) node[lab, above] {$12.5$};
  \node[note] at (3,-0.7) {Class mark (mid-point)};
}
\end{QAPair}

\begin{QAPair}{Question 1 (ii) --- MCQ}
\textcolor{gold}{\bfseries Question:} What is size of class interval $(4-7)$?\par
(a) 4 \quad (b) 5 \quad (c) 6 \quad (d) 7
\tcblower
\textcolor{green}{\bfseries Answer:} \textbf{(a) 4}\par
\Step{1} The class $4$--$7$ is an \textbf{inclusive} class.\par
\Step{2} Convert to boundaries: $3.5$ to $7.5$.\par
\Step{3} Class size (width) $=7.5-3.5=4$.\par
\Step{4} So the correct option is \textbf{(a) 4}.\par
\textcolor{muted}{(Some students use $7-4+1=4$; it gives the same answer here.)}
\end{QAPair}

\begin{QAPair}{Question 1 (iii) --- MCQ}
\textcolor{gold}{\bfseries Question:} Which of the following is chart of adjacent rectangles?\par
(a) bar graph \quad (b) frequency polygon \quad (c) histogram \quad (d) ogive
\tcblower
\textcolor{green}{\bfseries Answer:} \textbf{(c) histogram}\par
\Step{1} A \textbf{histogram} is made of \textbf{rectangles} for class intervals.\par
\Step{2} In a histogram, rectangles are \textbf{adjacent} (they touch) because intervals are continuous.\par
\Step{3} Therefore, the correct option is \textbf{(c) histogram}.\par
\StepFig{4}{(Visual) Adjacent rectangles = histogram.}{%
  \draw[geom,->] (0,0) -- (6.2,0) node[lab, right] {Class intervals};
  \draw[geom,->] (0,0) -- (0,3.4) node[lab, above] {Frequency};
  \draw[strong, fill=cyan, fill opacity=0.12] (0.4,0) rectangle (1.6,1.6);
  \draw[strong, fill=cyan, fill opacity=0.12] (1.6,0) rectangle (2.8,2.6);
  \draw[strong, fill=cyan, fill opacity=0.12] (2.8,0) rectangle (4.0,1.9);
  \draw[strong, fill=cyan, fill opacity=0.12] (4.0,0) rectangle (5.2,3.0);
  \node[note] at (3.1,-0.7) {Rectangles touch each other};
}
\end{QAPair}

\begin{QAPair}{Question 1 (iv) --- MCQ}
\textcolor{gold}{\bfseries Question:} Which of the following is measure of central tendency?\par
(a) variance \quad (b) standard deviation \quad (c) range \quad (d) arithmetic mean
\tcblower
\textcolor{green}{\bfseries Answer:} \textbf{(d) arithmetic mean}\par
\Step{1} Measures of \textbf{central tendency} tell the \textbf{center} of data (mean/median/mode).\par
\Step{2} Variance, standard deviation, and range are measures of \textbf{dispersion} (spread).\par
\Step{3} Therefore, the only central tendency option here is \textbf{arithmetic mean}.\par
\end{QAPair}

\begin{QAPair}{Question 1 (v) --- MCQ}
\textcolor{gold}{\bfseries Question:} Which of the following is a formula of arithmetic mean for grouped data?\par
(a) $\dfrac{\sum x}{n}$ \quad (b) (median-type) \quad (c) $\dfrac{\sum fx}{\sum f}$ \quad (d) (mode-type)
\tcblower
\textcolor{green}{\bfseries Answer:} \textbf{(c) $\dfrac{\sum fx}{\sum f}$}\par
\Step{1} In grouped/discrete frequency data, each value $x$ occurs $f$ times.\par
\Step{2} Total of all observations becomes $\sum fx$.\par
\Step{3} Total number of observations becomes $\sum f$.\par
\Step{4} Mean $=\dfrac{\sum fx}{\sum f}$, so option \textbf{(c)} is correct.\par
\end{QAPair}

\begin{QAPair}{Question 1 (vi) --- MCQ}
\textcolor{gold}{\bfseries Question:} If arithmetic mean of 25 values is 10, then what is sum of values?\par
(a) 250 \quad (b) 125 \quad (c) 25 \quad (d) 2.5
\tcblower
\textcolor{green}{\bfseries Answer:} \textbf{(a) 250}\par
\Step{1} Formula: $\bar{x}=\dfrac{\sum x}{n}$.\par
\Step{2} Rearrange: $\sum x = \bar{x}\,n$.\par
\Step{3} $\sum x = 10\times 25 = 250$.\par
\Step{4} Correct option is \textbf{(a)}.\par
\end{QAPair}

\begin{QAPair}{Question 1 (vii) --- MCQ}
\textcolor{gold}{\bfseries Question:} What is median of the data $4,3,0,2,1$?\par
(a) 0 \quad (b) 2 \quad (c) 3 \quad (d) 4
\tcblower
\textcolor{green}{\bfseries Answer:} \textbf{(b) 2}\par
\Step{1} Arrange data in ascending order: $0,1,2,3,4$.\par
\Step{2} There are 5 values, so the median is the \textbf{middle (3rd)} value.\par
\Step{3} The 3rd value is $2$.\par
\Step{4} Correct option is \textbf{(b)}.\par
\end{QAPair}

\begin{QAPair}{Question 1 (viii) --- MCQ}
\textcolor{gold}{\bfseries Question:} Which measure of central tendency can have more than one value?\par
(a) weighted mean \quad (b) median \quad (c) mode \quad (d) arithmetic mean
\tcblower
\textcolor{green}{\bfseries Answer:} \textbf{(c) mode}\par
\Step{1} Mode means the value(s) that occur \textbf{most frequently}.\par
\Step{2} Sometimes two values can occur with the same highest frequency.\par
\Step{3} Then data has two modes (bi-modal) or more (multi-modal).\par
\Step{4} So, correct option is \textbf{(c)}.\par
\end{QAPair}

\begin{QAPair}{Question 1 (ix) --- MCQ}
\textcolor{gold}{\bfseries Question:} The probability of getting M in ``MUHAMMAD'' is\par
(a) $\frac{1}{8}$ \quad (b) $\frac{3}{8}$ \quad (c) $\frac{3}{5}$ \quad (d) none of these
\tcblower
\textcolor{green}{\bfseries Answer:} \textbf{(b) $\frac{3}{8}$}\par
\Step{1} Total letters in ``MUHAMMAD'' $=8$.\par
\Step{2} Count M's: M occurs $3$ times.\par
\Step{3} Probability $=\dfrac{\text{favourable}}{\text{total}}=\dfrac{3}{8}$.\par
\Step{4} Correct option is \textbf{(b)}.\par
\end{QAPair}

\begin{QAPair}{Question 1 (x) --- MCQ}
\textcolor{gold}{\bfseries Question:} Probability of picking an ace from a well shuffled pack of 52 playing cards is\par
(a) $\frac{1}{52}$ \quad (b) $\frac{1}{13}$ \quad (c) $\frac{4}{13}$ \quad (d) none of these
\tcblower
\textcolor{green}{\bfseries Answer:} \textbf{(b) $\frac{1}{13}$}\par
\Step{1} Total cards $=52$.\par
\Step{2} Total aces in a deck $=4$.\par
\Step{3} Probability $=\dfrac{4}{52}=\dfrac{1}{13}$.\par
\Step{4} Correct option is \textbf{(b)}.\par
\end{QAPair}

\begin{QAPair}{Question 1 (xi) --- MCQ}
\textcolor{gold}{\bfseries Question:} A normal fair die is rolled 6000 times. The expected number of 5 is\par
(a) 100 \quad (b) 5000 \quad (c) 1000 \quad (d) none of these
\tcblower
\textcolor{green}{\bfseries Answer:} \textbf{(c) 1000}\par
\Step{1} For a fair die, $P(5)=\frac{1}{6}$.\par
\Step{2} Total trials $n=6000$.\par
\Step{3} Expected number $E=np=6000\times\frac{1}{6}=1000$.\par
\Step{4} Correct option is \textbf{(c)}.\par
\end{QAPair}

\begin{QAPair}{Question 1 (xii) --- MCQ}
\textcolor{gold}{\bfseries Question:} A fair coin is tossed 500 times. Expected number of tails is\par
(a) 100 \quad (b) 250 \quad (c) 1000 \quad (d) none of these
\tcblower
\textcolor{green}{\bfseries Answer:} \textbf{(b) 250}\par
\Step{1} For a fair coin, $P(\text{tail})=\frac12$.\par
\Step{2} Total tosses $n=500$.\par
\Step{3} Expected tails $E=np=500\times\frac12=250$.\par
\Step{4} Correct option is \textbf{(b)}.\par
\end{QAPair}

\begin{QAPair}{Question 1 (xiii) --- MCQ}
\textcolor{gold}{\bfseries Question:} In a group of 5 people, 4 like peach juice. The expected no. of people in a population of 1200 who like peach juice is\par
(a) 240 \quad (b) 600 \quad (c) 960 \quad (d) none of these
\tcblower
\textcolor{green}{\bfseries Answer:} \textbf{(c) 960}\par
\Step{1} Proportion liking peach juice $=\dfrac{4}{5}$.\par
\Step{2} Apply same proportion to 1200.\par
\Step{3} Expected $=1200\times\dfrac{4}{5}=1200\times 0.8=960$.\par
\Step{4} Correct option is \textbf{(c)}.\par
\end{QAPair}

\begin{QAPair}{Question 1 (xiv) --- MCQ}
\textcolor{gold}{\bfseries Question:} How many times Haani should toss a fair coin if he expects to get 100 tails?\par
(a) 100 \quad (b) 200 \quad (c) 1000 \quad (d) none of these
\tcblower
\textcolor{green}{\bfseries Answer:} \textbf{(b) 200}\par
\Step{1} For a fair coin, expected tails $=\dfrac{n}{2}$.\par
\Step{2} Given expected tails $=100$, so $\dfrac{n}{2}=100$.\par
\Step{3} Multiply both sides by 2: $n=200$.\par
\Step{4} Correct option is \textbf{(b)}.\par
\end{QAPair}

\begin{QAPair}{Question 1 (xv) --- MCQ}
\textcolor{gold}{\bfseries Question:} An event which can never happen is called\par
(a) sure event \quad (b) certain event \quad (c) possible event \quad (d) impossible event
\tcblower
\textcolor{green}{\bfseries Answer:} \textbf{(d) impossible event}\par
\Step{1} ``Can never happen'' means probability is $0$.\par
\Step{2} Such an event is called an \textbf{impossible event}.\par
\Step{3} Correct option is \textbf{(d)}.\par
\end{QAPair}

\begin{QAPair}{Question 1 (xvi) --- MCQ}
\textcolor{gold}{\bfseries Question:} Two such events whose probabilities are $\tfrac12$ each are called\par
(a) likely \quad (b) equally likely \quad (c) unlikely \quad (d) none of these
\tcblower
\textcolor{green}{\bfseries Answer:} \textbf{(b) equally likely}\par
\Step{1} If two events have the same probability, they are \textbf{equally likely}.\par
\Step{2} Here both probabilities are $\tfrac12$, so they are equal.\par
\Step{3} Correct option is \textbf{(b)}.\par
\end{QAPair}

\begin{QAPair}{Question 1 (xvii) --- MCQ}
\textcolor{gold}{\bfseries Question:} The probability of a certain event is\par
(a) 0 \quad (b) $\tfrac12$ \quad (c) $\tfrac34$ \quad (d) 1
\tcblower
\textcolor{green}{\bfseries Answer:} \textbf{(d) 1}\par
\Step{1} A certain (sure) event \textbf{always happens}.\par
\Step{2} Therefore its probability is the maximum possible, i.e. $1$.\par
\Step{3} Correct option is \textbf{(d)}.\par
\end{QAPair}

\begin{QAPair}{Question 1 (xviii) --- MCQ}
\textcolor{gold}{\bfseries Question:} The probability of an event can take the value\par
(a) 0 \quad (b) 1 \quad (c) $\tfrac12$ \quad (d) all of these
\tcblower
\textcolor{green}{\bfseries Answer:} \textbf{(d) all of these}\par
\Step{1} Any probability must satisfy $0\le P(E)\le 1$.\par
\Step{2} So $0$ is possible (impossible event), $1$ is possible (certain event), and $\tfrac12$ is also possible.\par
\Step{3} Therefore all listed values are possible.\par
\Step{4} Correct option is \textbf{(d)}.\par
\end{QAPair}

\begin{QAPair}{Question 1 (xix) --- MCQ}
\textcolor{gold}{\bfseries Question:} The probability of an event cannot take the value\par
(a) 0 \quad (b) 1 \quad (c) 2 \quad (d) all of these
\tcblower
\textcolor{green}{\bfseries Answer:} \textbf{(c) 2}\par
\Step{1} Probabilities are always between 0 and 1.\par
\Step{2} The value 2 is greater than 1, so it is impossible for a probability.\par
\Step{3} Therefore, correct option is \textbf{(c)}.\par
\end{QAPair}

\begin{QAPair}{Question 1 (xx) --- MCQ}
\textcolor{gold}{\bfseries Question:} The probability of an impossible event is\par
(a) 0 \quad (b) 1 \quad (c) $\tfrac12$ \quad (d) all of these
\tcblower
\textcolor{green}{\bfseries Answer:} \textbf{(a) 0}\par
\Step{1} An impossible event means it \textbf{never happens}.\par
\Step{2} ``Never happens'' corresponds to probability $0$.\par
\Step{3} Correct option is \textbf{(a)}.\par
\end{QAPair}

% ============================================================
% Q2
\begin{QAPair}{Question 2 --- For the following data, find mean, mode and median}
\textcolor{gold}{\bfseries Given:}\par
\[
\begin{array}{c|cccccccc}
x & 5 & 10 & 15 & 20 & 25 & 30 & 35 & 40\\ \hline
f & 2 & 4 & 6 & 8 & 10 & 7 & 5 & 3
\end{array}
\]

\Step{1} Compute $\sum f$ and $\sum fx$.\par
\[
\sum f = 2+4+6+8+10+7+5+3 = 45
\]
\[
\sum fx = 5(2)+10(4)+15(6)+20(8)+25(10)+30(7)+35(5)+40(3)=1055
\]

\Step{2} Mean:
\[
\bar{x}=\frac{\sum fx}{\sum f}=\frac{1055}{45}=\frac{211}{9}\approx 23.44
\]

\Step{3} Median position $=\dfrac{N+1}{2}=\dfrac{45+1}{2}=23$rd.\par
Cumulative frequencies: $2,6,12,20,30,37,42,45$.\par
So, $\text{Median}=25$.

\Step{4} Mode is the value with greatest frequency.\par
Largest $f=10$ at $x=25$, so $\text{Mode}=25$.
\tcblower
\textcolor{green}{\bfseries Answers:}\;
\textbf{Mean $\approx 23.44$, Median $=25$, Mode $=25$.}
\end{QAPair}

% ============================================================
% Q3
\begin{QAPair}{Question 3 --- Find the median for the following frequency distribution}
\textcolor{gold}{\bfseries Given:}\par
\[
\begin{array}{c|ccccc}
\text{Marks} & 30\!-\!39 & 40\!-\!49 & 50\!-\!59 & 60\!-\!69 & 70\!-\!79\\ \hline
f & 3 & 8 & 11 & 5 & 4
\end{array}
\]

\Step{1} Total frequency $N=3+8+11+5+4=31$, so $\frac{N}{2}=15.5$.

\Step{2} Cumulative frequencies:
\[
3,\; 11,\; 22,\; 27,\; 31
\]
Median class is $50$--$59$.

\Step{3} Use boundaries: $49.5$--$59.5$.\par
$l=49.5,\; h=10,\; c=11,\; f=11$.
\[
\text{Median}=49.5+\left(\frac{15.5-11}{11}\right)10 \approx 53.59
\]

\tcblower
\textcolor{green}{\bfseries Answer:} \textbf{Median $\approx 53.6$ marks.}
\end{QAPair}

% ============================================================
% Q4
\begin{QAPair}{Question 4 --- The arithmetic mean of 10 values is 35.5. If nine values are $20, 23, 37, 48, 29, 33, 45, 40, 45$, find the tenth value}
\Step{1} Sum of all 10 values:
\[
\sum_{10} x = \bar{x}\,n = 35.5\times 10 = 355
\]

\Step{2} Sum of given 9 values:
\[
20+23+37+48+29+33+45+40+45 = 320
\]

\Step{3} Tenth value:
\[
x_{10} = 355-320 = 35
\]
\tcblower
\textcolor{green}{\bfseries Answer:} \textbf{The tenth value is $35$.}
\end{QAPair}

% ============================================================
% Q5
\begin{QAPair}{Question 5 --- Find mean, median and mode for the following frequency distribution}
\textcolor{gold}{\bfseries Given:}\par
\[
\begin{array}{c|ccccc}
\text{Classes} & 0\!-\!10 & 10\!-\!20 & 20\!-\!30 & 30\!-\!40 & 40\!-\!50\\ \hline
f & 1 & 5 & 7 & 5 & 1
\end{array}
\]

\Step{1} Midpoints $m$: $5,15,25,35,45$.\par
\[
\sum f=19,\quad \sum fm = 1(5)+5(15)+7(25)+5(35)+1(45)=475
\]
Mean:
\[
\bar{x}=\frac{475}{19}=25
\]

\Step{2} Median:\par
$N=19$, so $\frac{N}{2}=9.5$.\par
C.F.: $1,6,13,18,19$ so median class is $20$--$30$.\par
Using boundaries: $l=19.5,\;h=10,\;c=6,\;f=7$:
\[
\text{Median}=19.5+\left(\frac{9.5-6}{7}\right)10=24.5
\]
\textcolor{muted}{(If your book uses lower limit $l=20$, it gives median $=25$.)}

\Step{3} Mode:\par
Modal class is $20$--$30$ ($f_1=7$), with $f_0=5,\;f_2=5$.\par
Using $l=20,\;h=10$:
\[
\text{Mode}=20+\left(\frac{7-5}{2(7)-5-5}\right)10=25
\]

\tcblower
\textcolor{green}{\bfseries Answers:}\;
\textbf{Mean $=25$, Mode $=25$, Median $=25$ (lower-limit method).}\par
\textcolor{muted}{If boundary method is used for median, it is $24.5$. Follow your book/teacher.}
\end{QAPair}

% ============================================================
% Q6
\begin{QAPair}{Question 6 --- Aaffan plays dart 20 times each day and probability of hitting bull's eye is $1/10$. Find (a) expected hits in 25 days (b) expected misses in 25 days}
\Step{1} Total trials in 25 days:
\[
n = 20\times 25 = 500
\]
\Step{2} Expected hits:
\[
E(\text{hits}) = np = 500\times \frac{1}{10} = 50
\]
\Step{3} Expected misses:
\[
E(\text{misses}) = n(1-p)=500\times\frac{9}{10}=450
\]
\tcblower
\textcolor{green}{\bfseries Answers:}\; \textbf{(a) 50 hits \quad (b) 450 misses.}
\end{QAPair}

% ============================================================
% Q7
\begin{QAPair}{Question 7 --- Shifaa usually gets late for her school 2 days out of 6 working days of a week. What is the expected no. of days of her late arrival in 4 weeks?}
\Step{1} Probability of late on a working day:
\[
p=\frac{2}{6}=\frac{1}{3}
\]
\Step{2} Total working days in 4 weeks:
\[
n = 6\times 4 = 24
\]
\Step{3} Expected late days:
\[
E = np = 24\times \frac{1}{3} = 8
\]
\tcblower
\textcolor{green}{\bfseries Answer:} \textbf{Expected late days $=8$.}
\end{QAPair}

% ============================================================
% Q8
\begin{QAPair}{Question 8 --- As'ha tries her best to score 100\% in each maths assessment, but the probability of getting this is 0.8. Find the expected number (out of 20) in which she will score 100\%}
\Step{1} Use $E=np$.\par
\Step{2} Here $n=20$, $p=0.8$.\par
\[
E = 20\times 0.8 = 16
\]
\tcblower
\textcolor{green}{\bfseries Answer:} \textbf{Expected number $=16$.}
\end{QAPair}

% ============================================================
% Q9
\begin{QAPair}{Question 9 --- In a lab shelf there are 50 reports of Covid-19. Among these 13 are females and 2 of the females are Covid positive. If a report is picked at random, find: (a) $P(\text{female})$ (b) $P(\text{male})$ (c) $P(\text{female and Covid negative})$}
\Step{1} Total reports $=50$.\par
\Step{2} Females $=13$ $\Rightarrow$ males $=50-13=37$.\par
\Step{3} Female Covid positive $=2$ $\Rightarrow$ female Covid negative $=13-2=11$.\par
\Step{4} Use $P=\dfrac{\text{favourable}}{\text{total}}$:\par
\[
(a)\;P(\text{female})=\frac{13}{50},\qquad
(b)\;P(\text{male})=\frac{37}{50},\qquad
(c)\;P(\text{female and Covid negative})=\frac{11}{50}.
\]
\tcblower
\textcolor{green}{\bfseries Answers:}\;
\textbf{(a) $\frac{13}{50}$ \quad (b) $\frac{37}{50}$ \quad (c) $\frac{11}{50}$.}
\end{QAPair}

\end{document}
