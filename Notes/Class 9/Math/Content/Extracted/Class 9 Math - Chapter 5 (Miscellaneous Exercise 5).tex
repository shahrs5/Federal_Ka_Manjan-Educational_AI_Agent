% !TEX TS-program = pdflatex
\documentclass[11pt]{article}

% -------------------- Packages --------------------
\usepackage[a4paper,margin=1in]{geometry}
\usepackage{amsmath,amssymb}
\usepackage[T1]{fontenc}
\usepackage{lmodern}
\usepackage{xcolor}
\usepackage{tcolorbox}
\tcbuselibrary{skins,breakable}
\usepackage{enumitem}
\usepackage{hyperref}
\usepackage{tikz}
\usetikzlibrary{calc,patterns}
\usepackage{microtype}

\pagestyle{empty}

% -------------------- Dark Theme Colors --------------------
\definecolor{bg}{HTML}{000000}
\definecolor{pairbg}{HTML}{121212}
\definecolor{solbg}{HTML}{0A0A0A}
\definecolor{border}{HTML}{2A2A2A}
\definecolor{text}{HTML}{FFFFFF}
\definecolor{muted}{HTML}{C9CDD3}
\definecolor{gold}{HTML}{FFD700}
\definecolor{green}{HTML}{4ADE80}
\definecolor{cyan}{HTML}{38BDF8}

\pagecolor{bg}
\color{text}

\hypersetup{
  colorlinks=true,
  linkcolor=cyan,
  urlcolor=cyan
}

\setlength{\parindent}{0pt}
\setlength{\parskip}{10pt}

\setlist[itemize]{left=1.4em,itemsep=6pt,topsep=6pt}
\setlist[enumerate]{left=1.6em,itemsep=4pt,topsep=4pt}

% ---- overflow safety ----
\sloppy
\setlength{\emergencystretch}{2em}

% -------------------- tcolorbox Base --------------------
\tcbset{
  enhanced,
  breakable,
  arc=12pt,
  boxrule=0.8pt,
  left=16pt,right=16pt,top=12pt,bottom=12pt
}

\newtcolorbox{QAPair}[1]{%
  colback=pairbg,
  colbacklower=solbg,
  colframe=border,
  coltext=text,
  title=\textcolor{gold}{\bfseries #1},
  fonttitle=\bfseries,
  coltitle=text,
  segmentation style={draw=border, dashed, line width=0.6pt},
  before upper=\sloppy,
  before lower=\sloppy,
}

\newtcolorbox{QuickBox}{%
  colback=pairbg,
  colframe=cyan,
  coltext=text,
  fontupper=\color{text},
  borderline north={4pt}{0pt}{cyan},
  arc=14pt,
  boxrule=0.8pt
}

% Helper for step headings
\newcommand{\Step}[1]{\textcolor{muted}{\textbf{Step #1:}}}

% ============================================================
\begin{document}

\begin{center}
{\LARGE\bfseries \textcolor{gold}{Miscellaneous Exercise 5 --- Solutions}}\\[-2pt]
\end{center}

\begin{QuickBox}
{\color{cyan}\bfseries Quick facts (useful)}\par\medskip
\begin{itemize}
\item \textbf{Linear equation (standard form):} $ax+b=0$ where $a\ne 0$.
\item \textbf{Absolute value:} $|u|=c$ gives $u=c$ or $u=-c$ (only if $c\ge 0$).
\item \textbf{Radicals:} $\sqrt{A}\ge 0$. If $\sqrt{A}=k$ with $k<0$, then \textbf{no real solution}.
\item \textbf{Inequalities:} multiplying/dividing by a \textbf{negative} number reverses the sign.
\item \textbf{OR in inequalities:} take the \textbf{union} of solution sets.
\end{itemize}
\end{QuickBox}

% ============================================================
% Q1 MCQs (i) to (xvi)

\begin{QAPair}{Question 1 (i) --- MCQ}
\textcolor{gold}{\bfseries Question:} Which one is the standard form of linear equation?
\begin{itemize}[leftmargin=*,labelsep=0.6em]
\item[(a)] $ax^2+b=0$
\item[(b)] $ax+b=-c$
\item[(c)] $ax+b>0$
\item[(d)] $ax+b=0$
\end{itemize}
\tcblower
\textcolor{green}{\bfseries Answer:} \textbf{(d)} \quad because standard linear form is $ax+b=0$.
\end{QAPair}

\begin{QAPair}{Question 1 (ii) --- MCQ}
\textcolor{gold}{\bfseries Question:} The exponent of the variable in a linear equation is?
\begin{itemize}[leftmargin=*,labelsep=0.6em]
\item[(a)] $1$
\item[(b)] $2$
\item[(c)] $-1$
\item[(d)] $0$
\end{itemize}
\tcblower
\textcolor{green}{\bfseries Answer:} \textbf{(a)} \quad linear means power of variable is $1$.
\end{QAPair}

\begin{QAPair}{Question 1 (iii) --- MCQ}
\textcolor{gold}{\bfseries Question:} Which one is the linear equation in one variable?
\begin{itemize}[leftmargin=*,labelsep=0.6em]
\item[(a)] $ax+y=0$
\item[(b)] $xy+3=0$
\item[(c)] $2x+3=0$
\item[(d)] $2x^2+3=0$
\end{itemize}
\tcblower
\textcolor{green}{\bfseries Answer:} \textbf{(c)} \quad only $2x+3=0$ is linear and has one variable.
\end{QAPair}

\begin{QAPair}{Question 1 (iv) --- MCQ}
\textcolor{gold}{\bfseries Question:} Which one is the solution of $12x+17=65$?
\begin{itemize}[leftmargin=*,labelsep=0.6em]
\item[(a)] $48$
\item[(b)] $\dfrac{82}{12}$
\item[(c)] $4$
\item[(d)] $\dfrac{65}{12}$
\end{itemize}
\tcblower
\textcolor{green}{\bfseries Answer:} \textbf{(c)} \quad since $12x=48 \Rightarrow x=4$.
\end{QAPair}

\begin{QAPair}{Question 1 (v) --- MCQ}
\textcolor{gold}{\bfseries Question:} What number must be subtracted from RHS of $7x=30$ so that $x=4$ becomes a solution?
\begin{itemize}[leftmargin=*,labelsep=0.6em]
\item[(a)] $7$
\item[(b)] $2$
\item[(c)] $4$
\item[(d)] $-2$
\end{itemize}
\tcblower
\textcolor{green}{\bfseries Answer:} \textbf{(b)} \quad because $7(4)=28$ so RHS must become $28$: $30-2=28$.
\end{QAPair}

\begin{QAPair}{Question 1 (vi) --- MCQ}
\textcolor{gold}{\bfseries Question:} Which property of equality will be applied to solve $-2x=\dfrac{2}{5}$?
\begin{itemize}[leftmargin=*,labelsep=0.6em]
\item[(a)] Addition
\item[(b)] Subtraction
\item[(c)] Division
\item[(d)] Addition and subtraction
\end{itemize}
\tcblower
\textcolor{green}{\bfseries Answer:} \textbf{(c)} \quad divide both sides by $-2$.
\end{QAPair}

\begin{QAPair}{Question 1 (vii) --- MCQ}
\textcolor{gold}{\bfseries Question:} Which one is the solution set of $5|x|=25$?
\begin{itemize}[leftmargin=*,labelsep=0.6em]
\item[(a)] $\{-25,25\}$
\item[(b)] $\{-25\}$
\item[(c)] $\{5\}$
\item[(d)] $\{-5,5\}$
\end{itemize}
\tcblower
\textcolor{green}{\bfseries Answer:} \textbf{(d)} \quad $|x|=5 \Rightarrow x=\pm 5$.
\end{QAPair}

\begin{QAPair}{Question 1 (viii) --- MCQ}
\textcolor{gold}{\bfseries Question:} Which is the solution set of $|x|+7=3$?
\begin{itemize}[leftmargin=*,labelsep=0.6em]
\item[(a)] $\{-4\}$
\item[(b)] $\{4,-4\}$
\item[(c)] $\{\}$
\item[(d)] $\{-7,-3\}$
\end{itemize}
\tcblower
\textcolor{green}{\bfseries Answer:} \textbf{(c)} \quad because $|x|= -4$ is impossible in reals.
\end{QAPair}

\begin{QAPair}{Question 1 (ix) --- MCQ}
\textcolor{gold}{\bfseries Question:} Which one is the solution set of $\sqrt{5x}=-10$?
\begin{itemize}[leftmargin=*,labelsep=0.6em]
\item[(a)] $\{\}$
\item[(b)] $\{-20\}$
\item[(c)] $\{20\}$
\item[(d)] $\{-2\}$
\end{itemize}
\tcblower
\textcolor{green}{\bfseries Answer:} \textbf{(a)} \quad since $\sqrt{5x}\ge 0$ cannot be $-10$.
\end{QAPair}

\begin{QAPair}{Question 1 (x) --- MCQ}
\textcolor{gold}{\bfseries Question:} Which one is the solution set of $\sqrt{3x+1}=5$?
\begin{itemize}[leftmargin=*,labelsep=0.6em]
\item[(a)] $25$
\item[(b)] $8$
\item[(c)] $24$
\item[(d)] $\dfrac{26}{3}$
\end{itemize}
\tcblower
\textcolor{green}{\bfseries Answer:} \textbf{(b)} \quad $3x+1=25 \Rightarrow x=8$.
\end{QAPair}

\begin{QAPair}{Question 1 (xi) --- MCQ}
\textcolor{gold}{\bfseries Question:} Which one is the solution of $\dfrac{4}{x}-\dfrac{2}{x}=5$?
\begin{itemize}[leftmargin=*,labelsep=0.6em]
\item[(a)] $-1$
\item[(b)] $\dfrac{2}{5}$
\item[(c)] $\dfrac{5}{2}$
\item[(d)] $0$
\end{itemize}
\tcblower
\textcolor{green}{\bfseries Answer:} \textbf{(b)} \quad $\dfrac{2}{x}=5 \Rightarrow x=\dfrac{2}{5}$.
\end{QAPair}

\begin{QAPair}{Question 1 (xii) --- MCQ}
\textcolor{gold}{\bfseries Question:} Which one is a strict inequality?
\begin{itemize}[leftmargin=*,labelsep=0.6em]
\item[(a)] $x+3\ne 0$
\item[(b)] $12x>5$
\item[(c)] $2y-3\le 0$
\item[(d)] $4x+5\ge 0$
\end{itemize}
\tcblower
\textcolor{green}{\bfseries Answer:} \textbf{(b)} \quad strict inequalities use $<$ or $>$ only.
\end{QAPair}

\begin{QAPair}{Question 1 (xiii) --- MCQ}
\textcolor{gold}{\bfseries Question:} What should be the value of $k$ if $x<y$ shows $kx>ky$?
\begin{itemize}[leftmargin=*,labelsep=0.6em]
\item[(a)] $k=0$
\item[(b)] $k>0$
\item[(c)] $k<0$
\item[(d)] $k\ge 0$
\end{itemize}
\tcblower
\textcolor{green}{\bfseries Answer:} \textbf{(c)} \quad multiplying by a negative reverses the inequality.
\end{QAPair}

\begin{QAPair}{Question 1 (xiv) --- MCQ}
\textcolor{gold}{\bfseries Question:} Which one is the compound relation?
\begin{itemize}[leftmargin=*,labelsep=0.6em]
\item[(a)] $1+2x<4+x$
\item[(b)] $4x+3>5\dfrac{3}{5}$
\item[(c)] $x+y>5\dfrac{1}{2}$
\item[(d)] $x\le 0$
\end{itemize}
\tcblower
\textcolor{green}{\bfseries Answer:} \textbf{(c)} \quad it relates \textbf{two variables} ($x$ and $y$).
\end{QAPair}

\begin{QAPair}{Question 1 (xv) --- MCQ}
\textcolor{gold}{\bfseries Question:} Which one is the solution of $3-\dfrac{1}{2}x\ge 0$?
\begin{itemize}[leftmargin=*,labelsep=0.6em]
\item[(a)] $x\ge -6$
\item[(b)] $x\le 6$
\item[(c)] $x\ge 6$
\item[(d)] $x\le -6$
\end{itemize}
\tcblower
\textcolor{green}{\bfseries Answer:} \textbf{(b)} \quad $3\ge \dfrac{x}{2}\Rightarrow 6\ge x$.
\end{QAPair}

\begin{QAPair}{Question 1 (xvi) --- MCQ}
\textcolor{gold}{\bfseries Question:} Which one is the solution of $2(x+6)\le 3(x+4)$?
\begin{itemize}[leftmargin=*,labelsep=0.6em]
\item[(a)] $x\le 0$
\item[(b)] $x\ge 0$
\item[(c)] $x\le 24$
\item[(d)] $x\ge 24$
\end{itemize}
\tcblower
\textcolor{green}{\bfseries Answer:} \textbf{(b)} \quad $2x+12\le 3x+12\Rightarrow 0\le x$.
\end{QAPair}

% ============================================================
% Solve the following (2) to (9)

\begin{QAPair}{Question 2}
\textcolor{gold}{\bfseries Question:}
\[
\frac{2x-11}{12}
=
\frac{2x+10}{12}
-
\left(\frac{28-2x}{4}-\frac14\right)
\]
\tcblower
\textcolor{green}{\bfseries Answer:}
\[
\begin{aligned}
\Step{1}\;& \left(\frac{28-2x}{4}-\frac14\right)=\frac{28-2x-1}{4}=\frac{27-2x}{4}\\
\Step{2}\;& \text{Multiply whole equation by }12:\\
&2x-11=(2x+10)-12\cdot\frac{27-2x}{4}\\
\Step{3}\;& 12\cdot\frac{27-2x}{4}=3(27-2x)=81-6x\\
\Step{4}\;& 2x-11=2x+10-(81-6x)=8x-71\\
\Step{5}\;& 2x-11=8x-71 \Rightarrow 60=6x \Rightarrow x=10\\[4pt]
&\textcolor{muted}{\textbf{Solution:}} x=10
\end{aligned}
\]
\end{QAPair}

\begin{QAPair}{Question 3}
\textcolor{gold}{\bfseries Question:}
\[
\sqrt{a-\frac12}=\sqrt{\frac{2a}{5}+\frac{2}{5}}
\]
\tcblower
\textcolor{green}{\bfseries Answer:}
\[
\begin{aligned}
\Step{1}\;& a-\frac12=\frac{2a}{5}+\frac{2}{5}\\
\Step{2}\;& \text{Multiply by }10:\quad 10a-5=4a+4\\
\Step{3}\;& 6a=9 \Rightarrow a=\frac{3}{2}\\
\Step{4}\;& \text{Check: both sides }=\sqrt{1}\;\checkmark\\[4pt]
&\textcolor{muted}{\textbf{Solution:}} a=\frac{3}{2}
\end{aligned}
\]
\end{QAPair}

\begin{QAPair}{Question 4}
\textcolor{gold}{\bfseries Question:}
\[
\frac{\sqrt{5x-4}-4}{10}=-1
\]
\tcblower
\textcolor{green}{\bfseries Answer:}
\[
\begin{aligned}
\Step{1}\;& \sqrt{5x-4}-4=-10\\
\Step{2}\;& \sqrt{5x-4}=-6\\
\Step{3}\;& \text{But }\sqrt{5x-4}\ge 0 \text{ (real roots), impossible.}\\[4pt]
&\textcolor{muted}{\textbf{Solution set:}} \phi
\end{aligned}
\]
\end{QAPair}

\begin{QAPair}{Question 5}
\textcolor{gold}{\bfseries Question:}
\[
5-|5y+1|=-9
\]
\tcblower
\textcolor{green}{\bfseries Answer:}
\[
\begin{aligned}
\Step{1}\;& -|5y+1|=-14 \Rightarrow |5y+1|=14\\
\Step{2}\;& 5y+1=14 \Rightarrow y=\frac{13}{5}\\
\Step{3}\;& 5y+1=-14 \Rightarrow y=-3\\[4pt]
&\textcolor{muted}{\textbf{Solution set:}} \left\{-3,\frac{13}{5}\right\}
\end{aligned}
\]
\end{QAPair}

\begin{QAPair}{Question 6}
\textcolor{gold}{\bfseries Question:}
\[
\frac{3}{4}x-1\ge x+1,\quad x\in\mathbb{R}
\]
\tcblower
\textcolor{green}{\bfseries Answer:}
\[
\begin{aligned}
\Step{1}\;& \frac{3}{4}x-x\ge 2\\
\Step{2}\;& -\frac14 x\ge 2\\
\Step{3}\;& x\le -8 \quad (\text{multiply by }-4\text{ flips sign})\\[4pt]
&\textcolor{muted}{\textbf{Solution:}} (-\infty,-8]
\end{aligned}
\]
\end{QAPair}

\begin{QAPair}{Question 7}
\textcolor{gold}{\bfseries Question:}
\[
4(2y+3)-(6y-1)>10,\quad y\in\mathbb{R}
\]
\tcblower
\textcolor{green}{\bfseries Answer:}
\[
\begin{aligned}
\Step{1}\;& 8y+12-6y+1>10\\
\Step{2}\;& 2y+13>10\\
\Step{3}\;& 2y>-3 \Rightarrow y>-\frac{3}{2}\\[4pt]
&\textcolor{muted}{\textbf{Solution:}} \left(-\frac{3}{2},\infty\right)
\end{aligned}
\]
\end{QAPair}

\begin{QAPair}{Question 8}
\textcolor{gold}{\bfseries Question:}
\[
\frac{3}{2}x\le -3 \ \ \textbf{or}\ \ \frac{2}{3}x\ge 4,\quad x\in\mathbb{R}
\]
\tcblower
\textcolor{green}{\bfseries Answer:}
\[
\begin{aligned}
\Step{1}\;& \frac{3}{2}x\le -3 \Rightarrow x\le -2\\
\Step{2}\;& \frac{2}{3}x\ge 4 \Rightarrow x\ge 6\\[4pt]
&\textcolor{muted}{\textbf{Solution:}} (-\infty,-2]\cup[6,\infty)
\end{aligned}
\]
\end{QAPair}

\begin{QAPair}{Question 9}
\textcolor{gold}{\bfseries Question:}
The difference between three times a number $y$ and $18$ is less than $12$ or greater than $39$.\\
Find all real numbers $y$.
\tcblower
\textcolor{green}{\bfseries Answer:}

“The difference” means absolute value:
\[
|3y-18|<12 \quad \textbf{or} \quad |3y-18|>39
\]

\[
\begin{aligned}
\Step{1}\;& |3y-18|<12 \Rightarrow -12<3y-18<12\\
&\Rightarrow 6<3y<30 \Rightarrow 2<y<10\\[6pt]
\Step{2}\;& |3y-18|>39 \Rightarrow
\begin{cases}
3y-18>39\\
3y-18<-39
\end{cases}\\
&\Rightarrow
\begin{cases}
y>19\\
y<-7
\end{cases}\\[6pt]
\Step{3}\;& \text{Union (because OR): } y<-7 \ \text{or}\ 2<y<10 \ \text{or}\ y>19.
\end{aligned}
\]

\[
\textcolor{muted}{\textbf{Solution:}}\quad (-\infty,-7)\cup(2,10)\cup(19,\infty)
\]
\end{QAPair}

\end{document}
