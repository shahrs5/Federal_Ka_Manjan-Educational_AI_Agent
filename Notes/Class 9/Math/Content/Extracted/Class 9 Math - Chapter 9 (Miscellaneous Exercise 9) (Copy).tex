% !TEX TS-program = pdflatex
\documentclass[11pt]{article}

% -------------------- Packages --------------------
\usepackage[a4paper,margin=1in]{geometry}
\usepackage{amsmath,amssymb}
\usepackage[T1]{fontenc}
\usepackage{lmodern}
\usepackage{xcolor}
\usepackage{tcolorbox}
\tcbuselibrary{skins,breakable}
\usepackage{enumitem}
\usepackage{hyperref}
\usepackage{tikz}
\usetikzlibrary{calc,patterns,angles,quotes,arrows.meta,intersections}

\pagestyle{empty}

% -------------------- Dark Theme Colors --------------------
\definecolor{bg}{HTML}{000000}
\definecolor{pairbg}{HTML}{121212}
\definecolor{solbg}{HTML}{0A0A0A}
\definecolor{border}{HTML}{2A2A2A}
\definecolor{text}{HTML}{FFFFFF}
\definecolor{muted}{HTML}{C9CDD3}
\definecolor{gold}{HTML}{FFD700}
\definecolor{green}{HTML}{4ADE80}
\definecolor{cyan}{HTML}{38BDF8}

\pagecolor{bg}
\color{text}

\hypersetup{
  colorlinks=true,
  linkcolor=cyan,
  urlcolor=cyan
}

\setlength{\parindent}{0pt}
\setlength{\parskip}{10pt}

\setlist[itemize]{left=1.4em,itemsep=6pt,topsep=6pt}
\setlist[enumerate]{left=1.6em,itemsep=4pt,topsep=4pt}

% -------------------- tcolorbox Base --------------------
\tcbset{
  enhanced,
  breakable,
  arc=12pt,
  boxrule=0.8pt,
  left=16pt,right=16pt,top=12pt,bottom=12pt
}

\newtcolorbox{QAPair}[1]{%
  colback=pairbg,
  colbacklower=solbg,
  colframe=border,
  coltext=text,
  title=\textcolor{gold}{\bfseries #1},
  fonttitle=\bfseries,
  coltitle=text,
  segmentation style={draw=border, dashed, line width=0.6pt}
}

\newtcolorbox{QuickBox}{%
  colback=pairbg,
  colframe=cyan,
  coltext=text,
  fontupper=\color{text},
  borderline north={4pt}{0pt}{cyan},
  arc=14pt,
  boxrule=0.8pt
}

% Helper for step headings
\newcommand{\Step}[1]{\textcolor{muted}{\textbf{Step #1:}}}

% -------------------- TikZ Styles --------------------
\tikzset{
  geoLine/.style={draw=muted, line width=0.9pt},
  geoBold/.style={draw=text, line width=1.1pt},
  geoDash/.style={draw=muted, dashed, line width=0.9pt},
  geoFill/.style={fill=cyan, fill opacity=0.18, draw=none},
  geoPoint/.style={fill=text, draw=none}, % used with \fill[geoPoint] ... circle(...)
  geoLabel/.style={text=text, font=\small},
  geoSmall/.style={text=text, font=\scriptsize},
}

% ============================================================
\begin{document}

\begin{center}
{\LARGE\bfseries \textcolor{gold}{Miscellaneous Exercise 9 --- Solutions}}\\[-2pt]
\end{center}

\begin{QuickBox}
{\color{cyan}\bfseries Quick formulas (useful)}\par\medskip
\begin{itemize}
\item \textbf{Sum of interior angles (any $n$-gon):} $(n-2)\times 180^\circ$.
\item \textbf{Each interior angle (regular $n$-gon):} $\displaystyle \frac{(n-2)180^\circ}{n}$.
\item \textbf{Sum of exterior angles (any polygon):} $360^\circ$.
\item \textbf{Each exterior angle (regular $n$-gon):} $\displaystyle \frac{360^\circ}{n}$.
\item \textbf{Similar triangles:} corresponding sides are proportional.
\item \textbf{Scaling for similar solids:} linear ratio $k \Rightarrow$ surface area ratio $k^2$, volume ratio $k^3$.
\item \textbf{Angle bisector theorem:} if a line bisects a triangle angle, it divides the opposite side in the ratio of adjacent sides.
\end{itemize}
\end{QuickBox}

% ============================================================
% Q1 (MCQs)

\begin{QAPair}{Question 1 (i) --- MCQ}
\textcolor{gold}{\bfseries Question:} Which of the following is polygon?
\begin{itemize}
\item[(a)] circle \hfill (b) pyramid \hfill (c) quadrilateral \hfill (d) sphere
\end{itemize}
\tcblower
\textcolor{green}{\bfseries Answer:} \textbf{(c) quadrilateral}
\[
\begin{aligned}
\Step{1}\;& \text{A polygon is a \emph{closed plane figure} made of straight line segments.}\\
\Step{2}\;& \text{A quadrilateral is a 4-sided polygon.}
\end{aligned}
\]
\end{QAPair}

\begin{QAPair}{Question 1 (ii) --- MCQ}
\textcolor{gold}{\bfseries Question:} Which of the following is regular polygon?
\begin{itemize}
\item[(a)] kite \hfill (b) rhombus \hfill (c) rectangle \hfill (d) square
\end{itemize}
\tcblower
\textcolor{green}{\bfseries Answer:} \textbf{(d) square}
\[
\begin{aligned}
\Step{1}\;& \text{Regular polygon: all sides equal and all angles equal.}\\
\Step{2}\;& \text{Only a square has both equal sides and equal angles among the options.}
\end{aligned}
\]
\end{QAPair}

\begin{QAPair}{Question 1 (iii) --- MCQ}
\textcolor{gold}{\bfseries Question:} If two triangles are similar, their corresponding sides are:
\begin{itemize}
\item[(a)] proportional \hfill (b) equal \hfill (c) congruent \hfill (d) parallel
\end{itemize}
\tcblower
\textcolor{green}{\bfseries Answer:} \textbf{(a) proportional}
\[
\begin{aligned}
\Step{1}\;& \triangle_1 \sim \triangle_2 \Rightarrow \frac{a_1}{a_2}=\frac{b_1}{b_2}=\frac{c_1}{c_2}.\\
\Step{2}\;& \text{So corresponding sides are proportional.}
\end{aligned}
\]
\end{QAPair}

\begin{QAPair}{Question 1 (iv) --- MCQ}
\textcolor{gold}{\bfseries Question:} What is the sum of interior angles for an irregular hexagon?
\begin{itemize}
\item[(a)] $120^\circ$ \hfill (b) $720^\circ$ \hfill (c) $135^\circ$ \hfill (d) $360^\circ$
\end{itemize}
\tcblower
\textcolor{green}{\bfseries Answer:} \textbf{(b) $720^\circ$}
\[
\begin{aligned}
\Step{1}\;& \text{Sum of interior angles of an $n$-gon}=(n-2)\cdot 180^\circ.\\
\Step{2}\;& n=6 \Rightarrow (6-2)\cdot 180^\circ=4\cdot 180^\circ=720^\circ.
\end{aligned}
\]
\end{QAPair}

\begin{QAPair}{Question 1 (v) --- MCQ}
\textcolor{gold}{\bfseries Question:} What is the sum of interior angles for a regular 12 sided polygon?
\begin{itemize}
\item[(a)] $1800^\circ$ \hfill (b) $2160^\circ$ \hfill (c) $1980^\circ$ \hfill (d) $360^\circ$
\end{itemize}
\tcblower
\textcolor{green}{\bfseries Answer:} \textbf{(a) $1800^\circ$}
\[
\begin{aligned}
\Step{1}\;& (n-2)\cdot 180^\circ \text{ with } n=12.\\
\Step{2}\;& (12-2)\cdot 180^\circ=10\cdot 180^\circ=1800^\circ.
\end{aligned}
\]
\end{QAPair}

\begin{QAPair}{Question 1 (vi) --- MCQ}
\textcolor{gold}{\bfseries Question:} How many sides a regular polygon has if its exterior angle is $15^\circ$?
\begin{itemize}
\item[(a)] 20 \hfill (b) 21 \hfill (c) 24 \hfill (d) 25
\end{itemize}
\tcblower
\textcolor{green}{\bfseries Answer:} \textbf{(c) 24}
\[
\begin{aligned}
\Step{1}\;& \text{Each exterior angle of a regular $n$-gon}=\frac{360^\circ}{n}.\\
\Step{2}\;& \frac{360^\circ}{n}=15^\circ \Rightarrow n=\frac{360}{15}=24.
\end{aligned}
\]
\end{QAPair}

\begin{center}
\includegraphics[width=16cm, height=4cm]{restricted.jpeg}
\end{center}

\end{document}
