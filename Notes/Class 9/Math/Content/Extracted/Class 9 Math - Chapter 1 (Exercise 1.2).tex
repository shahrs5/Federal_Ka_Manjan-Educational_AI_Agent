% !TEX TS-program = pdflatex
\documentclass[11pt]{article}

% -------------------- Packages --------------------
\usepackage[a4paper,margin=1in]{geometry}
\usepackage{amsmath,amssymb}
\usepackage[T1]{fontenc}
\usepackage{lmodern}
\usepackage{xcolor}
\usepackage{tcolorbox}
\tcbuselibrary{skins,breakable}
\usepackage{enumitem}
\usepackage{hyperref}

\pagestyle{empty}

% -------------------- Dark Theme Colors --------------------
\definecolor{bg}{HTML}{000000}
\definecolor{pairbg}{HTML}{121212}
\definecolor{solbg}{HTML}{0A0A0A}
\definecolor{border}{HTML}{2A2A2A}
\definecolor{text}{HTML}{FFFFFF}
\definecolor{muted}{HTML}{C9CDD3}
\definecolor{gold}{HTML}{FFD700}
\definecolor{green}{HTML}{4ADE80}
\definecolor{cyan}{HTML}{38BDF8}

\pagecolor{bg}
\color{text}

\hypersetup{
  colorlinks=true,
  linkcolor=cyan,
  urlcolor=cyan
}

\setlength{\parindent}{0pt}
\setlength{\parskip}{10pt}

\setlist[itemize]{left=1.4em,itemsep=6pt,topsep=6pt}
\setlist[enumerate]{left=1.6em,itemsep=4pt,topsep=4pt}

% -------------------- tcolorbox Base --------------------
\tcbset{
  enhanced,
  breakable,
  arc=12pt,
  boxrule=0.8pt,
  left=16pt,right=16pt,top=12pt,bottom=12pt
}

\newtcolorbox{QAPair}[1]{%
  colback=pairbg,
  colbacklower=solbg,
  colframe=border,
  coltext=text,
  title=\textcolor{gold}{\bfseries #1},
  fonttitle=\bfseries,
  coltitle=text,
  segmentation style={draw=border, dashed, line width=0.6pt},
}

% Visible text inside this box (fix)
\newtcolorbox{QuickBox}{%
  colback=pairbg,
  colframe=cyan,
  coltext=text,
  fontupper=\color{text},
  borderline north={4pt}{0pt}{cyan},
  arc=14pt,
  boxrule=0.8pt
}

% Helper for step headings
\newcommand{\Step}[1]{\textcolor{muted}{\textbf{Step #1:}}}

% ============================================================
\begin{document}

\begin{center}
{\LARGE\bfseries \textcolor{gold}{Exercise 1.2 --- Solutions}}\\[-2pt]
\end{center}

\begin{QuickBox}
{\color{cyan}\bfseries Quick formulas (useful)}\par\medskip
\begin{itemize}
\item \textbf{Product rule (same index):} $\sqrt[n]{a}\,\sqrt[n]{b}=\sqrt[n]{ab}$.
\item \textbf{Quotient rule (same index):} $\dfrac{\sqrt[n]{a}}{\sqrt[n]{b}}=\sqrt[n]{\dfrac{a}{b}}$.
\item \textbf{Exponent to radical:} $a^{\frac{m}{n}}=\sqrt[n]{a^m}$.
\item \textbf{Radical to exponent:} $\sqrt[n]{a^m}=a^{\frac{m}{n}}$.
\item \textbf{Exponent laws:} $a^m a^n=a^{m+n}$,\;
$\dfrac{a^m}{a^n}=a^{m-n}$,\;
$(a^m)^n=a^{mn}$.
\end{itemize}
\end{QuickBox}

% ============================================================
% Q1
\begin{QAPair}{Question 1 (i)}
\textcolor{gold}{\bfseries Question:} $\sqrt[3]{6}\cdot \sqrt[3]{6}$\\
\tcblower
\textcolor{green}{\bfseries Answer:}
\[
\begin{aligned}
\Step{1}\;& \sqrt[3]{6}\cdot \sqrt[3]{6} = \sqrt[3]{6\cdot 6} \qquad (\text{product rule, same index})\\
\Step{2}\;&= \sqrt[3]{36}.
\end{aligned}
\]
\end{QAPair}

\begin{QAPair}{Question 1 (ii)}
\textcolor{gold}{\bfseries Question:} $\sqrt[5]{4}\cdot \sqrt[5]{8}$\\
\tcblower
\textcolor{green}{\bfseries Answer:}
\[
\begin{aligned}
\Step{1}\;& \sqrt[5]{4}\cdot \sqrt[5]{8} = \sqrt[5]{4\cdot 8} \qquad (\text{product rule})\\
\Step{2}\;&= \sqrt[5]{32}\\
\Step{3}\;&= \sqrt[5]{2^5} \qquad (32=2^5)\\
\Step{4}\;&= 2.
\end{aligned}
\]
\end{QAPair}

\begin{QAPair}{Question 1 (iii)}
\textcolor{gold}{\bfseries Question:} $\sqrt[4]{x}\cdot \sqrt[4]{x^3}$\\
\tcblower
\textcolor{green}{\bfseries Answer:}
\[
\begin{aligned}
\Step{1}\;& \sqrt[4]{x}\cdot \sqrt[4]{x^3}=\sqrt[4]{x\cdot x^3} \qquad (\text{product rule})\\
\Step{2}\;&= \sqrt[4]{x^4}\\
\Step{3}\;&= x. \qquad (\text{since } \sqrt[4]{x^4}=x \text{ for } x>0)
\end{aligned}
\]
\end{QAPair}

\begin{QAPair}{Question 1 (iv)}
\textcolor{gold}{\bfseries Question:} $\sqrt{10}\cdot \sqrt[3]{11}$\\
\tcblower
\textcolor{green}{\bfseries Answer:}
\[
\begin{aligned}
\Step{1}\;& \sqrt{10}=10^{\frac12},\quad \sqrt[3]{11}=11^{\frac13}.\\
\Step{2}\;& 10^{\frac12}\cdot 11^{\frac13}
= 10^{\frac{3}{6}}\cdot 11^{\frac{2}{6}} \qquad (\text{LCM}(2,3)=6)\\
\Step{3}\;&= \sqrt[6]{10^3}\cdot \sqrt[6]{11^2}\\
\Step{4}\;&= \sqrt[6]{10^3\cdot 11^2} \qquad (\text{product rule})\\
\Step{5}\;&= \sqrt[6]{121000}.
\end{aligned}
\]
\end{QAPair}

\begin{QAPair}{Question 1 (v)}
\textcolor{gold}{\bfseries Question:} $\dfrac{\sqrt[4]{x^7}}{\sqrt[4]{x^5}}$\\
\tcblower
\textcolor{green}{\bfseries Answer:}
\[
\begin{aligned}
\Step{1}\;& \frac{\sqrt[4]{x^7}}{\sqrt[4]{x^5}}
= \sqrt[4]{\frac{x^7}{x^5}} \qquad (\text{quotient rule, same index})\\
\Step{2}\;&= \sqrt[4]{x^{7-5}}=\sqrt[4]{x^2}\\
\Step{3}\;&= x^{\frac{2}{4}}=x^{\frac12}=\sqrt{x}.
\end{aligned}
\]
\end{QAPair}

\begin{QAPair}{Question 1 (vi)}
\textcolor{gold}{\bfseries Question:} $\dfrac{\sqrt[3]{5000}}{\sqrt[3]{5}}$\\
\tcblower
\textcolor{green}{\bfseries Answer:}
\[
\begin{aligned}
\Step{1}\;& \frac{\sqrt[3]{5000}}{\sqrt[3]{5}}
= \sqrt[3]{\frac{5000}{5}} \qquad (\text{quotient rule})\\
\Step{2}\;&= \sqrt[3]{1000}\\
\Step{3}\;&= \sqrt[3]{10^3}=10.
\end{aligned}
\]
\end{QAPair}

\begin{QAPair}{Question 1 (vii)}
\textcolor{gold}{\bfseries Question:} $\dfrac{\sqrt{500}}{\sqrt{5}}$\\
\tcblower
\textcolor{green}{\bfseries Answer:}
\[
\begin{aligned}
\Step{1}\;& \frac{\sqrt{500}}{\sqrt{5}}=\sqrt{\frac{500}{5}} \qquad (\text{quotient rule})\\
\Step{2}\;&= \sqrt{100}\\
\Step{3}\;&= 10.
\end{aligned}
\]
\end{QAPair}

\begin{QAPair}{Question 1 (viii)}
\textcolor{gold}{\bfseries Question:} $\sqrt{10}\cdot \sqrt[3]{7}$\\
\tcblower
\textcolor{green}{\bfseries Answer:}
\[
\begin{aligned}
\Step{1}\;& \sqrt{10}=10^{\frac12},\quad \sqrt[3]{7}=7^{\frac13}.\\
\Step{2}\;& 10^{\frac12}\cdot 7^{\frac13}
=10^{\frac{3}{6}}\cdot 7^{\frac{2}{6}} \qquad (\text{LCM}=6)\\
\Step{3}\;&=\sqrt[6]{10^3}\cdot \sqrt[6]{7^2}\\
\Step{4}\;&=\sqrt[6]{10^3\cdot 7^2}=\sqrt[6]{49000}.
\end{aligned}
\]
\end{QAPair}

% ============================================================
% Q2
\begin{QAPair}{Question 2 (i)}
\textcolor{gold}{\bfseries Question:} $(216)^{\frac{2}{3}}$\\
\tcblower
\textcolor{green}{\bfseries Answer:}
\[
\begin{aligned}
\Step{1}\;& 216^{\frac{2}{3}}=\left(\sqrt[3]{216}\right)^2 \qquad \left(a^{\frac{m}{n}}=\left(\sqrt[n]{a}\right)^m\right)\\
\Step{2}\;&= 6^2 \qquad (\sqrt[3]{216}=6)\\
\Step{3}\;&= 36.
\end{aligned}
\]
\end{QAPair}

\begin{QAPair}{Question 2 (ii)}
\textcolor{gold}{\bfseries Question:} $(29)^{\frac{1}{2}}$\\
\tcblower
\textcolor{green}{\bfseries Answer:}
\[
\Step{1}\; 29^{\frac12}=\sqrt{29}.
\]
\end{QAPair}

\begin{QAPair}{Question 2 (iii)}
\textcolor{gold}{\bfseries Question:} $\left(\dfrac{1}{32}\right)^{\frac{1}{5}}$\\
\tcblower
\textcolor{green}{\bfseries Answer:}
\[
\begin{aligned}
\Step{1}\;& \left(\frac{1}{32}\right)^{\frac15}=\sqrt[5]{\frac{1}{32}}\\
\Step{2}\;&= \frac{1}{\sqrt[5]{32}}\\
\Step{3}\;&= \frac{1}{\sqrt[5]{2^5}}=\frac{1}{2}.
\end{aligned}
\]
\end{QAPair}

\begin{QAPair}{Question 2 (iv)}
\textcolor{gold}{\bfseries Question:} $(216)^{-\frac{2}{3}}$\\
\tcblower
\textcolor{green}{\bfseries Answer:}
\[
\begin{aligned}
\Step{1}\;&216^{-\frac23}=\frac{1}{216^{\frac23}} \qquad (a^{-k}=\tfrac{1}{a^k})\\
\Step{2}\;&=\frac{1}{36}. \qquad (\text{from Q2(i)})
\end{aligned}
\]
\end{QAPair}

\begin{QAPair}{Question 2 (v)}
\textcolor{gold}{\bfseries Question:} $(1000)^{\frac{1}{3}}$\\
\tcblower
\textcolor{green}{\bfseries Answer:}
\[
\begin{aligned}
\Step{1}\;& 1000^{\frac13}=\sqrt[3]{1000}\\
\Step{2}\;&=\sqrt[3]{10^3}=10.
\end{aligned}
\]
\end{QAPair}

\begin{QAPair}{Question 2 (vi)}
\textcolor{gold}{\bfseries Question:} $\left(\dfrac{1}{39}\right)^{\frac{1}{2}}$\\
\tcblower
\textcolor{green}{\bfseries Answer:}
\[
\begin{aligned}
\Step{1}\;&\left(\frac{1}{39}\right)^{\frac12}=\sqrt{\frac{1}{39}}\\
\Step{2}\;&=\frac{1}{\sqrt{39}}.
\end{aligned}
\]
\end{QAPair}

% ============================================================
% Q3
\begin{QAPair}{Question 3 (i)}
\textcolor{gold}{\bfseries Question:} $(\sqrt[3]{5})^2$\\
\tcblower
\textcolor{green}{\bfseries Answer:}
\[
\begin{aligned}
\Step{1}\;& \sqrt[3]{5}=5^{\frac13}\\
\Step{2}\;& (\sqrt[3]{5})^2=(5^{\frac13})^2=5^{\frac{2}{3}}.
\end{aligned}
\]
\end{QAPair}

\begin{QAPair}{Question 3 (ii)}
\textcolor{gold}{\bfseries Question:} $(\sqrt[4]{10})^8$\\
\tcblower
\textcolor{green}{\bfseries Answer:}
\[
\begin{aligned}
\Step{1}\;& \sqrt[4]{10}=10^{\frac14}\\
\Step{2}\;& (\sqrt[4]{10})^8=(10^{\frac14})^8=10^{\frac{8}{4}}=10^2=100.
\end{aligned}
\]
\end{QAPair}

\begin{QAPair}{Question 3 (iii)}
\textcolor{gold}{\bfseries Question:} $-(\sqrt[3]{6})^6$\\
\tcblower
\textcolor{green}{\bfseries Answer:}
\[
\begin{aligned}
\Step{1}\;& (\sqrt[3]{6})^6=(6^{\frac13})^6=6^{\frac{6}{3}}=6^2=36\\
\Step{2}\;& -(\sqrt[3]{6})^6=-36.
\end{aligned}
\]
\end{QAPair}

\begin{QAPair}{Question 3 (iv)}
\textcolor{gold}{\bfseries Question:} $(\sqrt[3]{6})^6$\\
\tcblower
\textcolor{green}{\bfseries Answer:}
\[
(\sqrt[3]{6})^6=(6^{\frac13})^6=6^{\frac{6}{3}}=6^2=36.
\]
\end{QAPair}

\begin{QAPair}{Question 3 (v)}
\textcolor{gold}{\bfseries Question:} $-(\sqrt[3]{5})^2$\\
\tcblower
\textcolor{green}{\bfseries Answer:}
\[
\begin{aligned}
\Step{1}\;& (\sqrt[3]{5})^2=(5^{\frac13})^2=5^{\frac23}\\
\Step{2}\;& -(\sqrt[3]{5})^2=-5^{\frac23}.
\end{aligned}
\]
\end{QAPair}

\begin{QAPair}{Question 3 (vi)}
\textcolor{gold}{\bfseries Question:} $-(\sqrt[4]{10})^8$\\
\tcblower
\textcolor{green}{\bfseries Answer:}
\[
\begin{aligned}
\Step{1}\;& (\sqrt[4]{10})^8=(10^{\frac14})^8=10^2=100\\
\Step{2}\;& -(\sqrt[4]{10})^8=-100.
\end{aligned}
\]
\end{QAPair}

% ============================================================
% Q4
\begin{QAPair}{Question 4 (i)}
\textcolor{gold}{\bfseries Question:} $\dfrac{16^{\frac15}\cdot 16^{\frac14}}{16^{-\frac{3}{10}}}$\\
\tcblower
\textcolor{green}{\bfseries Answer:}
\[
\begin{aligned}
\Step{1}\;& \frac{16^{\frac15}\cdot 16^{\frac14}}{16^{-\frac{3}{10}}}
=16^{\frac15+\frac14-(-\frac{3}{10})}\\
\Step{2}\;&=16^{\frac15+\frac14+\frac{3}{10}}
=16^{\frac{4}{20}+\frac{5}{20}+\frac{6}{20}}=16^{\frac{15}{20}}=16^{\frac34}\\
\Step{3}\;&= (16^{\frac14})^3 = 2^3 = 8.
\end{aligned}
\]
\end{QAPair}

\begin{QAPair}{Question 4 (ii)}
\textcolor{gold}{\bfseries Question:} $7^{-\frac13}\bigl(7^{\frac53}-7^{\frac43}\bigr)$\\
\tcblower
\textcolor{green}{\bfseries Answer:}
\[
\begin{aligned}
\Step{1}\;& 7^{-\frac13}\cdot 7^{\frac53}=7^{-\frac13+\frac53}=7^{\frac43}\\
\Step{2}\;& 7^{-\frac13}\cdot 7^{\frac43}=7^{-\frac13+\frac43}=7^{1}=7\\
\Step{3}\;& \Rightarrow\; 7^{-\frac13}(7^{\frac53}-7^{\frac43})=7^{\frac43}-7.
\end{aligned}
\]
\end{QAPair}

\begin{QAPair}{Question 4 (iii)}
\textcolor{gold}{\bfseries Question:} $\dfrac{2^{\frac23}\cdot 2^{\frac17}}{2^{\frac12}}$\\
\tcblower
\textcolor{green}{\bfseries Answer:}
\[
\begin{aligned}
\Step{1}\;& \frac{2^{\frac23}\cdot 2^{\frac17}}{2^{\frac12}}
=2^{\frac23+\frac17-\frac12}\\
\Step{2}\;&=2^{\frac{28}{42}+\frac{6}{42}-\frac{21}{42}}
=2^{\frac{13}{42}}.
\end{aligned}
\]
\end{QAPair}

\begin{QAPair}{Question 4 (iv)}
\textcolor{gold}{\bfseries Question:} $\dfrac{3^{-\frac12}\cdot 3^{\frac12}}{3^{\frac12}}$\\
\tcblower
\textcolor{green}{\bfseries Answer:}
\[
\begin{aligned}
\Step{1}\;& \frac{3^{-\frac12}\cdot 3^{\frac12}}{3^{\frac12}}
=3^{-\frac12+\frac12-\frac12}=3^{-\frac12}\\
\Step{2}\;&= \frac{1}{3^{\frac12}}=\frac{1}{\sqrt{3}}.
\end{aligned}
\]
\end{QAPair}

\begin{QAPair}{Question 4 (v)}
\textcolor{gold}{\bfseries Question:} $\left(\dfrac{36^{\frac12}\cdot 6^{\frac12}}{8^{\frac12}\cdot 27^{\frac12}}\right)^3$\\
\tcblower
\textcolor{green}{\bfseries Answer:}
\[
\begin{aligned}
\Step{1}\;& \frac{36^{\frac12}\cdot 6^{\frac12}}{8^{\frac12}\cdot 27^{\frac12}}
= \frac{\sqrt{36}\,\sqrt{6}}{\sqrt{8}\,\sqrt{27}}\\
\Step{2}\;&= \frac{6\sqrt{6}}{(2\sqrt{2})(3\sqrt{3})}
= \frac{6\sqrt{6}}{6\sqrt{6}}=1\\
\Step{3}\;& \Rightarrow \left(\frac{36^{\frac12}\cdot 6^{\frac12}}{8^{\frac12}\cdot 27^{\frac12}}\right)^3
=1^3=1.
\end{aligned}
\]
\end{QAPair}

\begin{QAPair}{Question 4 (vi)}
\textcolor{gold}{\bfseries Question:} $\left(\dfrac{2187\,a^5 b^{17}}{a^{12} b^3}\right)^{\frac17}$\\
\tcblower
\textcolor{green}{\bfseries Answer:}
\[
\begin{aligned}
\Step{1}\;& \frac{2187\,a^5 b^{17}}{a^{12} b^3}
=2187\cdot a^{5-12}\cdot b^{17-3}
=2187\cdot a^{-7}\cdot b^{14}\\
\Step{2}\;& 2187=3^7 \;\Rightarrow\; 2187\cdot a^{-7}\cdot b^{14}
=\frac{3^7\,b^{14}}{a^7}\\
\Step{3}\;& \left(\frac{3^7\,b^{14}}{a^7}\right)^{\frac17}
=\frac{(3^7)^{\frac17}\,(b^{14})^{\frac17}}{(a^7)^{\frac17}}
=\frac{3\,b^2}{a}.
\end{aligned}
\]
\end{QAPair}

\begin{QAPair}{Question 4 (vii)}
\textcolor{gold}{\bfseries Question:} $\sqrt[4]{\dfrac{a^3}{b^3}}\cdot \sqrt[4]{\dfrac{b^3}{c^3}}\cdot \sqrt[4]{\dfrac{c^3}{a^3}}$\\
\tcblower
\textcolor{green}{\bfseries Answer:}
\[
\begin{aligned}
\Step{1}\;& \sqrt[4]{\frac{a^3}{b^3}}\cdot \sqrt[4]{\frac{b^3}{c^3}}\cdot \sqrt[4]{\frac{c^3}{a^3}}
=\sqrt[4]{\frac{a^3}{b^3}\cdot \frac{b^3}{c^3}\cdot \frac{c^3}{a^3}}\\
\Step{2}\;&=\sqrt[4]{1}=1.
\end{aligned}
\]
\end{QAPair}

% ============================================================
% Q5
\begin{QAPair}{Question 5}
\textcolor{gold}{\bfseries Question:} Use laws of exponents to show that
\[
\left(\frac{x^p}{x^q}\right)^{p+q}
\left(\frac{y^q}{y^r}\right)^{q+r}
\left(\frac{z^r}{z^p}\right)^{r+p}
x^{q^2}y^{r^2}z^{p^2}
= x^{p^2}y^{q^2}z^{r^2}.
\]
\tcblower
\textcolor{green}{\bfseries Answer:}
\[
\begin{aligned}
\Step{1}\;& \left(\frac{x^p}{x^q}\right)^{p+q}=(x^{p-q})^{p+q}=x^{(p-q)(p+q)}=x^{p^2-q^2}.\\
\Step{2}\;& \left(\frac{y^q}{y^r}\right)^{q+r}=(y^{q-r})^{q+r}=y^{(q-r)(q+r)}=y^{q^2-r^2}.\\
\Step{3}\;& \left(\frac{z^r}{z^p}\right)^{r+p}=(z^{r-p})^{r+p}=z^{(r-p)(r+p)}=z^{r^2-p^2}.\\
\Step{4}\;& x^{p^2-q^2}\cdot x^{q^2}=x^{p^2},\quad
y^{q^2-r^2}\cdot y^{r^2}=y^{q^2},\quad
z^{r^2-p^2}\cdot z^{p^2}=z^{r^2}.\\
\Step{5}\;& \Rightarrow\; \text{LHS}=x^{p^2}y^{q^2}z^{r^2}=\text{RHS}.
\end{aligned}
\]
\end{QAPair}

\end{document}
