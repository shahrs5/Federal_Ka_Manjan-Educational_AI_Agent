% !TEX TS-program = pdflatex
\documentclass[11pt]{article}

% -------------------- Packages --------------------
\usepackage[a4paper,margin=1in]{geometry}
\usepackage{amsmath,amssymb}
\usepackage[T1]{fontenc}
\usepackage{lmodern}
\usepackage{xcolor}
\usepackage{tcolorbox}
\tcbuselibrary{skins,breakable}
\usepackage{enumitem}
\usepackage{hyperref}
\usepackage{tikz}
\usetikzlibrary{arrows.meta}

\pagestyle{empty}

% -------------------- Dark Theme Colors --------------------
\definecolor{bg}{HTML}{000000}
\definecolor{pairbg}{HTML}{121212}
\definecolor{solbg}{HTML}{0A0A0A}
\definecolor{border}{HTML}{2A2A2A}
\definecolor{text}{HTML}{FFFFFF}
\definecolor{muted}{HTML}{C9CDD3}
\definecolor{gold}{HTML}{FFD700}
\definecolor{green}{HTML}{4ADE80}
\definecolor{cyan}{HTML}{38BDF8}

\pagecolor{bg}
\color{text}

\hypersetup{
  colorlinks=true,
  linkcolor=cyan,
  urlcolor=cyan
}

\setlength{\parindent}{0pt}
\setlength{\parskip}{10pt}

\setlist[itemize]{left=1.4em,itemsep=6pt,topsep=6pt}
\setlist[enumerate]{left=1.6em,itemsep=4pt,topsep=4pt}

% -------------------- tcolorbox Base --------------------
\tcbset{
  enhanced,
  breakable,
  arc=12pt,
  boxrule=0.8pt,
  left=16pt,right=16pt,top=12pt,bottom=12pt
}

\newtcolorbox{QAPair}[1]{%
  colback=pairbg,
  colbacklower=solbg,
  colframe=border,
  coltext=text,
  title=\textcolor{gold}{\bfseries #1},
  fonttitle=\bfseries,
  coltitle=text,
  segmentation style={draw=border, dashed, line width=0.6pt},
}

% Visible text inside this box (fix)
\newtcolorbox{QuickBox}{%
  colback=pairbg,
  colframe=cyan,
  coltext=text,
  fontupper=\color{text},
  borderline north={4pt}{0pt}{cyan},
  arc=14pt,
  boxrule=0.8pt
}

% Helper for step headings
\newcommand{\Step}[1]{\textcolor{muted}{\textbf{Step #1:}}}

% -------------------- TikZ: Number line helpers --------------------
\tikzset{
  axis/.style={draw=text, very thick, -{Latex[length=3mm]}},
  tick/.style={draw=text, thick},
  ticklbl/.style={text=muted, font=\small},
  point/.style={fill=gold, draw=gold},
  openpt/.style={draw=gold, line width=1.2pt, fill=none},
  shade/.style={draw=green, line width=3.2pt, line cap=round},
}

% xmin,xmax,point x, caption
\newcommand{\NumLinePoint}[4]{%
\begin{center}
\begin{tikzpicture}[x=0.85cm,y=1cm]
  \draw[axis] (#1,0) -- (#2,0);
  \foreach \x in {#1,...,#2} {
    \draw[tick] (\x,0.12)--(\x,-0.12);
    \node[ticklbl, below] at (\x,-0.12) {\x};
  }
  \fill[point] (#3,0) circle (0.10);
  \node[above, text=gold, font=\small] at (#3,0.20) {$#3$};
\end{tikzpicture}

{\small\color{muted}#4}
\end{center}
}

% xmin,xmax, left end x, right end x, left open? (0/1), right open? (0/1), caption
\newcommand{\NumLineInterval}[7]{%
\begin{center}
\begin{tikzpicture}[x=0.85cm,y=1cm]
  \draw[axis] (#1,0) -- (#2,0);
  \foreach \x in {#1,...,#2} {
    \draw[tick] (\x,0.12)--(\x,-0.12);
    \node[ticklbl, below] at (\x,-0.12) {\x};
  }
  \draw[shade] (#3,0) -- (#4,0);

  % left endpoint
  \ifnum#5=1
    \draw[openpt] (#3,0) circle (0.12);
  \else
    \fill[point] (#3,0) circle (0.12);
  \fi

  % right endpoint
  \ifnum#6=1
    \draw[openpt] (#4,0) circle (0.12);
  \else
    \fill[point] (#4,0) circle (0.12);
  \fi
\end{tikzpicture}

{\small\color{muted}#7}
\end{center}
}

% xmin,xmax, endpoint x, direction (L/R), open flag (0/1), caption
\newcommand{\NumLineRay}[6]{%
\begingroup
\def\Dir{#4}\def\Ldir{L}%
\begin{center}
\begin{tikzpicture}[x=0.85cm,y=1cm]
  \draw[axis] (#1,0) -- (#2,0);
  \foreach \x in {#1,...,#2} {
    \draw[tick] (\x,0.12)--(\x,-0.12);
    \node[ticklbl, below] at (\x,-0.12) {\x};
  }

  % endpoint
  \ifnum#5=1
    \draw[openpt] (#3,0) circle (0.12);
  \else
    \fill[point] (#3,0) circle (0.12);
  \fi

  % ray
  \ifx\Dir\Ldir
    \draw[shade] (#3,0) -- (#1,0);
    \draw[shade, -{Latex[length=3mm]}] (#1,0) -- (#1-0.6,0);
  \else
    \draw[shade] (#3,0) -- (#2,0);
    \draw[shade, -{Latex[length=3mm]}] (#2,0) -- (#2+0.6,0);
  \fi
\end{tikzpicture}

{\small\color{muted}#6}
\end{center}
\endgroup
}

% ============================================================
\begin{document}

\begin{center}
{\LARGE\bfseries \textcolor{gold}{Exercise 1.1 --- Solutions}}\\[-2pt]
\end{center}

\begin{QuickBox}
{\color{cyan}\bfseries Quick formulas (useful)}\par\medskip
\begin{itemize}
\item \textbf{Fraction to decimal:} $\dfrac{a}{b}=a\div b$.
\item \textbf{Mixed number:} $m\dfrac{a}{b}=\dfrac{mb+a}{b}$.
\item \textbf{Root approximation:} $\sqrt{8}\approx 2.828$.
\item \textbf{Inequality on number line:} open circle for $<$ or $>$; closed circle for $\le$ or $\ge$.
\item \textbf{Useful properties:} Reflexive ($a=a$), Symmetric ($a=b\Rightarrow b=a$), Transitive ($a=b,b=c\Rightarrow a=c$), Distributive ($a(b-c)=ab-ac$).
\end{itemize}
\end{QuickBox}

% ============================================================
% Q1
\begin{QAPair}{Question 1 (i)}
\textcolor{gold}{\bfseries Question:} Represent $\dfrac{3}{4}$ on the number line.\\
\tcblower
\textcolor{green}{\bfseries Answer:}
\[
\begin{aligned}
\Step{1}\;& \frac{3}{4}=3\div 4=0.75.\\
\Step{2}\;& 0.75 \text{ lies between } 0 \text{ and } 1 \text{ (closer to }1\text{).}
\end{aligned}
\]
\NumLinePoint{-1}{2}{0.75}{Point at $0.75$ (i.e., $\frac34$).}
\end{QAPair}

\begin{QAPair}{Question 1 (ii)}
\textcolor{gold}{\bfseries Question:} Represent $-\dfrac{1}{3}$ on the number line.\\
\tcblower
\textcolor{green}{\bfseries Answer:}
\[
\begin{aligned}
\Step{1}\;& -\frac{1}{3}=-(1\div 3)\approx -0.333.\\
\Step{2}\;& -0.333 \text{ lies between } -1 \text{ and } 0.
\end{aligned}
\]
\NumLinePoint{-2}{1}{-0.333}{Point at approximately $-0.333$ (i.e., $-\frac13$).}
\end{QAPair}

\begin{QAPair}{Question 1 (iii)}
\textcolor{gold}{\bfseries Question:} Represent $4\dfrac{1}{2}$ on the number line.\\
\tcblower
\textcolor{green}{\bfseries Answer:}
\[
\begin{aligned}
\Step{1}\;& 4\frac12=\frac{4\cdot 2+1}{2}=\frac{9}{2}.\\
\Step{2}\;& \frac{9}{2}=9\div 2=4.5.
\end{aligned}
\]
\NumLinePoint{3}{6}{4.5}{Point exactly halfway between $4$ and $5$.}
\end{QAPair}

\begin{QAPair}{Question 1 (iv)}
\textcolor{gold}{\bfseries Question:} Represent $-\sqrt{8}$ on the number line.\\
\tcblower
\textcolor{green}{\bfseries Answer:}
\[
\begin{aligned}
\Step{1}\;& \sqrt{8}\approx 2.828.\\
\Step{2}\;& -\sqrt{8}\approx -2.828.
\end{aligned}
\]
\NumLinePoint{-4}{1}{-2.828}{Point at approximately $-2.828$ (i.e., $-\sqrt{8}$).}
\end{QAPair}

\begin{QAPair}{Question 1 (v)}
\textcolor{gold}{\bfseries Question:} Represent $\sqrt{8}$ on the number line.\\
\tcblower
\textcolor{green}{\bfseries Answer:}
\[
\begin{aligned}
\Step{1}\;& \sqrt{8}\approx 2.828.\\
\Step{2}\;& 2.828 \text{ lies between } 2 \text{ and } 3.
\end{aligned}
\]
\NumLinePoint{-1}{4}{2.828}{Point at approximately $2.828$ (i.e., $\sqrt{8}$).}
\end{QAPair}

\begin{QAPair}{Question 1 (vi)}
\textcolor{gold}{\bfseries Question:} Represent $-4\dfrac{1}{2}$ on the number line.\\
\tcblower
\textcolor{green}{\bfseries Answer:}
\[
\begin{aligned}
\Step{1}\;& -4\frac12=-\frac{9}{2}.\\
\Step{2}\;& -\frac{9}{2}=-4.5.
\end{aligned}
\]
\NumLinePoint{-6}{-3}{-4.5}{Point exactly halfway between $-5$ and $-4$.}
\end{QAPair}

\begin{QAPair}{Question 1 (vii)}
\textcolor{gold}{\bfseries Question:} Represent $\dfrac{1}{3}$ on the number line.\\
\tcblower
\textcolor{green}{\bfseries Answer:}
\[
\begin{aligned}
\Step{1}\;& \frac{1}{3}=1\div 3\approx 0.333.\\
\Step{2}\;& 0.333 \text{ lies between } 0 \text{ and } 1.
\end{aligned}
\]
\NumLinePoint{-1}{2}{0.333}{Point at approximately $0.333$ (i.e., $\frac13$).}
\end{QAPair}

\begin{QAPair}{Question 1 (viii)}
\textcolor{gold}{\bfseries Question:} Represent $-\dfrac{7}{8}$ on the number line.\\
\tcblower
\textcolor{green}{\bfseries Answer:}
\[
\begin{aligned}
\Step{1}\;& -\frac{7}{8}=-(7\div 8)=-0.875.\\
\Step{2}\;& -0.875 \text{ lies between } -1 \text{ and } 0 \text{ (closer to }-1\text{).}
\end{aligned}
\]
\NumLinePoint{-2}{1}{-0.875}{Point at $-0.875$ (i.e., $-\frac78$).}
\end{QAPair}

% ============================================================
% Q2
\begin{QAPair}{Question 2 (i)}
\textcolor{gold}{\bfseries Question:} Identify the property: $1\times (y-2)=y-2$.\\
\tcblower
\textcolor{green}{\bfseries Answer:}
\[
\Step{1}\; 1\cdot a=a \quad \Rightarrow\quad \textbf{Multiplicative Identity Property.}
\]
\end{QAPair}

\begin{QAPair}{Question 2 (ii)}
\textcolor{gold}{\bfseries Question:} Identify the property: $(0.2)\,5=1$.\\
\tcblower
\textcolor{green}{\bfseries Answer:}
\[
\begin{aligned}
\Step{1}\;& 0.2=\frac{1}{5}.\\
\Step{2}\;& \frac{1}{5}\cdot 5=1 \quad \Rightarrow\quad \textbf{Multiplicative Inverse Property.}
\end{aligned}
\]
\end{QAPair}

\begin{QAPair}{Question 2 (iii)}
\textcolor{gold}{\bfseries Question:} Identify the property: $(x+2)+y=y+(x+2)$.\\
\tcblower
\textcolor{green}{\bfseries Answer:}
\[
\Step{1}\; a+b=b+a \quad \Rightarrow\quad \textbf{Commutative Property of Addition.}
\]
\end{QAPair}

\begin{QAPair}{Question 2 (iv)}
\textcolor{gold}{\bfseries Question:} Identify the property: $-(3b)+(3b)=0$.\\
\tcblower
\textcolor{green}{\bfseries Answer:}
\[
\Step{1}\; (-a)+a=0 \quad \Rightarrow\quad \textbf{Additive Inverse Property.}
\]
\end{QAPair}

\begin{QAPair}{Question 2 (v)}
\textcolor{gold}{\bfseries Question:} Identify the property: $(x+5)-1=x+(5-1)$.\\
\tcblower
\textcolor{green}{\bfseries Answer:}
\[
\begin{aligned}
\Step{1}\;& (x+5)-1=(x+5)+(-1).\\
\Step{2}\;&=(x)+(5+(-1))=x+(5-1) \Rightarrow \textbf{Associative Property of Addition.}
\end{aligned}
\]
\end{QAPair}

\begin{QAPair}{Question 2 (vi)}
\textcolor{gold}{\bfseries Question:} Identify the property: $-3(2-y)=-6+3y$.\\
\tcblower
\textcolor{green}{\bfseries Answer:}
\[
\begin{aligned}
\Step{1}\;& -3(2-y)=(-3)\cdot 2+(-3)\cdot(-y).\\
\Step{2}\;&=-6+3y \Rightarrow \textbf{Distributive Property.}
\end{aligned}
\]
\end{QAPair}

% ============================================================
% Q3
\begin{QAPair}{Question 3 (i)}
\textcolor{gold}{\bfseries Question:} Represent $x<0$ on a number line.\\
\tcblower
\textcolor{green}{\bfseries Answer:}
\[
\Step{1}\; x<0 \Rightarrow (-\infty,0)\ \text{(open at 0, shade left).}
\]
\NumLineRay{-5}{5}{0}{L}{1}{Open circle at $0$, ray to the left.}
\end{QAPair}

\begin{QAPair}{Question 3 (ii)}
\textcolor{gold}{\bfseries Question:} Represent $-3<x<3$ on a number line.\\
\tcblower
\textcolor{green}{\bfseries Answer:}
\[
\Step{1}\; -3<x<3 \Rightarrow (-3,3)\ \text{(open at both ends, shade between).}
\]
\NumLineInterval{-5}{5}{-3}{3}{1}{1}{Open at $-3$ and $3$, shaded between.}
\end{QAPair}

\begin{QAPair}{Question 3 (iii)}
\textcolor{gold}{\bfseries Question:} Represent $x\ge -8$ on a number line.\\
\tcblower
\textcolor{green}{\bfseries Answer:}
\[
\Step{1}\; x\ge -8 \Rightarrow [-8,\infty)\ \text{(closed at $-8$, shade right).}
\]
\NumLineRay{-10}{2}{-8}{R}{0}{Closed circle at $-8$, ray to the right.}
\end{QAPair}

\begin{QAPair}{Question 3 (iv)}
\textcolor{gold}{\bfseries Question:} Represent $x>0$ on a number line.\\
\tcblower
\textcolor{green}{\bfseries Answer:}
\[
\Step{1}\; x>0 \Rightarrow (0,\infty)\ \text{(open at 0, shade right).}
\]
\NumLineRay{-5}{5}{0}{R}{1}{Open circle at $0$, ray to the right.}
\end{QAPair}

\begin{QAPair}{Question 3 (v)}
\textcolor{gold}{\bfseries Question:} Represent $x<-3$ on a number line.\\
\tcblower
\textcolor{green}{\bfseries Answer:}
\[
\Step{1}\; x<-3 \Rightarrow (-\infty,-3)\ \text{(open at $-3$, shade left).}
\]
\NumLineRay{-8}{2}{-3}{L}{1}{Open circle at $-3$, ray to the left.}
\end{QAPair}

\begin{QAPair}{Question 3 (vi)}
\textcolor{gold}{\bfseries Question:} Represent $-4<x\le 4$ on a number line.\\
\tcblower
\textcolor{green}{\bfseries Answer:}
\[
\Step{1}\; -4<x\le 4 \Rightarrow (-4,4]\ \text{(open at $-4$, closed at $4$).}
\]
\NumLineInterval{-6}{6}{-4}{4}{1}{0}{Open at $-4$, closed at $4$, shaded between.}
\end{QAPair}

% ============================================================
% Q4
\begin{QAPair}{Question 4 (i)}
\textcolor{gold}{\bfseries Question:} Identify the property: $9x=9x$.\\
\tcblower
\textcolor{green}{\bfseries Answer:}
\[
\Step{1}\; a=a \Rightarrow \textbf{Reflexive Property of Equality.}
\]
\end{QAPair}

\begin{QAPair}{Question 4 (ii)}
\textcolor{gold}{\bfseries Question:} If $x+2=y$ and $y=2x-3$, then $x+2=2x-3$.\\
\tcblower
\textcolor{green}{\bfseries Answer:}
\[
\begin{aligned}
\Step{1}\;& x+2=y,\ \ y=2x-3.\\
\Step{2}\;& \text{Replace } y \text{ with } (2x-3):\ x+2=2x-3.\\
&\Rightarrow\ \textbf{Transitive Property of Equality (Substitution).}
\end{aligned}
\]
\end{QAPair}

\begin{QAPair}{Question 4 (iii)}
\textcolor{gold}{\bfseries Question:} If $2x+3=y$, then $y=2x+3$.\\
\tcblower
\textcolor{green}{\bfseries Answer:}
\[
\Step{1}\; a=b \Rightarrow b=a \Rightarrow \textbf{Symmetric Property of Equality.}
\]
\end{QAPair}

\begin{QAPair}{Question 4 (iv)}
\textcolor{gold}{\bfseries Question:} If $3<4$, then $-3>-4$.\\
\tcblower
\textcolor{green}{\bfseries Answer:}
\[
\begin{aligned}
\Step{1}\;& 3<4.\\
\Step{2}\;& \text{Multiply both sides by } -1 \text{ (sign flips): } -3>-4.\\
&\Rightarrow\ \textbf{Multiplication Property of Inequality (negative factor reverses sign).}
\end{aligned}
\]
\end{QAPair}

\begin{QAPair}{Question 4 (v)}
\textcolor{gold}{\bfseries Question:} If $2y+2w=p$ and $p=50$, then $2y+2w=50$.\\
\tcblower
\textcolor{green}{\bfseries Answer:}
\[
\begin{aligned}
\Step{1}\;& 2y+2w=p,\ \ p=50.\\
\Step{2}\;& \text{Replace } p \text{ with } 50:\ 2y+2w=50.\\
&\Rightarrow\ \textbf{Transitive Property of Equality (Substitution).}
\end{aligned}
\]
\end{QAPair}

\begin{QAPair}{Question 4 (vi)}
\textcolor{gold}{\bfseries Question:} If $x+4>y+4$, then $x>y$.\\
\tcblower
\textcolor{green}{\bfseries Answer:}
\[
\begin{aligned}
\Step{1}\;& x+4>y+4.\\
\Step{2}\;& \text{Subtract } 4 \text{ from both sides: } x>y.\\
&\Rightarrow\ \textbf{Subtraction Property of Inequality.}
\end{aligned}
\]
\end{QAPair}

\begin{QAPair}{Question 4 (vii)}
\textcolor{gold}{\bfseries Question:} If $2<5$ and $5<9$, then $2<9$.\\
\tcblower
\textcolor{green}{\bfseries Answer:}
\[
\Step{1}\; 2<5,\ 5<9 \Rightarrow 2<9 \Rightarrow \textbf{Transitive Property of Inequality.}
\]
\end{QAPair}

\begin{QAPair}{Question 4 (viii)}
\textcolor{gold}{\bfseries Question:} If $-18<-16$, then $9>8$.\\
\tcblower
\textcolor{green}{\bfseries Answer:}
\[
\begin{aligned}
\Step{1}\;& -18<-16.\\
\Step{2}\;& \times(-1)\ \Rightarrow\ 18>16 \quad (\text{sign flips}).\\
\Step{3}\;& \div 2\ \Rightarrow\ 9>8. 
\end{aligned}
\]
\end{QAPair}

\end{document}
