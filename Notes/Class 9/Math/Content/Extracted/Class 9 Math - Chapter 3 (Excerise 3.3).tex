% !TEX TS-program = pdflatex
\documentclass[11pt]{article}

% -------------------- Packages --------------------
\usepackage[a4paper,margin=1in]{geometry}
\usepackage{amsmath,amssymb}
\usepackage[T1]{fontenc}
\usepackage{lmodern}
\usepackage{xcolor}
\usepackage{tcolorbox}
\tcbuselibrary{skins,breakable}
\usepackage{enumitem}
\usepackage{hyperref}

\pagestyle{empty}

% -------------------- Dark Theme Colors --------------------
\definecolor{bg}{HTML}{000000}
\definecolor{pairbg}{HTML}{121212}
\definecolor{solbg}{HTML}{0A0A0A}
\definecolor{border}{HTML}{2A2A2A}
\definecolor{text}{HTML}{FFFFFF}
\definecolor{muted}{HTML}{C9CDD3}
\definecolor{gold}{HTML}{FFD700}
\definecolor{green}{HTML}{4ADE80}
\definecolor{cyan}{HTML}{38BDF8}

\pagecolor{bg}
\color{text}

\hypersetup{
  colorlinks=true,
  linkcolor=cyan,
  urlcolor=cyan
}

\setlength{\parindent}{0pt}
\setlength{\parskip}{10pt}

\setlist[itemize]{left=1.4em,itemsep=6pt,topsep=6pt}
\setlist[enumerate]{left=1.6em,itemsep=4pt,topsep=4pt}

% -------------------- tcolorbox Base --------------------
\tcbset{
  enhanced,
  breakable,
  arc=12pt,
  boxrule=0.8pt,
  left=16pt,right=16pt,top=12pt,bottom=12pt
}

\newtcolorbox{QAPair}[1]{%
  colback=pairbg,
  colbacklower=solbg,
  colframe=border,
  coltext=text,
  title=\textcolor{gold}{\bfseries #1},
  fonttitle=\bfseries,
  coltitle=text,
  segmentation style={draw=border, dashed, line width=0.6pt},
}

% Visible text inside this box (fix)
\newtcolorbox{QuickBox}{%
  colback=pairbg,
  colframe=cyan,
  coltext=text,
  fontupper=\color{text},
  borderline north={4pt}{0pt}{cyan},
  arc=14pt,
  boxrule=0.8pt
}

% Helper for step headings
\newcommand{\Step}[1]{\textcolor{muted}{\textbf{Step #1:}}}

% --- Helper to allow line breaks inside long sets ---
\newcommand{\SetBreak}{,\allowbreak\ }

% ============================================================
\begin{document}

\begin{center}
{\LARGE\bfseries \textcolor{gold}{Exercise 3.3 --- Solutions}}\\[-2pt]
\end{center}

\begin{QuickBox}
{\color{cyan}\bfseries Quick formulas (useful)}\par\medskip
\begin{itemize}
\item \textbf{Equality of ordered pairs:} $(a,b)=(c,d)\iff a=c \text{ and } b=d$.
\item \textbf{Cartesian product:} $A\times B=\{(a,b)\mid a\in A,\ b\in B\}$.
\item \textbf{Number of ordered pairs:} $|A\times B|=|A|\cdot|B|$.
\item \textbf{Distributive laws:} $A\times(B\cup C)=(A\times B)\cup(A\times C)$.
\item \textbf{With intersection:} $D\times(E\cap F)=(D\times E)\cap(D\times F)$.
\item \textbf{With difference:} $A\times(B-C)=(A\times B)-(A\times C)$,\quad
$(A-B)\times C=(A\times C)-(B\times C)$.
\end{itemize}
\end{QuickBox}

% ============================================================
% Q1
\begin{QAPair}{Question 1 (i)}
\textcolor{gold}{\bfseries Question:} $(a,-b)=(7,1)$\\
\tcblower
\textcolor{green}{\bfseries Answer:}
\[
\begin{aligned}
\Step{1}\;& (a,-b)=(7,1)\ \Rightarrow\ a=7,\ -b=1.\\
\Step{2}\;& -b=1 \Rightarrow b=-1.\\
&\boxed{a=7,\ b=-1}
\end{aligned}
\]
\end{QAPair}

\begin{QAPair}{Question 1 (ii)}
\textcolor{gold}{\bfseries Question:} $(2a,\,2b+3)=(-10,\,-b)$\\
\tcblower
\textcolor{green}{\bfseries Answer:}
\[
\begin{aligned}
\Step{1}\;& 2a=-10 \Rightarrow a=-5.\\
\Step{2}\;& 2b+3=-b \Rightarrow 3b=-3 \Rightarrow b=-1.\\
&\boxed{a=-5,\ b=-1}
\end{aligned}
\]
\end{QAPair}

\begin{QAPair}{Question 1 (iii)}
\textcolor{gold}{\bfseries Question:} $(2a-4,\,6)=(8,\,-b+1)$\\
\tcblower
\textcolor{green}{\bfseries Answer:}
\[
\begin{aligned}
\Step{1}\;& 2a-4=8 \Rightarrow 2a=12 \Rightarrow a=6.\\
\Step{2}\;& 6=-b+1 \Rightarrow -b=5 \Rightarrow b=-5.\\
&\boxed{a=6,\ b=-5}
\end{aligned}
\]
\end{QAPair}

\begin{QAPair}{Question 1 (iv)}
\textcolor{gold}{\bfseries Question:} $(x+2y,\,y-3)=(2,5)$\\
\tcblower
\textcolor{green}{\bfseries Answer:}
\[
\begin{aligned}
\Step{1}\;& y-3=5 \Rightarrow y=8.\\
\Step{2}\;& x+2y=2 \Rightarrow x+16=2 \Rightarrow x=-14.\\
&\boxed{x=-14,\ y=8}
\end{aligned}
\]
\end{QAPair}

\begin{QAPair}{Question 1 (v)}
\textcolor{gold}{\bfseries Question:} $(2x-y,\,y-3x)=(4,2)$\\
\tcblower
\textcolor{green}{\bfseries Answer:}
\[
\begin{aligned}
\Step{1}\;& 2x-y=4 \Rightarrow y=2x-4.\\
\Step{2}\;& y-3x=2 \Rightarrow (2x-4)-3x=2 \Rightarrow -x-4=2 \Rightarrow x=-6.\\
\Step{3}\;& y=2(-6)-4=-16.\\
&\boxed{x=-6,\ y=-16}
\end{aligned}
\]
\end{QAPair}

\begin{QAPair}{Question 1 (vi)}
\textcolor{gold}{\bfseries Question:} $(4x+6y,\,x-12y)=(6,-3)$\\
\tcblower
\textcolor{green}{\bfseries Answer:}
\[
\begin{aligned}
\Step{1}\;& x-12y=-3 \Rightarrow x=-3+12y.\\
\Step{2}\;& 4x+6y=6 \Rightarrow 4(-3+12y)+6y=6\\
&\Rightarrow -12+48y+6y=6 \Rightarrow 54y=18 \Rightarrow y=\frac{1}{3}.\\
\Step{3}\;& x=-3+12\left(\frac{1}{3}\right)=-3+4=1.\\
&\boxed{x=1,\ y=\frac{1}{3}}
\end{aligned}
\]
\end{QAPair}

\begin{QAPair}{Question 1 (vii)}
\textcolor{gold}{\bfseries Question:} $(5x+y,\,-x+y)=(6,1)$\\
\tcblower
\textcolor{green}{\bfseries Answer:}
\[
\begin{aligned}
\Step{1}\;& -x+y=1 \Rightarrow y=1+x.\\
\Step{2}\;& 5x+y=6 \Rightarrow 5x+(1+x)=6 \Rightarrow 6x=5 \Rightarrow x=\frac{5}{6}.\\
\Step{3}\;& y=1+\frac{5}{6}=\frac{11}{6}.\\
&\boxed{x=\frac{5}{6},\ y=\frac{11}{6}}
\end{aligned}
\]
\end{QAPair}

% ============================================================
% Q2 (WRAPPED)
\begin{QAPair}{Question 2}
\textcolor{gold}{\bfseries Question:} Let $A=\{1,4,8\}$ and $B=\{1,0\}$. Find:
\[
(i)\ A\times B,\quad (ii)\ B\times A,\quad (iii)\ A\times A,\quad (iv)\ B\times B
\]
Also find how many elements are there in each.\\
\tcblower
\textcolor{green}{\bfseries Answer:}
{\small
\[
\begin{aligned}
\Step{1}\;& A\times B=\{(1,1)\SetBreak(1,0)\SetBreak(4,1)\SetBreak(4,0)\SetBreak(8,1)\SetBreak(8,0)\},\quad |A\times B|=3\cdot2=6.\\[2pt]
\Step{2}\;& B\times A=\{(1,1)\SetBreak(1,4)\SetBreak(1,8)\SetBreak(0,1)\SetBreak(0,4)\SetBreak(0,8)\},\quad |B\times A|=2\cdot3=6.\\[2pt]
\Step{3}\;& A\times A=\{(1,1)\SetBreak(1,4)\SetBreak(1,8)\SetBreak(4,1)\SetBreak(4,4)\SetBreak(4,8)\SetBreak(8,1)\SetBreak(8,4)\SetBreak(8,8)\},\\
&\qquad |A\times A|=3\cdot3=9.\\[2pt]
\Step{4}\;& B\times B=\{(1,1)\SetBreak(1,0)\SetBreak(0,1)\SetBreak(0,0)\},\quad |B\times B|=2\cdot2=4.
\end{aligned}
\]
}
\end{QAPair}

% ============================================================
% Q3
\begin{QAPair}{Question 3}
\textcolor{gold}{\bfseries Question:} Let $E=\{1,3\}$ and $F=\{4,6,8\}$. Express $E\times F$, $F\times E$, $E\times E$, $F\times F$ graphically.\\
\tcblower
\textcolor{green}{\bfseries Answer:} (Graphically means: plot each ordered pair $(x,y)$ as a point on the Cartesian plane.)
\[
\begin{aligned}
\Step{1}\;& E\times F=\{(1,4),(1,6),(1,8),(3,4),(3,6),(3,8)\}.\\
\Step{2}\;& F\times E=\{(4,1),(4,3),(6,1),(6,3),(8,1),(8,3)\}.\\
\Step{3}\;& E\times E=\{(1,1),(1,3),(3,1),(3,3)\}.\\
\Step{4}\;& F\times F=\{(4,4),(4,6),(4,8),(6,4),(6,6),(6,8),(8,4),(8,6),(8,8)\}.
\end{aligned}
\]
\end{QAPair}

% ============================================================
% Q4
\begin{QAPair}{Question 4}
\textcolor{gold}{\bfseries Question:} If
\[
L\times M=\{(0,2),(0,3),(0,4),(1,2),(1,3),(1,4)\},
\]
then find sets $L$, $M$ and $M\times L$.\\
\tcblower
\textcolor{green}{\bfseries Answer:}
\[
\begin{aligned}
\Step{1}\;& \text{First coordinates are } 0,1 \Rightarrow L=\{0,1\}.\\
\Step{2}\;& \text{Second coordinates are } 2,3,4 \Rightarrow M=\{2,3,4\}.\\
\Step{3}\;& M\times L=\{(2,0),(2,1),(3,0),(3,1),(4,0),(4,1)\}.
\end{aligned}
\]
\end{QAPair}

% ============================================================
% Q5 (WRAPPED)
\begin{QAPair}{Question 5 (i)--(iii)}
\textcolor{gold}{\bfseries Question:} Given $A=\{1,3,5\}$, $B=\{2,4\}$, $C=\{6,7\}$:
\begin{enumerate}[label=(\roman*)]
\item Find $A\times(B\cup C)$
\item Find $(A\times B)\cup(A\times C)$
\item Verify $A\times(B\cup C)=(A\times B)\cup(A\times C)$
\end{enumerate}
\tcblower
\textcolor{green}{\bfseries Answer:}
{\small
\[
\begin{aligned}
\Step{1}\;& B\cup C=\{2\SetBreak4\SetBreak6\SetBreak7\}.\\
\Step{2}\;& A\times(B\cup C)=\{(1,2)\SetBreak(1,4)\SetBreak(1,6)\SetBreak(1,7)\SetBreak(3,2)\SetBreak(3,4)\\
&\qquad\SetBreak(3,6)\SetBreak(3,7)\SetBreak(5,2)\SetBreak(5,4)\SetBreak(5,6)\SetBreak(5,7)\}.\\
\Step{3}\;& A\times B=\{(1,2)\SetBreak(1,4)\SetBreak(3,2)\SetBreak(3,4)\SetBreak(5,2)\SetBreak(5,4)\}.\\
\Step{4}\;& A\times C=\{(1,6)\SetBreak(1,7)\SetBreak(3,6)\SetBreak(3,7)\SetBreak(5,6)\SetBreak(5,7)\}.\\
\Step{5}\;& (A\times B)\cup(A\times C)=\{(1,2)\SetBreak(1,4)\SetBreak(1,6)\SetBreak(1,7)\SetBreak(3,2)\SetBreak(3,4)\\
&\qquad\SetBreak(3,6)\SetBreak(3,7)\SetBreak(5,2)\SetBreak(5,4)\SetBreak(5,6)\SetBreak(5,7)\}.\\
\Step{6}\;& \Rightarrow\ A\times(B\cup C)=(A\times B)\cup(A\times C).
\end{aligned}
\]
}
\end{QAPair}

% ============================================================
% Q6
\begin{QAPair}{Question 6 (i)--(iii)}
\textcolor{gold}{\bfseries Question:} Given $D=\{a,e,i\}$, $E=\{a,c\}$, $F=\{b,c\}$:
\begin{enumerate}[label=(\roman*)]
\item Find $D\times(E\cap F)$
\item Find $(D\times E)\cap(D\times F)$
\item Verify $D\times(E\cap F)=(D\times E)\cap(D\times F)$
\end{enumerate}
\tcblower
\textcolor{green}{\bfseries Answer:}
\[
\begin{aligned}
\Step{1}\;& E\cap F=\{c\}.\\
\Step{2}\;& D\times(E\cap F)=D\times\{c\}=\{(a,c),(e,c),(i,c)\}.\\
\Step{3}\;& D\times E=\{(a,a),(a,c),(e,a),(e,c),(i,a),(i,c)\}.\\
\Step{4}\;& D\times F=\{(a,b),(a,c),(e,b),(e,c),(i,b),(i,c)\}.\\
\Step{5}\;& (D\times E)\cap(D\times F)=\{(a,c),(e,c),(i,c)\}.\\
\Step{6}\;& \Rightarrow\ D\times(E\cap F)=(D\times E)\cap(D\times F).
\end{aligned}
\]
\end{QAPair}

% ============================================================
% Q7
\begin{QAPair}{Question 7 (i)--(ii)}
\textcolor{gold}{\bfseries Question:} Given
\[
A=\{x\mid x\in \mathbb{N},\,x<3\},\quad
B=\{y\mid y\in W,\,y<2\},\quad
C=\{0,2,4\},
\]
\begin{enumerate}[label=(\roman*)]
\item Verify $A\times(B-C)=(A\times B)-(A\times C)$
\item Verify $(A-B)\times C=(A\times C)-(B\times C)$
\end{enumerate}
\tcblower
\textcolor{green}{\bfseries Answer:} Take $\mathbb{N}=\{1,2,3,\dots\}$ and $W=\{0,1,2,\dots\}$. Then
\[
A=\{1,2\},\quad B=\{0,1\},\quad C=\{0,2,4\}.
\]
\[
\begin{aligned}
\Step{1}\;& B-C=\{0,1\}-\{0,2,4\}=\{1\}.\\
\Step{2}\;& A\times(B-C)=A\times\{1\}=\{(1,1),(2,1)\}.\\
\Step{3}\;& A\times B=\{(1,0),(1,1),(2,0),(2,1)\}.\\
\Step{4}\;& A\times C=\{(1,0),(1,2),(1,4),(2,0),(2,2),(2,4)\}.\\
\Step{5}\;& (A\times B)-(A\times C)=\{(1,1),(2,1)\}=A\times(B-C).\\[6pt]
\Step{6}\;& A-B=\{1,2\}-\{0,1\}=\{2\}.\\
\Step{7}\;& (A-B)\times C=\{2\}\times\{0,2,4\}=\{(2,0),(2,2),(2,4)\}.\\
\Step{8}\;& B\times C=\{(0,0),(0,2),(0,4),(1,0),(1,2),(1,4)\}.\\
\Step{9}\;& (A\times C)-(B\times C)=\{(2,0),(2,2),(2,4)\}=(A-B)\times C.
\end{aligned}
\]
\end{QAPair}

% ============================================================
% Q8 (WRAPPED)
\begin{QAPair}{Question 8}
\textcolor{gold}{\bfseries Question:} Let $X=\{x\mid x\in W,\,x\le 2\}$ and $Y=\{-1,-2,-3\}$. Exhibit $X\times Y$ and $Y\times X$ by arrow diagram.\\
\tcblower
\textcolor{green}{\bfseries Answer:} Since $W=\{0,1,2,\dots\}$, we have $X=\{0,1,2\}$.
{\small
\[
\begin{aligned}
\Step{1}\;& X\times Y=\{(0,-1)\SetBreak(0,-2)\SetBreak(0,-3)\SetBreak(1,-1)\SetBreak(1,-2)\SetBreak(1,-3)\\
&\qquad\SetBreak(2,-1)\SetBreak(2,-2)\SetBreak(2,-3)\}.\\[2pt]
\Step{2}\;& Y\times X=\{(-1,0)\SetBreak(-1,1)\SetBreak(-1,2)\SetBreak(-2,0)\SetBreak(-2,1)\SetBreak(-2,2)\\
&\qquad\SetBreak(-3,0)\SetBreak(-3,1)\SetBreak(-3,2)\}.
\end{aligned}
\]
}
\textcolor{muted}{\textbf{Arrow diagram description:}}
\begin{itemize}
\item For $X\times Y$: from each of $0,1,2$ draw arrows to each of $-1,-2,-3$.
\item For $Y\times X$: from each of $-1,-2,-3$ draw arrows to each of $0,1,2$.
\end{itemize}
\end{QAPair}

\end{document}
