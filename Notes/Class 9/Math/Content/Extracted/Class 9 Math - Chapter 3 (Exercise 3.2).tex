% !TEX TS-program = pdflatex
\documentclass[11pt]{article}

% -------------------- Packages --------------------
\usepackage[a4paper,margin=1in]{geometry}
\usepackage{amsmath,amssymb}
\usepackage[T1]{fontenc}
\usepackage{lmodern}
\usepackage{xcolor}
\usepackage{tcolorbox}
\tcbuselibrary{skins,breakable}
\usepackage{enumitem}
\usepackage{hyperref}

\pagestyle{empty}

% -------------------- Dark Theme Colors --------------------
\definecolor{bg}{HTML}{000000}
\definecolor{pairbg}{HTML}{121212}
\definecolor{solbg}{HTML}{0A0A0A}
\definecolor{border}{HTML}{2A2A2A}
\definecolor{text}{HTML}{FFFFFF}
\definecolor{muted}{HTML}{C9CDD3}
\definecolor{gold}{HTML}{FFD700}
\definecolor{green}{HTML}{4ADE80}
\definecolor{cyan}{HTML}{38BDF8}

\pagecolor{bg}
\color{text}

\hypersetup{
  colorlinks=true,
  linkcolor=cyan,
  urlcolor=cyan
}

\setlength{\parindent}{0pt}
\setlength{\parskip}{10pt}

\setlist[itemize]{left=1.4em,itemsep=6pt,topsep=6pt}
\setlist[enumerate]{left=1.6em,itemsep=4pt,topsep=4pt}

% -------------------- tcolorbox Base --------------------
\tcbset{
  enhanced,
  breakable,
  arc=12pt,
  boxrule=0.8pt,
  left=16pt,right=16pt,top=12pt,bottom=12pt
}

\newtcolorbox{QAPair}[1]{%
  colback=pairbg,
  colbacklower=solbg,
  colframe=border,
  coltext=text,
  title=\textcolor{gold}{\bfseries #1},
  fonttitle=\bfseries,
  coltitle=text,
  segmentation style={draw=border, dashed, line width=0.6pt},
}

% Visible text inside this box (fix)
\newtcolorbox{QuickBox}{%
  colback=pairbg,
  colframe=cyan,
  coltext=text,
  fontupper=\color{text},
  borderline north={4pt}{0pt}{cyan},
  arc=14pt,
  boxrule=0.8pt
}

% Helper for step headings
\newcommand{\Step}[1]{\textcolor{muted}{\textbf{Step #1:}}}

% ============================================================
\begin{document}

\begin{center}
{\LARGE\bfseries \textcolor{gold}{Exercise 3.2 --- Solutions}}\\[-2pt]
\end{center}

\begin{QuickBox}
{\color{cyan}\bfseries Quick formulas (Sets \& Venn)}\par\medskip
\begin{itemize}
\item \textbf{Two sets:} $n(A\cup B)=n(A)+n(B)-n(A\cap B)$.
\item \textbf{Difference:} $n(A-B)=n(A)-n(A\cap B)$ and $n(B-A)=n(B)-n(A\cap B)$.
\item \textbf{Neither (two sets):} $n(\text{neither})=n(U)-n(A\cup B)$.
\item \textbf{Three sets:} $n(A\cup B\cup C)=\sum n(\text{single})-\sum n(\text{pair})+n(A\cap B\cap C)$.
\item \textbf{Only one (three sets):} e.g.\ $n(A\ \text{only})=n(A)-n(A\cap B)-n(A\cap C)+n(A\cap B\cap C)$.
\end{itemize}
\end{QuickBox}

% ============================================================
% Q1
\begin{QAPair}{Question 1}
\textcolor{gold}{\bfseries Question:} Let $A$ and $B$ be two finite sets such that $n(A)=24$, $n(B)=18$ and $n(A\cup B)=31$. Find $n(A\cap B)$.\\
\tcblower
\textcolor{green}{\bfseries Answer:}
\[
\begin{aligned}
\Step{1}\;& n(A\cup B)=n(A)+n(B)-n(A\cap B).\\
\Step{2}\;& 31=24+18-n(A\cap B).\\
\Step{3}\;& n(A\cap B)=24+18-31=11.
\end{aligned}
\]
\end{QAPair}

% ============================================================
% Q2
\begin{QAPair}{Question 2}
\textcolor{gold}{\bfseries Question:} If $n(A-B)=23$, $n(A\cup B)=44$ and $n(A\cap B)=2$, then find $n(B-A)$. Also find $n(B)$.\\
\tcblower
\textcolor{green}{\bfseries Answer:}
\[
\begin{aligned}
\Step{1}\;& n(A)=n(A-B)+n(A\cap B)=23+2=25.\\
\Step{2}\;& n(A\cup B)=n(A)+n(B)-n(A\cap B)\\
&\Rightarrow 44=25+n(B)-2=23+n(B).\\
\Step{3}\;& n(B)=44-23=21.\\
\Step{4}\;& n(B-A)=n(B)-n(A\cap B)=21-2=19.
\end{aligned}
\]
\end{QAPair}

% ============================================================
% Q3
\begin{QAPair}{Question 3}
\textcolor{gold}{\bfseries Question:} In a group of $30$ Mathematics students, $20$ like Algebra and $15$ like both Geometry and Algebra. Show the data by Venn diagram. Also find how many students like Geometry.\\
\tcblower
\textcolor{green}{\bfseries Answer:} Let $A=$ Algebra, $G=$ Geometry. (Assume every student likes at least one of the two, so $n(A\cup G)=30$.)
\[
\begin{aligned}
\Step{1}\;& n(A\cap G)=15,\quad n(A)=20.\\
\Step{2}\;& n(A\ \text{only})=n(A)-n(A\cap G)=20-15=5.\\
\Step{3}\;& n(A\cup G)=n(A)+n(G)-n(A\cap G)\\
&\Rightarrow 30=20+n(G)-15.\\
\Step{4}\;& n(G)=30-5=25.\\[2pt]
&\text{Venn regions: } \text{Algebra only }=5,\; \text{Both }=15,\; \text{Geometry only }=25-15=10.
\end{aligned}
\]
\end{QAPair}

% ============================================================
% Q4
\begin{QAPair}{Question 4}
\textcolor{gold}{\bfseries Question:} In a street with $50$ houses, $25$ houses have lawns, $32$ houses have car porch and $15$ houses have both lawn and car porch. Show the data by Venn diagram. Also find how many houses have neither lawn nor porch.\\
\tcblower
\textcolor{green}{\bfseries Answer:} Let $L=$ lawns, $P=$ car porch.
\[
\begin{aligned}
\Step{1}\;& n(L\cup P)=n(L)+n(P)-n(L\cap P)=25+32-15=42.\\
\Step{2}\;& n(\text{neither})=50-n(L\cup P)=50-42=8.\\[2pt]
&\text{Venn regions: } L\text{ only}=25-15=10,\; P\text{ only}=32-15=17,\; \text{Both}=15,\; \text{Neither}=8.
\end{aligned}
\]
\end{QAPair}

% ============================================================
% Q5
\begin{QAPair}{Question 5}
\textcolor{gold}{\bfseries Question:} In a survey of $940$ children, $400$ students were found studying at primary level, $240$ at elementary and $175$ at secondary level. Create a Venn diagram to illustrate this information. How many children were found out of school?\\
\tcblower
\textcolor{green}{\bfseries Answer:} These three levels are treated as \emph{separate} (no overlap).
\[
\begin{aligned}
\Step{1}\;& n(\text{in school})=400+240+175=815.\\
\Step{2}\;& n(\text{out of school})=940-815=125.
\end{aligned}
\]
\end{QAPair}

% ============================================================
% Q6
\begin{QAPair}{Question 6}
\textcolor{gold}{\bfseries Question:} ABC Dairy polls its customers on their favorite flavor: chocolate, vanilla or mango. $100$ customers said they like mango flavor, $90$ said they like vanilla, $40$ polled for chocolate, $20$ liked both mango and vanilla while $14$ liked both chocolate and vanilla. How many customers said they like: (i) only mango? (ii) only vanilla? (iii) only chocolate?\\
\tcblower
\textcolor{green}{\bfseries Answer:} Let $M=$ mango, $V=$ vanilla, $C=$ chocolate. (Assume no other overlap is given, so $n(M\cap C)=0$ and no triple overlap.)
\[
\begin{aligned}
\Step{1}\;& n(M\ \text{only})=n(M)-n(M\cap V)=100-20=80.\\
\Step{2}\;& n(V\ \text{only})=n(V)-n(M\cap V)-n(C\cap V)=90-20-14=56.\\
\Step{3}\;& n(C\ \text{only})=n(C)-n(C\cap V)=40-14=26.
\end{aligned}
\]
\end{QAPair}

% ============================================================
% Q7
\begin{QAPair}{Question 7 (a)--(c)}
\textcolor{gold}{\bfseries Question:} In a survey of university $200$ students were interviewed. It was found that: $42$ have laptops, $80$ have cell phones, $100$ have iPods, $23$ have both a laptop and a cell phone, $10$ have both a laptop and iPod, $14$ have both a cell phone and iPod and $8$ have all three items.
\begin{enumerate}[label=(\alph*)]
\item How many students have only cell phone?
\item How many students have none of the three items?
\item How many students have both iPod and laptop but not cellphone?
\end{enumerate}
\tcblower
\textcolor{green}{\bfseries Answer:} Let $L=$ laptop, $C=$ cell phone, $I=$ iPod.
\[
\begin{aligned}
\Step{1}\;& \text{Only cell phone: } n(C\ \text{only})=n(C)-n(L\cap C)-n(C\cap I)+n(L\cap C\cap I)\\
&=80-23-14+8=51.\\[4pt]
\Step{2}\;& n(L\cup C\cup I)=n(L)+n(C)+n(I)-n(L\cap C)-n(L\cap I)-n(C\cap I)+n(L\cap C\cap I)\\
&=42+80+100-23-10-14+8=183.\\
&\text{None: } 200-183=17.\\[4pt]
\Step{3}\;& \text{Both iPod and laptop but not cellphone: } n((L\cap I)-C)=n(L\cap I)-n(L\cap C\cap I)=10-8=2.
\end{aligned}
\]
\end{QAPair}

% ============================================================
% Q8
\begin{QAPair}{Question 8}
\textcolor{gold}{\bfseries Question:} In a girl college, every student plays either badminton or table tennis or both. If $350$ play badminton, $280$ play table tennis and $150$ play both. Find how many students are there in the college?\\
\tcblower
\textcolor{green}{\bfseries Answer:} Let $B=$ badminton, $T=$ table tennis. Since everyone plays at least one, total $=n(B\cup T)$.
\[
\begin{aligned}
\Step{1}\;& n(B\cup T)=n(B)+n(T)-n(B\cap T)\\
\Step{2}\;&=350+280-150=480.
\end{aligned}
\]
\end{QAPair}

% ============================================================
% Q9
\begin{QAPair}{Question 9}
\textcolor{gold}{\bfseries Question:} Among $50$ students, $8$ are learning both English and Chinese languages. A total of $26$ students are learning English. If every student is learning at least one language, how many students are learning Chinese?\\
\tcblower
\textcolor{green}{\bfseries Answer:} Let $E=$ English, $C=$ Chinese. Since everyone learns at least one, $n(E\cup C)=50$.
\[
\begin{aligned}
\Step{1}\;& n(E\cup C)=n(E)+n(C)-n(E\cap C).\\
\Step{2}\;& 50=26+n(C)-8.\\
\Step{3}\;& n(C)=50-18=32.
\end{aligned}
\]
\end{QAPair}

% ============================================================
% Q10
\begin{QAPair}{Question 10}
\textcolor{gold}{\bfseries Question:} Out of $70$ people, $48$ like tea and $40$ like coffee and each person likes at least one of the two drinks. How many like both tea and coffee?\\
\tcblower
\textcolor{green}{\bfseries Answer:} Let $T=$ tea, $C=$ coffee. Since everyone likes at least one, $n(T\cup C)=70$.
\[
\begin{aligned}
\Step{1}\;& n(T\cup C)=n(T)+n(C)-n(T\cap C).\\
\Step{2}\;& 70=48+40-n(T\cap C).\\
\Step{3}\;& n(T\cap C)=88-70=18.
\end{aligned}
\]
\end{QAPair}

% ============================================================
% Q11
\begin{QAPair}{Question 11}
\textcolor{gold}{\bfseries Question:} There are $46$ students in science group and $50$ students in arts group. Find the number of students who are either in science or arts group.\\
\tcblower
\textcolor{green}{\bfseries Answer:} Science and Arts groups are treated as separate (no overlap).
\[
\begin{aligned}
\Step{1}\;& n(\text{science or arts})=46+50=96.
\end{aligned}
\]
\end{QAPair}

% ============================================================
% Q12
\begin{QAPair}{Question 12}
\textcolor{gold}{\bfseries Question:} In a group of people, $52$ people can speak Arabic and $112$ can speak French. How many can speak Arabic only? How many can speak French only if $12$ of them can speak both languages? How many people were in the group?\\
\tcblower
\textcolor{green}{\bfseries Answer:} Let $A=$ Arabic, $F=$ French, with $n(A\cap F)=12$.
\[
\begin{aligned}
\Step{1}\;& \text{Arabic only }=n(A)-n(A\cap F)=52-12=40.\\
\Step{2}\;& \text{French only }=n(F)-n(A\cap F)=112-12=100.\\
\Step{3}\;& \text{Total }=n(A\cup F)=n(A)+n(F)-n(A\cap F)=52+112-12=152.
\end{aligned}
\]
\end{QAPair}

% ============================================================
% Q13
\begin{QAPair}{Question 13 (i)--(iii)}
\textcolor{gold}{\bfseries Question:} In a high school, $360$ students like reading story books, $170$ like practical activities and $150$ like both. Find:
\begin{enumerate}[label=(\roman*)]
\item The number of students who like reading story books only.
\item The number of students who like only practical activities.
\item The total number of students in the school.
\end{enumerate}
\tcblower
\textcolor{green}{\bfseries Answer:} Let $S=$ story books, $P=$ practical activities, with $n(S\cap P)=150$.
\[
\begin{aligned}
\Step{1}\;& \text{Story books only }=n(S)-n(S\cap P)=360-150=210.\\
\Step{2}\;& \text{Practical only }=n(P)-n(S\cap P)=170-150=20.\\
\Step{3}\;& \text{Total }=n(S\cup P)=n(S)+n(P)-n(S\cap P)=360+170-150=380.
\end{aligned}
\]
\end{QAPair}

% ============================================================
% Q14
\begin{QAPair}{Question 14}
\textcolor{gold}{\bfseries Question:} In a survey of $60$ people, it was found that $25$ people watch channel $A$, $16$ watch channel $B$, $13$ watch channel $C$, $4$ watch both $A$ and $B$, $7$ watch both $B$ and $C$, $8$ watch both $A$ and $C$, $3$ watch all three channels. Find the number of people who watch at least one of the channels. Also find number of people who do not watch these channels.\\
\tcblower
\textcolor{green}{\bfseries Answer:} Use inclusion--exclusion for three sets $A,B,C$.
\[
\begin{aligned}
\Step{1}\;& n(A\cup B\cup C)=n(A)+n(B)+n(C)-n(A\cap B)-n(B\cap C)-n(A\cap C)+n(A\cap B\cap C)\\
\Step{2}\;&=25+16+13-4-7-8+3=38.\\
\Step{3}\;& \text{Do not watch any }=60-38=22.
\end{aligned}
\]
\end{QAPair}

\end{document}
