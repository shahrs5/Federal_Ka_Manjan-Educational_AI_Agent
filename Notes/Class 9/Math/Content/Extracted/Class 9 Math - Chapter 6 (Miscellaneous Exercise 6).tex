% !TEX TS-program = pdflatex
\documentclass[11pt]{article}

% -------------------- Packages --------------------
\usepackage[a4paper,margin=1in]{geometry}
\usepackage{amsmath,amssymb}
\usepackage[T1]{fontenc}
\usepackage{lmodern}
\usepackage{xcolor}
\usepackage{tcolorbox}
\tcbuselibrary{skins,breakable}
\usepackage{enumitem}
\usepackage{hyperref}
\usepackage{tikz}
\usetikzlibrary{calc,patterns,angles,quotes}

\pagestyle{empty}

% -------------------- Dark Theme Colors --------------------
\definecolor{bg}{HTML}{000000}
\definecolor{pairbg}{HTML}{121212}
\definecolor{solbg}{HTML}{0A0A0A}
\definecolor{border}{HTML}{2A2A2A}
\definecolor{text}{HTML}{FFFFFF}
\definecolor{muted}{HTML}{C9CDD3}
\definecolor{gold}{HTML}{FFD700}
\definecolor{green}{HTML}{4ADE80}
\definecolor{cyan}{HTML}{38BDF8}

\pagecolor{bg}
\color{text}

\hypersetup{
  colorlinks=true,
  linkcolor=cyan,
  urlcolor=cyan
}

\setlength{\parindent}{0pt}
\setlength{\parskip}{10pt}

\setlist[itemize]{left=1.4em,itemsep=6pt,topsep=6pt}
\setlist[enumerate]{left=1.6em,itemsep=4pt,topsep=4pt}

% -------------------- tcolorbox Base --------------------
\tcbset{
  enhanced,
  breakable,
  arc=12pt,
  boxrule=0.8pt,
  left=16pt,right=16pt,top=12pt,bottom=12pt
}

\newtcolorbox{QAPair}[1]{%
  colback=pairbg,
  colbacklower=solbg,
  colframe=border,
  coltext=text,
  title=\textcolor{gold}{\bfseries #1},
  fonttitle=\bfseries,
  coltitle=text,
  segmentation style={draw=border, dashed, line width=0.6pt},
}

\newtcolorbox{QuickBox}{%
  colback=pairbg,
  colframe=cyan,
  coltext=text,
  fontupper=\color{text},
  borderline north={4pt}{0pt}{cyan},
  arc=14pt,
  boxrule=0.8pt
}

% Helper for step headings
\newcommand{\Step}[1]{\textcolor{muted}{\textbf{Step #1:}}}

% -------------------- TikZ Styles --------------------
\tikzset{
  diagLine/.style={draw=muted, line width=0.9pt},
  diagBold/.style={draw=text, line width=1.1pt},
  diagLabel/.style={text=text, font=\small},
  diagNote/.style={text=muted, font=\small}
}

% ============================================================
\begin{document}

\begin{center}
{\LARGE\bfseries \textcolor{gold}{Miscellaneous Exercise 6 --- Solutions}}\\[-2pt]
\end{center}

\begin{QuickBox}
{\color{cyan}\bfseries Quick formulas (useful)}\par\medskip
\begin{itemize}
\item \textbf{Pythagorean identity:} $\sin^2\theta+\cos^2\theta=1$.
\item \textbf{Basic identities:} $1+\tan^2\theta=\sec^2\theta,\quad 1+\cot^2\theta=\csc^2\theta$.
\item \textbf{From identities:} $\csc^2\theta-\cot^2\theta=1$.
\item \textbf{Degree--radian:} $180^\circ=\pi\ \text{rad}$, so $\theta^\circ=\dfrac{\pi\theta}{180}$.
\item \textbf{Bearings:} measured clockwise from North (N = $0^\circ$, E = $90^\circ$, S = $180^\circ$, W = $270^\circ$).
\item \textbf{Right triangle:} if opposite $=a$, adjacent $=b$, then hypotenuse $=\sqrt{a^2+b^2}$.
\end{itemize}
\end{QuickBox}

% ============================================================
% Q1 MCQs (i) to (xv)
\begin{QAPair}{Question 1 (i) --- MCQ}
\textcolor{gold}{\bfseries Question:} If $\sin x=\dfrac14$, what is the value of $\cos x$?\par
\begin{itemize}
\item[(a)] $\dfrac{\sqrt{15}}{4}$
\item[(b)] $\dfrac{3}{4}$
\item[(c)] $\dfrac{\sqrt{17}}{4}$
\item[(d)] $\dfrac{15}{4}$
\end{itemize}
\tcblower
\textcolor{green}{\bfseries Answer:} \textbf{(a)}\par
\[
\begin{aligned}
\Step{1}\;& \sin^2x+\cos^2x=1.\\
\Step{2}\;& \cos^2x=1-\sin^2x=1-\left(\frac14\right)^2=1-\frac{1}{16}=\frac{15}{16}.\\
\Step{3}\;& \cos x=\sqrt{\frac{15}{16}}=\frac{\sqrt{15}}{4}\quad(\text{taking the standard acute value in MCQ}). 
\end{aligned}
\]
\end{QAPair}

\begin{QAPair}{Question 1 (ii) --- MCQ}
\textcolor{gold}{\bfseries Question:} $\cos\left(\dfrac{2\pi}{3}\right)=?$ \par
\begin{itemize}
\item[(a)] $\dfrac12$
\item[(b)] $-\dfrac12$
\item[(c)] $-\dfrac{\sqrt3}{2}$
\item[(d)] $\dfrac{\sqrt3}{2}$
\end{itemize}
\tcblower
\textcolor{green}{\bfseries Answer:} \textbf{(b)}\par
\[
\begin{aligned}
\Step{1}\;& \frac{2\pi}{3}=120^\circ.\\
\Step{2}\;& \cos 120^\circ=-\frac12. 
\end{aligned}
\]
\end{QAPair}

\begin{QAPair}{Question 1 (iii) --- MCQ}
\textcolor{gold}{\bfseries Question:} $\dfrac{1}{1-\sin\theta}+\dfrac{1}{1+\sin\theta}=?$\par
\begin{itemize}
\item[(a)] $\csc^2\theta$
\item[(b)] $2\csc^2\theta$
\item[(c)] $\sec^2\theta$
\item[(d)] $2\sec^2\theta$
\end{itemize}
\tcblower
\textcolor{green}{\bfseries Answer:} \textbf{(d)}\par
\[
\begin{aligned}
\Step{1}\;& \frac{1}{1-\sin\theta}+\frac{1}{1+\sin\theta}
= \frac{(1+\sin\theta)+(1-\sin\theta)}{(1-\sin\theta)(1+\sin\theta)}\\
\Step{2}\;&=\frac{2}{1-\sin^2\theta}=\frac{2}{\cos^2\theta}=2\sec^2\theta.
\end{aligned}
\]
\end{QAPair}

\begin{QAPair}{Question 1 (iv) --- MCQ}
\textcolor{gold}{\bfseries Question:} $-3-3\tan^2\theta=?$\par
\begin{itemize}
\item[(a)] $3\csc^2\theta$
\item[(b)] $-3\csc^2\theta$
\item[(c)] $-3\sec^2\theta$
\item[(d)] $3\sec^2\theta$
\end{itemize}
\tcblower
\textcolor{green}{\bfseries Answer:} \textbf{(c)}\par
\[
\begin{aligned}
\Step{1}\;& -3-3\tan^2\theta=-3(1+\tan^2\theta).\\
\Step{2}\;& 1+\tan^2\theta=\sec^2\theta\ \Rightarrow\ -3(1+\tan^2\theta)=-3\sec^2\theta.
\end{aligned}
\]
\end{QAPair}

\begin{QAPair}{Question 1 (v) --- MCQ}
\textcolor{gold}{\bfseries Question:} If $\cos x=\sin x$, then the value of $x$ is\par
\begin{itemize}
\item[(a)] $30^\circ$
\item[(b)] $45^\circ$
\item[(c)] $60^\circ$
\item[(d)] $90^\circ$
\end{itemize}
\tcblower
\textcolor{green}{\bfseries Answer:} \textbf{(b)}\par
\[
\begin{aligned}
\Step{1}\;& \cos x=\sin x \Rightarrow \tan x=1.\\
\Step{2}\;& \tan 45^\circ=1 \Rightarrow x=45^\circ \ (\text{principal acute angle}).
\end{aligned}
\]
\end{QAPair}

\begin{QAPair}{Question 1 (vi) --- MCQ}
\textcolor{gold}{\bfseries Question:} $4\csc^2\theta-4\cot^2\theta-4\cos 0^\circ=?$\par
\begin{itemize}
\item[(a)] $0$
\item[(b)] $1$
\item[(c)] $-1$
\item[(d)] $4$
\end{itemize}
\tcblower
\textcolor{green}{\bfseries Answer:} \textbf{(a)}\par
\[
\begin{aligned}
\Step{1}\;& \csc^2\theta-\cot^2\theta=1.\\
\Step{2}\;& 4\csc^2\theta-4\cot^2\theta=4(\csc^2\theta-\cot^2\theta)=4.\\
\Step{3}\;& \cos 0^\circ=1 \Rightarrow 4-4\cos 0^\circ=4-4(1)=0.
\end{aligned}
\]
\end{QAPair}

\begin{QAPair}{Question 1 (vii) --- MCQ}
\textcolor{gold}{\bfseries Question:} If $\tan x=\dfrac{a}{b}$, then $\sin x=?$\par
\begin{itemize}
\item[(a)] $\dfrac{a}{\sqrt{a^2-b^2}}$
\item[(b)] $\dfrac{b}{\sqrt{a^2-b^2}}$
\item[(c)] $\dfrac{a}{\sqrt{a^2+b^2}}$
\item[(d)] $\dfrac{b}{\sqrt{a^2+b^2}}$
\end{itemize}
\tcblower
\textcolor{green}{\bfseries Answer:} \textbf{(c)}\par
\[
\begin{aligned}
\Step{1}\;& \tan x=\frac{\text{opp}}{\text{adj}}=\frac{a}{b}\Rightarrow \text{opp}=a,\ \text{adj}=b.\\
\Step{2}\;& \text{hyp}=\sqrt{a^2+b^2}.\\
\Step{3}\;& \sin x=\frac{\text{opp}}{\text{hyp}}=\frac{a}{\sqrt{a^2+b^2}}.
\end{aligned}
\]
\end{QAPair}

\begin{QAPair}{Question 1 (viii) --- MCQ}
\textcolor{gold}{\bfseries Question:} $50^\circ 30'=?$\par
\begin{itemize}
\item[(a)] $50.2^\circ$
\item[(b)] $50.3^\circ$
\item[(c)] $50.4^\circ$
\item[(d)] $50.5^\circ$
\end{itemize}
\tcblower
\textcolor{green}{\bfseries Answer:} \textbf{(d)}\par
\[
\begin{aligned}
\Step{1}\;& 30'=\frac{30}{60}^\circ=0.5^\circ.\\
\Step{2}\;& 50^\circ 30'=50^\circ+0.5^\circ=50.5^\circ.
\end{aligned}
\]
\end{QAPair}

\begin{QAPair}{Question 1 (ix) --- MCQ}
\textcolor{gold}{\bfseries Question:} $(1-\cos^2\theta)\sec^2\theta=?$\par
\begin{itemize}
\item[(a)] $\csc^2\theta$
\item[(b)] $\cot^2\theta$
\item[(c)] $\sec^2\theta$
\item[(d)] $\tan^2\theta$
\end{itemize}
\tcblower
\textcolor{green}{\bfseries Answer:} \textbf{(d)}\par
\[
\begin{aligned}
\Step{1}\;& 1-\cos^2\theta=\sin^2\theta.\\
\Step{2}\;& (1-\cos^2\theta)\sec^2\theta=\sin^2\theta\cdot\frac{1}{\cos^2\theta}
=\frac{\sin^2\theta}{\cos^2\theta}=\tan^2\theta.
\end{aligned}
\]
\end{QAPair}

\begin{QAPair}{Question 1 (x) --- MCQ}
\textcolor{gold}{\bfseries Question:} In which quadrant do the angles between $90^\circ$ and $180^\circ$ lie?\par
\begin{itemize}
\item[(a)] 1st
\item[(b)] 2nd
\item[(c)] 3rd
\item[(d)] 4th
\end{itemize}
\tcblower
\textcolor{green}{\bfseries Answer:} \textbf{(b)}\par
\[
\begin{aligned}
\Step{1}\;& (90^\circ,180^\circ)\ \text{is exactly Quadrant II}.
\end{aligned}
\]
\end{QAPair}

\begin{QAPair}{Question 1 (xi) --- MCQ}
\textcolor{gold}{\bfseries Question:} The system in which the angles are measured in radians is called\par
\begin{itemize}
\item[(a)] CGS
\item[(b)] sexagesimal
\item[(c)] circular
\item[(d)] centesimal
\end{itemize}
\tcblower
\textcolor{green}{\bfseries Answer:} \textbf{(c)}\par

\end{QAPair}

\begin{QAPair}{Question 1 (xii) --- MCQ}
\textcolor{gold}{\bfseries Question:} $270^\circ=?$\par
\begin{itemize}
\item[(a)] $\dfrac{3\pi}{2}$
\item[(b)] $\dfrac{\pi}{2}$
\item[(c)] $-\dfrac{\pi}{2}$
\item[(d)] $-\dfrac{3\pi}{2}$
\end{itemize}
\tcblower
\textcolor{green}{\bfseries Answer:} \textbf{(a)}\par
\[
\begin{aligned}
\Step{1}\;& 180^\circ=\pi \Rightarrow 270^\circ=\frac{270}{180}\pi=\frac{3\pi}{2}.
\end{aligned}
\]
\end{QAPair}

\begin{QAPair}{Question 1 (xiii) --- MCQ}
\textcolor{gold}{\bfseries Question:} What is the bearing of south-west direction line?\par
\begin{itemize}
\item[(a)] $45^\circ$
\item[(b)] $135^\circ$
\item[(c)] $225^\circ$
\item[(d)] $270^\circ$
\end{itemize}
\tcblower
\textcolor{green}{\bfseries Answer:} \textbf{(c)}\par
\[
\begin{aligned}
\Step{1}\;& \text{SW is halfway between South }(180^\circ)\text{ and West }(270^\circ).\\
\Step{2}\;& \Rightarrow \text{bearing}=180^\circ+45^\circ=225^\circ.
\end{aligned}
\]
\end{QAPair}

\begin{QAPair}{Question 1 (xiv) --- MCQ}
\textcolor{gold}{\bfseries Question:} If the bearing from city A to city B is $070^\circ$, then the bearing from city B to city A is\par
\begin{itemize}
\item[(a)] $070^\circ$
\item[(b)] $110^\circ$
\item[(c)] $250^\circ$
\item[(d)] $290^\circ$
\end{itemize}
\tcblower
\textcolor{green}{\bfseries Answer:} \textbf{(c)}\par
\[
\begin{aligned}
\Step{1}\;& \text{Reverse bearing} = 070^\circ+180^\circ=250^\circ.
\end{aligned}
\]
\end{QAPair}

\begin{QAPair}{Question 1 (xv) --- MCQ}
\textcolor{gold}{\bfseries Question:} A boat sails $4$ km north from $L$ to $M$, then $3$ km west from $M$ to $P$.
What is the bearing of $P$ from $L$ (nearest degree)?\par
\begin{itemize}
\item[(a)] $037^\circ$
\item[(b)] $053^\circ$
\item[(c)] $217^\circ$
\item[(d)] $323^\circ$
\end{itemize}
\tcblower
\textcolor{green}{\bfseries Answer:} \textbf{(d)}\par
\[
\begin{aligned}
\Step{1}\;& \overrightarrow{LP}: \text{North }=4,\ \text{West }=3 \Rightarrow \text{direction is NW.}\\
\Step{2}\;& \text{Angle west of North }=\tan^{-1}\!\left(\frac{3}{4}\right)\approx 36.87^\circ.\\
\Step{3}\;& \text{Bearing} = 360^\circ-36.87^\circ \approx 323.13^\circ \approx 323^\circ.
\end{aligned}
\]
\end{QAPair}

% ============================================================
% Q2
\begin{QAPair}{Question 2}
\textcolor{gold}{\bfseries Question:} Find $\sin\alpha$, $\tan\alpha$ and $\sec\alpha$, if $\cos\alpha=-\dfrac{5}{13}$,
where $\dfrac{\pi}{2}<\alpha<\pi$.\\
\tcblower
\textcolor{green}{\bfseries Answer:}
\[
\begin{aligned}
\Step{1}\;& \frac{\pi}{2}<\alpha<\pi \Rightarrow \alpha \text{ is in Quadrant II }(\sin>0,\ \cos<0,\ \tan<0).\\
\Step{2}\;& \sin^2\alpha=1-\cos^2\alpha
=1-\left(\frac{5}{13}\right)^2
=1-\frac{25}{169}=\frac{144}{169}.\\
\Step{3}\;& \sin\alpha=\frac{12}{13}\quad(\text{positive in QII}).\\
\Step{4}\;& \tan\alpha=\frac{\sin\alpha}{\cos\alpha}
=\frac{\frac{12}{13}}{-\frac{5}{13}}=-\frac{12}{5}.\\
\Step{5}\;& \sec\alpha=\frac{1}{\cos\alpha}=-\frac{13}{5}.
\end{aligned}
\]
\end{QAPair}

% ============================================================
% Q3
\begin{QAPair}{Question 3}
\textcolor{gold}{\bfseries Question:} The angle of elevation of a hot air balloon $500$ m away, climbing up vertically,
changes from $30^\circ$ to $60^\circ$ in $100$ seconds. What is the upward speed (m/s)?\\
\tcblower
\textcolor{green}{\bfseries Answer:}
\[
\begin{aligned}
\Step{1}\;& \text{Horizontal distance is }500\text{ m. Height }h=500\tan\theta.\\
\Step{2}\;& h_1=500\tan 30^\circ=500\cdot\frac{1}{\sqrt3}=\frac{500}{\sqrt3}.\\
\Step{3}\;& h_2=500\tan 60^\circ=500\cdot\sqrt3=500\sqrt3.\\
\Step{4}\;& \Delta h=h_2-h_1
=500\left(\sqrt3-\frac{1}{\sqrt3}\right)
=500\cdot\frac{3-1}{\sqrt3}=\frac{1000}{\sqrt3}.\\
\Step{5}\;& \text{Speed}=\frac{\Delta h}{100}=\frac{1}{100}\cdot\frac{1000}{\sqrt3}
=\frac{10}{\sqrt3}\ \text{m/s}\approx 5.77\ \text{m/s}.
\end{aligned}
\]
\end{QAPair}

% ============================================================
% Q4
\begin{QAPair}{Question 4}
\textcolor{gold}{\bfseries Question:} Find the value of $a$ in the given figure.\\
\tcblower
\textcolor{green}{\bfseries Answer:}

\textcolor{muted}{The figure shows two right triangles sharing the same height $AB=10$, with hypotenuses $AC=14$ and $AD=16$, and $CD=a$.}\par

\[
\begin{aligned}
\Step{1}\;& \text{In }\triangle ABC:\quad BC=\sqrt{AC^2-AB^2}=\sqrt{14^2-10^2}=\sqrt{196-100}=\sqrt{96}=4\sqrt6.\\
\Step{2}\;& \text{In }\triangle ABD:\quad BD=\sqrt{AD^2-AB^2}=\sqrt{16^2-10^2}=\sqrt{256-100}=\sqrt{156}=2\sqrt{39}.\\
\Step{3}\;& CD=BD-BC \Rightarrow a=2\sqrt{39}-4\sqrt6.
\end{aligned}
\]
\[
\boxed{a=2\sqrt{39}-4\sqrt6 \ \approx\ 2.74}
\]

\textcolor{muted}{(Sketch)}\par
\begin{center}
\begin{tikzpicture}[scale=0.75]
  \coordinate (B) at (0,0);
  \coordinate (A) at (0,4);
  \coordinate (D) at (8.1,0);
  % Choose C so that AC ~ 14 when AB scaled; this is just a sketch
  \coordinate (C) at (5.0,0);

  \draw[diagBold] (A)--(B)--(D);
  \draw[diagLine] (A)--(C);
  \draw[diagBold] (A)--(D);

  \node[diagLabel] at ($(A)+(0.2,0.2)$) {$A$};
  \node[diagLabel] at ($(B)+(-0.2,-0.2)$) {$B$};
  \node[diagLabel] at ($(C)+(0,-0.4)$) {$C$};
  \node[diagLabel] at ($(D)+(0.2,-0.2)$) {$D$};

  \draw[diagLine] (B) ++(0,0) rectangle ++(0.5,0.5);

  \node[diagNote] at (-0.6,2.0) {$10$};
  \node[diagNote] at ($(A)!0.5!(C)+(0.2,0.3)$) {$14$};
  \node[diagNote] at ($(A)!0.5!(D)+(0.2,0.3)$) {$16$};

  \node[diagNote] at ($(C)!0.5!(D)+(0,-0.6)$) {$a$};
\end{tikzpicture}
\end{center}
\end{QAPair}

% ============================================================
% Q5
\begin{QAPair}{Question 5}
\textcolor{gold}{\bfseries Question:} The area of a right triangle is $50\text{ cm}^2$. One of its angles is $45^\circ$.
Find the lengths of the sides and hypotenuse.\\
\tcblower
\textcolor{green}{\bfseries Answer:}
\[
\begin{aligned}
\Step{1}\;& \text{A right triangle with a }45^\circ\text{ angle is a }45^\circ\!-\!45^\circ\!-\!90^\circ\text{ triangle, so legs are equal.}\\
\Step{2}\;& \text{Let each leg }=s.\ \text{Area}=\frac12 s^2=50 \Rightarrow s^2=100 \Rightarrow s=10\text{ cm}.\\
\Step{3}\;& \text{Hypotenuse} = s\sqrt2 = 10\sqrt2\text{ cm}.
\end{aligned}
\]
\[
\boxed{\text{Legs: }10\text{ cm and }10\text{ cm, \quad Hypotenuse: }10\sqrt2\text{ cm}}
\]
\end{QAPair}

% ============================================================
% Q6
\begin{QAPair}{Question 6}
\textcolor{gold}{\bfseries Question:} If the shadow of a building increases by $12$ m when the angle of elevation of the sun rays
decreases from $60^\circ$ to $45^\circ$, what is the height of that building?\\
\tcblower
\textcolor{green}{\bfseries Answer:}
\[
\begin{aligned}
\Step{1}\;& \text{Let height }=h. \ \text{Shadow length }=\frac{h}{\tan\theta}.\\
\Step{2}\;& \text{At }60^\circ:\ x=\frac{h}{\tan60^\circ}=\frac{h}{\sqrt3}.\\
\Step{3}\;& \text{At }45^\circ:\ y=\frac{h}{\tan45^\circ}=h.\\
\Step{4}\;& y-x=12 \Rightarrow h-\frac{h}{\sqrt3}=12
\Rightarrow h\left(1-\frac{1}{\sqrt3}\right)=12.\\
\Step{5}\;& h=\frac{12}{1-\frac{1}{\sqrt3}}
= \frac{12\sqrt3}{\sqrt3-1}
=\frac{12\sqrt3(\sqrt3+1)}{(\sqrt3-1)(\sqrt3+1)}
=\frac{12\sqrt3(\sqrt3+1)}{2}\\
&=6\sqrt3(\sqrt3+1)=6(3+\sqrt3)=18+6\sqrt3.
\end{aligned}
\]
\[
\boxed{h=18+6\sqrt3\ \text{m}\ \approx\ 28.39\ \text{m}}
\]
\end{QAPair}

% ============================================================
% Q7
\begin{QAPair}{Question 7}
\textcolor{gold}{\bfseries Question:} From the top of a $100$ m high building, the angle of depression to the bottom of a second building is $30^\circ$.
From the same point, the angle of elevation to the top of the second building is $20^\circ$.
Calculate the height of the second building.\\
\tcblower
\textcolor{green}{\bfseries Answer:}
\[
\begin{aligned}
\Step{1}\;& \text{Let the horizontal distance between buildings be }d.\\
\Step{2}\;& \text{Angle of depression to bottom is }30^\circ \Rightarrow \tan30^\circ=\frac{100}{d}.\\
\Step{3}\;& d=\frac{100}{\tan30^\circ}=100\sqrt3.\\
\Step{4}\;& \text{Angle of elevation to the top of second building is }20^\circ\\
&\Rightarrow \tan20^\circ=\frac{H-100}{d}\ \Rightarrow\ H-100=d\tan20^\circ.\\
\Step{5}\;& H=100+100\sqrt3\,\tan20^\circ.
\end{aligned}
\]
\[
\boxed{H=100+100\sqrt3\,\tan20^\circ\ \text{m}\ \approx\ 163.0\ \text{m}}
\]
\end{QAPair}

% ============================================================
% Q8
\begin{QAPair}{Question 8}
\textcolor{gold}{\bfseries Question:} A plane is $119$ km west and $100$ km south of an airport.
At what bearing does the pilot want to fly directly back to the airport?\\
\tcblower
\textcolor{green}{\bfseries Answer:}

\textcolor{muted}{From the plane to the airport: move }$119$ km \textbf{east} and $100$ km \textbf{north} (direction is NE).\par
\[
\begin{aligned}
\Step{1}\;& \text{Let }\phi=\text{angle east of North. Then }\tan\phi=\frac{\text{East}}{\text{North}}=\frac{119}{100}.\\
\Step{2}\;& \phi=\tan^{-1}\!\left(\frac{119}{100}\right)\approx 49.96^\circ.\\
\Step{3}\;& \text{Bearing (clockwise from North)}\approx 50^\circ.
\end{aligned}
\]
\[
\boxed{\text{Bearing} \approx 050^\circ\ (\text{to the nearest degree})}
\]
\end{QAPair}

\end{document}
