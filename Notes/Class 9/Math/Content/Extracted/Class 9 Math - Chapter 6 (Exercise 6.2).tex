% !TEX TS-program = pdflatex
\documentclass[11pt]{article}

% -------------------- Packages --------------------
\usepackage[a4paper,margin=1in]{geometry}
\usepackage{amsmath,amssymb}
\usepackage[T1]{fontenc}
\usepackage{lmodern}
\usepackage{xcolor}
\usepackage{tcolorbox}
\tcbuselibrary{skins,breakable}
\usepackage{enumitem}
\usepackage{hyperref}

\pagestyle{empty}

% -------------------- Dark Theme Colors --------------------
\definecolor{bg}{HTML}{000000}
\definecolor{pairbg}{HTML}{121212}
\definecolor{solbg}{HTML}{0A0A0A}
\definecolor{border}{HTML}{2A2A2A}
\definecolor{text}{HTML}{FFFFFF}
\definecolor{muted}{HTML}{C9CDD3}
\definecolor{gold}{HTML}{FFD700}
\definecolor{green}{HTML}{4ADE80}
\definecolor{cyan}{HTML}{38BDF8}

\pagecolor{bg}
\color{text}

\hypersetup{
  colorlinks=true,
  linkcolor=cyan,
  urlcolor=cyan
}

\setlength{\parindent}{0pt}
\setlength{\parskip}{10pt}

\setlist[itemize]{left=1.4em,itemsep=6pt,topsep=6pt}
\setlist[enumerate]{left=1.6em,itemsep=4pt,topsep=4pt}

% -------------------- tcolorbox Base --------------------
\tcbset{
  enhanced,
  breakable,
  arc=12pt,
  boxrule=0.8pt,
  left=16pt,right=16pt,top=12pt,bottom=12pt
}

\newtcolorbox{QAPair}[1]{%
  colback=pairbg,
  colbacklower=solbg,
  colframe=border,
  coltext=text,
  title=\textcolor{gold}{\bfseries #1},
  fonttitle=\bfseries,
  coltitle=text,
  segmentation style={draw=border, dashed, line width=0.6pt},
}

% Visible text inside this box (fix)
\newtcolorbox{QuickBox}{%
  colback=pairbg,
  colframe=cyan,
  coltext=text,
  fontupper=\color{text},
  borderline north={4pt}{0pt}{cyan},
  arc=14pt,
  boxrule=0.8pt
}

% Helper for step headings
\newcommand{\Step}[1]{\textcolor{muted}{\textbf{Step #1:}}}

% ============================================================
\begin{document}

\begin{center}
{\LARGE\bfseries \textcolor{gold}{Exercise 6.2 --- Solutions}}\\[-2pt]
\end{center}

\begin{QuickBox}
{\color{cyan}\bfseries Quick formulas (useful)}\par\medskip
{\small
\begin{itemize}
\item \textbf{Arc length (radians):} $l=r\theta$.
\item \textbf{Sector area (radians):} $A=\dfrac12 r^2\theta$.
\item \textbf{Degrees to radians:} $\theta_{\text{rad}}=\theta^\circ\cdot \dfrac{\pi}{180}$.
\item \textbf{DMS to degrees:} $D^\circ M' S''= \left(D+\dfrac{M}{60}+\dfrac{S}{3600}\right)^\circ$.
\item \textbf{From arc length:} $\theta=\dfrac{l}{r}$.
\item \textbf{From area:} $r=\sqrt{\dfrac{2A}{\theta}}$ (if $\theta$ is known in radians).
\item \textbf{Using $l$ and $A$:} since $\theta=\dfrac{l}{r}$, we get $A=\dfrac12 rl$.
\item \textbf{Clock hands (degrees):} Hour hand $=30h+0.5m$, Minute hand $=6m$,
then take the \emph{smaller} angle and convert to radians.
\end{itemize}
}
\end{QuickBox}

% ============================================================
% Q1
\begin{QAPair}{Question 1 (i)}
\textcolor{gold}{\bfseries Question:} Find arc length and sector area. \; $r=5\,\text{cm},\;\theta=\dfrac{\pi}{3}$ (radians).\\
\tcblower
\textcolor{green}{\bfseries Answer:}
\[
\begin{aligned}
\Step{1}\;& l=r\theta=5\cdot \frac{\pi}{3}=\frac{5\pi}{3}\ \text{cm}.\\
\Step{2}\;& A=\frac12 r^2\theta=\frac12(5^2)\cdot\frac{\pi}{3}
=\frac{25\pi}{6}\ \text{cm}^2.
\end{aligned}
\]
\end{QAPair}

\begin{QAPair}{Question 1 (ii)}
\textcolor{gold}{\bfseries Question:} Find arc length and sector area. \; $r=12\,\text{m},\;\theta=120^\circ$.\\
\tcblower
\textcolor{green}{\bfseries Answer:}
\[
\begin{aligned}
\Step{1}\;& 120^\circ=\frac{120\pi}{180}=\frac{2\pi}{3}\ \text{rad}.\\
\Step{2}\;& l=r\theta=12\cdot \frac{2\pi}{3}=8\pi\ \text{m}.\\
\Step{3}\;& A=\frac12 r^2\theta=\frac12(12^2)\cdot\frac{2\pi}{3}
=48\pi\ \text{m}^2.
\end{aligned}
\]
\end{QAPair}

\begin{QAPair}{Question 1 (iii)}
\textcolor{gold}{\bfseries We convert DMS to radians first.}\\
\textcolor{gold}{\bfseries Question:} Find arc length and sector area. \; $r=6\,\text{dm},\;\theta=60^\circ45'30''$.\\
\tcblower
\textcolor{green}{\bfseries Answer:}
\[
\begin{aligned}
\Step{1}\;& 60^\circ45'30''=60+\frac{45}{60}+\frac{30}{3600}
=60+\frac{3}{4}+\frac{1}{120}=\frac{7291}{120}^\circ.\\
\Step{2}\;& \theta=\frac{7291}{120}\cdot\frac{\pi}{180}=\frac{7291\pi}{21600}\ \text{rad}.\\
\Step{3}\;& l=r\theta=6\cdot \frac{7291\pi}{21600}
=\frac{7291\pi}{3600}\ \text{dm}.\\
\Step{4}\;& A=\frac12 r^2\theta=\frac12(6^2)\cdot\frac{7291\pi}{21600}
=\frac{7291\pi}{1200}\ \text{dm}^2.
\end{aligned}
\]
\end{QAPair}

% ============================================================
% Q2
\begin{QAPair}{Question 2}
\textcolor{gold}{\bfseries Question:} A central angle is $75^\circ$ in a circle of radius $4\,\text{cm}$. Find the intercepted arc length and the sector area.\\
\tcblower
\textcolor{green}{\bfseries Answer:}
\[
\begin{aligned}
\Step{1}\;& 75^\circ=\frac{75\pi}{180}=\frac{5\pi}{12}\ \text{rad}.\\
\Step{2}\;& l=r\theta=4\cdot\frac{5\pi}{12}=\frac{5\pi}{3}\ \text{cm}.\\
\Step{3}\;& A=\frac12 r^2\theta=\frac12(4^2)\cdot\frac{5\pi}{12}
=\frac{10\pi}{3}\ \text{cm}^2.
\end{aligned}
\]
\end{QAPair}

% ============================================================
% Q3
\begin{QAPair}{Question 3 (i)}
\textcolor{gold}{\bfseries Question:} Find $\theta$ (radians). \; $l=8\,\text{cm},\ r=4\,\text{cm}$.\\
\tcblower
\textcolor{green}{\bfseries Answer:}
\[
\begin{aligned}
\Step{1}\;& \theta=\frac{l}{r}=\frac{8}{4}=2\ \text{rad}.
\end{aligned}
\]
\end{QAPair}

\begin{QAPair}{Question 3 (ii)}
\textcolor{gold}{\bfseries Question:} Find $\theta$ (radians). \; $l=8.5\,\text{m},\ r=2.25\,\text{m}$.\\
\tcblower
\textcolor{green}{\bfseries Answer:}
\[
\begin{aligned}
\Step{1}\;& \theta=\frac{l}{r}=\frac{8.5}{2.25}
=\frac{85/10}{225/100}=\frac{850}{225}=\frac{34}{9}\ \text{rad}\\
&\approx 3.78\ \text{rad}.
\end{aligned}
\]
\end{QAPair}

\begin{QAPair}{Question 3 (iii)}
\textcolor{gold}{\bfseries Question:} Find $\theta$ (radians). \; $A=45\,\text{cm}^2,\ r=10.70\,\text{cm}$.\\
\tcblower
\textcolor{green}{\bfseries Answer:}
\[
\begin{aligned}
\Step{1}\;& A=\frac12 r^2\theta \ \Rightarrow\ \theta=\frac{2A}{r^2}.\\
\Step{2}\;& \theta=\frac{2(45)}{(10.70)^2}=\frac{90}{114.49}\approx 0.787\ \text{rad}.
\end{aligned}
\]
\end{QAPair}

\begin{QAPair}{Question 3 (iv)}
\textcolor{gold}{\bfseries Question:} Find $\theta$ (radians). \; $A=100\,\text{cm}^2,\ l=10\,\text{cm}$.\\
\tcblower
\textcolor{green}{\bfseries Answer:}
\[
\begin{aligned}
\Step{1}\;& \theta=\frac{l}{r}\ \Rightarrow\ A=\frac12 r^2\left(\frac{l}{r}\right)=\frac12 rl.\\
\Step{2}\;& r=\frac{2A}{l}=\frac{2(100)}{10}=20\ \text{cm}.\\
\Step{3}\;& \theta=\frac{l}{r}=\frac{10}{20}=\frac12=0.5\ \text{rad}.
\end{aligned}
\]
\end{QAPair}

% ============================================================
% Q4
\begin{QAPair}{Question 4 (i)}
\textcolor{gold}{\bfseries Question:} Find $r$. \; $l=4\,\text{cm},\ \theta=\pi$ (radians).\\
\tcblower
\textcolor{green}{\bfseries Answer:}
\[
\begin{aligned}
\Step{1}\;& l=r\theta \Rightarrow r=\frac{l}{\theta}=\frac{4}{\pi}\ \text{cm}.
\end{aligned}
\]
\end{QAPair}

\begin{QAPair}{Question 4 (ii)}
\textcolor{gold}{\bfseries Question:} Find $r$. \; $l=6\,\text{m},\ \theta=15^\circ$.\\
\tcblower
\textcolor{green}{\bfseries Answer:}
\[
\begin{aligned}
\Step{1}\;& 15^\circ=\frac{15\pi}{180}=\frac{\pi}{12}\ \text{rad}.\\
\Step{2}\;& r=\frac{l}{\theta}=\frac{6}{\pi/12}=\frac{72}{\pi}\ \text{m}.
\end{aligned}
\]
\end{QAPair}

\begin{QAPair}{Question 4 (iii)}
\textcolor{gold}{\bfseries Question:} Find $r$. \; $A=200\,\text{cm}^2,\ \theta=\dfrac{\pi}{4}$ (radians).\\
\tcblower
\textcolor{green}{\bfseries Answer:}
\[
\begin{aligned}
\Step{1}\;& A=\frac12 r^2\theta \Rightarrow r^2=\frac{2A}{\theta}.\\
\Step{2}\;& r^2=\frac{2(200)}{\pi/4}=\frac{400\cdot 4}{\pi}=\frac{1600}{\pi}.\\
\Step{3}\;& r=\sqrt{\frac{1600}{\pi}}=\frac{40}{\sqrt{\pi}}\ \text{cm}.
\end{aligned}
\]
\end{QAPair}

\begin{QAPair}{Question 4 (iv)}
\textcolor{gold}{\bfseries Question:} Find $r$. \; $A=100\,\text{dm}^2,\ l=10\,\text{dm}$.\\
\tcblower
\textcolor{green}{\bfseries Answer:}
\[
\begin{aligned}
\Step{1}\;& A=\frac12 rl \Rightarrow r=\frac{2A}{l}
=\frac{2(100)}{10}=20\ \text{dm}.
\end{aligned}
\]
\end{QAPair}

% ============================================================
% Q5
\begin{QAPair}{Question 5}
\textcolor{gold}{\bfseries Question:} A $30$ inch pendulum swings through an angle of $30^\circ$. Find the length of the arc (in inches) through which the tip swings.\\
\tcblower
\textcolor{green}{\bfseries Answer:}
\[
\begin{aligned}
\Step{1}\;& 30^\circ=\frac{30\pi}{180}=\frac{\pi}{6}\ \text{rad}.\\
\Step{2}\;& l=r\theta=30\cdot\frac{\pi}{6}=5\pi\ \text{inches}.
\end{aligned}
\]
\end{QAPair}

% ============================================================
% Q6
\begin{QAPair}{Question 6}
\textcolor{gold}{\bfseries Question:} The curve is an arc of a circle of radius $10\,\text{km}$. A motorcycle travels at $42\,\text{km/h}$. What is the angle (in degrees) through which it turns in $21$ minutes?\\
\tcblower
\textcolor{green}{\bfseries Answer:}
\[
\begin{aligned}
\Step{1}\;& 21\ \text{min}=\frac{21}{60}\ \text{h}=\frac{7}{20}\ \text{h}.\\
\Step{2}\;& \text{Arc length } l=vt=42\cdot\frac{7}{20}=\frac{294}{20}=14.7\ \text{km}.\\
\Step{3}\;& \theta_{\text{rad}}=\frac{l}{r}=\frac{14.7}{10}=1.47\ \text{rad}.\\
\Step{4}\;& \theta^\circ=\theta_{\text{rad}}\cdot\frac{180}{\pi}
=1.47\cdot\frac{180}{\pi}\approx 84.2^\circ.
\end{aligned}
\]
\end{QAPair}

% ============================================================
% Q7
\begin{QAPair}{Question 7}
\textcolor{gold}{\bfseries Question:} Diameter $d=12\,\text{cm}$ (so a semicircle). Find perimeter and area using $l=r\theta$ and $A=\dfrac12 r^2\theta$.\\
\tcblower
\textcolor{green}{\bfseries Answer:}
\[
\begin{aligned}
\Step{1}\;& r=\frac{d}{2}=\frac{12}{2}=6\ \text{cm},\qquad \theta=\pi\ \text{rad (semicircle)}.\\
\Step{2}\;& \text{Arc length } l=r\theta=6\pi\ \text{cm}.\\
\Step{3}\;& \text{Perimeter } P=\text{(arc)}+\text{(diameter)}=6\pi+12\ \text{cm}.\\
\Step{4}\;& \text{Area }=\frac12 r^2\theta=\frac12(6^2)\pi=18\pi\ \text{cm}^2.
\end{aligned}
\]
\end{QAPair}

% ============================================================
% Q8
\begin{QAPair}{Question 8 (i)}
\textcolor{gold}{\bfseries Question:} Find the circular measure (radians) of the angle between the hour and minute hands at $9$ o'clock.\\
\tcblower
\textcolor{green}{\bfseries Answer:}
\[
\begin{aligned}
\Step{1}\;& \text{At } 9{:}00:\ \text{hour hand}=270^\circ,\ \text{minute hand}=0^\circ.\\
\Step{2}\;& \text{Smaller angle}=90^\circ.\\
\Step{3}\;& 90^\circ=\frac{90\pi}{180}=\frac{\pi}{2}\ \text{rad}.
\end{aligned}
\]
\end{QAPair}

\begin{QAPair}{Question 8 (ii)}
\textcolor{gold}{\bfseries Question:} Find the circular measure (radians) of the angle between the hour and minute hands at $02{:}30$.\\
\tcblower
\textcolor{green}{\bfseries Answer:}
\[
\begin{aligned}
\Step{1}\;& \text{Hour hand}=30h+0.5m=30(2)+0.5(30)=60+15=75^\circ.\\
\Step{2}\;& \text{Minute hand}=6m=6(30)=180^\circ.\\
\Step{3}\;& \text{Difference}=|180-75|=105^\circ\ (\text{smaller angle}).\\
\Step{4}\;& 105^\circ=\frac{105\pi}{180}=\frac{7\pi}{12}\ \text{rad}.
\end{aligned}
\]
\end{QAPair}

\begin{QAPair}{Question 8 (iii)}
\textcolor{gold}{\bfseries Question:} Find the circular measure (radians) of the angle between the hour and minute hands at $06{:}45$.\\
\tcblower
\textcolor{green}{\bfseries Answer:}
\[
\begin{aligned}
\Step{1}\;& \text{Hour hand}=30(6)+0.5(45)=180+22.5=202.5^\circ.\\
\Step{2}\;& \text{Minute hand}=6(45)=270^\circ.\\
\Step{3}\;& \text{Difference}=|270-202.5|=67.5^\circ\ (\text{smaller angle}).\\
\Step{4}\;& 67.5^\circ=\frac{67.5\pi}{180}=\frac{3\pi}{8}\ \text{rad}.
\end{aligned}
\]
\end{QAPair}

\end{document}
