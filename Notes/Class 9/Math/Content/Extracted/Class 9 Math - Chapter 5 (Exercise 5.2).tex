% !TEX TS-program = pdflatex
\documentclass[11pt]{article}

% -------------------- Packages --------------------
\usepackage[a4paper,margin=1in]{geometry}
\usepackage{amsmath,amssymb}
\usepackage[T1]{fontenc}
\usepackage{lmodern}
\usepackage{xcolor}
\usepackage{tcolorbox}
\tcbuselibrary{skins,breakable}
\usepackage{enumitem}
\usepackage{hyperref}

\pagestyle{empty}

% -------------------- Dark Theme Colors --------------------
\definecolor{bg}{HTML}{000000}
\definecolor{pairbg}{HTML}{121212}
\definecolor{solbg}{HTML}{0A0A0A}
\definecolor{border}{HTML}{2A2A2A}
\definecolor{text}{HTML}{FFFFFF}
\definecolor{muted}{HTML}{C9CDD3}
\definecolor{gold}{HTML}{FFD700}
\definecolor{green}{HTML}{4ADE80}
\definecolor{cyan}{HTML}{38BDF8}

\pagecolor{bg}
\color{text}

\hypersetup{
  colorlinks=true,
  linkcolor=cyan,
  urlcolor=cyan
}

\setlength{\parindent}{0pt}
\setlength{\parskip}{10pt}

\setlist[itemize]{left=1.4em,itemsep=6pt,topsep=6pt}
\setlist[enumerate]{left=1.6em,itemsep=4pt,topsep=4pt}

% -------------------- tcolorbox Base --------------------
\tcbset{
  enhanced,
  breakable,
  arc=12pt,
  boxrule=0.8pt,
  left=16pt,right=16pt,top=12pt,bottom=12pt
}

\newtcolorbox{QAPair}[1]{%
  colback=pairbg,
  colbacklower=solbg,
  colframe=border,
  coltext=text,
  title=\textcolor{gold}{\bfseries #1},
  fonttitle=\bfseries,
  coltitle=text,
  segmentation style={draw=border, dashed, line width=0.6pt},
}

% Visible text inside this box (fix)
\newtcolorbox{QuickBox}{%
  colback=pairbg,
  colframe=cyan,
  coltext=text,
  fontupper=\color{text},
  borderline north={4pt}{0pt}{cyan},
  arc=14pt,
  boxrule=0.8pt
}

% Helper for step headings
\newcommand{\Step}[1]{\textcolor{muted}{\textbf{Step #1:}}}

% ============================================================
\begin{document}
\sloppy
\setlength{\emergencystretch}{2em}

\begin{center}
{\LARGE\bfseries \textcolor{gold}{Exercise 5.2 --- Solutions}}\\[-2pt]
\end{center}

\begin{QuickBox}
{\color{cyan}\bfseries Quick method (radical equations)}\par\medskip
\begin{itemize}
\item \textbf{Domain first:} every radicand must be $\ge 0$.
\item \textbf{Isolate the radical} (keep the $\sqrt{\cdot}$ alone on one side if possible).
\item \textbf{Square both sides} to remove $\sqrt{\cdot}$ and get a linear equation.
\item \textbf{Check the answer in the original equation} (to avoid \textbf{extraneous} solutions).
\item If you get only extraneous solutions, write the solution set as $\boldsymbol{\phi}$.
\end{itemize}
\end{QuickBox}

% ============================================================
% Q1
\begin{QAPair}{Question 1}
\textcolor{gold}{\bfseries Question:}\par
\[
\sqrt{2x}=4
\]
\tcblower
\textcolor{green}{\bfseries Answer:}
\[
\begin{aligned}
\Step{1}\;& \sqrt{2x}=4\\
\Step{2}\;& (\,\sqrt{2x}\,)^2=4^2 \;\Rightarrow\; 2x=16\\
\Step{3}\;& x=8\\
\Step{4}\;& \text{Check: }\sqrt{2(8)}=\sqrt{16}=4 \;\checkmark\\
&\Rightarrow\; \boxed{\text{Solution set }=\{8\}}.
\end{aligned}
\]
\end{QAPair}

% Q2
\begin{QAPair}{Question 2}
\textcolor{gold}{\bfseries Question:}\par
\[
\sqrt{x-3}=2
\]
\tcblower
\textcolor{green}{\bfseries Answer:}
\[
\begin{aligned}
\Step{1}\;& \sqrt{x-3}=2\\
\Step{2}\;& (\,\sqrt{x-3}\,)^2=2^2 \;\Rightarrow\; x-3=4\\
\Step{3}\;& x=7\\
\Step{4}\;& \text{Check: }\sqrt{7-3}=\sqrt{4}=2 \;\checkmark\\
&\Rightarrow\; \boxed{\text{Solution set }=\{7\}}.
\end{aligned}
\]
\end{QAPair}

% Q3
\begin{QAPair}{Question 3}
\textcolor{gold}{\bfseries Question:}\par
\[
\sqrt{x-5}=3
\]
\tcblower
\textcolor{green}{\bfseries Answer:}
\[
\begin{aligned}
\Step{1}\;& \sqrt{x-5}=3\\
\Step{2}\;& x-5=9\\
\Step{3}\;& x=14\\
\Step{4}\;& \text{Check: }\sqrt{14-5}=\sqrt{9}=3 \;\checkmark\\
&\Rightarrow\; \boxed{\text{Solution set }=\{14\}}.
\end{aligned}
\]
\end{QAPair}

% Q4
\begin{QAPair}{Question 4}
\textcolor{gold}{\bfseries Question:}\par
\[
\sqrt{2x+1}=9
\]
\tcblower
\textcolor{green}{\bfseries Answer:}
\[
\begin{aligned}
\Step{1}\;& \sqrt{2x+1}=9\\
\Step{2}\;& 2x+1=81\\
\Step{3}\;& 2x=80 \;\Rightarrow\; x=40\\
\Step{4}\;& \text{Check: }\sqrt{2(40)+1}=\sqrt{81}=9 \;\checkmark\\
&\Rightarrow\; \boxed{\text{Solution set }=\{40\}}.
\end{aligned}
\]
\end{QAPair}

% Q5
\begin{QAPair}{Question 5}
\textcolor{gold}{\bfseries Question:}\par
\[
\sqrt{5x-4}=14
\]
\tcblower
\textcolor{green}{\bfseries Answer:}
\[
\begin{aligned}
\Step{1}\;& \sqrt{5x-4}=14\\
\Step{2}\;& 5x-4=196\\
\Step{3}\;& 5x=200 \;\Rightarrow\; x=40\\
\Step{4}\;& \text{Check: }\sqrt{5(40)-4}=\sqrt{196}=14 \;\checkmark\\
&\Rightarrow\; \boxed{\text{Solution set }=\{40\}}.
\end{aligned}
\]
\end{QAPair}

% Q6
\begin{QAPair}{Question 6}
\textcolor{gold}{\bfseries Question:}\par
\[
\sqrt{3x-5}=-10
\]
\tcblower
\textcolor{green}{\bfseries Answer:}
\[
\begin{aligned}
\Step{1}\;& \sqrt{3x-5}\ge 0 \text{ for all real }x\text{ in its domain.}\\
\Step{2}\;& \text{But the RHS is }-10<0,\text{ so the equation cannot be true.}\\
&\Rightarrow\; \boxed{\text{Solution set }=\phi}.
\end{aligned}
\]
\end{QAPair}

% Q7
\begin{QAPair}{Question 7}
\textcolor{gold}{\bfseries Question:}\par
\[
\sqrt{y+4}-3=2
\]
\tcblower
\textcolor{green}{\bfseries Answer:}
\[
\begin{aligned}
\Step{1}\;& \sqrt{y+4}-3=2 \;\Rightarrow\; \sqrt{y+4}=5\\
\Step{2}\;& y+4=25\\
\Step{3}\;& y=21\\
\Step{4}\;& \text{Check: }\sqrt{21+4}-3=\sqrt{25}-3=5-3=2 \;\checkmark\\
&\Rightarrow\; \boxed{\text{Solution set }=\{21\}}.
\end{aligned}
\]
\end{QAPair}

% Q8
\begin{QAPair}{Question 8}
\textcolor{gold}{\bfseries Question:}\par
\[
5-\sqrt{2x-1}=0
\]
\tcblower
\textcolor{green}{\bfseries Answer:}
\[
\begin{aligned}
\Step{1}\;& 5-\sqrt{2x-1}=0 \;\Rightarrow\; \sqrt{2x-1}=5\\
\Step{2}\;& 2x-1=25\\
\Step{3}\;& 2x=26 \;\Rightarrow\; x=13\\
\Step{4}\;& \text{Check: }5-\sqrt{2(13)-1}=5-\sqrt{25}=0 \;\checkmark\\
&\Rightarrow\; \boxed{\text{Solution set }=\{13\}}.
\end{aligned}
\]
\end{QAPair}

% Q9
\begin{QAPair}{Question 9}
\textcolor{gold}{\bfseries Question:}\par
\[
\sqrt{y+1}-12=-10
\]
\tcblower
\textcolor{green}{\bfseries Answer:}
\[
\begin{aligned}
\Step{1}\;& \sqrt{y+1}-12=-10 \;\Rightarrow\; \sqrt{y+1}=2\\
\Step{2}\;& y+1=4\\
\Step{3}\;& y=3\\
\Step{4}\;& \text{Check: }\sqrt{3+1}-12=2-12=-10 \;\checkmark\\
&\Rightarrow\; \boxed{\text{Solution set }=\{3\}}.
\end{aligned}
\]
\end{QAPair}

% Q10
\begin{QAPair}{Question 10}
\textcolor{gold}{\bfseries Question:}\par
\[
\sqrt{5t-2}=\sqrt{3t+4}
\]
\tcblower
\textcolor{green}{\bfseries Answer:}
\[
\begin{aligned}
\Step{1}\;& \sqrt{5t-2}=\sqrt{3t+4}\\
\Step{2}\;& 5t-2=3t+4\\
\Step{3}\;& 2t=6 \;\Rightarrow\; t=3\\
\Step{4}\;& \text{Check: }\sqrt{5(3)-2}=\sqrt{13}=\sqrt{3(3)+4}\;\checkmark\\
&\Rightarrow\; \boxed{\text{Solution set }=\{3\}}.
\end{aligned}
\]
\end{QAPair}

% Q11
\begin{QAPair}{Question 11}
\textcolor{gold}{\bfseries Question:}\par
\[
\sqrt{9-2x}=\sqrt{5x-12}
\]
\tcblower
\textcolor{green}{\bfseries Answer:}
\[
\begin{aligned}
\Step{1}\;& \sqrt{9-2x}=\sqrt{5x-12}\\
\Step{2}\;& 9-2x=5x-12\\
\Step{3}\;& 21=7x \;\Rightarrow\; x=3\\
\Step{4}\;& \text{Check: }\sqrt{9-6}=\sqrt{3}=\sqrt{15-12}\;\checkmark\\
&\Rightarrow\; \boxed{\text{Solution set }=\{3\}}.
\end{aligned}
\]
\end{QAPair}

% Q12
\begin{QAPair}{Question 12}
\textcolor{gold}{\bfseries Question:}\par
\[
12-\sqrt{y+1}=14
\]
\tcblower
\textcolor{green}{\bfseries Answer:}
\[
\begin{aligned}
\Step{1}\;& 12-\sqrt{y+1}=14 \;\Rightarrow\; -\sqrt{y+1}=2\\
\Step{2}\;& \sqrt{y+1}=-2 \quad \text{(impossible since } \sqrt{y+1}\ge 0)\\
&\Rightarrow\; \boxed{\text{Solution set }=\phi}.
\end{aligned}
\]
\end{QAPair}

% Q13
\begin{QAPair}{Question 13}
\textcolor{gold}{\bfseries Question:}\par
\[
4\sqrt{z}+8=40
\]
\tcblower
\textcolor{green}{\bfseries Answer:}
\[
\begin{aligned}
\Step{1}\;& 4\sqrt{z}+8=40 \;\Rightarrow\; 4\sqrt{z}=32\\
\Step{2}\;& \sqrt{z}=8\\
\Step{3}\;& z=64\\
\Step{4}\;& \text{Check: }4\sqrt{64}+8=4(8)+8=40 \;\checkmark\\
&\Rightarrow\; \boxed{\text{Solution set }=\{64\}}.
\end{aligned}
\]
\end{QAPair}

% Q14
\begin{QAPair}{Question 14}
\textcolor{gold}{\bfseries Question:}\par
\[
\sqrt{\frac{a+6}{a+2}}=\sqrt{\frac{a+2}{a-1}}
\]
\tcblower
\textcolor{green}{\bfseries Answer:}
\[
\begin{aligned}
\Step{1}\;& \sqrt{\frac{a+6}{a+2}}=\sqrt{\frac{a+2}{a-1}}\\
\Step{2}\;& \frac{a+6}{a+2}=\frac{a+2}{a-1}\\
\Step{3}\;& (a+6)(a-1)=(a+2)^2\\
\Step{4}\;& a^2+5a-6=a^2+4a+4 \;\Rightarrow\; a=10\\
\Step{5}\;& \text{Check: } \frac{10+6}{10+2}=\frac{16}{12}=\frac{4}{3},\;
\frac{10+2}{10-1}=\frac{12}{9}=\frac{4}{3}\;\checkmark\\
&\Rightarrow\; \boxed{\text{Solution set }=\{10\}}.
\end{aligned}
\]
\end{QAPair}

% Q15
\begin{QAPair}{Question 15}
\textcolor{gold}{\bfseries Question:}\par
\[
\sqrt{\frac{z}{z+3}}=\sqrt{\frac{z+2}{z+6}}
\]
\tcblower
\textcolor{green}{\bfseries Answer:}
\[
\begin{aligned}
\Step{1}\;& \sqrt{\frac{z}{z+3}}=\sqrt{\frac{z+2}{z+6}}\\
\Step{2}\;& \frac{z}{z+3}=\frac{z+2}{z+6}\\
\Step{3}\;& z(z+6)=(z+2)(z+3)\\
\Step{4}\;& z^2+6z=z^2+5z+6 \;\Rightarrow\; z=6\\
\Step{5}\;& \text{Check: } \frac{6}{9}=\frac{2}{3},\;\frac{8}{12}=\frac{2}{3}\;\checkmark\\
&\Rightarrow\; \boxed{\text{Solution set }=\{6\}}.
\end{aligned}
\]
\end{QAPair}

% Q16
\begin{QAPair}{Question 16}
\textcolor{gold}{\bfseries Question:}\par
\[
\sqrt{5x-4}=\sqrt{7x+2}
\]
\tcblower
\textcolor{green}{\bfseries Answer:}
\[
\begin{aligned}
\Step{1}\;& \sqrt{5x-4}=\sqrt{7x+2}\\
\Step{2}\;& 5x-4=7x+2\\
\Step{3}\;& -6=2x \;\Rightarrow\; x=-3\\
\Step{4}\;& \text{Check domain: }5(-3)-4=-19<0 \text{ (radicand negative)}\\
&\Rightarrow\; x=-3 \text{ is extraneous, so } \boxed{\text{Solution set }=\phi}.
\end{aligned}
\]
\end{QAPair}

\end{document}
