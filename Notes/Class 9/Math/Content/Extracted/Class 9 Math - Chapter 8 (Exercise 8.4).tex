% !TEX TS-program = pdflatex
\documentclass[11pt]{article}

% -------------------- Packages --------------------
\usepackage[a4paper,margin=1in]{geometry}
\usepackage{amsmath,amssymb}
\usepackage[T1]{fontenc}
\usepackage{lmodern}
\usepackage{xcolor}
\usepackage{tcolorbox}
\tcbuselibrary{skins,breakable}
\usepackage{enumitem}
\usepackage{hyperref}

\pagestyle{empty}

% -------------------- Dark Theme Colors --------------------
\definecolor{bg}{HTML}{000000}
\definecolor{pairbg}{HTML}{121212}
\definecolor{solbg}{HTML}{0A0A0A}
\definecolor{border}{HTML}{2A2A2A}
\definecolor{text}{HTML}{FFFFFF}
\definecolor{muted}{HTML}{C9CDD3}
\definecolor{gold}{HTML}{FFD700}
\definecolor{green}{HTML}{4ADE80}
\definecolor{cyan}{HTML}{38BDF8}

\pagecolor{bg}
\color{text}

\hypersetup{
  colorlinks=true,
  linkcolor=cyan,
  urlcolor=cyan
}

\setlength{\parindent}{0pt}
\setlength{\parskip}{10pt}

\setlist[itemize]{left=1.4em,itemsep=6pt,topsep=6pt}
\setlist[enumerate]{left=1.6em,itemsep=4pt,topsep=4pt}

% -------------------- tcolorbox Base --------------------
\tcbset{
  enhanced,
  breakable,
  arc=12pt,
  boxrule=0.8pt,
  left=16pt,right=16pt,top=12pt,bottom=12pt
}

\newtcolorbox{QAPair}[1]{%
  colback=pairbg,
  colbacklower=solbg,
  colframe=border,
  coltext=text,
  title=\textcolor{gold}{\bfseries #1},
  fonttitle=\bfseries,
  coltitle=text,
  segmentation style={draw=border, dashed, line width=0.6pt},
}

% Visible text inside this box (fix)
\newtcolorbox{QuickBox}{%
  colback=pairbg,
  colframe=cyan,
  coltext=text,
  fontupper=\color{text},
  borderline north={4pt}{0pt}{cyan},
  arc=14pt,
  boxrule=0.8pt
}

% Helper for step headings
\newcommand{\Step}[1]{\textcolor{muted}{\textbf{Step #1:}}}

% ============================================================
\begin{document}

\begin{center}
{\LARGE\bfseries \textcolor{gold}{Exercise 8.4 --- Solutions}}\\[-2pt]
\end{center}

\begin{QuickBox}
{\color{cyan}\bfseries Quick formulas (useful)}\par\medskip
\begin{itemize}
\item \textbf{Slope:} $m=\dfrac{y_2-y_1}{x_2-x_1}$.
\item \textbf{Slope--intercept form:} $y=mx+c$.
\item \textbf{Point--slope form:} $y-y_1=m(x-x_1)$.
\item \textbf{Standard form:} $Ax+By=C$.
\item \textbf{Using a rate:} $\text{Total}=\text{fixed amount} + (\text{rate})\times(\text{quantity})$.
\end{itemize}
\end{QuickBox}

% ============================================================
% Q1
\begin{QAPair}{Question 1}
\textcolor{gold}{\bfseries Question:}
Nasir sold ``fruiters'' @ Rs.\ 120 per dozen and ``Shakri Malta'' @ Rs.\ 150 per dozen and earned Rs.\ 1200.
Write an equation in standard form that describes the situation.
If he sold 4 dozen of ``Shakri Malta'', how many dozens of ``fruiters'' did he sell?
\tcblower
\textcolor{green}{\bfseries Answer:}
Let $x$ = dozens of \emph{fruiters}, $y$ = dozens of \emph{Shakri Malta}.
\[
\begin{aligned}
\Step{1}\;& \text{Total earning} = 120x+150y.\\
\Step{2}\;& 120x+150y=1200 \quad (\text{standard form } Ax+By=C).\\
\Step{3}\;& \text{(Optional simplification)}\; \div 30:\; 4x+5y=40.\\[4pt]
\Step{4}\;& \text{If } y=4:\; 120x+150(4)=1200\\
&\Rightarrow 120x+600=1200\\
&\Rightarrow 120x=600\\
&\Rightarrow x=5.
\end{aligned}
\]
\[
\boxed{\text{He sold } 5 \text{ dozen fruiters.}}
\]
\end{QAPair}

% ============================================================
% Q2
\begin{QAPair}{Question 2 (i)}
\textcolor{gold}{\bfseries Question:}
The linear equation $y=1450x+2000$ gives the total hotel cost for one day.
Find the cost if a group of $7$ people stays for one day.
\tcblower
\textcolor{green}{\bfseries Answer:}
\[
\begin{aligned}
\Step{1}\;& x=7.\\
\Step{2}\;& y=1450(7)+2000=10150+2000=12150.
\end{aligned}
\]
\[
\boxed{y=\text{Rs.\ }12{,}150}
\]
\end{QAPair}

\begin{QAPair}{Question 2 (ii)}
\textcolor{gold}{\bfseries Question:}
How many people can stay for Rs.\ 13,600 for one day?
\tcblower
\textcolor{green}{\bfseries Answer:}
\[
\begin{aligned}
\Step{1}\;& 13600=1450x+2000\\
\Step{2}\;& 1450x=13600-2000=11600\\
\Step{3}\;& x=\frac{11600}{1450}=8.
\end{aligned}
\]
\[
\boxed{8 \text{ people}}
\]
\end{QAPair}

% ============================================================
% Q3
\begin{QAPair}{Question 3}
\textcolor{gold}{\bfseries Question:}
One company provides Rs.\ 5500 per week along with an extra bonus of Rs.\ 700.
The other offers Rs.\ 800 per day.
Convert into linear equations and tell which is better for two weeks.
\tcblower
\textcolor{green}{\bfseries Answer:}
Let $x$ = number of weeks, $y$ = total earning (in Rs.).
\[
\begin{aligned}
\Step{1}\;& \text{Company A: } y_A=5500x+700.\\
\Step{2}\;& \text{Company B: } 800 \text{ per day }=800(7)=5600 \text{ per week}\\
&\Rightarrow y_B=5600x.\\[4pt]
\Step{3}\;& \text{For } x=2:\\
&y_A=5500(2)+700=11000+700=11700,\\
&y_B=5600(2)=11200.
\end{aligned}
\]
\[
\boxed{\text{Company A is better for two weeks (Rs.\ 11,700 vs Rs.\ 11,200).}}
\]
\end{QAPair}

% ============================================================
% Q4
\begin{QAPair}{Question 4 (i)}
\textcolor{gold}{\bfseries Question:}
A man earns Rs.\ 120 per hour and has Rs.\ 500 initially.
Write a linear equation and find how much he will have after 12 hours.
\tcblower
\textcolor{green}{\bfseries Answer:}
Let $x$ = hours worked, $y$ = total money (Rs.).
\[
\begin{aligned}
\Step{1}\;& y=120x+500.\\
\Step{2}\;& \text{After } 12 \text{ hours: } y=120(12)+500=1440+500=1940.
\end{aligned}
\]
\[
\boxed{y=\text{Rs.\ }1{,}940}
\]
\end{QAPair}

\begin{QAPair}{Question 4 (ii)}
\textcolor{gold}{\bfseries Question:} What does slope show in this situation?
\tcblower
\textcolor{green}{\bfseries Answer:}
In $y=120x+500$, the slope is $120$.
\[
\boxed{\text{Slope }=120 \text{ means he earns Rs.\ 120 per hour (rate of earning).}}
\]
\end{QAPair}

\begin{QAPair}{Question 4 (iii)}
\textcolor{gold}{\bfseries Question:} What does $y$-intercept represent here?
\tcblower
\textcolor{green}{\bfseries Answer:}
The $y$-intercept is $500$ (value of $y$ when $x=0$).
\[
\boxed{\text{$y$-intercept }=500 \text{ means he already has Rs.\ 500 before working.}}
\]
\end{QAPair}

% ============================================================
% Q5
\begin{QAPair}{Question 5 (i)}
\textcolor{gold}{\bfseries Question:}
Ali shifted on 1st September. Meter reading was 44 units. Average use is 18 units/day.
Represent the situation through a linear equation.
\tcblower
\textcolor{green}{\bfseries Answer:}
Let $x$ = number of days after 1st September, $y$ = meter reading (units).
\[
\begin{aligned}
\Step{1}\;& \text{Initial reading }=44 \text{ at } x=0.\\
\Step{2}\;& \text{Daily increase }=18 \Rightarrow \text{slope}=18.\\
\Step{3}\;& y=18x+44.
\end{aligned}
\]
\[
\boxed{y=18x+44}
\]
\end{QAPair}

\begin{QAPair}{Question 5 (ii)}
\textcolor{gold}{\bfseries Question:} How many units are consumed till 30 September?
\tcblower
\textcolor{green}{\bfseries Answer:}
From 1st to 30th September is $30$ days of usage.
\[
\begin{aligned}
\Step{1}\;& \text{Units consumed}=18\times 30=540.
\end{aligned}
\]
\[
\boxed{540 \text{ units}}
\]
\end{QAPair}

\begin{QAPair}{Question 5 (iii)}
\textcolor{gold}{\bfseries Question:} What will be the bill after one month @ Rs.\ 20 per unit?
\tcblower
\textcolor{green}{\bfseries Answer:}
\[
\begin{aligned}
\Step{1}\;& \text{Monthly consumption}=540 \text{ units}.\\
\Step{2}\;& \text{Bill}=540\times 20=10800.
\end{aligned}
\]
\[
\boxed{\text{Bill}=\text{Rs.\ }10{,}800}
\]
\end{QAPair}

\begin{QAPair}{Question 5 (iv)}
\textcolor{gold}{\bfseries Question:} After how many days, the meter shows 404 units?
\tcblower
\textcolor{green}{\bfseries Answer:}
Using $y=18x+44$,
\[
\begin{aligned}
\Step{1}\;& 404=18x+44\\
\Step{2}\;& 18x=404-44=360\\
\Step{3}\;& x=\frac{360}{18}=20.
\end{aligned}
\]
\[
\boxed{20 \text{ days}}
\]
\end{QAPair}

% ============================================================
% Q6
\begin{QAPair}{Question 6 (i)}
\textcolor{gold}{\bfseries Question:}
Alia hired a taxi with fixed charge Rs.\ 1500 plus Rs.\ 450 per 30 minutes.
Represent the relation as a linear equation.
\tcblower
\textcolor{green}{\bfseries Answer:}
Let $x$ = time in hours, $y$ = total fare (Rs.).
\[
\begin{aligned}
\Step{1}\;& 450 \text{ per 30 min} = 450 \text{ per } \tfrac{1}{2}\text{ hour}\\
&\Rightarrow 900 \text{ per hour}.\\
\Step{2}\;& y=900x+1500.
\end{aligned}
\]
\[
\boxed{y=900x+1500}
\]
\end{QAPair}

\begin{QAPair}{Question 6 (ii)}
\textcolor{gold}{\bfseries Question:} What will be the taxi fare after 5 hours?
\tcblower
\textcolor{green}{\bfseries Answer:}
\[
\begin{aligned}
\Step{1}\;& x=5\\
\Step{2}\;& y=900(5)+1500=4500+1500=6000.
\end{aligned}
\]
\[
\boxed{\text{Rs.\ }6{,}000}
\]
\end{QAPair}

\begin{QAPair}{Question 6 (iii)}
\textcolor{gold}{\bfseries Question:} What is the slope of the equation in this case?
\tcblower
\textcolor{green}{\bfseries Answer:}
In $y=900x+1500$, the slope is $900$.
\[
\boxed{\text{Slope }=900 \text{ means Rs.\ 900 per hour (rate of fare).}}
\]
\end{QAPair}

% ============================================================
% Q7
\begin{QAPair}{Question 7 (i)}
\textcolor{gold}{\bfseries Question:}
Derive the relation between Fahrenheit and Celsius scales in slope-intercept form.
\tcblower
\textcolor{green}{\bfseries Answer:}
Let $C$ be Celsius and $F$ be Fahrenheit.
\[
\begin{aligned}
\Step{1}\;& 0^\circ C = 32^\circ F,\quad 100^\circ C = 212^\circ F.\\
\Step{2}\;& m=\frac{212-32}{100-0}=\frac{180}{100}=\frac{9}{5}.\\
\Step{3}\;& F = \frac{9}{5}C + 32.
\end{aligned}
\]
\[
\boxed{F=\frac{9}{5}C+32}
\]
\end{QAPair}

\begin{QAPair}{Question 7 (ii)}
\textcolor{gold}{\bfseries Question:} What do $y$-intercept and slope show in the equation?
\tcblower
\textcolor{green}{\bfseries Answer:}
For $F=\frac{9}{5}C+32$:
\[
\boxed{\text{Slope }=\frac{9}{5} \text{ means } 1^\circ C \text{ increase gives } 1.8^\circ F \text{ increase.}}
\]
\[
\boxed{\text{$y$-intercept }=32 \text{ means when } C=0^\circ,\; F=32^\circ.}
\]
\end{QAPair}

\begin{QAPair}{Question 7 (iii)}
\textcolor{gold}{\bfseries Question:} What is Fahrenheit temperature when Celsius is $5^\circ C$?
\tcblower
\textcolor{green}{\bfseries Answer:}
\[
\begin{aligned}
\Step{1}\;& F=\frac{9}{5}(5)+32\\
\Step{2}\;& F=9+32=41.
\end{aligned}
\]
\[
\boxed{41^\circ F}
\]
\end{QAPair}

% ============================================================
% Q8
\begin{QAPair}{Question 8 (i)}
\textcolor{gold}{\bfseries Question:}
A cricket team scores 96 runs in 16 overs and 180 runs in 30 overs.
Write an equation of the line for this situation.
\tcblower
\textcolor{green}{\bfseries Answer:}
Let $x$ = overs, $y$ = runs. Points: $(16,96)$ and $(30,180)$.
\[
\begin{aligned}
\Step{1}\;& m=\frac{180-96}{30-16}=\frac{84}{14}=6.\\
\Step{2}\;& y=mx+b \Rightarrow 96=6(16)+b \Rightarrow 96=96+b \Rightarrow b=0.\\
\Step{3}\;& y=6x.
\end{aligned}
\]
\[
\boxed{y=6x}
\]
\end{QAPair}

\begin{QAPair}{Question 8 (ii)}
\textcolor{gold}{\bfseries Question:} What does the gradient mean in terms of scores?
\tcblower
\textcolor{green}{\bfseries Answer:}
\[
\boxed{\text{Gradient }=6 \text{ means the team scores } 6 \text{ runs per over (run rate).}}
\]
\end{QAPair}

\begin{QAPair}{Question 8 (iii)}
\textcolor{gold}{\bfseries Question:} What does the $y$-intercept mean in terms of scores?
\tcblower
\textcolor{green}{\bfseries Answer:}
In $y=6x$, the $y$-intercept is $0$.
\[
\boxed{\text{$y$-intercept }=0 \text{ means at } 0 \text{ overs, predicted score is } 0 \text{ runs.}}
\]
\end{QAPair}

\begin{QAPair}{Question 8 (iv)}
\textcolor{gold}{\bfseries Question:} What will be the predicted score after 45 overs?
\tcblower
\textcolor{green}{\bfseries Answer:}
\[
\begin{aligned}
\Step{1}\;& y=6x,\; x=45\\
\Step{2}\;& y=6(45)=270.
\end{aligned}
\]
\[
\boxed{270 \text{ runs}}
\]
\end{QAPair}

\begin{QAPair}{Question 8 (v)}
\textcolor{gold}{\bfseries Question:} After how many overs will the predicted score be 240?
\tcblower
\textcolor{green}{\bfseries Answer:}
\[
\begin{aligned}
\Step{1}\;& 240=6x\\
\Step{2}\;& x=\frac{240}{6}=40.
\end{aligned}
\]
\[
\boxed{40 \text{ overs}}
\]
\end{QAPair}

% ============================================================
% Q9
\begin{QAPair}{Question 9 (i)}
\textcolor{gold}{\bfseries Question:}
A truck company charges Rs.\ 5000 per day plus some money per kilometre.
Abdullah drove 125 km and paid Rs.\ 30,000.
Write an equation in point-slope form for this situation.
\tcblower
\textcolor{green}{\bfseries Answer:}
Let $x$ = kilometres driven, $y$ = total cost (Rs.).
We know one point is $(125,30000)$ and the fixed cost gives $(0,5000)$.
\[
\begin{aligned}
\Step{1}\;& m=\frac{30000-5000}{125-0}=\frac{25000}{125}=200.\\
\Step{2}\;& \text{Point-slope form (using }(125,30000)\text{):}\\
& y-30000=200(x-125).
\end{aligned}
\]
\[
\boxed{y-30000=200(x-125)}
\]
\end{QAPair}

\begin{QAPair}{Question 9 (ii)}
\textcolor{gold}{\bfseries Question:}
Find the amount per kilometre the company charges and relate it with slope.
\tcblower
\textcolor{green}{\bfseries Answer:}
From above, $m=200$.
\[
\boxed{\text{The company charges Rs.\ }200 \text{ per km, and this is the slope of the line.}}
\]
\end{QAPair}

\begin{QAPair}{Question 9 (iii)}
\textcolor{gold}{\bfseries Question:} How much would it cost if Abdullah drove 180 km?
\tcblower
\textcolor{green}{\bfseries Answer:}
Using $y=5000+200x$:
\[
\begin{aligned}
\Step{1}\;& x=180\\
\Step{2}\;& y=5000+200(180)=5000+36000=41000.
\end{aligned}
\]
\[
\boxed{\text{Rs.\ }41{,}000}
\]
\end{QAPair}

% ============================================================
% Q10
\begin{QAPair}{Question 10}
\textcolor{gold}{\bfseries Question:}
A ship starts from Karachi at latitude $25^\circ$N and longitude $67^\circ$E and reaches latitude $32^\circ$N and longitude $54^\circ$E.
Derive the equation of the line in point-slope form.
If the ship moves to latitude $39^\circ$N, what is the longitude?
\tcblower
\textcolor{green}{\bfseries Answer:}
Take longitude as $x$ and latitude as $y$.
Points: $(67,25)$ and $(54,32)$.
\[
\begin{aligned}
\Step{1}\;& m=\frac{32-25}{54-67}=\frac{7}{-13}=-\frac{7}{13}.\\
\Step{2}\;& \text{Point-slope (using Karachi }(67,25)\text{):}\\
& y-25=-\frac{7}{13}(x-67).\\[4pt]
\Step{3}\;& \text{For latitude } y=39:\\
& 39-25=-\frac{7}{13}(x-67)\\
& 14=-\frac{7}{13}(x-67)\\
& 14\cdot 13=-7(x-67)\\
& 182=-7x+469\\
& -7x=-287\\
& x=41.
\end{aligned}
\]
\[
\boxed{y-25=-\frac{7}{13}(x-67)},\qquad
\boxed{\text{Longitude }=41^\circ \text{E when latitude }=39^\circ \text{N}.}
\]
\end{QAPair}

% ============================================================
% Q11
\begin{QAPair}{Question 11 (i)}
\textcolor{gold}{\bfseries Question:}
Length and width of a plot are in the ratio $2:1$.
Write the equation of the line and find the length if the width is 30 feet.
\tcblower
\textcolor{green}{\bfseries Answer:}
Let $x$ = width (ft), $y$ = length (ft).
\[
\begin{aligned}
\Step{1}\;& \frac{y}{x}=\frac{2}{1}\Rightarrow y=2x.\\
\Step{2}\;& \text{If } x=30:\; y=2(30)=60.
\end{aligned}
\]
\[
\boxed{y=2x},\qquad \boxed{\text{Length}=60\text{ ft}}
\]
\end{QAPair}

\begin{QAPair}{Question 11 (ii)}
\textcolor{gold}{\bfseries Question:} What is slope in this case and what does it mean?
\tcblower
\textcolor{green}{\bfseries Answer:}
In $y=2x$, the slope is $2$.
\[
\boxed{\text{Slope }=2 \text{ means length is } 2 \text{ times the width (2 ft length per 1 ft width).}}
\]
\end{QAPair}

\end{document}
