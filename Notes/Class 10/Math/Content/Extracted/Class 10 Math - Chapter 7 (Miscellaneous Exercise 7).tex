% !TEX TS-program = pdflatex
\documentclass[11pt]{article}

% -------------------- Packages --------------------
\usepackage[a4paper,margin=1in]{geometry}
\usepackage{amsmath,amssymb}
\usepackage[T1]{fontenc}
\usepackage{lmodern}
\usepackage{xcolor}
\usepackage{tcolorbox}
\tcbuselibrary{skins,breakable}
\usepackage{enumitem}
\usepackage{hyperref}
\usepackage{tikz}
\usetikzlibrary{calc,patterns,angles,quotes,intersections}

\pagestyle{empty}

% -------------------- Dark Theme Colors --------------------
\definecolor{bg}{HTML}{000000}
\definecolor{pairbg}{HTML}{121212}
\definecolor{solbg}{HTML}{0A0A0A}
\definecolor{border}{HTML}{2A2A2A}
\definecolor{text}{HTML}{FFFFFF}
\definecolor{muted}{HTML}{C9CDD3}
\definecolor{gold}{HTML}{FFD700}
\definecolor{green}{HTML}{4ADE80}
\definecolor{cyan}{HTML}{38BDF8}

\pagecolor{bg}
\color{text}

\hypersetup{
  colorlinks=true,
  linkcolor=cyan,
  urlcolor=cyan
}

\setlength{\parindent}{0pt}
\setlength{\parskip}{10pt}

\setlist[itemize]{left=1.4em,itemsep=6pt,topsep=6pt}
\setlist[enumerate]{left=1.6em,itemsep=4pt,topsep=4pt}

% -------------------- tcolorbox Base --------------------
\tcbset{
  enhanced,
  breakable,
  arc=12pt,
  boxrule=0.8pt,
  left=16pt,right=16pt,top=12pt,bottom=12pt
}

\newtcolorbox{QAPair}[1]{%
  colback=pairbg,
  colbacklower=solbg,
  colframe=border,
  coltext=text,
  title=\textcolor{gold}{\bfseries #1},
  fonttitle=\bfseries,
  coltitle=text,
  segmentation style={draw=border, dashed, line width=0.6pt},
}

\newtcolorbox{QuickBox}{%
  colback=pairbg,
  colframe=cyan,
  coltext=text,
  fontupper=\color{text},
  borderline north={4pt}{0pt}{cyan},
  arc=14pt,
  boxrule=0.8pt
}

% Helper for step headings
\newcommand{\Step}[1]{\textcolor{muted}{\textbf{Step #1:}}}

% -------------------- TikZ Styles --------------------
\tikzset{
  geom/.style={draw=muted, line width=0.95pt},
  strong/.style={draw=cyan, line width=1.05pt, -latex},
  helper/.style={draw=muted, dashed, line width=0.75pt},
  pt/.style={circle, fill=cyan, inner sep=1.2pt},
  lab/.style={text=text, font=\small},
  ang/.style={draw=cyan, line width=0.9pt},
  note/.style={text=muted, font=\small}
}

% -------------------- Step + Diagram Macro --------------------
% Usage:
% \StepFig{1}{<text>}{<tikzpicture contents ONLY>}
\newcommand{\StepFig}[3]{%
  \Step{#1} #2\par\medskip
  \begin{center}
    \begin{tikzpicture}[scale=0.92]
      #3
    \end{tikzpicture}
  \end{center}
  \vspace{-2pt}
}

% tiny right-angle mark macro
\newcommand{\RightAngleMark}[2]{%
  % #1 = corner point, #2 = size
  \draw[ang] ($(#1)+(#2,0)$) -- ($(#1)+(#2,#2)$) -- ($(#1)+(0,#2)$);
}

% Axis helper (no arrows; clean)
\newcommand{\Axes}{%
  \draw[helper] (-0.2,0) -- (5.2,0);
  \draw[helper] (0,-0.2) -- (0,4.2);
  \node[note] at (5.1,-0.25) {$x$};
  \node[note] at (-0.25,4.1) {$y$};
}

% ============================================================
\begin{document}

\begin{center}
{\LARGE\bfseries \textcolor{gold}{Miscellaneous Exercise 7 --- Solutions}}\\[-2pt]
\end{center}

\begin{QuickBox}
{\color{cyan}\bfseries Quick facts (Vectors)}\par\medskip
\begin{itemize}
\item \textbf{Scalar vs vector:} scalar has \emph{magnitude only} (e.g.\ speed); vector has \emph{magnitude + direction} (e.g.\ velocity, force).
\item \textbf{Position vectors:} if $\overrightarrow{OA}=\vec a$ and $\overrightarrow{OB}=\vec b$, then $\overrightarrow{AB}=\vec b-\vec a$.
\item \textbf{Magnitude:} for $\vec u = xi+yj$, \; $|\vec u|=\sqrt{x^2+y^2}$.
\item \textbf{Unit vector:} $\hat u=\dfrac{\vec u}{|\vec u|}$ (same direction, magnitude $1$).
\item \textbf{Resultant of forces:} add components: $R_x=\sum F\cos\theta$, $R_y=\sum F\sin\theta$, \;
$R=\sqrt{R_x^2+R_y^2}$, \; $\tan\phi=\dfrac{R_y}{R_x}$.
\end{itemize}
\end{QuickBox}

% ============================================================
% Q1 MCQs (i) to (xiv)

\begin{QAPair}{Question 1 (i) --- MCQ}
\textcolor{gold}{\bfseries Question:} Which of the following is a scalar quantity?\par
(a) velocity \quad (b) speed \quad (c) torque \quad (d) force
\tcblower
\textcolor{green}{\bfseries Answer:} \textbf{(b) speed}\par
\StepFig{1}{Speed is only the \emph{magnitude} of velocity, so it is a scalar.}{%
  \Axes
  \coordinate (O) at (0,0);
  \coordinate (P) at (4.4,2.2);
  \draw[strong] (O) -- (P);
  \fill[pt] (O) node[lab, below left] {$O$};
  \fill[pt] (P) node[lab, above right] {$P$};
  \node[lab] at (2.3,1.35) {$\vec v$ (velocity)};
  \node[note] at (3.9,0.55) {speed $=|\vec v|$ (scalar)};
}
\end{QAPair}

\begin{QAPair}{Question 1 (ii) --- MCQ}
\textcolor{gold}{\bfseries Question:} Which of the following is a vector quantity?\par
(a) velocity \quad (b) speed \quad (c) distance \quad (d) work
\tcblower
\textcolor{green}{\bfseries Answer:} \textbf{(a) velocity}\par
\StepFig{1}{Velocity needs \emph{direction} and magnitude, so it is a vector.}{%
  \Axes
  \coordinate (O) at (0,0);
  \coordinate (V) at (4.2,3.0);
  \draw[strong] (O)--(V);
  \fill[pt] (O) node[lab, below left] {$O$};
  \fill[pt] (V) node[lab, above right] {$\vec v$};
  \node[note] at (2.2,3.6) {direction matters};
}
\end{QAPair}

\begin{QAPair}{Question 1 (iii) --- MCQ}
\textcolor{gold}{\bfseries Question:} If $\vec a$ and $\vec b$ are position vectors of points $A$ and $B$ respectively, then $\overrightarrow{AB}$ is:\par
(a) $\vec b+\vec a$ \quad (b) $\vec a-\vec b$ \quad (c) $\vec b-\vec a$ \quad (d) $-\vec b-\vec a$
\tcblower
\textcolor{green}{\bfseries Answer:} \textbf{(c) $\vec b-\vec a$}\par
\StepFig{1}{\;$\overrightarrow{AB}=\overrightarrow{OB}-\overrightarrow{OA}=\vec b-\vec a$.}{%
  \Axes
  \coordinate (O) at (0,0);
  \coordinate (A) at (1.4,2.8);
  \coordinate (B) at (4.4,1.2);

  \draw[strong] (O)--(A);
  \draw[strong] (O)--(B);
  \draw[geom, -latex, line width=1.05pt] (A)--(B);

  \fill[pt] (O) node[lab, below left] {$O$};
  \fill[pt] (A) node[lab, above left] {$A$};
  \fill[pt] (B) node[lab, above right] {$B$};

  \node[lab] at (0.9,1.6) {$\vec a$};
  \node[lab] at (2.8,0.8) {$\vec b$};
  \node[note] at (3.2,2.35) {$\overrightarrow{AB}=\vec b-\vec a$};
}
\end{QAPair}

\begin{QAPair}{Question 1 (iv) --- MCQ}
\textcolor{gold}{\bfseries Question:} Which of the following is not a symbol of vector $\vec a$?\par
(a) $\vec a$ \quad (b) $\underline{a}$ \quad (c) $a$ \quad (d) $|\vec a|$
\tcblower
\textcolor{green}{\bfseries Answer:} \textbf{(d) $|\vec a|$}\par
\StepFig{1}{$|\vec a|$ means the \emph{magnitude} (a scalar), not the vector itself.}{%
  \Axes
  \coordinate (O) at (0,0);
  \coordinate (A) at (3.8,2.4);
  \draw[strong] (O)--(A);
  \draw[helper] (0,0) -- (3.8,0) -- (3.8,2.4);
  \fill[pt] (O) node[lab, below left] {$O$};
  \fill[pt] (A) node[lab, above right] {};
  \node[lab] at (2.1,1.45) {$\vec a$};
  \node[note] at (4.2,2.9) {$|\vec a|$ is length};
}
\end{QAPair}

\begin{QAPair}{Question 1 (v) --- MCQ}
\textcolor{gold}{\bfseries Question:} If $\overrightarrow{OP}=[-6,\,7]$, then $-\overrightarrow{OP}$ is equal to:\par
(a) $[-6,\,7]$ \quad (b) $[6,\,7]$ \quad (c) $[6,\,-7]$ \quad (d) $[-6,\,-7]$
\tcblower
\textcolor{green}{\bfseries Answer:} \textbf{(c) $[6,\,-7]$}\par
\StepFig{1}{Negating a vector reverses direction: $-\,[x,y]=[-x,-y]$.}{%
  \coordinate (O) at (0,0);
  \coordinate (P) at (-2.6,3.0);
  \coordinate (Q) at (2.6,-3.0);

  \draw[helper] (-3.2,0)--(3.2,0);
  \draw[helper] (0,-3.4)--(0,3.4);

  \draw[strong] (O)--(P);
  \draw[strong] (O)--(Q);

  \fill[pt] (O) node[lab, below left] {$O$};
  \fill[pt] (P) node[lab, above left] {$P$};
  \fill[pt] (Q) node[lab, below right] {$Q$};

  \node[note] at (-2.1,2.2) {$\overrightarrow{OP}=[-6,7]$};
  \node[note] at (2.1,-2.2) {$-\overrightarrow{OP}=[6,-7]$};
}
\end{QAPair}

\begin{QAPair}{Question 1 (vi) --- MCQ}
\textcolor{gold}{\bfseries Question:} If $\vec u=-5\hat{i}+12\hat{j}$, then $|\vec u|$ is equal to:\par
(a) $17$ \quad (b) $7$ \quad (c) $169$ \quad (d) $13$
\tcblower
\textcolor{green}{\bfseries Answer:} \textbf{(d) $13$}\par
\StepFig{1}{\;$|\vec u|=\sqrt{(-5)^2+12^2}=\sqrt{25+144}=13$.}{%
  \coordinate (O) at (0,0);
  \coordinate (A) at (4.0,0);
  \coordinate (B) at (4.0,3.2);

  \draw[helper] (-0.2,0)--(5.2,0);
  \draw[helper] (0,-0.2)--(0,4.2);

  \draw[geom] (O)--(A)--(B)--cycle;
  \RightAngleMark{A}{0.18}

  \draw[strong] (O)--(B);

  \fill[pt] (O) node[lab, below left] {$O$};
  \fill[pt] (B) node[lab, above right] {$\vec u$};

  \node[note] at (2.0,-0.55) {$5$};
  \node[note] at (4.55,1.6) {$12$};
  \node[note] at (2.6,2.1) {$|\vec u|=13$};
}
\end{QAPair}

\begin{QAPair}{Question 1 (vii) --- MCQ}
\textcolor{gold}{\bfseries Question:} Given that $\vec u$ is any vector. Which of the following is true?\par
(a) $|-\vec u|=|\vec u|$ \quad (b) $-|\vec u|=|\vec u|$ \quad (c) $|\vec u|+|-\vec u|=0$ \quad (d) $|\vec u|=0$
\tcblower
\textcolor{green}{\bfseries Answer:} \textbf{(a) $|-\vec u|=|\vec u|$}\par
\StepFig{1}{A vector and its negative have the same length (only direction changes).}{%
  \Axes
  \coordinate (O) at (0,0);
  \coordinate (U) at (4.0,2.2);
  \coordinate (V) at (-4.0,-2.2);

  \draw[strong] (O)--(U);
  \draw[strong] (O)--(V);

  \fill[pt] (O) node[lab, below left] {$O$};
  \fill[pt] (U) node[lab, above right] {$\vec u$};
  \fill[pt] (V) node[lab, below left] {$-\vec u$};

  \node[note] at (2.7,3.6) {$|\vec u|=|-\vec u|$};
}
\end{QAPair}

\begin{QAPair}{Question 1 (viii) --- MCQ}
\textcolor{gold}{\bfseries Question:} The unit vector of the vector $\vec u=6\hat{i}+10\hat{j}-2\hat{j}$ is:\par
(a) $\frac{3\hat{i}}{5}-\frac{4\hat{j}}{5}$ \quad
(b) $\frac{3\hat{i}}{5}+\frac{4\hat{j}}{5}$ \quad
(c) $-\frac{3\hat{i}}{5}-\frac{4\hat{j}}{5}$ \quad
(d) $-\frac{3\hat{i}}{5}+\frac{4\hat{j}}{5}$
\tcblower
\textcolor{green}{\bfseries Answer:} \textbf{(b) $\frac{3\hat{i}}{5}+\frac{4\hat{j}}{5}$}\par
\StepFig{1}{First simplify: $\vec u=6\hat i+(10-2)\hat j=6\hat i+8\hat j$, \;$|\vec u|=10$, so $\hat u=\frac{1}{10}(6,8)=\left(\frac35,\frac45\right)$.}{%
  \Axes
  \coordinate (O) at (0,0);
  \coordinate (U) at (4.2,2.8); % scaled drawing
  \coordinate (Uh) at (2.1,1.4);

  \draw[strong] (O)--(U);
  \draw[geom, -latex, line width=1.05pt] (O)--(Uh);

  \fill[pt] (O) node[lab, below left] {$O$};

  \node[lab] at (3.3,2.2) {$\vec u$};
  \node[lab] at (1.3,1.0) {$\hat u$};

  \node[note] at (4.2,3.6) {$\hat u=\dfrac{\vec u}{|\vec u|}$};
}
\end{QAPair}

\begin{QAPair}{Question 1 (ix) --- MCQ}
\textcolor{gold}{\bfseries Question:} If $\vec a=\lambda \vec b$, $\vec a=12\hat{i}-18\hat{j}$ and $\vec b=-2\hat{i}+3\hat{j}$, then $\lambda$ is equal to:\par
(a) $3$ \quad (b) $-3$ \quad (c) $6$ \quad (d) $-6$
\tcblower
\textcolor{green}{\bfseries Answer:} \textbf{(d) $-6$}\par
\StepFig{1}{Match components: $12=\lambda(-2)\Rightarrow \lambda=-6$ (and $-18=\lambda(3)$ also gives $\lambda=-6$).}{%
  \Axes
  \coordinate (O) at (0,0);
  \coordinate (b) at (1.2, -1.8);
  \coordinate (a) at (-3.6, 5.4);

  \draw[geom, -latex, line width=1.05pt] (O)--(b);
  \draw[strong] (O)--(a);

  \fill[pt] (O) node[lab, below left] {$O$};
  \node[lab] at (0.9,-1.1) {$\vec b$};
  \node[lab] at (-2.1,3.8) {$\vec a=-6\vec b$};
}
\end{QAPair}

\begin{QAPair}{Question 1 (x) --- MCQ}
\textcolor{gold}{\bfseries Question:} If $\vec u=[-5x,\,8]$ and $\vec v=[10,\,4y]$ are equal vectors, then:\par
(a) $x=2,y=2$ \quad (b) $x=-2,y=-2$ \quad (c) $x=-2,y=2$ \quad (d) $x=2,y=-2$
\tcblower
\textcolor{green}{\bfseries Answer:} \textbf{(c) $x=-2,\;y=2$}\par
\StepFig{1}{Equal vectors have equal components: $-5x=10\Rightarrow x=-2$ and $8=4y\Rightarrow y=2$.}{%
  \Axes
  \coordinate (O) at (0,0);
  \coordinate (U) at (4.0,3.2);
  \draw[strong] (O)--(U);
  \fill[pt] (O) node[lab, below left] {$O$};
  \fill[pt] (U) node[lab, above right] {};
  \node[lab] at (2.4,2.2) {$\vec u=\vec v$};
  \node[note] at (3.8,0.6) {match components};
}
\end{QAPair}

\begin{QAPair}{Question 1 (xi) --- MCQ}
\textcolor{gold}{\bfseries Question:} If $\vec p=[5,\,-6]$ and $\vec q=[2,\,6]$, then $\vec p-2\vec q$ is:\par
(a) $[1,\,-18]$ \quad (b) $[9,\,18]$ \quad (c) $[1,\,18]$ \quad (d) $[-9,\,-18]$
\tcblower
\textcolor{green}{\bfseries Answer:} \textbf{(a) $[1,\,-18]$}\par
\StepFig{1}{\;$\vec p-2\vec q=[5,-6]-[4,12]=[1,-18]$.}{%
  \Axes
  \coordinate (O) at (0,0);
  \coordinate (P) at (2.2,1.6);
  \coordinate (Q) at (1.2,3.0);
  \coordinate (R) at (0.4,-3.4);

  \draw[geom, -latex, line width=1.05pt] (O)--(P);
  \draw[geom, -latex, line width=1.05pt] (O)--(Q);
  \draw[strong] (O)--(R);

  \fill[pt] (O) node[lab, below left] {$O$};
  \node[lab] at (1.6,1.0) {$\vec p$};
  \node[lab] at (0.9,2.3) {$\vec q$};
  \node[lab] at (0.8,-2.5) {$\vec p-2\vec q$};
}
\end{QAPair}

\begin{QAPair}{Question 1 (xii) --- MCQ}
\textcolor{gold}{\bfseries Question:} If $\vec u=5\hat{i}+10\hat{j}$ and $\vec v=4\hat{j}$, then $|\vec u-\vec v|$ is:\par
(a) $\sqrt{123}$ \quad (b) $\sqrt{61}$ \quad (c) $\sqrt{11}$ \quad (d) $-\sqrt{61}$
\tcblower
\textcolor{green}{\bfseries Answer:} \textbf{(b) $\sqrt{61}$}\par
\StepFig{1}{\;$\vec u-\vec v=(5,10)-(0,4)=(5,6)$, so $|\vec u-\vec v|=\sqrt{5^2+6^2}=\sqrt{61}$.}{%
  \Axes
  \coordinate (O) at (0,0);
  \coordinate (W) at (3.8,2.4); % (5,6) scaled
  \draw[strong] (O)--(W);
  \fill[pt] (O) node[lab, below left] {$O$};
  \fill[pt] (W) node[lab, above right] {};
  \draw[helper] (0,0)--(3.8,0)--(3.8,2.4);
  \node[note] at (2.0,-0.55) {$5$};
  \node[note] at (4.35,1.2) {$6$};
  \node[note] at (2.6,2.9) {$\sqrt{61}$};
}
\end{QAPair}

\begin{QAPair}{Question 1 (xiii) --- MCQ}
\textcolor{gold}{\bfseries Question:} Which of the following vectors represents a position vector?\par
(a) $\overrightarrow{OP}$ \quad (b) $-\overrightarrow{OP}$ \quad (c) $\overrightarrow{PO}$ \quad (d) $\overrightarrow{PQ}$
\tcblower
\textcolor{green}{\bfseries Answer:} \textbf{(a) $\overrightarrow{OP}$}\par
\StepFig{1}{A position vector starts at the origin $O$ and ends at the point (here $P$).}{%
  \Axes
  \coordinate (O) at (0,0);
  \coordinate (P) at (4.3,2.6);
  \draw[strong] (O)--(P);
  \fill[pt] (O) node[lab, below left] {$O$};
  \fill[pt] (P) node[lab, above right] {$P$};
  \node[note] at (2.6,1.9) {$\overrightarrow{OP}$};
}
\end{QAPair}

\begin{QAPair}{Question 1 (xiv) --- MCQ}
\textcolor{gold}{\bfseries Question:} What type of a quadrilateral $ABCD$ is, if $\overrightarrow{AB}=\frac{2}{3}\overrightarrow{DC}$?\par
(a) kite \quad (b) rectangle \quad (c) trapezium \quad (d) rhombus
\tcblower
\textcolor{green}{\bfseries Answer:} \textbf{(c) trapezium}\par
\StepFig{1}{If one side vector is a nonzero scalar multiple of another, the sides are \emph{parallel}. Thus $AB\parallel DC$, so it is a trapezium.}{%
  \coordinate (A) at (0,0);
  \coordinate (B) at (4.8,0);
  \coordinate (D) at (0.8,2.6);
  \coordinate (C) at (4.0,2.6);

  \draw[geom] (A)--(B)--(C)--(D)--cycle;

  \draw[strong] (A)--(B);
  \draw[strong] (D)--(C);

  \fill[pt] (A) node[lab, below] {$A$};
  \fill[pt] (B) node[lab, below] {$B$};
  \fill[pt] (C) node[lab, above] {$C$};
  \fill[pt] (D) node[lab, above] {$D$};

  \node[note] at (2.4,-0.65) {$AB \parallel DC$};
}
\end{QAPair}

% ============================================================
% Q2
\begin{QAPair}{Question 2 --- Given $\vec p=3\hat{i}-4\hat{j}$ and $\vec q=-3\hat{i}-4\hat{j}$. Prove that $|\vec p|=|\vec q|$. Is $\vec p=\vec q$?}
\tcblower
\textcolor{green}{\bfseries Solution:}\par

\StepFig{1}{Compute magnitudes.}{%
  \node[lab, align=left] at (0,0) {$\displaystyle |\vec p|=\sqrt{3^2+(-4)^2}=\sqrt{9+16}=5$};
  \node[lab, align=left] at (0,-0.8) {$\displaystyle |\vec q|=\sqrt{(-3)^2+(-4)^2}=\sqrt{9+16}=5$};
}

\StepFig{2}{Show they are not equal vectors (different directions/components).}{%
  \draw[helper] (-3.2,0)--(3.2,0);
  \draw[helper] (0,-3.2)--(0,3.2);
  \coordinate (O) at (0,0);
  \coordinate (P) at (2.4,-3.2);
  \coordinate (Q) at (-2.4,-3.2);

  \draw[strong] (O)--(P);
  \draw[strong] (O)--(Q);

  \fill[pt] (O) node[lab, above right] {$O$};
  \fill[pt] (P) node[lab, below right] {$\vec p$};
  \fill[pt] (Q) node[lab, below left] {$\vec q$};

  \node[note] at (0,2.6) {same length, different direction};
}

\textcolor{green}{\bfseries Conclusion:}\;
$|\vec p|=|\vec q|=5$, but $\vec p\neq \vec q$ because their $x$-components differ ($3\neq -3$).
\end{QAPair}

% ============================================================
% Q3
\begin{QAPair}{Question 3 --- If $\vec a=2\hat{i}-4\hat{j}$ and $\vec b=-2\hat{i}+x\hat{j}$, find $x$ if $|\vec a+2\vec b|=6$.}
\tcblower
\textcolor{green}{\bfseries Solution:}\par

\StepFig{1}{Form $\vec a+2\vec b$.}{%
  \node[lab, align=left] at (0,0) {$\displaystyle \vec a+2\vec b=(2,-4)+2(-2,x)=(2-4,\,-4+2x)=(-2,\;2x-4)$};
}

\StepFig{2}{Use the magnitude condition $|\vec a+2\vec b|=6$.}{%
  \node[lab, align=left] at (0,0.6) {$\displaystyle \sqrt{(-2)^2+(2x-4)^2}=6$};
  \node[lab, align=left] at (0,-0.2) {$\displaystyle 4+(2x-4)^2=36 \;\Rightarrow\; (2x-4)^2=32$};
  \node[lab, align=left] at (0,-1.0) {$\displaystyle 2x-4=\pm\sqrt{32}=\pm 4\sqrt2$};
}

\textcolor{green}{\bfseries Answer:}\;
$\displaystyle x=2\pm 2\sqrt2$.
\end{QAPair}

% ============================================================
% Q4
\begin{QAPair}{Question 4 --- In $\triangle OAB$, $\overrightarrow{OA}=\vec a$, $\overrightarrow{OB}=\vec b$ and $M$ is midpoint of $\overline{OA}$. Write the following vectors in terms of $\vec a$ and $\vec b$: (i) $\overrightarrow{OM}$ (ii) $\overrightarrow{AM}$ (iii) $\overrightarrow{BM}$ (iv) $\overrightarrow{OB}+\overrightarrow{BA}$.}
\tcblower
\textcolor{green}{\bfseries Solution:}\par

\StepFig{1}{Draw the figure and mark $M$ as midpoint of $OA$.}{%
  \coordinate (O) at (0,0);
  \coordinate (A) at (5.6,0);
  \coordinate (B) at (3.3,2.8);
  \coordinate (M) at ($(O)!0.5!(A)$);

  \draw[geom] (O)--(A);
  \draw[geom] (O)--(B);
  \draw[geom] (B)--(A);
  \draw[helper] (B)--(M);

  \fill[pt] (O) node[lab, below left] {$O$};
  \fill[pt] (A) node[lab, below right] {$A$};
  \fill[pt] (B) node[lab, above] {$B$};
  \fill[pt] (M) node[lab, below] {$M$};

  \node[note] at (2.8,-0.75) {$M$ is midpoint of $OA$};
}

\StepFig{2}{Use midpoint and position-vector ideas.}{%
  \node[lab, align=left] at (0,0.9) {$\displaystyle (i)\;\overrightarrow{OM}=\frac12\,\overrightarrow{OA}=\frac12\vec a$};
  \node[lab, align=left] at (0,0.1) {$\displaystyle (ii)\;\overrightarrow{AM}=\overrightarrow{OM}-\overrightarrow{OA}=\frac12\vec a-\vec a=-\frac12\vec a$};
  \node[lab, align=left] at (0,-0.7) {$\displaystyle (iii)\;\overrightarrow{BM}=\overrightarrow{OM}-\overrightarrow{OB}=\frac12\vec a-\vec b$};
  \node[lab, align=left] at (0,-1.5) {$\displaystyle (iv)\;\overrightarrow{OB}+\overrightarrow{BA}=\vec b+(\overrightarrow{OA}-\overrightarrow{OB})=\vec b+(\vec a-\vec b)=\vec a$};
}

\textcolor{green}{\bfseries Answers:}\;
(i) $\dfrac12\vec a$;\;
(ii) $-\dfrac12\vec a$;\;
(iii) $\dfrac12\vec a-\vec b$;\;
(iv) $\vec a$.
\end{QAPair}

% ============================================================
% Q5
\begin{QAPair}{Question 5 --- Two tractors pull with forces $250$ N at $50^\circ$ and $300$ N at $40^\circ$ to the horizontal. Find the magnitude and direction of the resultant force.}
\tcblower
\textcolor{green}{\bfseries Solution:}\par

\StepFig{1}{Resolve forces into components and add.}{%
  \node[lab, align=left] at (0,1.0) {$\displaystyle R_x=250\cos50^\circ+300\cos40^\circ \approx 391.85$};
  \node[lab, align=left] at (0,0.2) {$\displaystyle R_y=250\sin50^\circ+300\sin40^\circ \approx 383.44$};
  \node[lab, align=left] at (0,-0.6) {$\displaystyle R=\sqrt{R_x^2+R_y^2}\approx \sqrt{391.85^2+383.44^2}\approx 548.1\ \text{N}$};
  \node[lab, align=left] at (0,-1.4) {$\displaystyle \phi=\tan^{-1}\!\left(\frac{R_y}{R_x}\right)\approx \tan^{-1}(0.9785)\approx 44.5^\circ$};
}

\StepFig{2}{Vector diagram (tail-to-tail) with resultant.}{%
  \Axes
  \coordinate (O) at (0,0);
  \coordinate (X) at ($(O)+(1,0)$); % helper point on +x-axis for angle marking

  % scaled arrows
  \coordinate (Fone) at ($(O)+(50:3.0)$);
  \coordinate (Ftwo) at ($(O)+(40:3.6)$);

  \draw[strong] (O)--(Fone);
  \draw[strong] (O)--(Ftwo);

  % resultant (approx)
  \coordinate (R) at ($(O)+(44.5:4.2)$);
  \draw[geom, -latex, line width=1.2pt] (O)--(R);

  \fill[pt] (O) node[lab, below left] {$O$};
  \node[lab] at (2.1,2.7) {$250\text{ N}$};
  \node[lab] at (2.9,2.2) {$300\text{ N}$};
  \node[lab] at (3.0,3.3) {$\vec R$};

  % FIXED: use a named helper point for the +x-axis ray
  \pic[ang, angle radius=0.65cm, "$50^\circ$"] {angle = X--O--Fone};
  \pic[ang, angle radius=1.00cm, "$40^\circ$"] {angle = X--O--Ftwo};
}

\textcolor{green}{\bfseries Answer:}\;
Resultant force $\boxed{R\approx 548.1\ \text{N}}$ at $\boxed{\phi\approx 44.5^\circ}$ above the horizontal.
\end{QAPair}

% ============================================================
% Q6
\begin{QAPair}{Question 6 --- The paths are $\vec V_1=120\hat{i}+12\hat{j}$ and $\vec V_2=90\hat{i}-30\hat{j}$ (meters). (i) How much farther did the first ball travel? (ii) What is the distance between the two balls thrown?}
\tcblower
\textcolor{green}{\bfseries Solution:}\par

\StepFig{1}{Compute distances (magnitudes).}{%
  \node[lab, align=left] at (0,1.0) {$\displaystyle |\vec V_1|=\sqrt{120^2+12^2}=\sqrt{14544}\approx 120.60\ \text{m}$};
  \node[lab, align=left] at (0,0.2) {$\displaystyle |\vec V_2|=\sqrt{90^2+(-30)^2}=\sqrt{9000}\approx 94.87\ \text{m}$};
  \node[lab, align=left] at (0,-0.6) {$\displaystyle \text{Difference}=|\vec V_1|-|\vec V_2|\approx 25.74\ \text{m}$};
}

\StepFig{2}{Distance between balls (separation) if thrown from the same point.}{%
  \node[lab, align=left] at (0,0.8) {$\displaystyle \vec V_1-\vec V_2=(120-90,\;12-(-30))=(30,\;42)$};
  \node[lab, align=left] at (0,0.0) {$\displaystyle \text{Separation}=\sqrt{30^2+42^2}=\sqrt{2664}\approx 51.61\ \text{m}$};
}

\StepFig{3}{Diagram of $\vec V_1,\vec V_2$ and separation.}{%
  \Axes
  \coordinate (O) at (0,0);
  \coordinate (A) at (4.6,0.5);  % V1 scaled
  \coordinate (B) at (3.5,-1.25);% V2 scaled
  \draw[strong] (O)--(A);
  \draw[strong] (O)--(B);
  \draw[geom, -latex, line width=1.05pt] (B)--(A);

  \fill[pt] (O) node[lab, below left] {$O$};
  \fill[pt] (A) node[lab, above right] {$V_1$};
  \fill[pt] (B) node[lab, below right] {$V_2$};

  \node[note] at (4.1,2.9) {separation $=|V_1-V_2|$};
}

\textcolor{green}{\bfseries Answers:}\;
(i) $\boxed{\approx 25.74\ \text{m}}$ farther.\quad
(ii) $\boxed{\approx 51.61\ \text{m}}$ between the two balls.
\end{QAPair}

% ============================================================
% Q7
\begin{QAPair}{Question 7 --- Ahmad swims at $6$ m/s in still water. River current flows west at $1.5$ m/s. Find the resultant velocity if: (i) he swims due west, (ii) due east, (iii) due north across the river.}
\tcblower
\textcolor{green}{\bfseries Solution:}\par

\StepFig{1}{Set axes: East = $+x$, North = $+y$. Current is west $\Rightarrow -x$ direction.}{%
  \draw[helper] (-3.2,0)--(3.2,0);
  \draw[helper] (0,-2.8)--(0,2.8);
  \node[note] at (3.05,-0.25) {East $(+x)$};
  \node[note] at (-3.05,-0.25) {West $(-x)$};
  \node[note] at (0.55,2.55) {North $(+y)$};
  \node[note] at (0.55,-2.55) {South $(-y)$};

  \draw[strong] (0,0) -- (-2.2,0);
  \node[lab] at (-1.5,0.35) {current $1.5$};
}

\StepFig{2}{(i) Swim due west: velocities add in the same direction.}{%
  \node[lab, align=left] at (0,0.2) {$\displaystyle v_{\text{result}}=6+1.5=7.5\ \text{m/s west}$};
}

\StepFig{3}{(ii) Swim due east: subtract the west current.}{%
  \node[lab, align=left] at (0,0.2) {$\displaystyle v_{\text{result}}=6-1.5=4.5\ \text{m/s east}$};
}

\StepFig{4}{(iii) Swim due north: perpendicular addition of (north 6) and (west 1.5).}{%
  \node[lab, align=left] at (0,1.0) {$\displaystyle |\vec v|=\sqrt{6^2+1.5^2}=\sqrt{38.25}\approx 6.19\ \text{m/s}$};
  \node[lab, align=left] at (0,0.2) {$\displaystyle \theta=\tan^{-1}\!\left(\frac{1.5}{6}\right)\approx 14.0^\circ$ west of north};
  \draw[helper] (-3.2,0)--(3.2,0);
  \draw[helper] (0,-2.8)--(0,2.8);

  \coordinate (O) at (0,0);
  \coordinate (N) at (0,2.4);
  \coordinate (R) at (-0.6,2.4);

  \draw[geom] (O)--(N);
  \draw[geom] (N)--(R);
  \draw[strong] (O)--(R);

  \RightAngleMark{N}{0.14}
  \fill[pt] (O) node[lab, below right] {$O$};
  \node[note] at (-1.2,1.6) {$\vec v$};
}

\textcolor{green}{\bfseries Answers:}\;
(i) $\boxed{7.5\ \text{m/s west}}$;\;
(ii) $\boxed{4.5\ \text{m/s east}}$;\;
(iii) $\boxed{\approx 6.19\ \text{m/s at }14.0^\circ\text{ west of north}}$.
\end{QAPair}

% ============================================================
% Q8
\begin{QAPair}{Question 8 --- A plane travels north at $150$ km/h. Wind blows due east at $50$ km/h. (i) Find the resultant speed of the plane. (ii) How far is the plane from its starting point after $10$ hours?}
\tcblower
\textcolor{green}{\bfseries Solution:}\par

\StepFig{1}{Resultant ground speed (perpendicular velocities).}{%
  \node[lab, align=left] at (0,1.0) {$\displaystyle v=\sqrt{150^2+50^2}=\sqrt{25000}\approx 158.11\ \text{km/h}$};
  \node[lab, align=left] at (0,0.2) {$\displaystyle \text{Direction: } \alpha=\tan^{-1}\!\left(\frac{50}{150}\right)\approx 18.43^\circ\ \text{east of north}$};
}

\StepFig{2}{After $10$ hours, displacement components and distance.}{%
  \node[lab, align=left] at (0,1.0) {$\displaystyle \text{North distance}=150\times 10=1500\ \text{km}$};
  \node[lab, align=left] at (0,0.2) {$\displaystyle \text{East distance}=50\times 10=500\ \text{km}$};
  \node[lab, align=left] at (0,-0.6) {$\displaystyle \text{Distance from start}=\sqrt{1500^2+500^2}=10\sqrt{25000}\approx 1581.14\ \text{km}$};
}

\StepFig{3}{Diagram (north + east = resultant).}{%
  \draw[helper] (-0.2,0)--(5.2,0);
  \draw[helper] (0,-0.2)--(0,4.2);
  \node[note] at (5.1,-0.25) {East};
  \node[note] at (-0.35,4.1) {North};

  \coordinate (O) at (0,0);
  \coordinate (N) at (0,3.6);
  \coordinate (E) at (1.2,3.6);

  \draw[geom] (O)--(N);
  \draw[geom] (N)--(E);
  \draw[strong] (O)--(E);

  \RightAngleMark{N}{0.16}

  \fill[pt] (O) node[lab, below left] {$O$};
  \node[lab] at (0.35,2.2) {$150$};
  \node[lab] at (0.9,3.95) {$50$};
  \node[lab] at (1.0,2.2) {$\vec v$};
}

\textcolor{green}{\bfseries Answers:}\;
(i) $\boxed{v\approx 158.11\ \text{km/h}}$.\quad
(ii) $\boxed{\approx 1581.14\ \text{km}}$ from the starting point after $10$ hours.
\end{QAPair}

\end{document}
