% !TEX TS-program = pdflatex
\documentclass[11pt]{article}

% -------------------- Packages --------------------
\usepackage[a4paper,margin=1in]{geometry}
\usepackage{amsmath,amssymb}
\usepackage[T1]{fontenc}
\usepackage{lmodern}
\usepackage{xcolor}
\usepackage{tcolorbox}
\tcbuselibrary{skins,breakable}
\usepackage{enumitem}
\usepackage{hyperref}

\pagestyle{empty}

% -------------------- Dark Theme Colors --------------------
\definecolor{bg}{HTML}{000000}
\definecolor{pairbg}{HTML}{121212}
\definecolor{solbg}{HTML}{0A0A0A}
\definecolor{border}{HTML}{2A2A2A}
\definecolor{text}{HTML}{FFFFFF}
\definecolor{muted}{HTML}{C9CDD3}
\definecolor{gold}{HTML}{FFD700}
\definecolor{green}{HTML}{4ADE80}
\definecolor{cyan}{HTML}{38BDF8}

\pagecolor{bg}
\color{text}

\hypersetup{
  colorlinks=true,
  linkcolor=cyan,
  urlcolor=cyan
}

\setlength{\parindent}{0pt}
\setlength{\parskip}{10pt}

\setlist[itemize]{left=1.4em,itemsep=6pt,topsep=6pt}
\setlist[enumerate]{left=1.6em,itemsep=4pt,topsep=4pt}

% -------------------- tcolorbox Base --------------------
\tcbset{
  enhanced,
  breakable,
  arc=12pt,
  boxrule=0.8pt,
  left=16pt,right=16pt,top=12pt,bottom=12pt
}

\newtcolorbox{QAPair}[1]{%
  colback=pairbg,
  colbacklower=solbg,
  colframe=border,
  coltext=text,
  title=\textcolor{gold}{\bfseries #1},
  fonttitle=\bfseries,
  coltitle=text,
  segmentation style={draw=border, dashed, line width=0.6pt},
}

% Visible text inside this box
\newtcolorbox{QuickBox}{%
  colback=pairbg,
  colframe=cyan,
  coltext=text,
  fontupper=\color{text},
  borderline north={4pt}{0pt}{cyan},
  arc=14pt,
  boxrule=0.8pt
}

% Helper for step headings
\newcommand{\Step}[1]{\textcolor{muted}{\textbf{Step #1:}}}

% ============================================================
\begin{document}

\begin{center}
{\LARGE\bfseries \textcolor{gold}{Exercise 2.4 --- Solutions}}\\[-2pt]
\end{center}

\begin{QuickBox}
{\color{cyan}\bfseries Quick formulas (useful)}\par\medskip
\begin{itemize}
\item For $ax^2+bx+c=0$ ($a\neq 0$) with roots $S_1,S_2$:
\[
S_1+S_2=-\frac{b}{a},\qquad S_1S_2=\frac{c}{a}.
\]
\item Quadratic with roots $\alpha,\beta$:
\[
x^2-(\alpha+\beta)x+\alpha\beta=0.
\]
\item Handy identities:
\[
S_1^2+S_2^2=(S_1+S_2)^2-2S_1S_2,\quad
\frac{1}{S_1}+\frac{1}{S_2}=\frac{S_1+S_2}{S_1S_2},
\]
\[
(S_1-S_2)^2=(S_1+S_2)^2-4S_1S_2,\quad
S_1^3+S_2^3=(S_1+S_2)^3-3S_1S_2(S_1+S_2).
\]
\end{itemize}
\end{QuickBox}

% ============================================================
% Q1
\begin{QAPair}{Question 1 (i)}
\textcolor{gold}{\bfseries Question:} $x^2-5x+2=0$ (Find sum \& product of roots)\\
\tcblower
\textcolor{green}{\bfseries Answer:}
Let the roots be $S_1,S_2$. Here $a=1,b=-5,c=2$.
\[
\begin{aligned}
\Step{1}\;& S_1+S_2=-\frac{b}{a}=-\frac{-5}{1}=5,\\
\Step{2}\;& S_1S_2=\frac{c}{a}=\frac{2}{1}=2.
\end{aligned}
\]
\end{QAPair}

\begin{QAPair}{Question 1 (ii)}
\textcolor{gold}{\bfseries Question:} $-4x^2-6x-2=0$\\
\tcblower
\textcolor{green}{\bfseries Answer:}
Here $a=-4,b=-6,c=-2$.
\[
\begin{aligned}
\Step{1}\;& S_1+S_2=-\frac{b}{a}=-\frac{-6}{-4}=\frac{6}{-4}=-\frac{3}{2},\\
\Step{2}\;& S_1S_2=\frac{c}{a}=\frac{-2}{-4}=\frac{1}{2}.
\end{aligned}
\]
\end{QAPair}

\begin{QAPair}{Question 1 (iii)}
\textcolor{gold}{\bfseries Question:} $5x^2-2x+2=0$\\
\tcblower
\textcolor{green}{\bfseries Answer:}
Here $a=5,b=-2,c=2$.
\[
\begin{aligned}
\Step{1}\;& S_1+S_2=-\frac{b}{a}=-\frac{-2}{5}=\frac{2}{5},\\
\Step{2}\;& S_1S_2=\frac{c}{a}=\frac{2}{5}.
\end{aligned}
\]
\end{QAPair}

\begin{QAPair}{Question 1 (iv)}
\textcolor{gold}{\bfseries Question:} $-4x^2-8x-9=0$\\
\tcblower
\textcolor{green}{\bfseries Answer:}
Here $a=-4,b=-8,c=-9$.
\[
\begin{aligned}
\Step{1}\;& S_1+S_2=-\frac{b}{a}=-\frac{-8}{-4}=\frac{8}{-4}=-2,\\
\Step{2}\;& S_1S_2=\frac{c}{a}=\frac{-9}{-4}=\frac{9}{4}.
\end{aligned}
\]
\end{QAPair}

\begin{QAPair}{Question 1 (v)}
\textcolor{gold}{\bfseries Question:} $16y^2-17y-12=0$\\
\tcblower
\textcolor{green}{\bfseries Answer:}
Let roots be $S_1,S_2$ in $y$. Here $a=16,b=-17,c=-12$.
\[
\begin{aligned}
\Step{1}\;& S_1+S_2=-\frac{b}{a}=-\frac{-17}{16}=\frac{17}{16},\\
\Step{2}\;& S_1S_2=\frac{c}{a}=\frac{-12}{16}=-\frac{3}{4}.
\end{aligned}
\]
\end{QAPair}

\begin{QAPair}{Question 1 (vi)}
\textcolor{gold}{\bfseries Question:} $0.3x^2-7.7x+1.8=0$\\
\tcblower
\textcolor{green}{\bfseries Answer:}
Write decimals as fractions: $0.3=\frac{3}{10}$, $7.7=\frac{77}{10}$, $1.8=\frac{9}{5}$.
So $a=\frac{3}{10}$, $b=-\frac{77}{10}$, $c=\frac{9}{5}$.
\[
\begin{aligned}
\Step{1}\;& S_1+S_2=-\frac{b}{a}
= -\frac{-\frac{77}{10}}{\frac{3}{10}}
=\frac{77}{3},\\
\Step{2}\;& S_1S_2=\frac{c}{a}
=\frac{\frac{9}{5}}{\frac{3}{10}}
=\frac{9}{5}\cdot\frac{10}{3}=6.
\end{aligned}
\]
\end{QAPair}

% ============================================================
% Q2
\begin{QAPair}{Question 2 (i)}
\textcolor{gold}{\bfseries Question:} Form a quadratic equation with roots $1,\;-\dfrac{8}{3}$.\\
\tcblower
\textcolor{green}{\bfseries Answer:}
Let $\alpha=1,\ \beta=-\dfrac{8}{3}$.
\[
\begin{aligned}
\Step{1}\;& \alpha+\beta=1-\frac{8}{3}=-\frac{5}{3},\qquad \alpha\beta=-\frac{8}{3}.\\
\Step{2}\;& x^2-(\alpha+\beta)x+\alpha\beta=0\\
&\Rightarrow x^2+\frac{5}{3}x-\frac{8}{3}=0.\\
\Step{3}\;& \text{Multiply by }3:\quad 3x^2+5x-8=0.
\end{aligned}
\]
\end{QAPair}

\begin{QAPair}{Question 2 (ii)}
\textcolor{gold}{\bfseries Question:} Form a quadratic equation with roots $\sqrt{3},\;2\sqrt{3}$.\\
\tcblower
\textcolor{green}{\bfseries Answer:}
Let $\alpha=\sqrt{3},\ \beta=2\sqrt{3}$.
\[
\begin{aligned}
\Step{1}\;& \alpha+\beta=3\sqrt{3},\qquad \alpha\beta=2\cdot 3=6.\\
\Step{2}\;& x^2-(3\sqrt{3})x+6=0.
\end{aligned}
\]
\end{QAPair}

\begin{QAPair}{Question 2 (iii)}
\textcolor{gold}{\bfseries Question:} Form a quadratic equation with roots $2+\sqrt{3},\;2-\sqrt{3}$.\\
\tcblower
\textcolor{green}{\bfseries Answer:}
Let $\alpha=2+\sqrt{3},\ \beta=2-\sqrt{3}$.
\[
\begin{aligned}
\Step{1}\;& \alpha+\beta=4,\qquad \alpha\beta=(2+\sqrt{3})(2-\sqrt{3})=4-3=1.\\
\Step{2}\;& x^2-4x+1=0.
\end{aligned}
\]
\end{QAPair}

\begin{QAPair}{Question 2 (iv)}
\textcolor{gold}{\bfseries Question:} Form a quadratic equation with roots $5i,\;-5i$.\\
\tcblower
\textcolor{green}{\bfseries Answer:}
Let $\alpha=5i,\ \beta=-5i$.
\[
\begin{aligned}
\Step{1}\;& \alpha+\beta=0,\qquad \alpha\beta=(5i)(-5i)=-25i^2=25.\\
\Step{2}\;& x^2-0\cdot x+25=0 \;\Rightarrow\; x^2+25=0.
\end{aligned}
\]
\end{QAPair}

\begin{QAPair}{Question 2 (v)}
\textcolor{gold}{\bfseries Question:} Form a quadratic equation with roots $7+2i,\;7-2i$.\\
\tcblower
\textcolor{green}{\bfseries Answer:}
Let $\alpha=7+2i,\ \beta=7-2i$.
\[
\begin{aligned}
\Step{1}\;& \alpha+\beta=14,\\
\Step{2}\;& \alpha\beta=(7+2i)(7-2i)=7^2-(2i)^2=49-(-4)=53.\\
\Step{3}\;& x^2-14x+53=0.
\end{aligned}
\]
\end{QAPair}

% ============================================================
% Q3
\begin{QAPair}{Question 3 (Given)}
\textcolor{gold}{\bfseries Question:} If $S_1,S_2$ are roots of $3x^2-2x+4=0$, find the following values.\\
\tcblower
\textcolor{green}{\bfseries Answer:}
For $3x^2-2x+4=0$:
\[
S_1+S_2=-\frac{b}{a}=\frac{2}{3}=:s,\qquad S_1S_2=\frac{c}{a}=\frac{4}{3}=:p.
\]
Also,
\[
S_1^2+S_2^2=s^2-2p=\left(\frac{2}{3}\right)^2-2\cdot\frac{4}{3}=\frac{4}{9}-\frac{8}{3}=-\frac{20}{9}.
\]
\end{QAPair}

\begin{QAPair}{Question 3 (i)}
\textcolor{gold}{\bfseries Question:} $\dfrac{1}{S_1^2}+\dfrac{1}{S_2^2}$\\
\tcblower
\textcolor{green}{\bfseries Answer:}
\[
\begin{aligned}
\Step{1}\;& \frac{1}{S_1^2}+\frac{1}{S_2^2}=\frac{S_1^2+S_2^2}{S_1^2S_2^2}
=\frac{S_1^2+S_2^2}{(S_1S_2)^2}.\\
\Step{2}\;&=\frac{-\frac{20}{9}}{\left(\frac{4}{3}\right)^2}
=\frac{-\frac{20}{9}}{\frac{16}{9}}=-\frac{5}{4}.
\end{aligned}
\]
\end{QAPair}

\begin{QAPair}{Question 3 (ii)}
\textcolor{gold}{\bfseries Question:} $S_1^2+S_2^2$\\
\tcblower
\textcolor{green}{\bfseries Answer:}
\[
S_1^2+S_2^2=-\frac{20}{9}.
\]
\end{QAPair}

\begin{QAPair}{Question 3 (iii)}
\textcolor{gold}{\bfseries Question:} $2S_1+2S_2+4$\\
\tcblower
\textcolor{green}{\bfseries Answer:}
\[
\begin{aligned}
\Step{1}\;& 2S_1+2S_2+4=2(S_1+S_2)+4=2s+4.\\
\Step{2}\;&=2\cdot\frac{2}{3}+4=\frac{4}{3}+\frac{12}{3}=\frac{16}{3}.
\end{aligned}
\]
\end{QAPair}

\begin{QAPair}{Question 3 (iv)}
\textcolor{gold}{\bfseries Question:} $\dfrac{1}{S_1}+\dfrac{1}{S_2}$\\
\tcblower
\textcolor{green}{\bfseries Answer:}
\[
\begin{aligned}
\Step{1}\;& \frac{1}{S_1}+\frac{1}{S_2}=\frac{S_1+S_2}{S_1S_2}=\frac{s}{p}.\\
\Step{2}\;&=\frac{\frac{2}{3}}{\frac{4}{3}}=\frac{1}{2}.
\end{aligned}
\]
\end{QAPair}

\begin{QAPair}{Question 3 (v)}
\textcolor{gold}{\bfseries Question:} $\dfrac{S_1}{S_2}+\dfrac{S_2}{S_1}$\\
\tcblower
\textcolor{green}{\bfseries Answer:}
\[
\begin{aligned}
\Step{1}\;& \frac{S_1}{S_2}+\frac{S_2}{S_1}=\frac{S_1^2+S_2^2}{S_1S_2}=\frac{s^2-2p}{p}.\\
\Step{2}\;&=\frac{-\frac{20}{9}}{\frac{4}{3}}=-\frac{20}{9}\cdot\frac{3}{4}=-\frac{5}{3}.
\end{aligned}
\]
\end{QAPair}

\begin{QAPair}{Question 3 (vi)}
\textcolor{gold}{\bfseries Question:} $S_1S_2^2+S_1^2S_2$\\
\tcblower
\textcolor{green}{\bfseries Answer:}
\[
\begin{aligned}
\Step{1}\;& S_1S_2^2+S_1^2S_2=S_1S_2(S_1+S_2)=ps.\\
\Step{2}\;&=\frac{4}{3}\cdot\frac{2}{3}=\frac{8}{9}.
\end{aligned}
\]
\end{QAPair}

\begin{QAPair}{Question 3 (vii)}
\textcolor{gold}{\bfseries Question:} $S_1^3S_2+S_1S_2^3$\\
\tcblower
\textcolor{green}{\bfseries Answer:}
\[
\begin{aligned}
\Step{1}\;& S_1^3S_2+S_1S_2^3=S_1S_2(S_1^2+S_2^2)=p(s^2-2p).\\
\Step{2}\;&=\frac{4}{3}\cdot\left(-\frac{20}{9}\right)=-\frac{80}{27}.
\end{aligned}
\]
\end{QAPair}

\begin{QAPair}{Question 3 (viii)}
\textcolor{gold}{\bfseries Question:} $(S_1-3)(S_2-3)$\\
\tcblower
\textcolor{green}{\bfseries Answer:}
\[
\begin{aligned}
\Step{1}\;& (S_1-3)(S_2-3)=S_1S_2-3(S_1+S_2)+9=p-3s+9.\\
\Step{2}\;&=\frac{4}{3}-3\cdot\frac{2}{3}+9=\frac{4}{3}-2+9=\frac{25}{3}.
\end{aligned}
\]
\end{QAPair}

% ============================================================
% Q4
\begin{QAPair}{Question 4 (Given)}
\textcolor{gold}{\bfseries Question:} If $S_1,S_2$ are roots of $7x^2+10x+7=0$, form equations for the given new roots.\\
\tcblower
\textcolor{green}{\bfseries Answer:}
For $7x^2+10x+7=0$:
\[
S_1+S_2=-\frac{10}{7}=:s,\qquad S_1S_2=\frac{7}{7}=1=:p.
\]
Also,
\[
S_1^2+S_2^2=s^2-2p=\left(\frac{100}{49}\right)-2=\frac{2}{49}.
\]
\end{QAPair}

\begin{QAPair}{Question 4 (i)}
\textcolor{gold}{\bfseries Question:} Roots $S_1^2,\;S_2^2$\\
\tcblower
\textcolor{green}{\bfseries Answer:}
Sum $=S_1^2+S_2^2=\dfrac{2}{49}$, product $=S_1^2S_2^2=p^2=1$.
\[
\begin{aligned}
\Step{1}\;& x^2-\frac{2}{49}x+1=0.\\
\Step{2}\;& \text{Multiply by }49:\quad 49x^2-2x+49=0.
\end{aligned}
\]
\end{QAPair}

\begin{QAPair}{Question 4 (ii)}
\textcolor{gold}{\bfseries Question:} Roots $\dfrac{1}{S_1},\;\dfrac{1}{S_2}$\\
\tcblower
\textcolor{green}{\bfseries Answer:}
\[
\begin{aligned}
\Step{1}\;& \frac{1}{S_1}+\frac{1}{S_2}=\frac{S_1+S_2}{S_1S_2}=\frac{s}{p}=s=-\frac{10}{7},\\
\Step{2}\;& \frac{1}{S_1}\cdot\frac{1}{S_2}=\frac{1}{S_1S_2}=\frac{1}{p}=1.\\
\Step{3}\;& x^2-\left(-\frac{10}{7}\right)x+1=0
\Rightarrow x^2+\frac{10}{7}x+1=0\\
&\Rightarrow 7x^2+10x+7=0.
\end{aligned}
\]
\end{QAPair}

\begin{QAPair}{Question 4 (iii)}
\textcolor{gold}{\bfseries Question:} Roots $S_1^3S_2,\;S_1S_2^3$\\
\tcblower
\textcolor{green}{\bfseries Answer:}
Since $p=S_1S_2=1$,
\[
S_1^3S_2=S_1^2(S_1S_2)=S_1^2,\qquad S_1S_2^3=S_2^2.
\]
So it is the same as part (i):
\[
49x^2-2x+49=0.
\]
\end{QAPair}

\begin{QAPair}{Question 4 (iv)}
\textcolor{gold}{\bfseries Question:} Roots $S_1-\dfrac{1}{S_1},\;S_2-\dfrac{1}{S_2}$\\
\tcblower
\textcolor{green}{\bfseries Answer:}
Because $p=1$, we have $\dfrac{1}{S_1}=S_2$ and $\dfrac{1}{S_2}=S_1$.
Thus roots are $(S_1-S_2)$ and $(S_2-S_1)=-(S_1-S_2)$, so their sum is $0$.
\[
\begin{aligned}
\Step{1}\;& (S_1-S_2)^2=s^2-4p=\frac{100}{49}-4=-\frac{96}{49}.\\
\Step{2}\;& \text{Product}=(S_1-S_2)(S_2-S_1)=-(S_1-S_2)^2=\frac{96}{49}.\\
\Step{3}\;& x^2-0\cdot x+\frac{96}{49}=0 \;\Rightarrow\; 49x^2+96=0.
\end{aligned}
\]
\end{QAPair}

\begin{QAPair}{Question 4 (v)}
\textcolor{gold}{\bfseries Question:} Roots $2S_1+1,\;2S_2+1$\\
\tcblower
\textcolor{green}{\bfseries Answer:}
\[
\begin{aligned}
\Step{1}\;& \text{Sum}=2(S_1+S_2)+2=2s+2=2\left(-\frac{10}{7}\right)+2=-\frac{6}{7}.\\
\Step{2}\;& \text{Product}=(2S_1+1)(2S_2+1)=4p+2s+1\\
&=4(1)+2\left(-\frac{10}{7}\right)+1=5-\frac{20}{7}=\frac{15}{7}.\\
\Step{3}\;& x^2-\left(-\frac{6}{7}\right)x+\frac{15}{7}=0
\Rightarrow 7x^2+6x+15=0.
\end{aligned}
\]
\end{QAPair}

\begin{QAPair}{Question 4 (vi)}
\textcolor{gold}{\bfseries Question:} Roots $\dfrac{S_1}{S_2},\;\dfrac{S_2}{S_1}$\\
\tcblower
\textcolor{green}{\bfseries Answer:}
Since $p=1$, $\dfrac{S_1}{S_2}=S_1^2$ and $\dfrac{S_2}{S_1}=S_2^2$.
So it is again the same as part (i):
\[
49x^2-2x+49=0.
\]
\end{QAPair}

\begin{QAPair}{Question 4 (vii)}
\textcolor{gold}{\bfseries Question:} Roots $S_1+S_2,\;\dfrac{1}{S_1}+\dfrac{1}{S_2}$\\
\tcblower
\textcolor{green}{\bfseries Answer:}
\[
\Step{1}\; S_1+S_2=s=-\frac{10}{7},\qquad \frac{1}{S_1}+\frac{1}{S_2}=\frac{s}{p}=s=-\frac{10}{7}.
\]
Both roots are equal to $-\dfrac{10}{7}$.
\[
\begin{aligned}
\Step{2}\;& (x+\tfrac{10}{7})^2=0
\Rightarrow x^2+\frac{20}{7}x+\frac{100}{49}=0.\\
\Step{3}\;& \text{Multiply by }49:\quad 49x^2+140x+100=0.
\end{aligned}
\]
\end{QAPair}

\begin{QAPair}{Question 4 (viii)}
\textcolor{gold}{\bfseries Question:} Roots $S_1^2+1,\;S_2^2+1$\\
\tcblower
\textcolor{green}{\bfseries Answer:}
\[
\begin{aligned}
\Step{1}\;& \text{Sum}=(S_1^2+S_2^2)+2=\frac{2}{49}+2=\frac{100}{49}.\\
\Step{2}\;& \text{Product}=(S_1^2+1)(S_2^2+1)=S_1^2S_2^2+(S_1^2+S_2^2)+1\\
&=p^2+\frac{2}{49}+1=1+\frac{2}{49}+1=\frac{100}{49}.\\
\Step{3}\;& x^2-\frac{100}{49}x+\frac{100}{49}=0
\Rightarrow 49x^2-100x+100=0.
\end{aligned}
\]
\end{QAPair}

\begin{QAPair}{Question 4 (ix)}
\textcolor{gold}{\bfseries Question:} Roots $S_1^2+S_2,\;S_1+S_2^2$\\
\tcblower
\textcolor{green}{\bfseries Answer:}
\[
\begin{aligned}
\Step{1}\;& \text{Sum}=(S_1^2+S_2)+(S_1+S_2^2)=(S_1^2+S_2^2)+(S_1+S_2)\\
&=\frac{2}{49}-\frac{10}{7}=\frac{2}{49}-\frac{70}{49}=-\frac{68}{49}.\\[4pt]
\Step{2}\;& \text{Product}=(S_1^2+S_2)(S_1+S_2^2)=S_1^3+S_2^3+S_1S_2+S_1^2S_2^2.\\
\Step{3}\;& S_1^3+S_2^3=s^3-3ps=\left(-\frac{10}{7}\right)^3-3(1)\left(-\frac{10}{7}\right)
=-\frac{1000}{343}+\frac{30}{7}=\frac{470}{343}.\\
\Step{4}\;& \Rightarrow \text{Product}=\frac{470}{343}+1+1=\frac{1156}{343}.\\
\Step{5}\;& x^2-\left(-\frac{68}{49}\right)x+\frac{1156}{343}=0
\Rightarrow x^2+\frac{68}{49}x+\frac{1156}{343}=0.\\
\Step{6}\;& \text{Multiply by }343:\quad 343x^2+476x+1156=0.
\end{aligned}
\]
\end{QAPair}

% ============================================================
% Q5
\begin{QAPair}{Question 5}
\textcolor{gold}{\bfseries Question:} If $S_1,S_2$ are roots of $x^2+6x+3=0$, form the equation whose roots are $(S_1+S_2)^2,\ (S_1-S_2)^2$.\\
\tcblower
\textcolor{green}{\bfseries Answer:}
For $x^2+6x+3=0$:
\[
s=S_1+S_2=-6,\qquad p=S_1S_2=3.
\]
\[
\begin{aligned}
\Step{1}\;& r_1=(S_1+S_2)^2=s^2=36,\\
\Step{2}\;& r_2=(S_1-S_2)^2=s^2-4p=36-12=24.
\end{aligned}
\]
So the required equation (roots $36,24$) is:
\[
x^2-(36+24)x+(36)(24)=0 \;\Rightarrow\; x^2-60x+864=0.
\]
\end{QAPair}

% ============================================================
% Q6
\begin{QAPair}{Question 6}
\textcolor{gold}{\bfseries Question:} If $S_1,S_2$ are roots of $2x^2+6x-3=0$, form the equation whose roots are
\[
S_1-\frac{3}{S_2^2},\qquad S_2-\frac{3}{S_1^2}.
\]
\tcblower
\textcolor{green}{\bfseries Answer:}
For $2x^2+6x-3=0$:
\[
s=S_1+S_2=-\frac{6}{2}=-3,\qquad p=S_1S_2=\frac{-3}{2}.
\]
Let
\[
A=S_1-\frac{3}{S_2^2},\qquad B=S_2-\frac{3}{S_1^2}.
\]
Note: $\dfrac{1}{S_2^2}=\dfrac{S_1^2}{p^2}$ and $\dfrac{1}{S_1^2}=\dfrac{S_2^2}{p^2}$.
Also $S_1^2+S_2^2=s^2-2p=9-2\left(-\frac{3}{2}\right)=12$, and $p^2=\dfrac{9}{4}$.

\[
\begin{aligned}
\Step{1}\;& A+B=(S_1+S_2)-\frac{3}{p^2}(S_1^2+S_2^2)
=s-\frac{3}{\frac{9}{4}}\cdot 12
=-3-\frac{4}{3}\cdot 12=-19.\\[4pt]
\Step{2}\;& AB=p-\frac{3}{p^2}(S_1S_2^2+S_1^2S_2)+\frac{9}{p^4}(S_1^2S_2^2)\\
&=p-\frac{3}{p^2}(ps)+\frac{9}{p^2}
=p-\frac{3s}{p}+\frac{9}{p^2}.
\end{aligned}
\]
Now substitute $s=-3,\ p=-\dfrac{3}{2}$:
\[
AB=-\frac{3}{2}-\frac{3(-3)}{-\frac{3}{2}}+\frac{9}{\left(\frac{9}{4}\right)}
=-\frac{3}{2}-6+4=-\frac{7}{2}.
\]
Hence the required equation is
\[
x^2-(A+B)x+AB=0
\Rightarrow x^2+19x-\frac{7}{2}=0
\Rightarrow 2x^2+38x-7=0.
\]
\end{QAPair}

% ============================================================
% Q7
\begin{QAPair}{Question 7}
\textcolor{gold}{\bfseries Question:} Find $k$ if $S_1$ and $S_1-5$ are roots of $x^2-3kx+5=0$.\\
\tcblower
\textcolor{green}{\bfseries Answer:}
Let the roots be $r$ and $r-5$.
For $x^2-3kx+5=0$:
\[
r+(r-5)=3k,\qquad r(r-5)=5.
\]
\[
\begin{aligned}
\Step{1}\;& r(r-5)=5 \Rightarrow r^2-5r-5=0\\
\Step{2}\;& r=\frac{5\pm\sqrt{25+20}}{2}=\frac{5\pm 3\sqrt{5}}{2}.\\
\Step{3}\;& 3k=2r-5=\pm 3\sqrt{5}\;\Rightarrow\; k=\pm \sqrt{5}.
\end{aligned}
\]
\end{QAPair}

% ============================================================
% Q8
\begin{QAPair}{Question 8}
\textcolor{gold}{\bfseries Question:} Find $k$ such that $3$ is a root of $x^2+kx-21=0$.\\
\tcblower
\textcolor{green}{\bfseries Answer:}
If $x=3$ is a root, substitute $x=3$:
\[
\begin{aligned}
\Step{1}\;& 3^2+3k-21=0\\
\Step{2}\;& 9+3k-21=0 \Rightarrow 3k-12=0 \Rightarrow k=4.
\end{aligned}
\]
\end{QAPair}

\end{document}
