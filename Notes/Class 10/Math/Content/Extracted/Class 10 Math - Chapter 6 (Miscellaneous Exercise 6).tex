% !TEX TS-program = pdflatex
\documentclass[11pt]{article}

% -------------------- Packages --------------------
\usepackage[a4paper,margin=1in]{geometry}
\usepackage{amsmath,amssymb}
\usepackage[T1]{fontenc}
\usepackage{lmodern}
\usepackage{xcolor}
\usepackage{tcolorbox}
\tcbuselibrary{skins,breakable}
\usepackage{enumitem}
\usepackage{hyperref}
\usepackage{tikz}
\usetikzlibrary{calc,patterns,angles,quotes,intersections}

\pagestyle{empty}

% -------------------- Dark Theme Colors --------------------
\definecolor{bg}{HTML}{000000}
\definecolor{pairbg}{HTML}{121212}
\definecolor{solbg}{HTML}{0A0A0A}
\definecolor{border}{HTML}{2A2A2A}
\definecolor{text}{HTML}{FFFFFF}
\definecolor{muted}{HTML}{C9CDD3}
\definecolor{gold}{HTML}{FFD700}
\definecolor{green}{HTML}{4ADE80}
\definecolor{cyan}{HTML}{38BDF8}

\pagecolor{bg}
\color{text}

\hypersetup{
  colorlinks=true,
  linkcolor=cyan,
  urlcolor=cyan
}

\setlength{\parindent}{0pt}
\setlength{\parskip}{10pt}

\setlist[itemize]{left=1.4em,itemsep=6pt,topsep=6pt}
\setlist[enumerate]{left=1.6em,itemsep=4pt,topsep=4pt}

% -------------------- tcolorbox Base --------------------
\tcbset{
  enhanced,
  breakable,
  arc=12pt,
  boxrule=0.8pt,
  left=16pt,right=16pt,top=12pt,bottom=12pt
}

\newtcolorbox{QAPair}[1]{%
  colback=pairbg,
  colbacklower=solbg,
  colframe=border,
  coltext=text,
  title=\textcolor{gold}{\bfseries #1},
  fonttitle=\bfseries,
  coltitle=text,
  segmentation style={draw=border, dashed, line width=0.6pt},
}

\newtcolorbox{QuickBox}{%
  colback=pairbg,
  colframe=cyan,
  coltext=text,
  fontupper=\color{text},
  borderline north={4pt}{0pt}{cyan},
  arc=14pt,
  boxrule=0.8pt
}

% Helper for step headings
\newcommand{\Step}[1]{\textcolor{muted}{\textbf{Step #1:}}}

% -------------------- TikZ Styles --------------------
\tikzset{
  geom/.style={draw=muted, line width=0.95pt},
  strong/.style={draw=cyan, line width=1.05pt},
  helper/.style={draw=muted, dashed, line width=0.75pt},
  arcH/.style={draw=muted, dashed, line width=0.75pt},
  pt/.style={circle, fill=cyan, inner sep=1.2pt},
  lab/.style={text=text, font=\small},
  ang/.style={draw=cyan, line width=0.9pt},
  note/.style={text=muted, font=\small}
}

% -------------------- Step + Diagram Macro --------------------
% Usage:
% \StepFig{1}{<text>}{<tikzpicture contents ONLY>}
\newcommand{\StepFig}[3]{%
  \Step{#1} #2\par\medskip
  \begin{center}
    \begin{tikzpicture}[scale=0.92]
      #3
    \end{tikzpicture}
  \end{center}
  \vspace{-2pt}
}

% tiny right-angle mark macro
\newcommand{\RightAngleMark}[2]{%
  \draw[ang] ($(#1)+(#2,0)$) -- ($(#1)+(#2,#2)$) -- ($(#1)+(0,#2)$);
}

% Simple axes macro (TikZ only)
% \Axes{xmin}{xmax}{ymin}{ymax}{x label}{y label}
\newcommand{\Axes}[6]{%
  \draw[geom,->] (#1,0) -- (#2,0) node[lab, right] {#5};
  \draw[geom,->] (0,#3) -- (0,#4) node[lab, above] {#6};
}

% ============================================================
\begin{document}

\begin{center}
{\LARGE\bfseries \textcolor{gold}{Miscellaneous Exercise 6 --- Solutions}}\\[-2pt]
\end{center}

\begin{QuickBox}
{\color{cyan}\bfseries Quick facts (very useful)}\par\medskip
\begin{itemize}
\item \textbf{Onto (surjective):} Range (image) $=$ codomain. \quad \textbf{Into:} Range is a proper subset of codomain.
\item \textbf{Injective:} different inputs give different outputs (no two arrows land on the same point).
\item \textbf{Binary relations from $Y$ to $X$:} all subsets of $Y\times X$, so count is $2^{|Y\times X|}=2^{|Y||X|}$.
\item \textbf{Domain of a relation:} set of first components of ordered pairs.
\item \textbf{Intercepts:} y-intercept at $x=0$; x-intercepts at $y=0$.
\item \textbf{Inverse function:} swap $x,y$ and solve for $y$; inverse is undefined where its denominator becomes $0$.
\item \textbf{Parabola $y=ax^2+bx+c$:} opens \textbf{up} if $a>0$, \textbf{down} if $a<0$. Vertex at $x=-\frac{b}{2a}$.
\item \textbf{Exponential decay:} $A\cdot b^n$ with $0<b<1$ decreases as $n$ increases.
\end{itemize}
\end{QuickBox}

% ============================================================
% Q1 (MCQs)
\begin{QAPair}{Question 1 (i) --- MCQ}
\textcolor{gold}{\bfseries Question:}
If $A=\{-2,0,2\}$ and $B=\{0,2\}$ and $f:A\to B$ is defined as
$f=\{(-2,2),(0,0),(2,0)\}$, what type of function is $f$?\par
(a) into \quad (b) onto \quad (c) injective \quad (d) bijective
\tcblower
\textcolor{green}{\bfseries Answer:} \textbf{(b) onto}\par
\StepFig{1}{The image of $f$ is $\{0,2\}$ which equals $B$, so $f$ is \emph{onto}. It is not injective because $0$ and $2$ both map to $0$.}{%
  \draw[geom] (-3,0) ellipse (1.3 and 2.2);
  \draw[geom] (3,0) ellipse (1.3 and 1.8);
  \node[lab] at (-3,2.1) {$A$};
  \node[lab] at (3,1.7) {$B$};

  \node[lab] (a1) at (-3,1.0) {$-2$};
  \node[lab] (a2) at (-3,0.0) {$0$};
  \node[lab] (a3) at (-3,-1.0) {$2$};

  \node[lab] (b1) at (3,0.6) {$2$};
  \node[lab] (b2) at (3,-0.6) {$0$};

  \draw[strong,->] (a1) -- (b1);
  \draw[strong,->] (a2) -- (b2);
  \draw[strong,->] (a3) -- (b2); 

  \node[note] at (0,-2.2) {Range $=\{0,2\}=B$ (onto), but two arrows land on $0$ (not one-to-one).};
}
\end{QAPair}

\begin{QAPair}{Question 1 (ii) --- MCQ}
\textcolor{gold}{\bfseries Question:}
If number of elements in set $X$ is $3$ and number of elements in set $Y$ is $2$,
how many binary relations are possible from $Y$ to $X$?\par
(a) $4$ \quad (b) $6$ \quad (c) $2^6$ \quad (d) $2^9$
\tcblower
\textcolor{green}{\bfseries Answer:} \textbf{(c) $2^6$}\par
\StepFig{1}{A relation is any subset of $Y\times X$. Here $|Y\times X|=|Y||X|=2\cdot3=6$, so number of subsets is $2^6$.}{%
  % draw a 2x3 grid representing YxX
  \draw[geom] (-1.2,0.2) rectangle (4.2,2.8);
  \draw[geom] (0.6,0.2) -- (0.6,2.8);
  \draw[geom] (2.4,0.2) -- (2.4,2.8);
  \draw[geom] ( -1.2,1.5) -- (4.2,1.5);

  \node[lab] at (-0.3,3.2) {$x_1$};
  \node[lab] at (1.5,3.2) {$x_2$};
  \node[lab] at (3.3,3.2) {$x_3$};

  \node[lab] at (-2.1,2.15) {$y_1$};
  \node[lab] at (-2.1,0.85) {$y_2$};

  \node[note] at (1.5,-0.4) {$Y\times X$ has $2\times3=6$ ordered pairs $\Rightarrow$ relations $=2^6$.};
}
\end{QAPair}

\begin{QAPair}{Question 1 (iii) --- MCQ}
\textcolor{gold}{\bfseries Question:}
What is the domain of the relation $g=\{(1,0),(2,2),(3,4)\}$?\par
(a) $\{0,1,2,3\}$ \quad (b) $\{0,2,4\}$ \quad (c) $\{1,2,3\}$ \quad (d) $\{0,1,2,3,4\}$
\tcblower
\textcolor{green}{\bfseries Answer:} \textbf{(c) $\{1,2,3\}$}\par
\StepFig{1}{Domain means all \emph{first} coordinates of the ordered pairs: $1,2,3$.}{%
  \node[lab] at (0,1.6) {$g=\{(1,0),(2,2),(3,4)\}$};
  \node[lab] at (0,0.7) {Domain $=\{1,2,3\}$};
  \draw[strong] (-2.2,0.1) -- (2.2,0.1);
  \node[note] at (0,-0.4) {Take the left entries only: $1,2,3$.};
}
\end{QAPair}

\begin{QAPair}{Question 1 (iv) --- MCQ}
\textcolor{gold}{\bfseries Question:}
What is the x-intercept of every point on y-axis?\par
(a) $0$ \quad (b) $1$ \quad (c) $-1$ \quad (d) undefined
\tcblower
\textcolor{green}{\bfseries Answer:} \textbf{(a) $0$}\par
\StepFig{1}{Every point on the y-axis has $x=0$, so its x-intercept value is $0$.}{%
  \Axes{-3}{3}{-2}{3}{$x$}{$y$}
  \draw[strong] (0,-2) -- (0,3); % y-axis highlight
  \fill[pt] (0,2) circle(1.2pt) node[lab, right] {$(0,2)$};
  \node[note] at (0.6,-1.5) {On y-axis $\Rightarrow x=0$};
}
\end{QAPair}

\begin{QAPair}{Question 1 (v) --- MCQ}
\textcolor{gold}{\bfseries Question:}
At what point will the graph of $y=2x^2-1$ cut the y-axis?\par
(a) $\left(\pm\frac{1}{\sqrt2},0\right)$ \quad (b) $-1$ \quad (c) $(0,-1)$ \quad (d) $(-1,0)$
\tcblower
\textcolor{green}{\bfseries Answer:} \textbf{(c) $(0,-1)$}\par
\StepFig{1}{On y-axis, $x=0$. So $y=2(0)^2-1=-1$, hence point $(0,-1)$.}{%
  \Axes{-2.5}{2.5}{-2.5}{3}{$x$}{$y$}
  % parabola y=2x^2-1
  \draw[strong,domain=-1.6:1.6,samples=120] plot (\x,{2*(\x*\x)-1});
  \fill[pt] (0,-1) circle(1.2pt) node[lab, right] {$(0,-1)$};
  \node[note] at (0.2,-2.0) {y-intercept};
}
\end{QAPair}

\begin{QAPair}{Question 1 (vi) --- MCQ}
\textcolor{gold}{\bfseries Question:}
If $y=2x-1$, what is $f^{-1}(x)$?\par
(a) $\frac{1+y}{2}$ \quad (b) $2y-1$ \quad (c) $\frac{1+x}{2}$ \quad (d) $y+1$
\tcblower
\textcolor{green}{\bfseries Answer:} \textbf{(c) $\dfrac{1+x}{2}$}\par
\StepFig{1}{Swap $x,y$: $x=2y-1 \Rightarrow 2y=x+1 \Rightarrow y=\frac{x+1}{2}$.}{%
  \Axes{-2.5}{3}{-2.5}{3}{$x$}{$y$}
  \draw[helper] (-2.5,-2.5) -- (3,3) node[lab, above] {$y=x$};
  \draw[strong,domain=-2:3,samples=2] plot (\x,{2*\x-1});
  \draw[strong,domain=-2:3,samples=2] plot (\x,{(\x+1)/2});
  \node[note] at (1.4,-2.1) {Inverse is reflection in $y=x$};
}
\end{QAPair}

\begin{QAPair}{Question 1 (vii) --- MCQ}
\textcolor{gold}{\bfseries Question:}
If $f(x)=\frac{1}{2}x$, what is $f^2(x)$?\par
(a) $\frac{1}{4x}$ \quad (b) $\frac{1}{4}x$ \quad (c) $2x$ \quad (d) $\frac{1}{4}x^2$
\tcblower
\textcolor{green}{\bfseries Answer:} \textbf{(b) $\dfrac{1}{4}x$}\par
\StepFig{1}{$f^2(x)$ means $f(f(x))$: first multiply by $\frac12$, then again multiply by $\frac12$, so overall $\frac14x$.}{%
  \node[lab] (x) at (-3,0) {$x$};
  \node[lab] (fx) at (0,0) {$\dfrac{x}{2}$};
  \node[lab] (ffx) at (3,0) {$\dfrac{x}{4}$};

  \draw[strong,->] (x) -- (fx) node[midway, note, above] {$f$};
  \draw[strong,->] (fx) -- (ffx) node[midway, note, above] {$f$};
  \node[note] at (0,-1.2) {$f^2(x)=\dfrac{x}{4}$};
}
\end{QAPair}

\begin{QAPair}{Question 1 (viii) --- MCQ}
\textcolor{gold}{\bfseries Question:}
If $y=\dfrac{x}{x-2}$, for what value of $x$ will the function become undefined?\par
(a) $0$ \quad (b) $-2$ \quad (c) $2$ \quad (d) $\pm 2$
\tcblower
\textcolor{green}{\bfseries Answer:} \textbf{(c) $2$}\par
\StepFig{1}{A rational function is undefined where its denominator is $0$: $x-2=0\Rightarrow x=2$.}{%
  \Axes{-2.5}{4}{-2.5}{4}{$x$}{$y$}
  \draw[strong] (2,-2.5) -- (2,4) node[lab, above] {$x=2$};
  \node[note] at (0.3,3.2) {Vertical asymptote / undefined at $x=2$};
}
\end{QAPair}

\begin{QAPair}{Question 1 (ix) --- MCQ}
\textcolor{gold}{\bfseries Question:}
Which of the following is an exponential function?\par
(a) $\left(\frac13\right)^x$ \quad (b) $e^x$ \quad (c) $2^x$ \quad (d) All of these
\tcblower
\textcolor{green}{\bfseries Answer:} \textbf{(d) All of these}\par
\StepFig{1}{All have the variable in the exponent and a positive base not equal to $1$, so all are exponential.}{%
  \Axes{-2.2}{2.6}{-0.5}{4.2}{$x$}{$y$}
  \draw[strong,domain=-2:2.2,samples=120] plot (\x,{exp(\x)});
  \draw[strong,domain=-2:2.2,samples=120] plot (\x,{pow(2,\x)});
  \draw[strong,domain=-2:2.2,samples=120] plot (\x,{pow(1/3,\x)});
  \node[note] at (1.2,3.7) {$e^x$};
  \node[note] at (1.2,2.8) {$2^x$};
  \node[note] at (1.2,1.1) {$\left(\frac13\right)^x$};
}
\end{QAPair}

\begin{QAPair}{Question 1 (x) --- MCQ}
\textcolor{gold}{\bfseries Question:}
If $f(x)=\dfrac{2}{3}x^2-5$, what is the value of $f(-3)$?\par
(a) $0$ \quad (b) $1$ \quad (c) $-3$ \quad (d) $-1$
\tcblower
\textcolor{green}{\bfseries Answer:} \textbf{(b) $1$}\par
\StepFig{1}{Compute: $f(-3)=\frac{2}{3}(9)-5=6-5=1$.}{%
  \node[lab] at (0,1.0) {$f(-3)=\dfrac{2}{3}(-3)^2-5$};
  \node[lab] at (0,0.1) {$=\dfrac{2}{3}\cdot 9-5=6-5=1$};
  \fill[pt] (0,-0.8) circle(1.2pt);
  \node[note] at (0,-1.4) {So $f(-3)=1$};
}
\end{QAPair}

% ============================================================
% Q2
\begin{QAPair}{Question 2 --- Find inverse of $f(x)=\dfrac{3x}{2x-1}$. Where is $f^{-1}(x)$ undefined? Also find $f^{-1}(-1)$.}
\textcolor{gold}{\bfseries Working:}\par

\Step{1:} Let $y=\dfrac{3x}{2x-1}$. Multiply both sides by $(2x-1)$:
\[
y(2x-1)=3x.
\]

\Step{2:} Expand and collect $x$-terms:
\[
2xy-y=3x \quad\Rightarrow\quad 2xy-3x=y.
\]

\Step{3:} Factor out $x$ and solve:
\[
x(2y-3)=y \quad\Rightarrow\quad x=\frac{y}{2y-3}.
\]

\Step{4:} Replace $y$ by $x$:
\[
\boxed{\,f^{-1}(x)=\frac{x}{2x-3}\, }.
\]

\Step{5:} Undefined when denominator $2x-3=0$:
\[
2x-3=0 \Rightarrow \boxed{x=\frac{3}{2}}.
\]

\Step{6:} Evaluate $f^{-1}(-1)$:
\[
f^{-1}(-1)=\frac{-1}{2(-1)-3}=\frac{-1}{-5}=\boxed{\frac15}.
\]

\StepFig{7}{Sketch idea: $f^{-1}(x)$ has a vertical asymptote at $x=\frac32$.}{%
  \Axes{-1.0}{4.0}{-2.0}{3.0}{$x$}{$y$}
  \draw[strong] (1.5,-2.0) -- (1.5,3.0);
  \node[lab] at (1.5,3.3) {$x=\frac32$};
  \node[note] at (0.2,-1.6) {$f^{-1}(x)=\dfrac{x}{2x-3}$};
}

\tcblower
\textcolor{green}{\bfseries Final answers:}
\[
\boxed{f^{-1}(x)=\frac{x}{2x-3}},\qquad
\boxed{f^{-1}(x)\text{ undefined at }x=\frac32},\qquad
\boxed{f^{-1}(-1)=\frac15}.
\]
\end{QAPair}

% ============================================================
% Q3
\begin{QAPair}{Question 3 --- Sketch the graph of $y=-x^2-4x$. Will it open upward or downward?}
\textcolor{gold}{\bfseries Working:}\par
\Step{1:} Factor / complete square:
\[
y=-x^2-4x=-(x^2+4x)=-(x+2)^2+4.
\]
So the vertex is at $\boxed{(-2,4)}$.\par

\Step{2:} Intercepts:
\[
0=-x^2-4x=-x(x+4)\Rightarrow x=0 \text{ or } x=-4,
\]
so x-intercepts are $\boxed{(0,0)}$ and $\boxed{(-4,0)}$.\par

\Step{3:} Since the coefficient of $x^2$ is $-1<0$, the parabola opens \textbf{downward}.\par

\StepFig{4}{Sketch (key points and shape).}{%
  \begin{scope}[scale=0.85]
    \Axes{-6}{2.5}{-2}{5}{$x$}{$y$}
    \draw[strong,domain=-5.5:1.5,samples=160] plot (\x,{-(\x*\x)-4*\x});
    \fill[pt] (-2,4) circle(1.3pt) node[lab, above] {Vertex $(-2,4)$};
    \fill[pt] (0,0) circle(1.3pt) node[lab, right] {$(0,0)$};
    \fill[pt] (-4,0) circle(1.3pt) node[lab, left] {$(-4,0)$};
    \draw[helper] (-2,-2) -- (-2,5) node[note, above] {axis $x=-2$};
  \end{scope}
}

\tcblower
\textcolor{green}{\bfseries Answer:} The graph is a parabola opening \textbf{downward}, with vertex $(-2,4)$ and x-intercepts $(-4,0)$ and $(0,0)$.
\end{QAPair}

% ============================================================
% Q4
\begin{QAPair}{Question 4 --- Find $y=ax^2+bx+c$ cutting x-axis at $(-1,0)$ and $(1,0)$ and y-axis at $(0,10)$.}
\textcolor{gold}{\bfseries Working:}\par
\Step{1:} Roots at $x=-1$ and $x=1$ mean
\[
y=k(x+1)(x-1)=k(x^2-1).
\]

\Step{2:} Use y-intercept $(0,10)$:
\[
10=y(0)=k(0^2-1)=-k \Rightarrow k=-10.
\]

\Step{3:} Therefore
\[
\boxed{\,y=-10(x^2-1)= -10x^2+10\, }.
\]

\StepFig{4}{Sketch (crosses x-axis at $\pm1$ and y-axis at $10$).}{%
  \begin{scope}[xscale=1.2,yscale=0.22] % compress y to fit 10 nicely
    \Axes{-2}{2}{-2}{12}{$x$}{$y$}
    \draw[strong,domain=-1.6:1.6,samples=160] plot (\x,{-10*(\x*\x)+10});
    \fill[pt] (-1,0) circle(1.6pt) node[lab, below] {$(-1,0)$};
    \fill[pt] (1,0) circle(1.6pt) node[lab, below] {$(1,0)$};
    \fill[pt] (0,10) circle(1.6pt) node[lab, right] {$(0,10)$};
  \end{scope}
}

\tcblower
\textcolor{green}{\bfseries Answer:} \(\boxed{y=-10x^2+10}\).
\end{QAPair}

% ============================================================
% Q5
\begin{QAPair}{Question 5 --- For $f(x)=x^2-x-6$, find $x$ if $f(x)=f(3)$.}
\textcolor{gold}{\bfseries Working:}\par
\Step{1:} Compute $f(3)$:
\[
f(3)=3^2-3-6=9-3-6=0.
\]

\Step{2:} Set $f(x)=0$:
\[
x^2-x-6=0 \Rightarrow (x-3)(x+2)=0.
\]

\Step{3:} Solutions:
\[
\boxed{x=3 \text{ or } x=-2}.
\]

\StepFig{4}{Check on number line.}{%
  \draw[geom] (-4,0) -- (4,0);
  \foreach \x in {-4,-3,-2,-1,0,1,2,3,4}{
    \draw[geom] (\x,0.12)--(\x,-0.12);
    \node[lab] at (\x,-0.45) {\x};
  }
  \fill[pt] (-2,0) circle(1.4pt) node[lab, above] {$-2$};
  \fill[pt] (3,0) circle(1.4pt) node[lab, above] {$3$};
  \node[note] at (0,-1.2) {Both give $f(x)=0=f(3)$.};
}

\tcblower
\textcolor{green}{\bfseries Answer:} \(\boxed{x=-2 \text{ or } x=3}\).
\end{QAPair}

% ============================================================
% Q6
\begin{QAPair}{Question 6 --- Find the inverse function to convert Celsius to Fahrenheit if $C=\frac{5}{9}(F-32)$, and use it to convert $25^\circ C$ (graphically).}
\textcolor{gold}{\bfseries Working:}\par
\Step{1:} Start with
\[
C=\frac{5}{9}(F-32).
\]

\Step{2:} Solve for $F$:
\[
\frac{9}{5}C=F-32 \Rightarrow \boxed{F=\frac{9}{5}C+32}.
\]

\Step{3:} Convert $25^\circ C$:
\[
F=\frac{9}{5}(25)+32=45+32=\boxed{77^\circ F}.
\]

\StepFig{4}{Graphical idea: on the line $F=\frac95 C+32$, go to $C=25$ and read $F$.}{%
  \begin{scope}[xscale=0.12,yscale=0.05]
    % axes: C on x from 0 to 40, F on y from 20 to 110
    \draw[geom,->] (0,0) -- (42,0) node[lab, right] {$C$};
    \draw[geom,->] (0,0) -- (0,112) node[lab, above] {$F$};

    % line F = 9/5 C + 32
    \draw[strong,domain=0:40,samples=2] plot (\x,{(9/5)*\x+32});

    % mark C=25, F=77
    \draw[helper] (25,0) -- (25,77);
    \draw[helper] (0,77) -- (25,77);
    \fill[pt] (25,77) circle(2.0pt);
    \node[lab] at (27,80) {$(25,77)$};

    % labels
    \node[note] at (22,15) {$C=25$};
    \node[note] at (7,77) {$F=77$};
    \node[note] at (27,104) {$F=\frac95C+32$};
  \end{scope}
}

\tcblower
\textcolor{green}{\bfseries Answer:} \(\boxed{F=\frac{9}{5}C+32}\) and \(\boxed{25^\circ C=77^\circ F}\).
\end{QAPair}

% ============================================================
% Q7
\begin{QAPair}{Question 7 --- Find the point(s) of intersection of $f(x)=x-2$ and $g(x)=x^2-4x+2$.}
\textcolor{gold}{\bfseries Working:}\par
\Step{1:} Set $f(x)=g(x)$:
\[
x-2=x^2-4x+2 \Rightarrow 0=x^2-5x+4.
\]

\Step{2:} Factor:
\[
x^2-5x+4=(x-1)(x-4)=0 \Rightarrow x=1 \text{ or } x=4.
\]

\Step{3:} Find $y=x-2$:
\[
x=1\Rightarrow y=-1,\qquad x=4\Rightarrow y=2.
\]

\StepFig{4}{Sketch showing two intersection points.}{%
  \begin{scope}[scale=0.85]
    \Axes{-1}{6}{-3}{6}{$x$}{$y$}
    \draw[strong,domain=-0.5:5.5,samples=2] plot (\x,{\x-2}); % line
    \draw[strong,domain=-0.2:5.2,samples=180] plot (\x,{(\x*\x)-4*\x+2}); % parabola
    \fill[pt] (1,-1) circle(1.6pt) node[lab, left] {$(1,-1)$};
    \fill[pt] (4,2) circle(1.6pt) node[lab, right] {$(4,2)$};
  \end{scope}
}

\tcblower
\textcolor{green}{\bfseries Answer:} The graphs intersect at \(\boxed{(1,-1)}\) and \(\boxed{(4,2)}\).
\end{QAPair}

% ============================================================
% Q8
\begin{QAPair}{Question 8 --- Medicine remaining: $f(n)=120(0.9)^n$ mg. Find when there is $20$ mg left (using the graph).}
\textcolor{gold}{\bfseries Working:}\par
\Step{1:} Set $120(0.9)^n=20$:
\[
(0.9)^n=\frac{20}{120}=\frac{1}{6}.
\]

\Step{2:} Take logs:
\[
n=\frac{\ln(1/6)}{\ln(0.9)}\approx 17.01.
\]

So, approximately \(\boxed{n\approx 17\text{ hours}}\).\par

\StepFig{3}{Graph (decay) and the level $f(n)=20$.}{%
  \begin{scope}[xscale=0.18,yscale=0.02]
    \draw[geom,->] (0,0) -- (28,0) node[lab, right] {$n$ (hours)};
    \draw[geom,->] (0,0) -- (0,140) node[lab, above] {$f(n)$ (mg)};

    % curve f(n)=120*0.9^n
    \draw[strong,domain=0:27,samples=160] plot (\x,{120*exp(ln(0.9)*\x)});

    % horizontal line y=20
    \draw[helper] (0,20) -- (27,20);
    \node[lab] at (26,24) {$20$ mg};

    % mark approximate intersection near n=17.01
    \draw[helper] (17.01,0) -- (17.01,20);
    \fill[pt] (17.01,20) circle(2.0pt);
    \node[lab] at (18.2,28) {$\approx(17,20)$};

    % initial point
    \fill[pt] (0,120) circle(2.0pt) node[lab, right] {$(0,120)$};
  \end{scope}
}

\tcblower
\textcolor{green}{\bfseries Answer:} There will be about \(\boxed{20\text{ mg}}\) left after \(\boxed{17\text{ hours (approximately)}}\).
\end{QAPair}

% ============================================================
% Q9
\begin{QAPair}{Question 9 --- A sound wave amplitude is $A(t)=|2\cos(t)|$ (t in seconds). If $t=5$, what is the amplitude?}
\textcolor{gold}{\bfseries Working:}\par
\Step{1:} Substitute $t=5$:
\[
A(5)=|2\cos(5)|.
\]

\Step{2:} Using radians, $\cos(5)\approx 0.28366$:
\[
A(5)\approx |2(0.28366)|\approx |0.56732|=\boxed{0.567\text{ (approximately)}}.
\]

\StepFig{3}{Sketch of $A(t)=|2\cos t|$ near $t=5$ and the point $t=5$.}{%
  \begin{scope}[xscale=0.6,yscale=0.8]
    \Axes{0}{7}{-0.2}{2.4}{$t$}{$A$}
    \draw[strong,domain=0:7,samples=200] plot (\x,{abs(2*cos(\x r))}); % note: cos expects degrees, so use (\x r) for radians
    \draw[helper] (5,0) -- (5,{abs(2*cos(5 r))});
    \fill[pt] (5,{abs(2*cos(5 r))}) circle(1.6pt) node[lab, right] {$A(5)$};
    \node[note] at (4.5,2.2) {$A(t)=|2\cos t|$};
  \end{scope}
}

\tcblower
\textcolor{green}{\bfseries Answer:} \(\boxed{A(5)=|2\cos 5|\approx 0.567}\).
\end{QAPair}

\end{document}
