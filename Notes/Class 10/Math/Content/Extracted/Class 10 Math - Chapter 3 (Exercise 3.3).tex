% !TEX TS-program = pdflatex
\documentclass[11pt]{article}

% -------------------- Packages --------------------
\usepackage[a4paper,margin=1in]{geometry}
\usepackage{amsmath,amssymb}
\usepackage[T1]{fontenc} % (fixed: was "fontec")
\usepackage{lmodern}
\usepackage{xcolor}
\usepackage{tcolorbox}
\tcbuselibrary{skins,breakable}
\usepackage{enumitem}
\usepackage{hyperref}

\pagestyle{empty}

% -------------------- Dark Theme Colors --------------------
\definecolor{bg}{HTML}{000000}
\definecolor{pairbg}{HTML}{121212}
\definecolor{solbg}{HTML}{0A0A0A}
\definecolor{border}{HTML}{2A2A2A}
\definecolor{text}{HTML}{FFFFFF}
\definecolor{muted}{HTML}{C9CDD3}
\definecolor{gold}{HTML}{FFD700}
\definecolor{green}{HTML}{4ADE80}
\definecolor{cyan}{HTML}{38BDF8}

\pagecolor{bg}
\color{text}

\hypersetup{
  colorlinks=true,
  linkcolor=cyan,
  urlcolor=cyan
}

\setlength{\parindent}{0pt}
\setlength{\parskip}{10pt}

\setlist[itemize]{left=1.4em,itemsep=6pt,topsep=6pt}
\setlist[enumerate]{left=1.6em,itemsep=4pt,topsep=4pt}

% -------------------- tcolorbox Base --------------------
\tcbset{
  enhanced,
  breakable,
  arc=12pt,
  boxrule=0.8pt,
  left=16pt,right=16pt,top=12pt,bottom=12pt
}

\newtcolorbox{QAPair}[1]{%
  colback=pairbg,
  colbacklower=solbg,
  colframe=border,
  coltext=text,
  title=\textcolor{gold}{\bfseries #1},
  fonttitle=\bfseries,
  coltitle=text,
  segmentation style={draw=border, dashed, line width=0.6pt},
}

% Visible text inside this box (fix)
\newtcolorbox{QuickBox}{%
  colback=pairbg,
  colframe=cyan,
  coltext=text,
  fontupper=\color{text},
  borderline north={4pt}{0pt}{cyan},
  arc=14pt,
  boxrule=0.8pt
}

% Helper for step headings
\newcommand{\Step}[1]{\textcolor{muted}{\textbf{Step #1:}}}

% ============================================================
\begin{document}

\begin{center}
{\LARGE\bfseries \textcolor{gold}{Exercise 3.3 --- Solutions}}\\[-2pt]
\end{center}

\begin{QuickBox}
{\color{cyan}\bfseries Quick formulas (useful)}\par\medskip
\begin{itemize}
\item For $A=\begin{bmatrix}a&b\\ c&d\end{bmatrix}$: \quad $\det(A)=ad-bc$.
\item If $\det(A)\neq 0$ then $A^{-1}=\dfrac{1}{\det(A)}\begin{bmatrix}d&-b\\ -c&a\end{bmatrix}$.
\item $\operatorname{adj}(A)=\begin{bmatrix}d&-b\\ -c&a\end{bmatrix}$ and \ $\operatorname{adj}(A)=\det(A)\,A^{-1}$ (when $\det(A)\neq 0$).
\item For $2\times 2$: \ $\det(\operatorname{adj}(A))=\det(A)$.
\item Cramer's rule (for $2\times2$): if $\det(A)\neq0$, then $x=\dfrac{D_x}{D},\ y=\dfrac{D_y}{D}$.
\end{itemize}
\end{QuickBox}

% ============================================================
% Q1
\begin{QAPair}{Question 1 (a)}
\textcolor{gold}{\bfseries Question:} Mention singular and nonsingular matrices from the given ones.\\
$A=\begin{bmatrix}6&4\\2&3\end{bmatrix},\ 
B=\begin{bmatrix}-5&7\\-9&4\end{bmatrix},\ 
C=\begin{bmatrix}9&18\\2&4\end{bmatrix},\ 
D=\begin{bmatrix}a&b+\frac1a\\ a&b\end{bmatrix},\ 
E=\begin{bmatrix}a&b+c\\ a&d+c\end{bmatrix}\ (a\neq 0,\ b=d).$
\tcblower
\textcolor{green}{\bfseries Answer:}
\[
\begin{aligned}
\Step{1}\;& \det(A)=6\cdot3-4\cdot2=18-8=10\neq0 \ \Rightarrow\ A\ \text{is nonsingular}.\\[4pt]
\Step{2}\;& \det(B)=(-5)\cdot4-7\cdot(-9)=-20+63=43\neq0 \ \Rightarrow\ B\ \text{is nonsingular}.\\[4pt]
\Step{3}\;& \det(C)=9\cdot4-18\cdot2=36-36=0 \ \Rightarrow\ C\ \text{is singular}.\\[4pt]
\Step{4}\;& \det(D)=a\cdot b-a\left(b+\frac1a\right)=ab-(ab+1)=-1\neq0 \ \Rightarrow\ D\ \text{is nonsingular}.\\[4pt]
\Step{5}\;& \det(E)=a(d+c)-a(b+c)=a(d-b)=0\ \ (\because d=b)\ \Rightarrow\ E\ \text{is singular}.
\end{aligned}
\]
\[
\boxed{\text{Nonsingular: }A,B,D \qquad \text{Singular: }C,E.}
\]
\end{QAPair}

\begin{QAPair}{Question 1 (b)}
\textcolor{gold}{\bfseries Question:} If $\lvert P\rvert=9$ and $P=\begin{bmatrix}3&3\\1&4k\end{bmatrix}$, find $k$.\\
\tcblower
\textcolor{green}{\bfseries Answer:}
\[
\begin{aligned}
\Step{1}\;& \det(P)=3(4k)-3(1)=12k-3.\\
\Step{2}\;& 12k-3=9 \ \Rightarrow\ 12k=12 \ \Rightarrow\ k=1.
\end{aligned}
\]
\[
\boxed{k=1.}
\]
\end{QAPair}

\begin{QAPair}{Question 1 (c)}
\textcolor{gold}{\bfseries Question:} If $\lvert T\rvert=3$ and $\operatorname{adj}T=\begin{bmatrix}5&x\\3&2\end{bmatrix}$, find $x$.\\
\tcblower
\textcolor{green}{\bfseries Answer:}
\[
\begin{aligned}
\Step{1}\;& \text{For a }2\times2\text{ matrix, }\det(\operatorname{adj}T)=\det(T).\\
\Step{2}\;& \det\!\left(\begin{bmatrix}5&x\\3&2\end{bmatrix}\right)=5\cdot2-x\cdot3=10-3x.\\
\Step{3}\;& 10-3x=3 \ \Rightarrow\ 3x=7 \ \Rightarrow\ x=\frac{7}{3}.
\end{aligned}
\]
\[
\boxed{x=\frac{7}{3}.}
\]
\end{QAPair}

% ============================================================
% Q2
\begin{QAPair}{Question 2 (a)}
\textcolor{gold}{\bfseries Question:} Find the multiplicative inverses of these matrices if possible:\\
$R=\begin{bmatrix}4&-1\\-6&2\end{bmatrix},\ 
S=\begin{bmatrix}5\\3\end{bmatrix},\ 
T=\begin{bmatrix}25&2\\50&4\end{bmatrix},\ 
U=\begin{bmatrix}x&x+1\\y&y+1\end{bmatrix}\ \text{if }x=y.$
\tcblower
\textcolor{green}{\bfseries Answer:}
\[
\begin{aligned}
\Step{1}\;& \det(R)=4\cdot2-(-1)(-6)=8-6=2\neq0 \Rightarrow R^{-1}\ \text{exists}.\\
& R^{-1}=\frac{1}{2}\begin{bmatrix}2&1\\6&4\end{bmatrix}.\\[6pt]
\Step{2}\;& S\ \text{is }2\times1\ (\text{not square})\Rightarrow \boxed{S^{-1}\ \text{does not exist}}.\\[6pt]
\Step{3}\;& \det(T)=25\cdot4-2\cdot50=100-100=0 \Rightarrow \boxed{T^{-1}\ \text{does not exist}}.\\[6pt]
\Step{4}\;& \det(U)=x(y+1)-(x+1)y=x-y. \ \text{If }x=y,\ \det(U)=0\\
& \Rightarrow \boxed{U^{-1}\ \text{does not exist (when }x=y\text{)}}.
\end{aligned}
\]
\end{QAPair}

\begin{QAPair}{Question 2 (b)}
\textcolor{gold}{\bfseries Question:} If $R=\begin{bmatrix}4&1\\-6&2\end{bmatrix}$, verify that $RR^{-1}=R^{-1}R=I$.\\
\tcblower
\textcolor{green}{\bfseries Answer:}
\[
\begin{aligned}
\Step{1}\;& \det(R)=4\cdot2-1(-6)=8+6=14\neq0.\\
\Step{2}\;& R^{-1}=\frac{1}{14}\begin{bmatrix}2&-1\\6&4\end{bmatrix}.\\
\Step{3}\;& RR^{-1}
=\frac{1}{14}\begin{bmatrix}4&1\\-6&2\end{bmatrix}\begin{bmatrix}2&-1\\6&4\end{bmatrix}
=\frac{1}{14}\begin{bmatrix}14&0\\0&14\end{bmatrix}
=\begin{bmatrix}1&0\\0&1\end{bmatrix}=I.\\
\Step{4}\;& R^{-1}R
=\frac{1}{14}\begin{bmatrix}2&-1\\6&4\end{bmatrix}\begin{bmatrix}4&1\\-6&2\end{bmatrix}
=\frac{1}{14}\begin{bmatrix}14&0\\0&14\end{bmatrix}=I.
\end{aligned}
\]
\[
\boxed{RR^{-1}=R^{-1}R=I.}
\]
\end{QAPair}

% ============================================================
% Q3
\begin{QAPair}{Question 3}
\textcolor{gold}{\bfseries Question:} If $Y=\begin{bmatrix}1&2\\2&3\end{bmatrix}$ and $Z=\begin{bmatrix}2&3\\1&2\end{bmatrix}$, verify that $(YZ)^{-1}=Z^{-1}Y^{-1}$.\\
\tcblower
\textcolor{green}{\bfseries Answer:}
\[
\begin{aligned}
\Step{1}\;& YZ=\begin{bmatrix}1&2\\2&3\end{bmatrix}\begin{bmatrix}2&3\\1&2\end{bmatrix}
=\begin{bmatrix}4&7\\7&12\end{bmatrix}.\\
\Step{2}\;& \det(YZ)=4\cdot12-7\cdot7=48-49=-1,\\
& (YZ)^{-1}=\frac{1}{-1}\begin{bmatrix}12&-7\\-7&4\end{bmatrix}
=\begin{bmatrix}-12&7\\7&-4\end{bmatrix}.\\
\Step{3}\;& \det(Y)=1\cdot3-2\cdot2=-1 \Rightarrow
Y^{-1}=\frac{1}{-1}\begin{bmatrix}3&-2\\-2&1\end{bmatrix}
=\begin{bmatrix}-3&2\\2&-1\end{bmatrix}.\\
\Step{4}\;& \det(Z)=2\cdot2-3\cdot1=1 \Rightarrow
Z^{-1}=\begin{bmatrix}2&-3\\-1&2\end{bmatrix}.\\
\Step{5}\;& Z^{-1}Y^{-1}
=\begin{bmatrix}2&-3\\-1&2\end{bmatrix}\begin{bmatrix}-3&2\\2&-1\end{bmatrix}
=\begin{bmatrix}-12&7\\7&-4\end{bmatrix}.
\end{aligned}
\]
\[
\boxed{(YZ)^{-1}=\begin{bmatrix}-12&7\\7&-4\end{bmatrix}=Z^{-1}Y^{-1}.}
\]
\end{QAPair}

% ============================================================
% Q4
\begin{QAPair}{Question 4}
\textcolor{gold}{\bfseries Question:} What is relation among $\lvert A\rvert$, $\lvert A^{-1}\rvert$ and $\lvert \operatorname{adj}A\rvert$ if
$A=\begin{bmatrix}5&7\\2&3\end{bmatrix}$?\\
\tcblower
\textcolor{green}{\bfseries Answer:}
\[
\begin{aligned}
\Step{1}\;& \det(A)=5\cdot3-7\cdot2=15-14=1.\\
\Step{2}\;& \det(A^{-1})=\frac{1}{\det(A)}=\frac{1}{1}=1.\\
\Step{3}\;& \operatorname{adj}(A)=\begin{bmatrix}3&-7\\-2&5\end{bmatrix},\quad
\det(\operatorname{adj}(A))=3\cdot5-(-7)(-2)=15-14=1.\\
\end{aligned}
\]
\[
\boxed{\det(A)=1,\quad \det(A^{-1})=\frac{1}{\det(A)}=1,\quad \det(\operatorname{adj}(A))=\det(A)=1.}
\]
\end{QAPair}

% ============================================================
% Q5
\begin{QAPair}{Question 5}
\textcolor{gold}{\bfseries Question:} Write these matrix equations into system of linear equations if possible.\\
(i) $\begin{bmatrix}4&2\\3&1\end{bmatrix}\begin{bmatrix}x\\y\end{bmatrix}=\begin{bmatrix}6\\4\end{bmatrix}$ \qquad
(ii) $\begin{bmatrix}5&0\\0&4\end{bmatrix}\begin{bmatrix}x\\y\end{bmatrix}=\begin{bmatrix}10\\20\end{bmatrix}$\\
(iii) $\begin{bmatrix}0\\3\end{bmatrix}\times\begin{bmatrix}x\\y\end{bmatrix}=\begin{bmatrix}12\\24\end{bmatrix}$ \qquad
(iv) $\begin{bmatrix}x&y\end{bmatrix}\begin{bmatrix}5\\3\end{bmatrix}=\begin{bmatrix}2\end{bmatrix}$\\
\tcblower
\textcolor{green}{\bfseries Answer:}
\[
\begin{aligned}
\Step{1}\;& \text{(i)}\quad
\begin{bmatrix}4&2\\3&1\end{bmatrix}\begin{bmatrix}x\\y\end{bmatrix}
=\begin{bmatrix}6\\4\end{bmatrix}
\ \Rightarrow\
\begin{cases}
4x+2y=6,\\
3x+y=4.
\end{cases}\\[8pt]
\Step{2}\;& \text{(ii)}\quad
\begin{bmatrix}5&0\\0&4\end{bmatrix}\begin{bmatrix}x\\y\end{bmatrix}
=\begin{bmatrix}10\\20\end{bmatrix}
\ \Rightarrow\
\begin{cases}
5x=10,\\
4y=20.
\end{cases}\\[8pt]
\Step{3}\;& \text{(iii)}\quad
\begin{bmatrix}0\\3\end{bmatrix}\times\begin{bmatrix}x\\y\end{bmatrix}
\ \text{is }(2\times1)\times(2\times1)\ \text{which is not defined.}\\
&\Rightarrow\ \boxed{\text{Not possible (incompatible dimensions).}}\\[8pt]
\Step{4}\;& \text{(iv)}\quad
\begin{bmatrix}x&y\end{bmatrix}\begin{bmatrix}5\\3\end{bmatrix}=\begin{bmatrix}2\end{bmatrix}
\ \Rightarrow\ 5x+3y=2.
\end{aligned}
\]
\end{QAPair}

% ============================================================
% Q6
\begin{QAPair}{Question 6 (a)}
\textcolor{gold}{\bfseries Question:} Write the systems of linear equations in matrix form.\\
(i) $x+y=2,\ x-y=4$ \qquad
(ii) $2x+y=90,\ 5y-x=10$\\
(iii) $y=3,\ x=4$ \qquad
(iv) $\frac{5}{2}x-3y=1,\ \frac{1}{2}y-4x=2$\\
\tcblower
\textcolor{green}{\bfseries Answer:}
\[
\begin{aligned}
\Step{1}\;& \text{(i)}\quad
\begin{bmatrix}1&1\\1&-1\end{bmatrix}\begin{bmatrix}x\\y\end{bmatrix}
=\begin{bmatrix}2\\4\end{bmatrix}.\\[8pt]
\Step{2}\;& \text{(ii)}\quad
\begin{bmatrix}2&1\\-1&5\end{bmatrix}\begin{bmatrix}x\\y\end{bmatrix}
=\begin{bmatrix}90\\10\end{bmatrix}.\\[8pt]
\Step{3}\;& \text{(iii)}\quad
\begin{bmatrix}0&1\\1&0\end{bmatrix}\begin{bmatrix}x\\y\end{bmatrix}
=\begin{bmatrix}3\\4\end{bmatrix}.\\[8pt]
\Step{4}\;& \text{(iv)}\quad
\begin{bmatrix}\frac{5}{2}&-3\\-4&\frac{1}{2}\end{bmatrix}\begin{bmatrix}x\\y\end{bmatrix}
=\begin{bmatrix}1\\2\end{bmatrix}.
\end{aligned}
\]
\end{QAPair}

\begin{QAPair}{Question 6 (b)}
\textcolor{gold}{\bfseries Question:} If the matrix of coefficients of $5x-4y=30$ and $10x-ky=60$ is singular, find $k$.\\
\tcblower
\textcolor{green}{\bfseries Answer:}
\[
\begin{aligned}
\Step{1}\;& A=\begin{bmatrix}5&-4\\10&-k\end{bmatrix},\quad
\det(A)=5(-k)-(-4)(10)=-5k+40.\\
\Step{2}\;& \text{Singular } \Rightarrow \det(A)=0 \Rightarrow -5k+40=0 \Rightarrow k=8.
\end{aligned}
\]
\[
\boxed{k=8.}
\]
\end{QAPair}

% ============================================================
% Q7(a)
\begin{QAPair}{Question 7 (a) (i)}
\textcolor{gold}{\bfseries Question:} Solve by Matrix Inversion Method:
\[
4x+3y=-6,\qquad x+2y=1.
\]
\tcblower
\textcolor{green}{\bfseries Answer:}
\[
\begin{aligned}
\Step{1}\;&
\begin{bmatrix}4&3\\1&2\end{bmatrix}\begin{bmatrix}x\\y\end{bmatrix}
=\begin{bmatrix}-6\\1\end{bmatrix},
\quad
\det=4\cdot2-3\cdot1=5\neq0.\\
\Step{2}\;&
\begin{bmatrix}4&3\\1&2\end{bmatrix}^{-1}
=\frac{1}{5}\begin{bmatrix}2&-3\\-1&4\end{bmatrix}.\\
\Step{3}\;&
\begin{bmatrix}x\\y\end{bmatrix}
=\frac{1}{5}\begin{bmatrix}2&-3\\-1&4\end{bmatrix}\begin{bmatrix}-6\\1\end{bmatrix}
=\frac{1}{5}\begin{bmatrix}-15\\10\end{bmatrix}
=\begin{bmatrix}-3\\2\end{bmatrix}.
\end{aligned}
\]
\[
\boxed{x=-3,\ y=2.}
\]
\end{QAPair}

\begin{QAPair}{Question 7 (a) (ii)}
\textcolor{gold}{\bfseries Question:} Solve by Matrix Inversion Method:
\[
-x+2y=1.5,\qquad 5x+4y=3.
\]
\tcblower
\textcolor{green}{\bfseries Answer:}
\[
\begin{aligned}
\Step{1}\;&
\begin{bmatrix}-1&2\\5&4\end{bmatrix}\begin{bmatrix}x\\y\end{bmatrix}
=\begin{bmatrix}\frac32\\3\end{bmatrix},
\quad
\det=(-1)\cdot4-2\cdot5=-14\neq0.\\
\Step{2}\;&
\begin{bmatrix}-1&2\\5&4\end{bmatrix}^{-1}
=\frac{1}{-14}\begin{bmatrix}4&-2\\-5&-1\end{bmatrix}
=\frac{1}{14}\begin{bmatrix}-4&2\\5&1\end{bmatrix}.\\
\Step{3}\;&
\begin{bmatrix}x\\y\end{bmatrix}
=\frac{1}{14}\begin{bmatrix}-4&2\\5&1\end{bmatrix}\begin{bmatrix}\frac32\\3\end{bmatrix}
=\frac{1}{14}\begin{bmatrix}0\\ \frac{21}{2}\end{bmatrix}
=\begin{bmatrix}0\\\frac34\end{bmatrix}.
\end{aligned}
\]
\[
\boxed{x=0,\ y=\frac34.}
\]
\end{QAPair}

\begin{QAPair}{Question 7 (a) (iii)}
\textcolor{gold}{\bfseries Question:} Solve by Matrix Inversion Method:
\[
2y=10-16x,\qquad 24x=15-3y.
\]
\tcblower
\textcolor{green}{\bfseries Answer:}
\[
\begin{aligned}
\Step{1}\;& 2y=10-16x \Rightarrow 16x+2y=10,\\
& 24x=15-3y \Rightarrow 24x+3y=15.\\
\Step{2}\;& \text{But } \ (24x+3y=15)=\frac{3}{2}(16x+2y=10), \ \text{so the equations are dependent}.\\
\Step{3}\;& \det\!\begin{bmatrix}16&2\\24&3\end{bmatrix}=16\cdot3-2\cdot24=48-48=0
\Rightarrow \text{inverse does not exist.}
\end{aligned}
\]
\[
\boxed{\text{No unique solution (infinitely many solutions): } 8x+y=5 \ \Rightarrow\ y=5-8x.}
\]
\end{QAPair}

\begin{QAPair}{Question 7 (a) (iv)}
\textcolor{gold}{\bfseries Question:} Solve by Matrix Inversion Method:
\[
9-x=7y,\qquad 14y+2x=18.
\]
\tcblower
\textcolor{green}{\bfseries Answer:}
\[
\begin{aligned}
\Step{1}\;& 9-x=7y \Rightarrow x+7y=9,\\
& 14y+2x=18 \Rightarrow x+7y=9.\\
\Step{2}\;& \text{Both equations are the same }\Rightarrow \det=0 \text{ and inverse method is not applicable.}
\end{aligned}
\]
\[
\boxed{\text{Infinitely many solutions: } x+7y=9 \ \Rightarrow\ x=9-7y.}
\]
\end{QAPair}

% ============================================================
% Q7(b)
\begin{QAPair}{Question 7 (b) (i)}
\textcolor{gold}{\bfseries Question:} Use Cramer's rule:
\[
2x+3y=5,\qquad 5x+10y=10.
\]
\tcblower
\textcolor{green}{\bfseries Answer:}
\[
\begin{aligned}
\Step{1}\;& D=\begin{vmatrix}2&3\\5&10\end{vmatrix}=20-15=5\neq0.\\
\Step{2}\;& D_x=\begin{vmatrix}5&3\\10&10\end{vmatrix}=50-30=20,\quad
D_y=\begin{vmatrix}2&5\\5&10\end{vmatrix}=20-25=-5.\\
\Step{3}\;& x=\frac{D_x}{D}=\frac{20}{5}=4,\quad y=\frac{D_y}{D}=\frac{-5}{5}=-1.
\end{aligned}
\]
\[
\boxed{x=4,\ y=-1.}
\]
\end{QAPair}

\begin{QAPair}{Question 7 (b) (ii)}
\textcolor{gold}{\bfseries Question:} Use Cramer's rule:
\[
x=\frac{2}{3}-2y,\qquad 4y=3-3x.
\]
\tcblower
\textcolor{green}{\bfseries Answer:}
\[
\begin{aligned}
\Step{1}\;& x+2y=\frac{2}{3},\qquad 3x+4y=3.\\
\Step{2}\;& D=\begin{vmatrix}1&2\\3&4\end{vmatrix}=4-6=-2\neq0.\\
\Step{3}\;& D_x=\begin{vmatrix}\frac{2}{3}&2\\3&4\end{vmatrix}=\frac{8}{3}-6=-\frac{10}{3},\quad
D_y=\begin{vmatrix}1&\frac{2}{3}\\3&3\end{vmatrix}=3-2=1.\\
\Step{4}\;& x=\frac{D_x}{D}=\frac{-\frac{10}{3}}{-2}=\frac{5}{3},\quad
y=\frac{D_y}{D}=\frac{1}{-2}=-\frac{1}{2}.
\end{aligned}
\]
\[
\boxed{x=\frac{5}{3},\ y=-\frac{1}{2}.}
\]
\end{QAPair}

\begin{QAPair}{Question 7 (b) (iii)}
\textcolor{gold}{\bfseries Question:} Use Cramer's rule:
\[
\frac{6}{10}x+\frac{8}{10}y=20,\qquad \frac{8}{10}x-\frac{6}{10}y=10.
\]
\tcblower
\textcolor{green}{\bfseries Answer:}
\[
\begin{aligned}
\Step{1}\;& 6x+8y=200,\qquad 8x-6y=100.\\
\Step{2}\;& D=\begin{vmatrix}6&8\\8&-6\end{vmatrix}=6(-6)-8\cdot8=-36-64=-100\neq0.\\
\Step{3}\;& D_x=\begin{vmatrix}200&8\\100&-6\end{vmatrix}=200(-6)-8(100)=-1200-800=-2000.\\
& D_y=\begin{vmatrix}6&200\\8&100\end{vmatrix}=6(100)-200(8)=600-1600=-1000.\\
\Step{4}\;& x=\frac{D_x}{D}=\frac{-2000}{-100}=20,\quad y=\frac{D_y}{D}=\frac{-1000}{-100}=10.
\end{aligned}
\]
\[
\boxed{x=20,\ y=10.}
\]
\end{QAPair}

\begin{QAPair}{Question 7 (b) (iv)}
\textcolor{gold}{\bfseries Question:} Use Cramer's rule:
\[
16x-10+2y=0,\qquad 15-3y-24x=0.
\]
\tcblower
\textcolor{green}{\bfseries Answer:}
\[
\begin{aligned}
\Step{1}\;& 16x+2y=10,\qquad 24x+3y=15.\\
\Step{2}\;& D=\begin{vmatrix}16&2\\24&3\end{vmatrix}=48-48=0.\\
\Step{3}\;& D_x=\begin{vmatrix}10&2\\15&3\end{vmatrix}=30-30=0,\quad
D_y=\begin{vmatrix}16&10\\24&15\end{vmatrix}=240-240=0.
\end{aligned}
\]
\[
\boxed{D=0\ \text{and}\ D_x=D_y=0 \Rightarrow \text{infinitely many solutions: } 8x+y=5\ (\Rightarrow y=5-8x).}
\]
\end{QAPair}

\end{document}
