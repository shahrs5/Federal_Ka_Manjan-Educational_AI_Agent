% !TEX TS-program = pdflatex
\documentclass[11pt]{article}

% -------------------- Packages --------------------
\usepackage[a4paper,margin=1in]{geometry}
\usepackage{amsmath,amssymb}
\usepackage[T1]{fontenc}
\usepackage{lmodern}
\usepackage{xcolor}
\usepackage{tcolorbox}
\tcbuselibrary{skins,breakable}
\usepackage{enumitem}
\usepackage{hyperref}

\pagestyle{empty}

% -------------------- Dark Theme Colors --------------------
\definecolor{bg}{HTML}{000000}
\definecolor{pairbg}{HTML}{121212}
\definecolor{solbg}{HTML}{0A0A0A}
\definecolor{border}{HTML}{2A2A2A}
\definecolor{text}{HTML}{FFFFFF}
\definecolor{muted}{HTML}{C9CDD3}
\definecolor{gold}{HTML}{FFD700}
\definecolor{green}{HTML}{4ADE80}
\definecolor{cyan}{HTML}{38BDF8}

\pagecolor{bg}
\color{text}

\hypersetup{
  colorlinks=true,
  linkcolor=cyan,
  urlcolor=cyan
}

\setlength{\parindent}{0pt}
\setlength{\parskip}{10pt}

\setlist[itemize]{left=1.4em,itemsep=6pt,topsep=6pt}
\setlist[enumerate]{left=1.6em,itemsep=4pt,topsep=4pt}

% -------------------- tcolorbox Base --------------------
\tcbset{
  enhanced,
  breakable,
  arc=12pt,
  boxrule=0.8pt,
  left=16pt,right=16pt,top=12pt,bottom=12pt
}

\newtcolorbox{QAPair}[1]{%
  colback=pairbg,
  colbacklower=solbg,
  colframe=border,
  coltext=text,
  title=\textcolor{gold}{\bfseries #1},
  fonttitle=\bfseries,
  coltitle=text,
  segmentation style={draw=border, dashed, line width=0.6pt},
}

% --- FIX APPLIED HERE: Added 'enhanced' and 'width=\linewidth' ---
\newtcolorbox{QuickBox}{%
  enhanced, 
  colback=pairbg,
  colframe=cyan,
  coltext=text,
  fontupper=\color{text},
  borderline north={4pt}{0pt}{cyan},
  arc=14pt,
  boxrule=0.8pt,
  left=10pt,
  right=10pt,
  width=\linewidth,
  breakable
}

% Helper for step headings
\newcommand{\Step}[1]{\textcolor{muted}{\textbf{Step #1:}}}

% ============================================================
\begin{document}

\begin{center}
{\LARGE\bfseries \textcolor{gold}{Exercise 2.3 --- Solutions}}\\[-2pt]
\end{center}

\begin{QuickBox}
{\color{cyan}\bfseries Quick formulas (useful)}\par\medskip
\begin{itemize}
\item For $ax^2+bx+c=0$, the \textbf{discriminant} is $\Delta=b^2-4ac$.
\item \textbf{Nature of roots:}
% --- FIX APPLIED HERE: Used 'aligned' to split the long equation ---
\[
\begin{aligned}
\Delta>0 &\Rightarrow \text{two distinct real roots},\\
\Delta=0 &\Rightarrow \text{real and equal roots},\\
\Delta<0 &\Rightarrow \text{non-real (imaginary) roots}.
\end{aligned}
\]
\item A quadratic is a \textbf{perfect square} (like $(px+q)^2$) when $\Delta=0$.
\end{itemize}
\end{QuickBox}

% ============================================================
% Q1
\begin{QAPair}{Question 1 (i)}
\textcolor{gold}{\bfseries Question:} Find the discriminant of $x^2+6x-27=0$.\\
\tcblower
\textcolor{green}{\bfseries Answer:}
For $ax^2+bx+c=0$, $\Delta=b^2-4ac$.
\[
\begin{aligned}
\Step{1}\;& a=1,\; b=6,\; c=-27.\\
\Step{2}\;& \Delta = 6^2-4(1)(-27)=36+108=144.
\end{aligned}
\]
\[
\boxed{\Delta=144}
\]
\end{QAPair}

\begin{QAPair}{Question 1 (ii)}
\textcolor{gold}{\bfseries Question:} Find the discriminant of $x^2-x-12=0$.\\
\tcblower
\textcolor{green}{\bfseries Answer:}
\[
\begin{aligned}
\Step{1}\;& a=1,\; b=-1,\; c=-12.\\
\Step{2}\;& \Delta = (-1)^2-4(1)(-12)=1+48=49.
\end{aligned}
\]
\[
\boxed{\Delta=49}
\]
\end{QAPair}

\begin{QAPair}{Question 1 (iii)}
\textcolor{gold}{\bfseries Question:} Find the discriminant of $8x^2+2x+1=0$.\\
\tcblower
\textcolor{green}{\bfseries Answer:}
\[
\begin{aligned}
\Step{1}\;& a=8,\; b=2,\; c=1.\\
\Step{2}\;& \Delta = 2^2-4(8)(1)=4-32=-28.
\end{aligned}
\]
\[
\boxed{\Delta=-28}
\]
\end{QAPair}

\begin{QAPair}{Question 1 (iv)}
\textcolor{gold}{\bfseries Question:} Find the discriminant of $12x^2-11x-15=0$.\\
\tcblower
\textcolor{green}{\bfseries Answer:}
\[
\begin{aligned}
\Step{1}\;& a=12,\; b=-11,\; c=-15.\\
\Step{2}\;& \Delta = (-11)^2-4(12)(-15)=121+720=841.
\end{aligned}
\]
\[
\boxed{\Delta=841}
\]
\end{QAPair}

% ============================================================
% Q2
\begin{QAPair}{Question 2 (i)}
\textcolor{gold}{\bfseries Question:} Discuss the nature of roots of $x^2-2x-15=0$.\\
\tcblower
\textcolor{green}{\bfseries Answer:}
\[
\begin{aligned}
\Step{1}\;& a=1,\; b=-2,\; c=-15.\\
\Step{2}\;& \Delta=b^2-4ac=(-2)^2-4(1)(-15)=4+60=64>0.
\end{aligned}
\]
\[
\boxed{\text{Roots are real and distinct.}}
\]
\end{QAPair}

\begin{QAPair}{Question 2 (ii)}
\textcolor{gold}{\bfseries Question:} Discuss the nature of roots of $x^2+3x-4=0$.\\
\tcblower
\textcolor{green}{\bfseries Answer:}
\[
\begin{aligned}
\Step{1}\;& a=1,\; b=3,\; c=-4.\\
\Step{2}\;& \Delta=3^2-4(1)(-4)=9+16=25>0.
\end{aligned}
\]
\[
\boxed{\text{Roots are real and distinct.}}
\]
\end{QAPair}

\begin{QAPair}{Question 2 (iii)}
\textcolor{gold}{\bfseries Question:} Discuss the nature of roots of $12x^2+x-20=0$.\\
\tcblower
\textcolor{green}{\bfseries Answer:}
\[
\begin{aligned}
\Step{1}\;& a=12,\; b=1,\; c=-20.\\
\Step{2}\;& \Delta=1^2-4(12)(-20)=1+960=961>0.
\end{aligned}
\]
\[
\boxed{\text{Roots are real and distinct.}}
\]
\end{QAPair}

\begin{QAPair}{Question 2 (iv)}
\textcolor{gold}{\bfseries Question:} Discuss the nature of roots of $x^2+2x+8=0$.\\
\tcblower
\textcolor{green}{\bfseries Answer:}
\[
\begin{aligned}
\Step{1}\;& a=1,\; b=2,\; c=8.\\
\Step{2}\;& \Delta=2^2-4(1)(8)=4-32=-28<0.
\end{aligned}
\]
\[
\boxed{\text{Roots are non-real (imaginary).}}
\]
\end{QAPair}

\begin{QAPair}{Question 2 (v)}
\textcolor{gold}{\bfseries Question:} Discuss the nature of roots of $x^2+3x-9=0$.\\
\tcblower
\textcolor{green}{\bfseries Answer:}
\[
\begin{aligned}
\Step{1}\;& a=1,\; b=3,\; c=-9.\\
\Step{2}\;& \Delta=3^2-4(1)(-9)=9+36=45>0.
\end{aligned}
\]
\[
\boxed{\text{Roots are real and distinct.}}
\]
\end{QAPair}

% ============================================================
% Q3
\begin{QAPair}{Question 3}
\textcolor{gold}{\bfseries Question:} For what value of $k$, $9x^2-kx+16=0$ is a perfect square?\\
\tcblower
\textcolor{green}{\bfseries Answer:}
A quadratic is a perfect square when $\Delta=0$.
\[
\begin{aligned}
\Step{1}\;& a=9,\; b=-k,\; c=16.\\
\Step{2}\;& \Delta=b^2-4ac=(-k)^2-4(9)(16)=k^2-576.\\
\Step{3}\;& \Delta=0 \Rightarrow k^2-576=0 \Rightarrow k^2=576 \Rightarrow k=\pm 24.
\end{aligned}
\]
\[
\boxed{k=24\ \text{or}\ k=-24}
\]
\end{QAPair}

% ============================================================
% Q4
\begin{QAPair}{Question 4}
\textcolor{gold}{\bfseries Question:} If roots of $x^2+kx+9=0$ are equal, find $k$.\\
\tcblower
\textcolor{green}{\bfseries Answer:}
Equal roots $\Rightarrow \Delta=0$.
\[
\begin{aligned}
\Step{1}\;& a=1,\; b=k,\; c=9.\\
\Step{2}\;& \Delta=k^2-4(1)(9)=k^2-36.\\
\Step{3}\;& \Delta=0 \Rightarrow k^2-36=0 \Rightarrow k^2=36 \Rightarrow k=\pm 6.
\end{aligned}
\]
\[
\boxed{k=6\ \text{or}\ k=-6}
\]
\end{QAPair}

% ============================================================
% Q5
\begin{QAPair}{Question 5}
\textcolor{gold}{\bfseries Question:} Show that the roots of $2x^2+(mx-1)^2=3$ are equal if $3m^2+4=0$.\\
\tcblower
\textcolor{green}{\bfseries Answer:}
\[
\begin{aligned}
\Step{1}\;& 2x^2+(mx-1)^2=3\\
&\Rightarrow 2x^2+(m^2x^2-2mx+1)=3\\
&\Rightarrow (m^2+2)x^2-2mx-2=0.
\end{aligned}
\]
Now compute the discriminant:
\[
\begin{aligned}
\Step{2}\;& a=m^2+2,\; b=-2m,\; c=-2.\\
\Step{3}\;& \Delta=b^2-4ac=(-2m)^2-4(m^2+2)(-2)\\
&=4m^2+8(m^2+2)=12m^2+16=4(3m^2+4).
\end{aligned}
\]
\[
\Step{4}\; \text{If } 3m^2+4=0,\ \text{then } \Delta=0 \Rightarrow \text{roots are equal.}
\]
\end{QAPair}

% ============================================================
% Q6
\begin{QAPair}{Question 6 (i)}
\textcolor{gold}{\bfseries Question:} Find $m$ when roots of $x^2-6x+m=0$ are equal.\\
\tcblower
\textcolor{green}{\bfseries Answer:}
Equal roots $\Rightarrow \Delta=0$.
\[
\begin{aligned}
\Step{1}\;& a=1,\; b=-6,\; c=m.\\
\Step{2}\;& \Delta= (-6)^2-4(1)(m)=36-4m.\\
\Step{3}\;& 36-4m=0 \Rightarrow m=9.
\end{aligned}
\]
\[
\boxed{m=9}
\]
\end{QAPair}

\begin{QAPair}{Question 6 (ii)}
\textcolor{gold}{\bfseries Question:} Find $m$ when roots of $m^2x^2+(2m+1)x+1=0$ are equal.\\
\tcblower
\textcolor{green}{\bfseries Answer:}
\[
\begin{aligned}
\Step{1}\;& a=m^2,\; b=2m+1,\; c=1.\\
\Step{2}\;& \Delta=(2m+1)^2-4(m^2)(1)\\
&=4m^2+4m+1-4m^2=4m+1.\\
\Step{3}\;& \Delta=0 \Rightarrow 4m+1=0 \Rightarrow m=-\frac14.
\end{aligned}
\]
\[
\boxed{m=-\frac14}
\]
\end{QAPair}

\begin{QAPair}{Question 6 (iii)}
\textcolor{gold}{\bfseries Question:} Find $m$ when roots of $(m+3)x^2+(m+1)x+m+1=0$ are equal.\\
\tcblower
\textcolor{green}{\bfseries Answer:}
\[
\begin{aligned}
\Step{1}\;& a=m+3,\; b=m+1,\; c=m+1.\\
\Step{2}\;& \Delta=(m+1)^2-4(m+3)(m+1)\\
&=(m+1)\Big((m+1)-4(m+3)\Big)\\
&=(m+1)(m+1-4m-12)=(m+1)(-3m-11).
\end{aligned}
\]
\[
\begin{aligned}
\Step{3}\;& \Delta=0 \Rightarrow (m+1)(-3m-11)=0\\
&\Rightarrow m=-1 \quad \text{or}\quad m=-\frac{11}{3}.
\end{aligned}
\]
\[
\boxed{m=-1\ \text{or}\ m=-\frac{11}{3}}
\]
\end{QAPair}

% ============================================================
% Q7
\begin{QAPair}{Question 7}
\textcolor{gold}{\bfseries Question:} Show that the roots of
\[
(a^2+b^2)x^2+2(ac+bd)x+c^2+d^2=0
\]
are imaginary. Moreover, it shows repeated roots if $ad=bc$.\\
\tcblower
\textcolor{green}{\bfseries Answer:}
\[
\begin{aligned}
\Step{1}\;& a_1=a^2+b^2,\quad b_1=2(ac+bd),\quad c_1=c^2+d^2.\\
\Step{2}\;& \Delta=b_1^2-4a_1c_1
= \big(2(ac+bd)\big)^2-4(a^2+b^2)(c^2+d^2)\\
&=4\Big((ac+bd)^2-(a^2+b^2)(c^2+d^2)\Big).
\end{aligned}
\]
Now simplify the bracket:
\[
\begin{aligned}
(ac+bd)^2 &= a^2c^2+2abcd+b^2d^2,\\
(a^2+b^2)(c^2+d^2) &= a^2c^2+a^2d^2+b^2c^2+b^2d^2.
\end{aligned}
\]
So
\[
\begin{aligned}
(ac+bd)^2-(a^2+b^2)(c^2+d^2)
&=2abcd-a^2d^2-b^2c^2\\
&=-(ad-bc)^2.
\end{aligned}
\]
Hence
\[
\Step{3}\; \Delta =4\big(-(ad-bc)^2\big)= -4(ad-bc)^2\le 0.
\]
\[
\Step{4}\;
\begin{cases}
ad\ne bc \Rightarrow \Delta<0 \Rightarrow \text{roots are imaginary},\\
ad=bc \Rightarrow \Delta=0 \Rightarrow \text{roots are real and equal (repeated).}
\end{cases}
\]
\end{QAPair}

% ============================================================
% Q8
\begin{QAPair}{Question 8}
\textcolor{gold}{\bfseries Question:} Show that the roots of $(ax+c)^2=4bx$ will be equal, if $b=ac$.\\
\tcblower
\textcolor{green}{\bfseries Answer:}
\[
\begin{aligned}
\Step{1}\;& (ax+c)^2=4bx\\
&\Rightarrow a^2x^2+2acx+c^2-4bx=0\\
&\Rightarrow a^2x^2+2(ac-2b)x+c^2=0.
\end{aligned}
\]
Discriminant:
\[
\begin{aligned}
\Step{2}\;& \Delta=\big(2(ac-2b)\big)^2-4(a^2)(c^2)\\
&=4\Big((ac-2b)^2-a^2c^2\Big)\\
&=4\Big(a^2c^2-4abc+4b^2-a^2c^2\Big)=16b(b-ac).
\end{aligned}
\]
\[
\Step{3}\; \text{If } b=ac,\ \text{then } \Delta=16b(b-ac)=0 \Rightarrow \text{roots are equal.}
\]
\end{QAPair}

% ============================================================
% Q9
\begin{QAPair}{Question 9 (i)}
\textcolor{gold}{\bfseries Question:} Show that the roots of $mx^2-2mx+m-1=0$ are real.\\
\tcblower
\textcolor{green}{\bfseries Answer:}
\[
\begin{aligned}
\Step{1}\;& a=m,\; b=-2m,\; c=m-1.\\
\Step{2}\;& \Delta=b^2-4ac=(-2m)^2-4(m)(m-1)\\
&=4m^2-4m^2+4m=4m.
\end{aligned}
\]
\[
\Step{3}\; \text{For a quadratic, } m\ne 0.\ \text{If } m>0,\ \Delta=4m>0 \Rightarrow \text{roots are real (distinct).}
\]
\end{QAPair}

\begin{QAPair}{Question 9 (ii)}
\textcolor{gold}{\bfseries Question:} Show that the roots of $bx^2+ax+a-b=0$ are real.\\
\tcblower
\textcolor{green}{\bfseries Answer:}
\[
\begin{aligned}
\Step{1}\;& a_1=b,\; b_1=a,\; c_1=a-b.\\
\Step{2}\;& \Delta=a^2-4b(a-b)=a^2-4ab+4b^2=(a-2b)^2\ge 0.
\end{aligned}
\]
\[
\boxed{\text{Therefore, the roots are real.}}
\]
\end{QAPair}

% ============================================================
% Q10
\begin{QAPair}{Question 10}
\textcolor{gold}{\bfseries Question:} Show that the roots of $(a+b)x^2-ax-b=0$ are real.\\
\tcblower
\textcolor{green}{\bfseries Answer:}
\[
\begin{aligned}
\Step{1}\;& a_1=a+b,\; b_1=-a,\; c_1=-b.\\
\Step{2}\;& \Delta=b_1^2-4a_1c_1=(-a)^2-4(a+b)(-b)\\
&=a^2+4b(a+b)=a^2+4ab+4b^2=(a+2b)^2\ge 0.
\end{aligned}
\]
\[
\boxed{\text{Therefore, the roots are real.}}
\]
\end{QAPair}

\end{document}