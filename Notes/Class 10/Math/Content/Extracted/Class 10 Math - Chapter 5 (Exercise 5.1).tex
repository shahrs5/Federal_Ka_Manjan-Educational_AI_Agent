% !TEX TS-program = pdflatex
\documentclass[11pt]{article}

% -------------------- Packages --------------------
\usepackage[a4paper,margin=1in]{geometry}
\usepackage{amsmath,amssymb}
\usepackage[T1]{fontenc}
\usepackage{lmodern}
\usepackage{xcolor}
\usepackage{tcolorbox}
\tcbuselibrary{skins,breakable}
\usepackage{enumitem}
\usepackage{hyperref}

\pagestyle{empty}

% -------------------- Dark Theme Colors --------------------
\definecolor{bg}{HTML}{000000}
\definecolor{pairbg}{HTML}{121212}
\definecolor{solbg}{HTML}{0A0A0A}
\definecolor{border}{HTML}{2A2A2A}
\definecolor{text}{HTML}{FFFFFF}
\definecolor{muted}{HTML}{C9CDD3}
\definecolor{gold}{HTML}{FFD700}
\definecolor{green}{HTML}{4ADE80}
\definecolor{cyan}{HTML}{38BDF8}

\pagecolor{bg}
\color{text}

\hypersetup{
  colorlinks=true,
  linkcolor=cyan,
  urlcolor=cyan
}

\setlength{\parindent}{0pt}
\setlength{\parskip}{10pt}

\setlist[itemize]{left=1.4em,itemsep=6pt,topsep=6pt}
\setlist[enumerate]{left=1.6em,itemsep=4pt,topsep=4pt}

% -------------------- tcolorbox Base --------------------
\tcbset{
  enhanced,
  breakable,
  arc=12pt,
  boxrule=0.8pt,
  left=16pt,right=16pt,top=12pt,bottom=12pt
}

\newtcolorbox{QAPair}[1]{%
  colback=pairbg,
  colbacklower=solbg,
  colframe=border,
  coltext=text,
  title=\textcolor{gold}{\bfseries #1},
  fonttitle=\bfseries,
  coltitle=text,
  segmentation style={draw=border, dashed, line width=0.6pt},
}

% Visible text inside this box (fix)
\newtcolorbox{QuickBox}{%
  colback=pairbg,
  colframe=cyan,
  coltext=text,
  fontupper=\color{text},
  borderline north={4pt}{0pt}{cyan},
  arc=14pt,
  boxrule=0.8pt
}

% Helper for step headings
\newcommand{\Step}[1]{\textcolor{muted}{\textbf{Step #1:}}}

% ============================================================
\begin{document}

\begin{center}
{\LARGE\bfseries \textcolor{gold}{Exercise 5.1 --- Solutions}}\\[-2pt]
\end{center}

\begin{QuickBox}
{\color{cyan}\bfseries Quick formulas (useful)}\par\medskip
\begin{itemize}
\item \textbf{Indices:} $\dfrac{a^m}{a^n}=a^{m-n}$,\; $a^m a^n=a^{m+n}$.
\item \textbf{Common factors:} $\dfrac{ka}{kb}=\dfrac{a}{b}\ (k\neq 0)$.
\item \textbf{Difference of squares:} $x^2-a^2=(x-a)(x+a)$.
\item \textbf{Sign flip:} $3-x=-(x-3)$.
\item \textbf{Triangular number:} $T(n)=\dfrac{n(n+1)}{2}$.
\end{itemize}
\end{QuickBox}

% ============================================================
% Q1
\begin{QAPair}{Question 1 (i)}
\textcolor{gold}{\bfseries Question:} $\displaystyle \frac{15ax^3y^2}{25a^2xy^6}$\\
\tcblower
\textcolor{green}{\bfseries Answer:}
\[
\begin{aligned}
\Step{1}\;& \frac{15ax^3y^2}{25a^2xy^6}
= \frac{15}{25}\cdot \frac{a}{a^2}\cdot \frac{x^3}{x}\cdot \frac{y^2}{y^6}\\
\Step{2}\;&= \frac{3}{5}\cdot \frac{1}{a}\cdot x^2 \cdot \frac{1}{y^4}\\
\Step{3}\;&= \frac{3x^2}{5ay^4}.
\end{aligned}
\]
\end{QAPair}

\begin{QAPair}{Question 1 (ii)}
\textcolor{gold}{\bfseries Question:} $\displaystyle \frac{38k^2p^3m^4}{57k^3pm^2}$\\
\tcblower
\textcolor{green}{\bfseries Answer:}
\[
\begin{aligned}
\Step{1}\;& \frac{38k^2p^3m^4}{57k^3pm^2}
=\frac{38}{57}\cdot \frac{k^2}{k^3}\cdot \frac{p^3}{p}\cdot \frac{m^4}{m^2}\\
\Step{2}\;&=\frac{2}{3}\cdot \frac{1}{k}\cdot p^2 \cdot m^2\\
\Step{3}\;&= \frac{2p^2m^2}{3k}.
\end{aligned}
\]
\end{QAPair}

\begin{QAPair}{Question 1 (iii)}
\textcolor{gold}{\bfseries Question:} $\displaystyle \frac{mn^4pq^4}{m^2n^3p^4}$\\
\tcblower
\textcolor{green}{\bfseries Answer:}
\[
\begin{aligned}
\Step{1}\;& \frac{mn^4pq^4}{m^2n^3p^4}
= m^{1-2}\,n^{4-3}\,p^{1-4}\,q^4\\
\Step{2}\;&= \frac{nq^4}{mp^3}.
\end{aligned}
\]
\end{QAPair}

\begin{QAPair}{Question 1 (iv)}
\textcolor{gold}{\bfseries Question:} $\displaystyle \frac{3abc}{15a^2b^2c}$\\
\tcblower
\textcolor{green}{\bfseries Answer:}
\[
\begin{aligned}
\Step{1}\;& \frac{3abc}{15a^2b^2c}
=\frac{3}{15}\cdot \frac{a}{a^2}\cdot \frac{b}{b^2}\cdot \frac{c}{c}\\
\Step{2}\;&=\frac{1}{5}\cdot \frac{1}{a}\cdot \frac{1}{b}\\
\Step{3}\;&=\frac{1}{5ab}.
\end{aligned}
\]
\end{QAPair}

\begin{QAPair}{Question 1 (v)}
\textcolor{gold}{\bfseries Question:} $\displaystyle \frac{46l^3m^4n^5}{69l^2m^3n^4}$\\
\tcblower
\textcolor{green}{\bfseries Answer:}
\[
\begin{aligned}
\Step{1}\;& \frac{46l^3m^4n^5}{69l^2m^3n^4}
=\frac{46}{69}\cdot l^{3-2}\cdot m^{4-3}\cdot n^{5-4}\\
\Step{2}\;&=\frac{2}{3}\cdot lmn\\
\Step{3}\;&=\frac{2lmn}{3}.
\end{aligned}
\]
\end{QAPair}

\begin{QAPair}{Question 1 (vi)}
\textcolor{gold}{\bfseries Question:} $\displaystyle \frac{x-3}{3-x}$\\
\tcblower
\textcolor{green}{\bfseries Answer:}
\[
\begin{aligned}
\Step{1}\;& 3-x = -(x-3)\\
\Step{2}\;& \frac{x-3}{3-x}=\frac{x-3}{-(x-3)}=-1 \quad (x\neq 3).
\end{aligned}
\]
\end{QAPair}

\begin{QAPair}{Question 1 (vii)}
\textcolor{gold}{\bfseries Question:} $\displaystyle \frac{x^2-81}{x+9}$\\
\tcblower
\textcolor{green}{\bfseries Answer:}
\[
\begin{aligned}
\Step{1}\;& x^2-81=(x-9)(x+9)\\
\Step{2}\;& \frac{x^2-81}{x+9}=\frac{(x-9)(x+9)}{x+9}=x-9 \quad (x\neq -9).
\end{aligned}
\]
\end{QAPair}

\begin{QAPair}{Question 1 (viii)}
\textcolor{gold}{\bfseries Question:} $\displaystyle \frac{(r+3)(r+4)}{r^2-16}$\\
\tcblower
\textcolor{green}{\bfseries Answer:}
\[
\begin{aligned}
\Step{1}\;& r^2-16=(r-4)(r+4)\\
\Step{2}\;& \frac{(r+3)(r+4)}{(r-4)(r+4)}=\frac{r+3}{r-4}\quad (r\neq \pm 4).
\end{aligned}
\]
\end{QAPair}

% ============================================================
% Q2
\begin{QAPair}{Question 2 (i)}
\textcolor{gold}{\bfseries Question:} Evaluate $3(r^2-s^2)$ if $r=2,\ s=-1$.\\
\tcblower
\textcolor{green}{\bfseries Answer:}
\[
\begin{aligned}
\Step{1}\;& r^2-s^2=2^2-(-1)^2=4-1=3\\
\Step{2}\;& 3(r^2-s^2)=3\cdot 3=9.
\end{aligned}
\]
\end{QAPair}

\begin{QAPair}{Question 2 (ii)}
\textcolor{gold}{\bfseries Question:} Evaluate $\dfrac12 mv^2$ at $m=18.75,\ v=5.6$.\\
\tcblower
\textcolor{green}{\bfseries Answer:}
\[
\begin{aligned}
\Step{1}\;& v^2=(5.6)^2=31.36\\
\Step{2}\;& \frac12 m=\frac12(18.75)=9.375\\
\Step{3}\;& \frac12 mv^2 = 9.375\times 31.36 = 294.
\end{aligned}
\]
\end{QAPair}

\begin{QAPair}{Question 2 (iii)}
\textcolor{gold}{\bfseries Question:} Evaluate $\sqrt{2gs}$ when $g=32.2,\ s=144.9$.\\
\tcblower
\textcolor{green}{\bfseries Answer:}
\[
\begin{aligned}
\Step{1}\;& 2gs = 2(32.2)(144.9)=64.4\times 144.9=9331.56\\
\Step{2}\;& \sqrt{2gs}=\sqrt{9331.56}=96.6.
\end{aligned}
\]
\end{QAPair}

\begin{QAPair}{Question 2 (iv)}
\textcolor{gold}{\bfseries Question:} Evaluate $3x-y+\dfrac{1}{z}$ if $x=-\dfrac12,\ y=3,\ z=-\dfrac13$.\\
\tcblower
\textcolor{green}{\bfseries Answer:}
\[
\begin{aligned}
\Step{1}\;& 3x=3\left(-\frac12\right)=-\frac32,\qquad \frac{1}{z}=\frac{1}{-\frac13}=-3\\
\Step{2}\;& 3x-y+\frac{1}{z}=-\frac32-3-3\\
\Step{3}\;&=-\frac32-6=-\frac{15}{2}.
\end{aligned}
\]
\end{QAPair}

\begin{QAPair}{Question 2 (v)}
\textcolor{gold}{\bfseries Question:} Evaluate $0.1d^2+0.01d+1$ if $d=-0.2$.\\
\tcblower
\textcolor{green}{\bfseries Answer:}
\[
\begin{aligned}
\Step{1}\;& d^2=(-0.2)^2=0.04\\
\Step{2}\;& 0.1d^2+0.01d+1 = 0.1(0.04)+0.01(-0.2)+1\\
\Step{3}\;&=0.004-0.002+1=1.002.
\end{aligned}
\]
\end{QAPair}

\begin{QAPair}{Question 2 (vi)}
\textcolor{gold}{\bfseries Question:} Evaluate $\dfrac{4}{7}b^3-3\dfrac12\,b^2+b-3$ if $b=\dfrac12$.\\
\tcblower
\textcolor{green}{\bfseries Answer:}
\[
\begin{aligned}
\Step{1}\;& b=\frac12,\quad b^2=\frac14,\quad b^3=\frac18,\quad 3\frac12=\frac72\\
\Step{2}\;& \frac{4}{7}b^3=\frac{4}{7}\cdot\frac18=\frac{1}{14},\qquad
\left(3\frac12\right)b^2=\frac72\cdot\frac14=\frac78\\
\Step{3}\;& \frac{4}{7}b^3-3\frac12\,b^2+b-3
=\frac{1}{14}-\frac78+\frac12-3\\
\Step{4}\;&=\frac{4}{56}-\frac{49}{56}+\frac{28}{56}-\frac{168}{56}
=-\frac{185}{56}.
\end{aligned}
\]
\end{QAPair}

% ============================================================
% Q3
\begin{QAPair}{Question 3}
\textcolor{gold}{\bfseries Question:} If $T(n)=\dfrac{n(n+1)}{2}$, find the $100^{\text{th}}$ triangular number.\\
\tcblower
\textcolor{green}{\bfseries Answer:}
\[
\begin{aligned}
\Step{1}\;& T(100)=\frac{100(100+1)}{2}=\frac{100\cdot 101}{2}\\
\Step{2}\;&=50\cdot 101=5050.
\end{aligned}
\]
\end{QAPair}

% ============================================================
% Q4
\begin{QAPair}{Question 4}
\textcolor{gold}{\bfseries Question:} If $P(x)=x^2+2x-15$, $D(x)=x-3$ and $Q(x)=x+5$, show that $\dfrac{P(2)}{Q(2)}=D(2)$.\\
\tcblower
\textcolor{green}{\bfseries Answer:}
\[
\begin{aligned}
\Step{1}\;& P(2)=2^2+2(2)-15=4+4-15=-7,\qquad Q(2)=2+5=7\\
\Step{2}\;& \frac{P(2)}{Q(2)}=\frac{-7}{7}=-1\\
\Step{3}\;& D(2)=2-3=-1\\
\Step{4}\;& \Rightarrow\; \frac{P(2)}{Q(2)}=D(2).
\end{aligned}
\]
\end{QAPair}

% ============================================================
% Q5
\begin{QAPair}{Question 5}
\textcolor{gold}{\bfseries Question:} If $g(x)=\dfrac{1}{2x^3}+\dfrac{x}{2}+2$, find $g\!\left(-\dfrac13\right)$.\\
\tcblower
\textcolor{green}{\bfseries Answer:}
\[
\begin{aligned}
\Step{1}\;& x=-\frac13 \;\Rightarrow\; x^3=-\frac{1}{27},\quad 2x^3=-\frac{2}{27}\\
\Step{2}\;& \frac{1}{2x^3}=\frac{1}{-\frac{2}{27}}=-\frac{27}{2},\qquad \frac{x}{2}=-\frac{1}{6}\\
\Step{3}\;& g\!\left(-\frac13\right)=-\frac{27}{2}-\frac{1}{6}+2\\
\Step{4}\;&=-\frac{81}{6}-\frac{1}{6}+\frac{12}{6}=-\frac{70}{6}=-\frac{35}{3}.
\end{aligned}
\]
\end{QAPair}

% ============================================================
% Q6
\begin{QAPair}{Question 6}
\textcolor{gold}{\bfseries Question:} The volume of a basketball is $v=38808\text{ cm}^3$. If
\[
r=\sqrt[3]{\frac{3v}{4\pi}},\quad \text{take } \pi=\frac{22}{7},
\]
determine the radius $r$.\\
\tcblower
\textcolor{green}{\bfseries Answer:}
\[
\begin{aligned}
\Step{1}\;& r=\sqrt[3]{\frac{3v}{4\pi}}
=\sqrt[3]{\frac{3(38808)}{4\cdot \frac{22}{7}}}
=\sqrt[3]{\frac{116424}{\frac{88}{7}}}\\
\Step{2}\;&=\sqrt[3]{116424\cdot\frac{7}{88}}
=\sqrt[3]{\left(\frac{116424}{88}\right)\cdot 7}
=\sqrt[3]{1323\cdot 7}\\
\Step{3}\;&=\sqrt[3]{9261}
=\sqrt[3]{21^3}=21.
\end{aligned}
\]
\[
\boxed{r=21\text{ cm}}
\]
\end{QAPair}

\end{document}
