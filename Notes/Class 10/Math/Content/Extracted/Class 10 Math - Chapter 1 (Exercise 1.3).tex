% !TEX TS-program = pdflatex
\documentclass[11pt]{article}

% -------------------- Packages --------------------
\usepackage[a4paper,margin=1in]{geometry}
\usepackage{amsmath,amssymb}
\usepackage[T1]{fontenc}
\usepackage{lmodern}
\usepackage{xcolor}
\usepackage{tcolorbox}
\tcbuselibrary{skins,breakable}
\usepackage{enumitem}
\usepackage{hyperref}

% --- Diagrams ---
\usepackage{tikz}
\usetikzlibrary{arrows.meta}

\pagestyle{empty}

% -------------------- Dark Theme Colors --------------------
\definecolor{bg}{HTML}{000000}
\definecolor{pairbg}{HTML}{121212}
\definecolor{solbg}{HTML}{0A0A0A}
\definecolor{border}{HTML}{2A2A2A}
\definecolor{text}{HTML}{FFFFFF}
\definecolor{muted}{HTML}{C9CDD3}
\definecolor{gold}{HTML}{FFD700}
\definecolor{green}{HTML}{4ADE80}
\definecolor{cyan}{HTML}{38BDF8}

\pagecolor{bg}
\color{text}

\hypersetup{
  colorlinks=true,
  linkcolor=cyan,
  urlcolor=cyan
}

\setlength{\parindent}{0pt}
\setlength{\parskip}{10pt}

\setlist[itemize]{left=1.4em,itemsep=6pt,topsep=6pt}
\setlist[enumerate]{left=1.6em,itemsep=4pt,topsep=4pt}

% -------------------- tcolorbox Base --------------------
\tcbset{
  enhanced,
  breakable,
  arc=12pt,
  boxrule=0.8pt,
  left=16pt,right=16pt,top=12pt,bottom=12pt
}

\newtcolorbox{QAPair}[1]{%
  colback=pairbg,
  colbacklower=solbg,
  colframe=border,
  coltext=text,
  title=\textcolor{gold}{\bfseries #1},
  fonttitle=\bfseries,
  coltitle=text,
  segmentation style={draw=border, dashed, line width=0.6pt},
}

\newtcolorbox{QuickBox}{%
  colback=pairbg,
  colframe=cyan,
  coltext=text,
  fontupper=\color{text},
  borderline north={4pt}{0pt}{cyan},
  arc=14pt,
  boxrule=0.8pt
}

% Helper for step headings
\newcommand{\Step}[1]{\textcolor{muted}{\textbf{Step #1:}}}

% A helper symbol for engineering imaginary unit
\newcommand{\J}{\mathrm{J}}

% ============================================================
\begin{document}

\begin{center}
{\LARGE\bfseries \textcolor{gold}{Exercise 1.3 --- Solutions}}\\[-2pt]
\end{center}

\begin{QuickBox}
{\color{cyan}\bfseries Quick formulas (useful)}\par\medskip
\begin{itemize}
\item \textbf{Imaginary unit:} $i^2=-1$ \quad (In electrical questions, $\J$ is used and $\J^2=-1$.)
\item \textbf{Solve:} $x^2+a=0 \Rightarrow x^2=-a \Rightarrow x=\pm i\sqrt{a}$ \quad (for $a>0$).
\item \textbf{Conjugate:} $\overline{a+bi}=a-bi$.
\item \textbf{Division rule:} $\displaystyle \frac{a+bi}{c+di}=\frac{(a+bi)(c-di)}{c^2+d^2}$ \quad (multiply by conjugate).
\item \textbf{Factorization:} $x^2+a^2=(x+ai)(x-ai)$.
\item \textbf{Circuit formula:} $E=IZ \Rightarrow I=\dfrac{E}{Z},\; Z=\dfrac{E}{I}$.
\end{itemize}
\end{QuickBox}

% ============================================================
% Q1
\begin{QAPair}{Question 1}
\textcolor{gold}{\bfseries Question:} Solve $x^2+7=0$.
\tcblower
\textcolor{green}{\bfseries Answer:}
\[
\Step{1}\; x^2+7=0 \;\Rightarrow\; x^2=-7.
\]
\[
\Step{2}\; x=\pm\sqrt{-7}=\pm i\sqrt{7}.
\]
\[
\boxed{x= i\sqrt{7}\;\; \text{or}\;\; x=-i\sqrt{7}}
\]
\end{QAPair}

% ============================================================
% Q2
\begin{QAPair}{Question 2}
\textcolor{gold}{\bfseries Question:} Solve $x^2+9=0$.
\tcblower
\textcolor{green}{\bfseries Answer:}
\[
\Step{1}\; x^2=-9
\qquad\Rightarrow\qquad
\Step{2}\; x=\pm\sqrt{-9}=\pm 3i.
\]
\[
\boxed{x=3i\;\;\text{or}\;\;x=-3i}
\]
\end{QAPair}

% ============================================================
% Q3
\begin{QAPair}{Question 3}
\textcolor{gold}{\bfseries Question:} Solve $x^2+100=0$.
\tcblower
\textcolor{green}{\bfseries Answer:}
\[
\Step{1}\; x^2=-100
\qquad\Rightarrow\qquad
\Step{2}\; x=\pm\sqrt{-100}=\pm 10i.
\]
\[
\boxed{x=10i\;\;\text{or}\;\;x=-10i}
\]
\end{QAPair}

% ---------------- Diagram (Argand plane for Q1--Q3 roots) ----------------
\begin{QAPair}{Argand Diagram (for roots of Q1--Q3)}
\textcolor{gold}{\bfseries Question:} Show where the solutions lie on the complex plane.
\tcblower
\textcolor{green}{\bfseries Answer:}
\textcolor{muted}{All roots are \emph{purely imaginary}, so they lie on the vertical (imaginary) axis.}

\begin{center}
\begin{tikzpicture}[scale=0.55]
% Axes
\draw[-{Stealth[length=3mm]}, color=text, line width=0.9pt] (-7,0) -- (7,0) node[right] {\textcolor{text}{$\Re$}};
\draw[-{Stealth[length=3mm]}, color=text, line width=0.9pt] (0,-12) -- (0,12) node[above] {\textcolor{text}{$\Im$}};

% Ticks and labels
\foreach \y/\lab in {-10/-10, -3/-3, 0/0, 3/3, 10/10}{
  \draw[color=border] (-0.18,\y) -- (0.18,\y);
  \node[left, text=text] at (-0.25,\y) {\small \lab};
}

% Points (approx positions for sqrt7)
\filldraw[cyan] (0,2.65) circle (2.3pt) node[right, text=text] {\small $i\sqrt{7}$};
\filldraw[cyan] (0,-2.65) circle (2.3pt) node[right, text=text] {\small $-i\sqrt{7}$};

\filldraw[cyan] (0,3) circle (2.3pt) node[right, text=text] {\small $3i$};
\filldraw[cyan] (0,-3) circle (2.3pt) node[right, text=text] {\small $-3i$};

\filldraw[cyan] (0,10) circle (2.3pt) node[right, text=text] {\small $10i$};
\filldraw[cyan] (0,-10) circle (2.3pt) node[right, text=text] {\small $-10i$};

\end{tikzpicture}
\end{center}

\textcolor{muted}{Note: $\sqrt{7}\approx 2.65$, so the points $\pm i\sqrt{7}$ are placed near $\pm 2.65$ on the imaginary axis.}
\end{QAPair}

% ============================================================
% Q4
\begin{QAPair}{Question 4}
\textcolor{gold}{\bfseries Question:} Determine whether $1+2i$ is a solution of $x^2-2x+5=0$.
\tcblower
\textcolor{green}{\bfseries Answer:}

Let $x=1+2i$.

\[
\Step{1}\; x^2=(1+2i)^2=1+4i+(2i)^2=1+4i-4=-3+4i.
\]
\[
\Step{2}\; -2x=-2(1+2i)=-2-4i.
\]
\[
\Step{3}\; x^2-2x+5=(-3+4i)+(-2-4i)+5=0.
\]
\[
\boxed{\text{$1+2i$ \;is a solution.}}
\]
\end{QAPair}

% ============================================================
% Q5
\begin{QAPair}{Question 5}
\textcolor{gold}{\bfseries Question:} Determine whether $1-2i$ is a solution of $x^2-2x+5=0$.
\tcblower
\textcolor{green}{\bfseries Answer:}

Let $x=1-2i$.

\[
\Step{1}\; x^2=(1-2i)^2=1-4i+(-2i)^2=1-4i-4=-3-4i.
\]
\[
\Step{2}\; -2x=-2(1-2i)=-2+4i.
\]
\[
\Step{3}\; x^2-2x+5=(-3-4i)+(-2+4i)+5=0.
\]
\[
\boxed{\text{$1-2i$ \;is a solution.}}
\]
\end{QAPair}

% ============================================================
% Q6
\begin{QAPair}{Question 6}
\textcolor{gold}{\bfseries Question:} Determine whether $1-i$ is a solution of $x^2+2x+2=0$.
\tcblower
\textcolor{green}{\bfseries Answer:}

Let $x=1-i$.

\[
\Step{1}\; x^2=(1-i)^2=1-2i+i^2=1-2i-1=-2i.
\]
\[
\Step{2}\; 2x=2(1-i)=2-2i.
\]
\[
\Step{3}\; x^2+2x+2=(-2i)+(2-2i)+2=4-4i\neq 0.
\]
\[
\boxed{\text{$1-i$ \;is NOT a solution.}}
\]
\end{QAPair}

% ============================================================
% Q7
\begin{QAPair}{Question 7}
\textcolor{gold}{\bfseries Question:} Determine whether $i$ is a solution of $x^2+1=0$.
\tcblower
\textcolor{green}{\bfseries Answer:}

Let $x=i$.

\[
\Step{1}\; x^2+1=i^2+1=-1+1=0.
\]
\[
\boxed{\text{$i$ \;is a solution.}}
\]
\end{QAPair}

% ============================================================
% Q8
\begin{QAPair}{Question 8}
\textcolor{gold}{\bfseries Question:} Factorize $x^2+16$.
\tcblower
\textcolor{green}{\bfseries Answer:}

\[
\Step{1}\; x^2+16=x^2+4^2.
\]
\[
\Step{2}\; x^2+4^2=(x+4i)(x-4i).
\]
\[
\boxed{x^2+16=(x+4i)(x-4i)}
\]
\end{QAPair}

% ============================================================
% Q9
\begin{QAPair}{Question 9}
\textcolor{gold}{\bfseries Question:} Factorize $a^2+b^2$.
\tcblower
\textcolor{green}{\bfseries Answer:}

\[
\Step{1}\; a^2+b^2=a^2+(b)^2.
\]
\[
\Step{2}\; a^2+b^2=(a+bi)(a-bi).
\]
\[
\boxed{a^2+b^2=(a+bi)(a-bi)}
\]
\end{QAPair}

% ============================================================
% Q10
\begin{QAPair}{Question 10}
\textcolor{gold}{\bfseries Question:} Factorize $x^2+25y^2$.
\tcblower
\textcolor{green}{\bfseries Answer:}

\[
\Step{1}\; x^2+25y^2=x^2+(5y)^2.
\]
\[
\Step{2}\; x^2+(5y)^2=(x+5yi)(x-5yi).
\]
\[
\boxed{x^2+25y^2=(x+5yi)(x-5yi)}
\]
\end{QAPair}

% ============================================================
% Q11
\begin{QAPair}{Question 11}
\textcolor{gold}{\bfseries Question:} Solve the system:
\[
\begin{cases}
z-4w=3i\\
2z+3w=11-5i
\end{cases}
\]
\tcblower
\textcolor{green}{\bfseries Answer:}

\[
\Step{1}\; \text{From } z-4w=3i \Rightarrow z=3i+4w.
\]
\[
\Step{2}\; \text{Substitute into } 2z+3w=11-5i:
\quad 2(3i+4w)+3w=11-5i.
\]
\[
\Step{3}\; 6i+8w+3w=11-5i
\Rightarrow 11w=11-11i
\Rightarrow w=1-i.
\]
\[
\Step{4}\; z=3i+4(1-i)=3i+4-4i=4-i.
\]

\[
\boxed{z=4-i,\qquad w=1-i}
\]
\end{QAPair}

% ============================================================
% Q12
\begin{QAPair}{Question 12}
\textcolor{gold}{\bfseries Question:} Solve the system:
\[
\begin{cases}
3z+(2+i)w=11-i\\
(2-i)z-w=1+i
\end{cases}
\]
\tcblower
\textcolor{green}{\bfseries Answer:}

\[
\Step{1}\; (2-i)z-w=1+i \Rightarrow w=(2-i)z-(1+i).
\]

\[
\Step{2}\; \text{Substitute into } 3z+(2+i)w=11-i:
\quad 3z+(2+i)\bigl((2-i)z-(1+i)\bigr)=11-i.
\]

\[
\Step{3}\; (2+i)(2-i)=4-2i+2i-i^2=5.
\]
So,
\[
3z+5z-(2+i)(1+i)=11-i.
\]

\[
\Step{4}\; (2+i)(1+i)=2+2i+i+i^2=1+3i.
\]
Thus,
\[
8z-(1+3i)=11-i
\Rightarrow 8z=12+2i
\Rightarrow z=\frac{12+2i}{8}=\frac{3}{2}+\frac{i}{4}.
\]

\[
\Step{5}\; w=(2-i)z-(1+i).
\]
Compute $(2-i)\left(\frac{3}{2}+\frac{i}{4}\right)$:
\[
(2-i)\left(\frac{3}{2}+\frac{i}{4}\right)=\frac{13}{4}-i.
\]
So,
\[
w=\left(\frac{13}{4}-i\right)-(1+i)=\frac{9}{4}-2i.
\]

\[
\boxed{z=\frac{3}{2}+\frac{i}{4},\qquad w=\frac{9}{4}-2i}
\]
\end{QAPair}

% ============================================================
% Q13a(i)
\begin{QAPair}{Question 13(a)(i)}
\textcolor{gold}{\bfseries Question:} In a circuit $E=IZ$. Find $I$ if\\
$E=(70+220\J)$ volts, $\; Z=(16+8\J)$ ohms.
\tcblower
\textcolor{green}{\bfseries Answer:}

\[
\Step{1}\; I=\frac{E}{Z}=\frac{70+220\J}{16+8\J}.
\]

\[
\Step{2}\; \text{Multiply top and bottom by conjugate }(16-8\J):
\quad I=\frac{(70+220\J)(16-8\J)}{16^2+8^2}.
\]

\[
\Step{3}\; \text{Denominator: } 16^2+8^2=256+64=320.
\]

\[
\Step{4}\; \text{Numerator: } (70+220\J)(16-8\J)
=1120-560\J+3520\J-1760\J^2
=2880+2960\J
\]
(because $\J^2=-1$, so $-1760\J^2=+1760$).

\[
\Step{5}\; I=\frac{2880}{320}+\frac{2960}{320}\J
=9+\frac{37}{4}\J.
\]

\[
\boxed{I=9+\frac{37}{4}\J\ \text{amp}\; \;(\text{i.e. } I=9+9.25\J)}
\]
\end{QAPair}

% ============================================================
% Q13a(ii)
\begin{QAPair}{Question 13(a)(ii)}
\textcolor{gold}{\bfseries Question:} Find $I$ if\\
$E=(85+110\J)$ volts, $\; Z=(3-4\J)$ ohms.
\tcblower
\textcolor{green}{\bfseries Answer:}

\[
\Step{1}\; I=\frac{E}{Z}=\frac{85+110\J}{3-4\J}.
\]

\[
\Step{2}\; \text{Multiply by conjugate }(3+4\J):
\quad I=\frac{(85+110\J)(3+4\J)}{3^2+4^2}.
\]

\[
\Step{3}\; \text{Denominator: } 3^2+4^2=9+16=25.
\]

\[
\Step{4}\; \text{Numerator: } (85+110\J)(3+4\J)
=255+340\J+330\J+440\J^2
=-185+670\J.
\]

\[
\Step{5}\; I=\frac{-185}{25}+\frac{670}{25}\J
=-\frac{37}{5}+\frac{134}{5}\J.
\]

\[
\boxed{I=-\frac{37}{5}+\frac{134}{5}\J\ \text{amp}\; \;(\text{i.e. } I=-7.4+26.8\J)}
\]
\end{QAPair}

% ============================================================
% Q13b(i)
\begin{QAPair}{Question 13(b)(i)}
\textcolor{gold}{\bfseries Question:} Find $Z$ if\\
$E=(-50+100\J)$ volts, $\; I=(-6-2\J)$ amp.
\tcblower
\textcolor{green}{\bfseries Answer:}

\[
\Step{1}\; Z=\frac{E}{I}=\frac{-50+100\J}{-6-2\J}.
\]

\[
\Step{2}\; \text{Multiply by conjugate }(-6+2\J):
\quad Z=\frac{(-50+100\J)(-6+2\J)}{(-6)^2+(-2)^2}.
\]

\[
\Step{3}\; \text{Denominator: } 36+4=40.
\]

\[
\Step{4}\; \text{Numerator: } (-50+100\J)(-6+2\J)
=300-100\J-600\J+200\J^2
=100-700\J.
\]

\[
\Step{5}\; Z=\frac{100}{40}-\frac{700}{40}\J
=\frac{5}{2}-\frac{35}{2}\J.
\]

\[
\boxed{Z=\frac{5}{2}-\frac{35}{2}\J\ \Omega\; \;(\text{i.e. } Z=2.5-17.5\J)}
\]
\end{QAPair}

% ============================================================
% Q13b(ii)
\begin{QAPair}{Question 13(b)(ii)}
\textcolor{gold}{\bfseries Question:} Find $Z$ if\\
$E=(100+10\J)$ volts, $\; I=(-8+3\J)$ amp.
\tcblower
\textcolor{green}{\bfseries Answer:}

\[
\Step{1}\; Z=\frac{E}{I}=\frac{100+10\J}{-8+3\J}.
\]

\[
\Step{2}\; \text{Multiply by conjugate }(-8-3\J):
\quad Z=\frac{(100+10\J)(-8-3\J)}{(-8)^2+3^2}.
\]

\[
\Step{3}\; \text{Denominator: } 64+9=73.
\]

\[
\Step{4}\; \text{Numerator: } (100+10\J)(-8-3\J)
=-800-300\J-80\J-30\J^2
=-770-380\J.
\]

\[
\Step{5}\; Z=-\frac{770}{73}-\frac{380}{73}\J.
\]

\[
\boxed{Z=-\frac{770}{73}-\frac{380}{73}\J\ \Omega\; \;(\text{approx } Z\approx -10.55-5.21\J)}
\]
\end{QAPair}

% ============================================================
% Q13c
\begin{QAPair}{Question 13(c)}
\textcolor{gold}{\bfseries Question:} Evaluate $\displaystyle \frac{1}{z-z^2}$ when $z=\frac{1-i}{10}$.
\tcblower
\textcolor{green}{\bfseries Answer:}

\[
\Step{1}\; z=\frac{1-i}{10}.
\]

\[
\Step{2}\; z^2=\left(\frac{1-i}{10}\right)^2=\frac{(1-i)^2}{100}
=\frac{1-2i+i^2}{100}
=\frac{-2i}{100}=-\frac{i}{50}.
\]

\[
\Step{3}\; z-z^2=\frac{1-i}{10}-\left(-\frac{i}{50}\right)
=\frac{1-i}{10}+\frac{i}{50}.
\]

\[
\Step{4}\; \text{Common denominator }50:
\quad \frac{1-i}{10}=\frac{5(1-i)}{50}=\frac{5-5i}{50}.
\]
So,
\[
z-z^2=\frac{5-5i+i}{50}=\frac{5-4i}{50}.
\]

\[
\Step{5}\; \frac{1}{z-z^2}=\frac{1}{\frac{5-4i}{50}}=\frac{50}{5-4i}.
\]

\[
\Step{6}\; \text{Multiply by conjugate }(5+4i):
\quad \frac{50}{5-4i}=\frac{50(5+4i)}{5^2+4^2}
=\frac{250+200i}{41}.
\]

\[
\boxed{\frac{1}{z-z^2}=\frac{250}{41}+\frac{200}{41}i}
\]
\end{QAPair}

\end{document}
