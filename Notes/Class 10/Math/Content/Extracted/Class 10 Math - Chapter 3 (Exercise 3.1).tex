% !TEX TS-program = pdflatex
\documentclass[11pt]{article}

% -------------------- Packages --------------------
\usepackage[a4paper,margin=1in]{geometry}
\usepackage{amsmath,amssymb}
\usepackage[T1]{fontenc}
\usepackage{lmodern}
\usepackage{xcolor}
\usepackage{tcolorbox}
\tcbuselibrary{skins,breakable}
\usepackage{enumitem}
\usepackage{hyperref}

\pagestyle{empty}

% -------------------- Dark Theme Colors --------------------
\definecolor{bg}{HTML}{000000}
\definecolor{pairbg}{HTML}{121212}
\definecolor{solbg}{HTML}{0A0A0A}
\definecolor{border}{HTML}{2A2A2A}
\definecolor{text}{HTML}{FFFFFF}
\definecolor{muted}{HTML}{C9CDD3}
\definecolor{gold}{HTML}{FFD700}
\definecolor{green}{HTML}{4ADE80}
\definecolor{cyan}{HTML}{38BDF8}

\pagecolor{bg}
\color{text}

\hypersetup{
  colorlinks=true,
  linkcolor=cyan,
  urlcolor=cyan
}

\setlength{\parindent}{0pt}
\setlength{\parskip}{10pt}

\setlist[itemize]{left=1.4em,itemsep=6pt,topsep=6pt}
\setlist[enumerate]{left=1.6em,itemsep=4pt,topsep=4pt}

% -------------------- tcolorbox Base --------------------
\tcbset{
  enhanced,
  breakable,
  arc=12pt,
  boxrule=0.8pt,
  left=16pt,right=16pt,top=12pt,bottom=12pt
}

\newtcolorbox{QAPair}[1]{%
  colback=pairbg,
  colbacklower=solbg,
  colframe=border,
  coltext=text,
  title=\textcolor{gold}{\bfseries #1},
  fonttitle=\bfseries,
  coltitle=text,
  segmentation style={draw=border, dashed, line width=0.6pt},
}

% Visible text inside this box (fix)
\newtcolorbox{QuickBox}{%
  colback=pairbg,
  colframe=cyan,
  coltext=text,
  fontupper=\color{text},
  borderline north={4pt}{0pt}{cyan},
  arc=14pt,
  boxrule=0.8pt
}

% Helper for step headings
\newcommand{\Step}[1]{\textcolor{muted}{\textbf{Step #1:}}}

% ============================================================
\begin{document}

\begin{center}
{\LARGE\bfseries \textcolor{gold}{Exercise 3.1 --- Solutions}}\\[-2pt]
\end{center}

\begin{QuickBox}
{\color{cyan}\bfseries Quick facts (Matrices)}\par\medskip
\begin{itemize}
\item \textbf{Order:} If a matrix has $m$ rows and $n$ columns, its order is $m\times n$.
\item \textbf{Equal matrices:} Same order \emph{and} corresponding entries are equal.
\item \textbf{Diagonal matrix:} All off-diagonal entries are $0$.
\item \textbf{Identity matrix $I$:} Diagonal entries are $1$, off-diagonal entries are $0$.
\item \textbf{Symmetric:} $A^T=A$. \quad \textbf{Skew-symmetric:} $A^T=-A$ (diagonal must be $0$).
\end{itemize}
\end{QuickBox}

% ============================================================
% Q1
\begin{QAPair}{Question 1 (a)}
\textcolor{gold}{\bfseries Question:} Write rows, columns and order of $\begin{bmatrix}1+2 & 4+5\end{bmatrix}$.\\
\tcblower
\textcolor{green}{\bfseries Answer:}
\[
\begin{aligned}
\Step{1}\;& \begin{bmatrix}1+2 & 4+5\end{bmatrix}
= \begin{bmatrix}3 & 9\end{bmatrix}.\\
\Step{2}\;& \text{Rows}=1,\ \text{Columns}=2,\ \text{Order}=1\times 2.
\end{aligned}
\]
\end{QAPair}

\begin{QAPair}{Question 1 (b)}
\textcolor{gold}{\bfseries Question:} Write rows, columns and order of
$\begin{bmatrix} a+s & 0\\ 0 & d+s \end{bmatrix}$.\\
\tcblower
\textcolor{green}{\bfseries Answer:}
\[
\Step{1}\; \text{Rows}=2,\ \text{Columns}=2,\ \text{Order}=2\times 2.
\]
\end{QAPair}

\begin{QAPair}{Question 1 (c)}
\textcolor{gold}{\bfseries Question:} Write rows, columns and order of
$\begin{bmatrix} 5\\ 7\end{bmatrix}$.\\
\tcblower
\textcolor{green}{\bfseries Answer:}
\[
\Step{1}\; \text{Rows}=2,\ \text{Columns}=1,\ \text{Order}=2\times 1.
\]
\end{QAPair}

\begin{QAPair}{Question 1 (d)}
\textcolor{gold}{\bfseries Question:} Write rows, columns and order of $\begin{bmatrix}4+2\times 1+5\end{bmatrix}$.\\
\tcblower
\textcolor{green}{\bfseries Answer:}
\[
\begin{aligned}
\Step{1}\;& 4+2\times 1+5 = 4+2+5=11.\\
\Step{2}\;& \Rightarrow \begin{bmatrix}4+2\times 1+5\end{bmatrix}
=\begin{bmatrix}11\end{bmatrix}.\\
\Step{3}\;& \text{Rows}=1,\ \text{Columns}=1,\ \text{Order}=1\times 1.
\end{aligned}
\]
\end{QAPair}

% ============================================================
% Q2
\begin{QAPair}{Question 2 (a)}
\textcolor{gold}{\bfseries Question:} Check whether the matrices are equal:
\[
\begin{bmatrix}1&0\\0&1\end{bmatrix},\quad
\begin{bmatrix}\sqrt{25}-\sqrt{25} & \dfrac{\sqrt{25}}{5}\\[6pt]
\dfrac{\sqrt{16}}{4} & \sqrt{16-16}\end{bmatrix}.
\]
\tcblower
\textcolor{green}{\bfseries Answer:}
\[
\begin{aligned}
\Step{1}\;& \sqrt{25}-\sqrt{25}=5-5=0,\quad \frac{\sqrt{25}}{5}=\frac{5}{5}=1,\\
&\frac{\sqrt{16}}{4}=\frac{4}{4}=1,\quad \sqrt{16-16}=\sqrt{0}=0.\\[4pt]
\Step{2}\;& \Rightarrow
\begin{bmatrix}\sqrt{25}-\sqrt{25} & \dfrac{\sqrt{25}}{5}\\[6pt]
\dfrac{\sqrt{16}}{4} & \sqrt{16-16}\end{bmatrix}
=
\begin{bmatrix}0&1\\1&0\end{bmatrix}.
\end{aligned}
\]
\[
\Step{3}\; \begin{bmatrix}1&0\\0&1\end{bmatrix}\neq
\begin{bmatrix}0&1\\1&0\end{bmatrix}
\Rightarrow \boxed{\text{Not equal}.}
\]
\end{QAPair}

\begin{QAPair}{Question 2 (b)}
\textcolor{gold}{\bfseries Question:} Check whether the matrices are equal:
\[
\begin{bmatrix}1&2&3\end{bmatrix},\quad
\begin{bmatrix}1+2+3\end{bmatrix}.
\]
\tcblower
\textcolor{green}{\bfseries Answer:}
\[
\begin{aligned}
\Step{1}\;& \begin{bmatrix}1&2&3\end{bmatrix}\ \text{is of order }1\times 3.\\
\Step{2}\;& \begin{bmatrix}1+2+3\end{bmatrix}=\begin{bmatrix}6\end{bmatrix}\ \text{is of order }1\times 1.
\end{aligned}
\]
\[
\Step{3}\; \text{Orders are different} \Rightarrow \boxed{\text{Not equal}.}
\]
\end{QAPair}

\begin{QAPair}{Question 2 (c)}
\textcolor{gold}{\bfseries Question:} Check whether the matrices are equal:
\[
\begin{bmatrix}1\\2\end{bmatrix},\quad
\begin{bmatrix}1\\2\end{bmatrix}.
\]
\tcblower
\textcolor{green}{\bfseries Answer:}
\[
\Step{1}\; \text{Same order }(2\times 1)\ \text{and same entries} \Rightarrow \boxed{\text{Equal}.}
\]
\end{QAPair}

\begin{QAPair}{Question 2 (d)}
\textcolor{gold}{\bfseries Question:} Check whether the matrices are equal:
\[
\begin{bmatrix}1+0&0+0\\0\times 0&1-0\end{bmatrix},\quad
\begin{bmatrix}\dfrac{5\times 5}{25}&\dfrac{3\times 0}{7}\\[6pt]
2-17+15&17\times \dfrac{1}{17}\end{bmatrix}.
\]
\tcblower
\textcolor{green}{\bfseries Answer:}
\[
\begin{aligned}
\Step{1}\;&
\begin{bmatrix}1+0&0+0\\0\times 0&1-0\end{bmatrix}
=
\begin{bmatrix}1&0\\0&1\end{bmatrix}.\\[6pt]
\Step{2}\;&
\begin{bmatrix}\dfrac{5\times 5}{25}&\dfrac{3\times 0}{7}\\[6pt]
2-17+15&17\times \dfrac{1}{17}\end{bmatrix}
=
\begin{bmatrix}1&0\\0&1\end{bmatrix}.
\end{aligned}
\]
\[
\Step{3}\; \boxed{\text{Equal}.}
\]
\end{QAPair}

% ============================================================
% Q3
\begin{QAPair}{Question 3}
\textcolor{gold}{\bfseries Question:} Hyder, Hassan and Ahmed scored $7$, $11$ and $10$ points. Display in a row matrix $(R)$ and a column matrix $(C)$.\\
\tcblower
\textcolor{green}{\bfseries Answer:}
\[
\Step{1}\; R=\begin{bmatrix}7&11&10\end{bmatrix},\qquad
C=\begin{bmatrix}7\\11\\10\end{bmatrix}.
\]
\end{QAPair}

% ============================================================
% Q4
\begin{QAPair}{Question 4}
\textcolor{gold}{\bfseries Question:} Write the most appropriate type of each matrix.\\
\tcblower
\textcolor{green}{\bfseries Answer:}

\textcolor{muted}{(a)}\;
$\begin{bmatrix}5&0\\0&5\end{bmatrix}$
\[
\Step{1}\; \text{Diagonal matrix (also a \textbf{scalar} matrix: }5I\text{).}
\]

\textcolor{muted}{(b)}\;
$\begin{bmatrix}0&0\\0&3+0\end{bmatrix}=\begin{bmatrix}0&0\\0&3\end{bmatrix}$
\[
\Step{2}\; \text{Diagonal matrix.}
\]

\textcolor{muted}{(c)}\;
$\begin{bmatrix}15\times 3\\ 9\times 5\end{bmatrix}=\begin{bmatrix}45\\45\end{bmatrix}$
\[
\Step{3}\; \text{Column matrix (order }2\times 1\text{).}
\]

\textcolor{muted}{(d)}\;
$\begin{bmatrix}\dfrac{\sqrt{625}}{25} & 17+15-32\\ 18-19+1 & \dfrac{3}{\sqrt{9}}\end{bmatrix}
=\begin{bmatrix}1&0\\0&1\end{bmatrix}$
\[
\Step{4}\; \text{Identity matrix.}
\]

\textcolor{muted}{(e)}\;
$\begin{bmatrix}6+2&0\\-3+3&8\end{bmatrix}
=\begin{bmatrix}8&0\\0&8\end{bmatrix}$
\[
\Step{5}\; \text{Diagonal matrix (also a \textbf{scalar} matrix: }8I\text{).}
\]
\end{QAPair}

% ============================================================
% Q5
\begin{QAPair}{Question 5}
\textcolor{gold}{\bfseries Question:} Check for symmetric and skew-symmetric matrices.\\
\tcblower
\textcolor{green}{\bfseries Answer:}

\textcolor{muted}{(a)}\;
$A=\begin{bmatrix}5&2&3\\2&9&6\\3&6&3\end{bmatrix}$
\[
\Step{1}\; A^T=\begin{bmatrix}5&2&3\\2&9&6\\3&6&3\end{bmatrix}=A
\Rightarrow \boxed{\text{Symmetric}.}
\]

\textcolor{muted}{(b)}\;
$B=\begin{bmatrix}0&9\\-9&0\end{bmatrix}$
\[
\Step{2}\; B^T=\begin{bmatrix}0&-9\\9&0\end{bmatrix}=-B
\Rightarrow \boxed{\text{Skew-symmetric}.}
\]

\textcolor{muted}{(c)}\;
$C=\begin{bmatrix}0&-2&-4\\2&0&-2\\4&2&0\end{bmatrix}$
\[
\Step{3}\; C^T=\begin{bmatrix}0&2&4\\-2&0&2\\-4&-2&0\end{bmatrix}=-C
\Rightarrow \boxed{\text{Skew-symmetric}.}
\]

\textcolor{muted}{(d)}\;
$D=\begin{bmatrix}0&-7\\7&0\end{bmatrix}$
\[
\Step{4}\; D^T=\begin{bmatrix}0&7\\-7&0\end{bmatrix}=-D
\Rightarrow \boxed{\text{Skew-symmetric}.}
\]

\textcolor{muted}{(e)}\;
$E=\begin{bmatrix}0&-5\\-5&0\end{bmatrix}$
\[
\Step{5}\; E^T=\begin{bmatrix}0&-5\\-5&0\end{bmatrix}=E
\Rightarrow \boxed{\text{Symmetric}.}
\]
\end{QAPair}

% ============================================================
% Q6
\begin{QAPair}{Question 6}
\textcolor{gold}{\bfseries Question:} Write a symmetric and a skew-symmetric matrix of order $3$.\\
\tcblower
\textcolor{green}{\bfseries Answer:}

\textcolor{muted}{\textbf{One symmetric matrix of order 3:}}
\[
S=
\begin{bmatrix}
1&2&3\\
2&0&4\\
3&4&5
\end{bmatrix}
\qquad (\text{since } S^T=S).
\]

\textcolor{muted}{\textbf{One skew-symmetric matrix of order 3:}}
\[
K=
\begin{bmatrix}
0&1&-2\\
-1&0&3\\
2&-3&0
\end{bmatrix}
\qquad (\text{since } K^T=-K).
\]
\end{QAPair}

\end{document}
