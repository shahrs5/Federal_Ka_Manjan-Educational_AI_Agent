% !TEX TS-program = pdflatex
\documentclass[11pt]{article}

% -------------------- Packages --------------------
\usepackage[a4paper,margin=1in]{geometry}
\usepackage{amsmath,amssymb}
\usepackage[T1]{fontenc}
\usepackage{lmodern}
\usepackage{xcolor}
\usepackage{tcolorbox}
\tcbuselibrary{skins,breakable}
\usepackage{enumitem}
\usepackage{hyperref}
\usepackage{tikz}
\usetikzlibrary{calc,patterns,angles,quotes,intersections}

\pagestyle{empty}

% -------------------- Dark Theme Colors --------------------
\definecolor{bg}{HTML}{000000}
\definecolor{pairbg}{HTML}{121212}
\definecolor{solbg}{HTML}{0A0A0A}
\definecolor{border}{HTML}{2A2A2A}
\definecolor{text}{HTML}{FFFFFF}
\definecolor{muted}{HTML}{C9CDD3}
\definecolor{gold}{HTML}{FFD700}
\definecolor{green}{HTML}{4ADE80}
\definecolor{cyan}{HTML}{38BDF8}

\pagecolor{bg}
\color{text}

\hypersetup{
  colorlinks=true,
  linkcolor=cyan,
  urlcolor=cyan
}

\setlength{\parindent}{0pt}
\setlength{\parskip}{10pt}

\setlist[itemize]{left=1.4em,itemsep=6pt,topsep=6pt}
\setlist[enumerate]{left=1.6em,itemsep=4pt,topsep=4pt}

% -------------------- tcolorbox Base --------------------
\tcbset{
  enhanced,
  breakable,
  arc=12pt,
  boxrule=0.8pt,
  left=16pt,right=16pt,top=12pt,bottom=12pt
}

\newtcolorbox{QAPair}[1]{%
  colback=pairbg,
  colbacklower=solbg,
  colframe=border,
  coltext=text,
  title=\textcolor{gold}{\bfseries #1},
  fonttitle=\bfseries,
  coltitle=text,
  segmentation style={draw=border, dashed, line width=0.6pt},
}

\newtcolorbox{QuickBox}{%
  colback=pairbg,
  colframe=cyan,
  coltext=text,
  fontupper=\color{text},
  borderline north={4pt}{0pt}{cyan},
  arc=14pt,
  boxrule=0.8pt
}

% Helper for step headings
\newcommand{\Step}[1]{\textcolor{muted}{\textbf{Step #1:}}}

% -------------------- TikZ Styles --------------------
\tikzset{
  geom/.style={draw=muted, line width=0.95pt},
  strong/.style={draw=cyan, line width=1.05pt},
  helper/.style={draw=muted, dashed, line width=0.75pt},
  arcH/.style={draw=muted, dashed, line width=0.75pt},
  pt/.style={circle, fill=cyan, inner sep=1.2pt},
  lab/.style={text=text, font=\small},
  ang/.style={draw=cyan, line width=0.9pt},
  note/.style={text=muted, font=\small},
  axes/.style={draw=muted, line width=0.95pt, ->},
  grid/.style={draw=border, line width=0.35pt},
  bline/.style={draw=gold, line width=1.05pt},
  blineD/.style={draw=gold, dashed, line width=1.05pt},
  sol/.style={draw=green, line width=1.25pt},
  region/.style={pattern=north east lines, pattern color=cyan},
}

% -------------------- Step + Diagram Macro --------------------
% Usage:
% \StepFig{1}{<text>}{<tikzpicture contents ONLY>}
\newcommand{\StepFig}[3]{%
  \Step{#1} #2\par\medskip
  \begin{center}
    \begin{tikzpicture}[scale=0.92]
      #3
    \end{tikzpicture}
  \end{center}
  \vspace{-2pt}
}

% tiny right-angle mark macro (not essential here, but kept from template)
\newcommand{\RightAngleMark}[2]{%
  % #1 = corner point, #2 = size
  \draw[ang] ($(#1)+(#2,0)$) -- ($(#1)+(#2,#2)$) -- ($(#1)+(0,#2)$);
}

% ============================================================
\begin{document}

\begin{center}
{\Large\bfseries \textcolor{gold}{Miscellaneous Exercise 4 --- Solutions}}\\[-2pt]
\end{center}

\begin{QuickBox}
{\color{cyan}\bfseries Quick facts (very useful)}\par\medskip
\begin{itemize}
\item \textbf{Solve linear inequalities} like equations, but \textbf{reverse} the sign when multiplying/dividing by a \textbf{negative}.
\item \textbf{Number line:} \textbf{closed dot} for $\le,\ge$; \textbf{open dot} for $<,>$; shade in the solution direction.
\item \textbf{Two-variable inequalities:} draw the boundary line:
  \textbf{solid} for $\le,\ge$ and \textbf{dashed} for $<,>$, then shade the correct half-plane using a \textbf{test point} (usually $(0,0)$ if it is not on the line).
\item \textbf{System of inequalities:} the solution is the \textbf{overlapping (common)} shaded region.
\end{itemize}
\end{QuickBox}

% ============================================================
% Q1 (MCQs)
\begin{QAPair}{Question 1 (i) --- MCQ}
\textcolor{gold}{\bfseries Question:} The solution of the inequality $6x-7 \ge 2x+17$ is:\par
(a) $x<6$ \quad (b) $x\le 6$ \quad (c) $x>6$ \quad (d) $x\ge 6$
\tcblower
\textcolor{green}{\bfseries Answer:} \textbf{(d) $x\ge 6$}\par

\Step{1:} Solve:
\[
6x-7 \ge 2x+17
\Rightarrow 4x \ge 24
\Rightarrow x \ge 6.
\]

\StepFig{2}{Graph on a number line (closed dot at $6$, shade to the right).}{%
  \begin{scope}[x=0.8cm,y=0.9cm]
    \draw[geom] (-1,0)--(11,0);
    \foreach \x in {0,1,...,10} {
      \draw[geom] (\x,0.12)--(\x,-0.12);
      \node[lab, below] at (\x,-0.22) {\scriptsize \x};
    }
    \fill[green] (6,0) circle (2.2pt);
    \draw[sol] (6,0)--(10.8,0);
    \draw[sol] (10.8,0)--(10.4,0.18);
    \draw[sol] (10.8,0)--(10.4,-0.18);
    \node[note] at (7.9,0.55) {$x\ge 6$};
  \end{scope}
}
\end{QAPair}

\begin{QAPair}{Question 1 (ii) --- MCQ}
\textcolor{gold}{\bfseries Question:} The solution of the inequality $x<3$ for $x\in\mathbb{N}$ is:\par
(a) $\{0,1,2\}$ \quad (b) $\{-2,-2,0,1,2\}$ \quad (c) $\{1,2\}$ \quad (d) $\{1\}$
\tcblower
\textcolor{green}{\bfseries Answer:} \textbf{(c) $\{1,2\}$}\par

\Step{1:} Since $\mathbb{N}=\{1,2,3,\dots\}$ in this context, numbers less than $3$ are $1,2$.

\StepFig{2}{Mark only the natural numbers $1$ and $2$ (discrete points).}{%
  \begin{scope}[x=1.0cm,y=1.0cm]
    \draw[geom] (-0.5,0)--(4.5,0);
    \foreach \x in {0,1,2,3,4} {
      \draw[geom] (\x,0.12)--(\x,-0.12);
      \node[lab, below] at (\x,-0.25) {\scriptsize \x};
    }
    \fill[green] (1,0) circle (2.0pt);
    \fill[green] (2,0) circle (2.0pt);
    \node[note] at (2.2,0.55) {$\{1,2\}$};
  \end{scope}
}
\end{QAPair}

\begin{QAPair}{Question 1 (iii) --- MCQ}
\textcolor{gold}{\bfseries Question:} In general, we use a test point when graphing the inequality:\par
(a) $(-1,-1)$ \quad (b) $(0,0)$ \quad (c) $(2,2)$ \quad (d) $(-3,-3)$
\tcblower
\textcolor{green}{\bfseries Answer:} \textbf{(b) $(0,0)$}\par

\Step{1:} We usually choose $(0,0)$ because it is easy to substitute (unless the boundary line passes through it).

\StepFig{2}{Example: For the boundary line $x+y=1$, $(0,0)$ is \emph{not} on the line, so it is a good test point.}{%
  \begin{scope}[x=0.65cm,y=0.65cm]
    \draw[grid] (-3,-3) grid (5,5);
    \draw[axes] (-3,0)--(5,0) node[lab, right] {$x$};
    \draw[axes] (0,-3)--(0,5) node[lab, above] {$y$};

    % boundary line x + y = 1  -> y = 1 - x
    \draw[bline] (-3,4) -- (5,-4);

    \fill[green] (0,0) circle (2.2pt);
    \node[lab, above right] at (0,0) {$(0,0)$};

    \node[note] at (2.2,4.5) {Use $(0,0)$ if not on boundary};
  \end{scope}
}
\end{QAPair}

\begin{QAPair}{Question 1 (iv) --- MCQ}
\textcolor{gold}{\bfseries Question:} The solution of the $\dfrac{x}{-2}<3-x$ is:\par
(a) $x<2$ \quad (b) $x<6$ \quad (c) $x>2$ \quad (d) $x<-2$
\tcblower
\textcolor{green}{\bfseries Answer:} \textbf{(b) $x<6$}\par

\Step{1:} Solve:
\[
\frac{x}{-2} < 3-x
\Rightarrow -\frac{x}{2}<3-x
\Rightarrow -x < 6-2x
\Rightarrow x < 6.
\]

\StepFig{2}{Graph (open dot at $6$, shade to the left).}{%
  \begin{scope}[x=0.8cm,y=0.9cm]
    \draw[geom] (-1,0)--(11,0);
    \foreach \x in {0,1,...,10} {
      \draw[geom] (\x,0.12)--(\x,-0.12);
      \node[lab, below] at (\x,-0.22) {\scriptsize \x};
    }
    \draw[sol] (-0.8,0)--(6,0);
    \draw[sol] (-0.8,0)--(-0.4,0.18);
    \draw[sol] (-0.8,0)--(-0.4,-0.18);
    \draw[green, line width=1.0pt] (6,0) circle (2.4pt); % open circle
    \node[note] at (3.1,0.55) {$x<6$};
  \end{scope}
}
\end{QAPair}

\begin{QAPair}{Question 1 (v) --- MCQ}
\textcolor{gold}{\bfseries Question:} Which ordered pair is a solution of the inequality $4x-y\ge 3$?\par
(a) $(0,0)$ \quad (b) $(-1,2)$ \quad (c) $(1,1)$ \quad (d) $(0,-2)$
\tcblower
\textcolor{green}{\bfseries Answer:} \textbf{(c) $(1,1)$}\par

\Step{1:} Check $(1,1)$:
\[
4(1)-1 = 3 \ge 3 \quad \Rightarrow \quad \text{true}.
\]

\StepFig{2}{Graph the boundary line $4x-y=3$ and mark $(1,1)$.}{%
  \begin{scope}[x=0.75cm,y=0.75cm]
    \draw[grid] (-1,-2) grid (5,5);
    \draw[axes] (-1,0)--(5,0) node[lab, right] {$x$};
    \draw[axes] (0,-2)--(0,5) node[lab, above] {$y$};

    % boundary: y = 4x - 3
    \draw[bline] (-0.2,-3.8) -- (2,5); % visible part

    % point (1,1)
    \fill[green] (1,1) circle (2.2pt);
    \node[lab, above right] at (1,1) {$(1,1)$};

    \node[note] at (3.1,4.4) {$4x-y=3$};
  \end{scope}
}
\end{QAPair}

\begin{QAPair}{Question 1 (vi) --- MCQ}
\textcolor{gold}{\bfseries Question:} Which ordered pair is a solution of the system $2x-y\le 5$ and $x+2y>2$?\par
(a) $(1,-1)$ \quad (b) $(4,1)$ \quad (c) $(2,0)$ \quad (d) $(3,2)$
\tcblower
\textcolor{green}{\bfseries Answer:} \textbf{(d) $(3,2)$}\par

\Step{1:} Check $(3,2)$:
\[
2(3)-2=4\le 5 \quad \text{and}\quad 3+2(2)=7>2.
\]
So it satisfies both.

\StepFig{2}{Graph both boundaries and mark $(3,2)$ (it lies in the common region).}{%
  \begin{scope}[x=0.65cm,y=0.65cm]
    \draw[grid] (-1,-1) grid (7,7);
    \draw[axes] (-1,0)--(7,0) node[lab, right] {$x$};
    \draw[axes] (0,-1)--(0,7) node[lab, above] {$y$};

    % 2x - y = 5  -> y = 2x - 5 (solid)
    \draw[bline] (-1,-7) -- (7,9);
    \node[note] at (5.9,6.4) {$2x-y=5$};

    % x + 2y = 2  -> y = 1 - x/2 (dashed for strict >)
    \draw[blineD] (-1,1.5) -- (7,-2.5);
    \node[note] at (4.7,0.1) {$x+2y=2$};

    % point (3,2)
    \fill[green] (3,2) circle (2.2pt);
    \node[lab, above right] at (3,2) {$(3,2)$};
  \end{scope}
}
\end{QAPair}

\begin{QAPair}{Question 1 (vii) --- MCQ}
\textcolor{gold}{\bfseries Question:} The solution for $-5x\ge -80$ is:\par
(a) $\{x\mid x<16\}$ \quad (b) $\{x\mid x\le 16\}$ \quad (c) $\{x\mid x>16\}$ \quad (d) $\{x\mid x\ge 16\}$
\tcblower
\textcolor{green}{\bfseries Answer:} \textbf{(b) $\{x\mid x\le 16\}$}\par

\Step{1:} Divide by $-5$ (reverse the sign):
\[
-5x\ge -80 \Rightarrow x \le 16.
\]

\StepFig{2}{Graph (closed dot at $16$, shade to the left).}{%
  \begin{scope}[x=0.45cm,y=0.9cm]
    \draw[geom] (-2,0)--(22,0);
    \foreach \x in {0,2,4,...,20} {
      \draw[geom] (\x,0.12)--(\x,-0.12);
      \node[lab, below] at (\x,-0.22) {\scriptsize \x};
    }
    \fill[green] (16,0) circle (2.2pt);
    \draw[sol] (-1.8,0)--(16,0);
    \draw[sol] (-1.8,0)--(-1.4,0.18);
    \draw[sol] (-1.8,0)--(-1.4,-0.18);
    \node[note] at (8,0.55) {$x\le 16$};
  \end{scope}
}
\end{QAPair}

\begin{QAPair}{Question 1 (viii) --- MCQ}
\textcolor{gold}{\bfseries Question:} $(7,-2)$ is a solution of the inequality;\par
(a) $x-y<-4$ \quad (b) $2x+y<10$ \quad (c) $x+10y<1$ \quad (d) $-x-y>-3$
\tcblower
\textcolor{green}{\bfseries Answer:} \textbf{(c) $x+10y<1$}\par

\Step{1:} Substitute $(7,-2)$ into (c):
\[
7+10(-2)=7-20=-13<1 \quad \Rightarrow \quad \text{true}.
\]

\StepFig{2}{Graph the boundary line $x+10y=1$ and mark $(7,-2)$ (it lies below the line, so it satisfies $<$).}{%
  \begin{scope}[x=0.55cm,y=0.55cm]
    \draw[grid] (-1,-4) grid (9,4);
    \draw[axes] (-1,0)--(9,0) node[lab, right] {$x$};
    \draw[axes] (0,-4)--(0,4) node[lab, above] {$y$};

    % x + 10y = 1 -> y = (1-x)/10 (dashed because strict <)
    \draw[blineD] (-1,0.2) -- (9,-0.8);
    \node[note] at (6.8,1.0) {$x+10y=1$};

    % point (7,-2)
    \fill[green] (7,-2) circle (2.2pt);
    \node[lab, above right] at (7,-2) {$(7,-2)$};
  \end{scope}
}
\end{QAPair}

% ============================================================
% Q2--Q5: Solve inequalities, graph the solution
\begin{QAPair}{Question 2 --- Solve $\dfrac{2}{3}x - 4 \ge 1$ and graph}
\textcolor{gold}{\bfseries Solution:}
\tcblower
\Step{1:} Add $4$ on both sides:
\[
\frac{2}{3}x \ge 5
\]
\Step{2:} Multiply by $\frac{3}{2}$:
\[
x \ge \frac{15}{2}.
\]

\StepFig{3}{Number line graph (closed dot at $\frac{15}{2}=7.5$, shade right).}{%
  \begin{scope}[x=0.9cm,y=0.9cm]
    \draw[geom] (-1,0)--(11,0);
    \foreach \x/\lab in {0/0,1/1,2/2,3/3,4/4,5/5,6/6,7/7,8/8,9/9,10/10} {
      \draw[geom] (\x,0.12)--(\x,-0.12);
      \node[lab, below] at (\x,-0.22) {\scriptsize \lab};
    }
    \fill[green] (7.5,0) circle (2.2pt);
    \draw[sol] (7.5,0)--(10.8,0);
    \draw[sol] (10.8,0)--(10.4,0.18);
    \draw[sol] (10.8,0)--(10.4,-0.18);
    \node[note] at (8.7,0.55) {$x\ge \frac{15}{2}$};
  \end{scope}
}
\end{QAPair}

\begin{QAPair}{Question 3 --- Solve $1-3x \le -14 + 2x$ and graph}
\textcolor{gold}{\bfseries Solution:}
\tcblower
\Step{1:} Add $3x$ to both sides:
\[
1 \le -14 + 5x
\]
\Step{2:} Add $14$:
\[
15 \le 5x
\Rightarrow x \ge 3.
\]

\StepFig{3}{Number line graph (closed dot at $3$, shade right).}{%
  \begin{scope}[x=0.9cm,y=0.9cm]
    \draw[geom] (-1,0)--(11,0);
    \foreach \x in {0,1,...,10} {
      \draw[geom] (\x,0.12)--(\x,-0.12);
      \node[lab, below] at (\x,-0.22) {\scriptsize \x};
    }
    \fill[green] (3,0) circle (2.2pt);
    \draw[sol] (3,0)--(10.8,0);
    \draw[sol] (10.8,0)--(10.4,0.18);
    \draw[sol] (10.8,0)--(10.4,-0.18);
    \node[note] at (6.2,0.55) {$x\ge 3$};
  \end{scope}
}
\end{QAPair}

\begin{QAPair}{Question 4 --- Solve $7(x-1) > -8 + 7x$ and graph}
\textcolor{gold}{\bfseries Solution:}
\tcblower
\Step{1:} Expand:
\[
7x-7 > -8 + 7x
\]
\Step{2:} Subtract $7x$ from both sides:
\[
-7 > -8
\]
This is \textbf{always true}, so \textbf{every real number} satisfies the inequality.\par
\textcolor{green}{\bfseries Answer:} $x\in\mathbb{R}$.

\StepFig{3}{Number line graph (all real numbers).}{%
  \begin{scope}[x=0.8cm,y=0.9cm]
    \draw[sol] (-0.5,0)--(10.5,0);
    \draw[sol] (-0.5,0)--(-0.1,0.18);
    \draw[sol] (-0.5,0)--(-0.1,-0.18);
    \draw[sol] (10.5,0)--(10.1,0.18);
    \draw[sol] (10.5,0)--(10.1,-0.18);

    \draw[geom] (-0.8,0)--(10.8,0);
    \foreach \x in {0,1,...,10} {
      \draw[geom] (\x,0.12)--(\x,-0.12);
    }
    \node[note] at (5,0.55) {All real numbers $\mathbb{R}$};
  \end{scope}
}
\end{QAPair}

\begin{QAPair}{Question 5 --- Solve $-3(2x-1)\ge 1-8x$ and graph}
\textcolor{gold}{\bfseries Solution:}
\tcblower
\Step{1:} Expand:
\[
-6x+3 \ge 1-8x
\]
\Step{2:} Add $8x$:
\[
2x+3 \ge 1
\]
\Step{3:} Subtract $3$:
\[
2x \ge -2
\Rightarrow x \ge -1.
\]

\StepFig{4}{Number line graph (closed dot at $-1$, shade right).}{%
  \begin{scope}[x=0.9cm,y=0.9cm]
    \draw[geom] (-6,0)--(6,0);
    \foreach \x in {-5,-4,...,5} {
      \draw[geom] (\x,0.12)--(\x,-0.12);
      \node[lab, below] at (\x,-0.22) {\scriptsize \x};
    }
    \fill[green] (-1,0) circle (2.2pt);
    \draw[sol] (-1,0)--(5.8,0);
    \draw[sol] (5.8,0)--(5.4,0.18);
    \draw[sol] (5.8,0)--(5.4,-0.18);
    \node[note] at (1.7,0.55) {$x\ge -1$};
  \end{scope}
}
\end{QAPair}

% ============================================================
% Q6--Q9: Graph on a plane
\begin{QAPair}{Question 6 --- Graph $y>2x$}
\textcolor{gold}{\bfseries Graph:} Boundary is $y=2x$ (dashed), shade \textbf{above}.
\tcblower
\Step{1:} Draw boundary line $y=2x$ (dashed because $>$).\par
\Step{2:} Test point $(0,0)$: $0>0$ is false, so shade the \textbf{other side} (above the line).

\StepFig{3}{Graph of $y>2x$.}{%
  \begin{scope}[x=0.75cm,y=0.75cm]
    \draw[grid] (-4,-4) grid (4,6);
    \draw[axes] (-4,0)--(4,0) node[lab, right] {$x$};
    \draw[axes] (0,-4)--(0,6) node[lab, above] {$y$};

    % shaded region within window (above y=2x)
    \fill[region] (-4,6) -- (3,6) -- (-2,-4) -- (-4,-4) -- cycle;

    % boundary segment inside window: from (-2,-4) to (3,6)
    \draw[blineD] (-2,-4) -- (3,6);
    \node[note] at (2.4,5.2) {$y=2x$};

    \fill[green] (0,0) circle (2.0pt);
    \node[note] at (0.8,-0.6) {$(0,0)$ not included};
  \end{scope}
}
\end{QAPair}

\begin{QAPair}{Question 7 --- Graph $y<3x$}
\textcolor{gold}{\bfseries Graph:} Boundary is $y=3x$ (dashed), shade \textbf{below}.
\tcblower
\Step{1:} Draw boundary line $y=3x$ (dashed because $<$).\par
\Step{2:} Test point $(0,0)$: $0<0$ is false, so shade the \textbf{other side} (below the line).

\StepFig{3}{Graph of $y<3x$.}{%
  \begin{scope}[x=0.75cm,y=0.75cm]
    \draw[grid] (-4,-4) grid (4,6);
    \draw[axes] (-4,0)--(4,0) node[lab, right] {$x$};
    \draw[axes] (0,-4)--(0,6) node[lab, above] {$y$};

    % boundary segment inside window: line y=3x hits bottom at x=-4/3 and top at x=2
    \coordinate (A) at (-4/3,-4);
    \coordinate (B) at (2,6);

    % shaded region below the line within window
    \fill[region] (A) -- (4,-4) -- (4,6) -- (B) -- cycle;

    \draw[blineD] (A) -- (B);
    \node[note] at (2.8,5.2) {$y=3x$};

    \fill[green] (0,0) circle (2.0pt);
    \node[note] at (0.9,0.6) {$(0,0)$ not included};
  \end{scope}
}
\end{QAPair}

\begin{QAPair}{Question 8 --- Graph $y\ge -5$}
\textcolor{gold}{\bfseries Graph:} Boundary is $y=-5$ (solid), shade \textbf{above}.
\tcblower
\Step{1:} Draw $y=-5$ (solid because $\ge$).\par
\Step{2:} Shade the region \textbf{above} the line.

\StepFig{3}{Graph of $y\ge -5$.}{%
  \begin{scope}[x=0.7cm,y=0.7cm]
    \draw[grid] (-6,-6) grid (6,6);
    \draw[axes] (-6,0)--(6,0) node[lab, right] {$x$};
    \draw[axes] (0,-6)--(0,6) node[lab, above] {$y$};

    \fill[region] (-6,-5) -- (6,-5) -- (6,6) -- (-6,6) -- cycle;
    \draw[bline] (-6,-5)--(6,-5);
    \node[note] at (3.6,-4.4) {$y=-5$};
  \end{scope}
}
\end{QAPair}

\begin{QAPair}{Question 9 --- Graph $x\le 6$}
\textcolor{gold}{\bfseries Graph:} Boundary is $x=6$ (solid), shade \textbf{left}.
\tcblower
\Step{1:} Draw $x=6$ (solid because $\le$).\par
\Step{2:} Shade all points with $x\le 6$ (left side, including the line).

\StepFig{3}{Graph of $x\le 6$.}{%
  \begin{scope}[x=0.6cm,y=0.6cm]
    \draw[grid] (-2,-6) grid (10,6);
    \draw[axes] (-2,0)--(10,0) node[lab, right] {$x$};
    \draw[axes] (0,-6)--(0,6) node[lab, above] {$y$};

    \fill[region] (-2,-6) -- (6,-6) -- (6,6) -- (-2,6) -- cycle;
    \draw[bline] (6,-6)--(6,6);
    \node[note] at (6.5,5.3) {$x=6$};
  \end{scope}
}
\end{QAPair}

% ============================================================
% Q10--Q13: Graph the system of inequalities
\begin{QAPair}{Question 10 --- Graph the system $3x+4y\ge 12$ and $5x+6y\le 30$}
\textcolor{gold}{\bfseries System:}
\[
3x+4y\ge 12 \quad \text{and}\quad 5x+6y\le 30
\]
\tcblower
\Step{1:} Write in slope-intercept form:
\[
3x+4y\ge 12 \Rightarrow y \ge 3-\frac{3}{4}x,
\qquad
5x+6y\le 30 \Rightarrow y \le 5-\frac{5}{6}x.
\]
\Step{2:} Shade \textbf{above} the first line and \textbf{below} the second line.\par
\textcolor{green}{\bfseries Solution region:} the overlap (a band between the two lines).

\StepFig{3}{Graph of the system (common shaded region).}{%
  \begin{scope}[x=0.55cm,y=0.55cm]
    \draw[grid] (-2,-6) grid (10,8);
    \draw[axes] (-2,0)--(10,0) node[lab, right] {$x$};
    \draw[axes] (0,-6)--(0,8) node[lab, above] {$y$};

    % Define endpoints at x=-2 and x=10 for both lines
    \coordinate (L1a) at (-2,9/2);      % 3x+4y=12
    \coordinate (L1b) at (10,-9/2);
    \coordinate (L2a) at (-2,20/3);     % 5x+6y=30
    \coordinate (L2b) at (10,-10/3);

    % Shade between lines within this window
    \fill[region] (L2a) -- (L2b) -- (L1b) -- (L1a) -- cycle;

    \draw[bline] (L1a)--(L1b);
    \node[note] at (7.5,-3.9) {$3x+4y=12$};

    \draw[bline] (L2a)--(L2b);
    \node[note] at (7.5,-2.1) {$5x+6y=30$};
  \end{scope}
}
\end{QAPair}

\begin{QAPair}{Question 11 --- Graph the system $y-x\ge 1$ and $y-4\le 4$}
\textcolor{gold}{\bfseries System:}
\[
y-x\ge 1 \quad \text{and}\quad y-4\le 4
\]
\tcblower
\Step{1:} Simplify:
\[
y-x\ge 1 \Rightarrow y\ge x+1,
\qquad
y-4\le 4 \Rightarrow y\le 8.
\]
\Step{2:} Shade \textbf{above} $y=x+1$ and \textbf{below} $y=8$.\par
\textcolor{green}{\bfseries Solution region:} common shaded part.

\StepFig{3}{Graph of the system.}{%
  \begin{scope}[x=0.55cm,y=0.55cm]
    \draw[grid] (-4,-4) grid (10,10);
    \draw[axes] (-4,0)--(10,0) node[lab, right] {$x$};
    \draw[axes] (0,-4)--(0,10) node[lab, above] {$y$};

    % Lines: y=x+1 and y=8
    \coordinate (A) at (-4,-3);
    \coordinate (B) at (7,8);      % intersection with y=8
    \coordinate (T1) at (-4,8);
    \coordinate (T2) at (7,8);

    % Shade region above y=x+1 and below y=8 within window
    \fill[region] (T1) -- (T2) -- (B) -- (A) -- cycle;

    \draw[bline] (-4,-3) -- (10,11); % y=x+1 (visible)
    \node[note] at (6.8,9.0) {$y=x+1$};

    \draw[bline] (-4,8)--(10,8);
    \node[note] at (8.6,8.5) {$y=8$};
  \end{scope}
}
\end{QAPair}

\begin{QAPair}{Question 12 --- Graph the system $8x+5y\le 40$ and $x\ge 0$}
\textcolor{gold}{\bfseries System:}
\[
8x+5y\le 40 \quad \text{and}\quad x\ge 0
\]
\tcblower
\Step{1:} Solve for $y$:
\[
8x+5y\le 40 \Rightarrow y\le 8-\frac{8}{5}x.
\]
\Step{2:} Shade \textbf{below} the line and to the \textbf{right} of the $y$-axis ($x\ge 0$).

\StepFig{3}{Graph of the system.}{%
  \begin{scope}[x=0.75cm,y=0.6cm]
    \draw[grid] (-1,-2) grid (8,10);
    \draw[axes] (-1,0)--(8,0) node[lab, right] {$x$};
    \draw[axes] (0,-2)--(0,10) node[lab, above] {$y$};

    % boundary line 8x+5y=40: intercepts (0,8), (5,0)
    \coordinate (P) at (0,8);
    \coordinate (Q) at (5,0);
    \draw[bline] (P)--(Q);
    \node[note] at (4.6,3.8) {$8x+5y=40$};

    % x=0 boundary
    \draw[bline] (0,-2)--(0,10);

    % Shade feasible region within window (triangle-ish clip)
    % Use a big polygon covering x>=0 under the line, within the window
    \fill[region] (0,-2) -- (6.25,-2) -- (0,8) -- cycle;

    \node[note] at (3.6,-1.3) {Common region};
  \end{scope}
}
\end{QAPair}

\begin{QAPair}{Question 13 --- Graph the system $2x+y\le 12$ and $y\ge 0$}
\textcolor{gold}{\bfseries System:}
\[
2x+y\le 12 \quad \text{and}\quad y\ge 0
\]
\tcblower
\Step{1:} Write the line:
\[
2x+y=12 \Rightarrow y=12-2x.
\]
\Step{2:} Shade \textbf{below} the line and \textbf{above} the $x$-axis ($y\ge 0$).\par
This forms a triangular region.

\StepFig{3}{Graph of the system.}{%
  \begin{scope}[x=0.6cm,y=0.6cm]
    \draw[grid] (-1,-1) grid (7,13);
    \draw[axes] (-1,0)--(7,0) node[lab, right] {$x$};
    \draw[axes] (0,-1)--(0,13) node[lab, above] {$y$};

    % line y=12-2x: intercepts (0,12) and (6,0)
    \coordinate (A) at (0,12);
    \coordinate (B) at (6,0);
    \draw[bline] (A)--(B);
    \node[note] at (4.8,9.5) {$2x+y=12$};

    % y=0 boundary (x-axis)
    \draw[bline] (-1,0)--(7,0);

    % Shade triangle
    \fill[region] (0,0) -- (0,12) -- (6,0) -- cycle;

    \node[note] at (2.6,3.2) {Solution region};
  \end{scope}
}
\end{QAPair}

% ============================================================
% Q14--Q15: Write a system of inequalities for the described region
\begin{QAPair}{Question 14 --- Rectangle with vertices $(2,1),(2,4),(6,4),(6,1)$}
\textcolor{gold}{\bfseries Required:} Write a system of inequalities for the shaded rectangle.
\tcblower
\Step{1:} The vertical sides are $x=2$ and $x=6$.\par
\Step{2:} The horizontal sides are $y=1$ and $y=4$.\par

\textcolor{green}{\bfseries System of inequalities:}
\[
2\le x \le 6,
\qquad
1\le y \le 4.
\]

\StepFig{3}{Graph of the rectangle region.}{%
  \begin{scope}[x=0.8cm,y=0.8cm]
    \draw[grid] (0,0) grid (8,6);
    \draw[axes] (0,0)--(8,0) node[lab, right] {$x$};
    \draw[axes] (0,0)--(0,6) node[lab, above] {$y$};

    % rectangle
    \fill[region] (2,1) rectangle (6,4);
    \draw[bline] (2,1) rectangle (6,4);

    % vertices
    \fill[green] (2,1) circle (2pt);
    \fill[green] (2,4) circle (2pt);
    \fill[green] (6,4) circle (2pt);
    \fill[green] (6,1) circle (2pt);

    \node[note] at (4,4.6) {$(2,1),(2,4),(6,4),(6,1)$};
  \end{scope}
}
\end{QAPair}

\begin{QAPair}{Question 15 --- Triangle with vertices $(3,0),(3,2),(0,-2)$}
\textcolor{gold}{\bfseries Required:} Write a system of inequalities for the shaded triangle.
\tcblower
\Step{1:} One side is vertical: $x=3$.\par
\Step{2:} Line through $(0,-2)$ and $(3,0)$:
\[
m=\frac{0-(-2)}{3-0}=\frac{2}{3}\quad\Rightarrow\quad y=\frac{2}{3}x-2.
\]
\Step{3:} Line through $(0,-2)$ and $(3,2)$:
\[
m=\frac{2-(-2)}{3-0}=\frac{4}{3}\quad\Rightarrow\quad y=\frac{4}{3}x-2.
\]

\textcolor{green}{\bfseries System of inequalities (inside the triangle):}
\[
x\le 3,
\qquad
y \ge \frac{2}{3}x-2,
\qquad
y \le \frac{4}{3}x-2.
\]

\StepFig{4}{Graph of the triangle region.}{%
  \begin{scope}[x=1.0cm,y=0.8cm]
    \draw[grid] (-1,-4) grid (4,3);
    \draw[axes] (-1,0)--(4,0) node[lab, right] {$x$};
    \draw[axes] (0,-4)--(0,3) node[lab, above] {$y$};

    % vertices
    \coordinate (C) at (0,-2);
    \coordinate (A) at (3,0);
    \coordinate (B) at (3,2);

    % shade triangle
    \fill[region] (C)--(A)--(B)--cycle;

    % draw boundaries
    \draw[bline] (A)--(B); % x=3 side
    \draw[bline] (C)--(A); % y = (2/3)x - 2
    \draw[bline] (C)--(B); % y = (4/3)x - 2

    % mark points
    \fill[green] (A) circle (2pt) node[lab, right] {$(3,0)$};
    \fill[green] (B) circle (2pt) node[lab, right] {$(3,2)$};
    \fill[green] (C) circle (2pt) node[lab, left]  {$(0,-2)$};

    \node[note] at (1.4,2.6) {Common region};
  \end{scope}
}
\end{QAPair}

\end{document}
