% !TEX TS-program = pdflatex
\documentclass[11pt]{article}

% -------------------- Packages --------------------
\usepackage[a4paper,margin=1in]{geometry}
\usepackage{amsmath,amssymb}
\usepackage[T1]{fontenc}
\usepackage{lmodern}
\usepackage{xcolor}
\usepackage{tcolorbox}
\tcbuselibrary{skins,breakable}
\usepackage{enumitem}
\usepackage{hyperref}
\usepackage{tikz}
\usetikzlibrary{calc,patterns,angles,quotes,intersections}

\pagestyle{empty}

% -------------------- Dark Theme Colors --------------------
\definecolor{bg}{HTML}{000000}
\definecolor{pairbg}{HTML}{121212}
\definecolor{solbg}{HTML}{0A0A0A}
\definecolor{border}{HTML}{2A2A2A}
\definecolor{text}{HTML}{FFFFFF}
\definecolor{muted}{HTML}{C9CDD3}
\definecolor{gold}{HTML}{FFD700}
\definecolor{green}{HTML}{4ADE80}
\definecolor{cyan}{HTML}{38BDF8}

\pagecolor{bg}
\color{text}

\hypersetup{
  colorlinks=true,
  linkcolor=cyan,
  urlcolor=cyan
}

\setlength{\parindent}{0pt}
\setlength{\parskip}{10pt}

\setlist[itemize]{left=1.4em,itemsep=6pt,topsep=6pt}
\setlist[enumerate]{left=1.6em,itemsep=4pt,topsep=4pt}

% -------------------- tcolorbox Base --------------------
\tcbset{
  enhanced,
  breakable,
  arc=12pt,
  boxrule=0.8pt,
  left=16pt,right=16pt,top=12pt,bottom=12pt
}

\newtcolorbox{QAPair}[1]{%
  colback=pairbg,
  colbacklower=solbg,
  colframe=border,
  coltext=text,
  title=\textcolor{gold}{\bfseries #1},
  fonttitle=\bfseries,
  coltitle=text,
  segmentation style={draw=border, dashed, line width=0.6pt},
}

\newtcolorbox{QuickBox}{%
  colback=pairbg,
  colframe=cyan,
  coltext=text,
  fontupper=\color{text},
  borderline north={4pt}{0pt}{cyan},
  arc=14pt,
  boxrule=0.8pt
}

% Helper for step headings
\newcommand{\Step}[1]{\textcolor{muted}{\textbf{Step #1:}}}

% -------------------- TikZ Styles --------------------
\tikzset{
  geom/.style={draw=muted, line width=0.95pt},
  strong/.style={draw=cyan, line width=1.05pt},
  helper/.style={draw=muted, dashed, line width=0.75pt},
  arcH/.style={draw=muted, dashed, line width=0.75pt},
  pt/.style={circle, fill=cyan, inner sep=1.2pt},
  lab/.style={text=text, font=\small},
  ang/.style={draw=cyan, line width=0.9pt},
  note/.style={text=muted, font=\small}
}

% -------------------- Step + Diagram Macro --------------------
% Usage:
% \StepFig{1}{<text>}{<tikzpicture contents ONLY>}
%
% FIX: add transform shape so nodes/text scale with coordinates (prevents overlap)
%      also remove negative vspace
\newcommand{\StepFig}[3]{%
  \Step{#1} #2\par\medskip
  \begin{center}
    \begin{tikzpicture}[
      scale=0.92,
      transform shape,
      every node/.style={font=\small}
    ]
      #3
    \end{tikzpicture}
  \end{center}
}

% tiny right-angle mark macro
\newcommand{\RightAngleMark}[2]{%
  % #1 = corner point, #2 = size
  \draw[ang] ($(#1)+(#2,0)$) -- ($(#1)+(#2,#2)$) -- ($(#1)+(0,#2)$);
}

% ============================================================
\begin{document}

\begin{center}
{\LARGE\bfseries \textcolor{gold}{Miscellaneous Exercise 5 --- Solutions}}\\[-2pt]
\end{center}

\begin{QuickBox}
{\color{cyan}\bfseries Quick facts (very useful)}\par\medskip
\begin{itemize}
\item \textbf{Rational expression:} $\dfrac{P(x)}{Q(x)}$ where $P,Q$ are polynomials and $Q(x)\neq 0$.
\item \textbf{Degree of a term:} sum of exponents in the term (e.g.\ $x^2y^3$ has degree $2+3=5$).
\item \textbf{Constant polynomial:} degree $0$.
\item \textbf{Difference of squares:} $x^2-y^2=(x-y)(x+y)$.
\item \textbf{Circumference:} $C=2\pi r$.
\item \textbf{Sphere surface area:} $S=4\pi r^2$.
\item \textbf{Right triangle:} $c^2=a^2+b^2$ (Pythagoras).
\end{itemize}
\end{QuickBox}

% ============================================================
% Q1 (MCQs)
\begin{QAPair}{Question 1 (i) --- MCQ}
\textcolor{gold}{\bfseries Question:} An expression which is the ratio of two polynomials but the polynomial in denominator is non-zero is called\par
(a) polynomial \quad (b) rational expression \quad (c) compound expression \quad (d) irrational expression
\tcblower
\textcolor{green}{\bfseries Answer:} \textbf{(b) rational expression}\par

\StepFig{1}{A rational expression is a \emph{fraction} of polynomials.}{%
  \node[draw=cyan, rounded corners=10pt, inner sep=8pt, text=text] (num) at (0,0.9) {$P(x)$};
  \draw[geom] (-1.2,0.4) -- (1.2,0.4);
  \node[draw=cyan, rounded corners=10pt, inner sep=8pt, text=text] (den) at (0,-0.1) {$Q(x)\neq 0$};
  \node[note] at (0,-0.9) {$\dfrac{P(x)}{Q(x)}$};
}
\end{QAPair}

\begin{QAPair}{Question 1 (ii) --- MCQ}
\textcolor{gold}{\bfseries Question:} The degree of $x^2y^3-\dfrac{xy^2z^3}{y}-\sqrt{25}\,z^5$ is\par
(a) 5 \quad (b) 6 \quad (c) 7 \quad (d) none
\tcblower
\textcolor{green}{\bfseries Answer:} \textbf{(a) 5}\par

\StepFig{1}{Find degrees of each term and take the maximum.}{%
  \node[draw=border, rounded corners=10pt, inner sep=8pt, text=text, align=left] at (0,1.2)
  {$x^2y^3:\ 2+3=5$};
  \node[draw=border, rounded corners=10pt, inner sep=8pt, text=text, align=left] at (0,0.2)
  {$\dfrac{xy^2z^3}{y}=xyz^3:\ 1+1+3=5$};
  \node[draw=border, rounded corners=10pt, inner sep=8pt, text=text, align=left] at (0,-0.8)
  {$\sqrt{25}\,z^5=5z^5:\ 5$};
  \node[note] at (0,-1.55) {Maximum degree $=5$};
}
\end{QAPair}

\begin{QAPair}{Question 1 (iii) --- MCQ}
\textcolor{gold}{\bfseries Question:} Constant polynomial is also called\par
(a) linear polynomial \quad (b) no degree polynomial \quad (c) expression \quad (d) zero degree polynomial
\tcblower
\textcolor{green}{\bfseries Answer:} \textbf{(d) zero degree polynomial}\par

\StepFig{1}{A constant polynomial has a horizontal graph (degree $0$).}{%
  \draw[geom,->] (-3,0) -- (3,0) node[lab, right] {$x$};
  \draw[geom,->] (0,-1.4) -- (0,1.8) node[lab, above] {$y$};
  \draw[strong] (-2.6,1.0) -- (2.6,1.0);
  \node[note] at (0,1.35) {$y=c$ (constant)};
}
\end{QAPair}

\begin{QAPair}{Question 1 (iv) --- MCQ}
\textcolor{gold}{\bfseries Question:} Ali is 2 years younger than his sister Ayesha. If Ayesha's present age is $x$ years, then the age of Ali after 5 years will be\par
(a) $(x+7)$ \quad (b) $(x-2)$ \quad (c) $(x+3)$ \quad (d) $(x-7)$
\tcblower
\textcolor{green}{\bfseries Answer:} \textbf{(c) $(x+3)$ years}\par

\StepFig{1}{Timeline idea: Ali is $x-2$ now, so after 5 years $\Rightarrow x-2+5=x+3$.}{%
  \draw[geom,->] (-3.2,0) -- (3.2,0);
  \node[lab] at (-2.6,0.35) {now};
  \node[lab] at (2.6,0.35) {after 5 yrs};
  \draw[strong] (-2.6,0) -- (2.6,0);
  \node[note] at (-2.6,-0.55) {Ali: $x-2$};
  \node[note] at (2.6,-0.55) {Ali: $x+3$};
}
\end{QAPair}

\begin{QAPair}{Question 1 (v) --- MCQ}
\textcolor{gold}{\bfseries Question:} The value of $2\Big\{x^3-(x^2-3-2x)^2\Big\}$ at $x=2$ is\par
(a) 2 \quad (b) 14 \quad (c) $-2$ \quad (d) 6
\tcblower
\textcolor{green}{\bfseries Answer:} \textbf{(c) $-2$}\par
\textcolor{muted}{\small (In the scan, the inner bracket looks like it is squared; that gives the MCQ option.)}\par

\StepFig{1}{Substitute $x=2$ carefully.}{%
  \node[draw=border, rounded corners=10pt, inner sep=8pt, text=text, align=left] at (0,0.9)
  {$2\{2^3-(2^2-3-2\cdot 2)^2\}$};
  \node[draw=border, rounded corners=10pt, inner sep=8pt, text=text, align=left] at (0,-0.05)
  {$=2\{8-(4-3-4)^2\}=2\{8-(-3)^2\}$};
  \node[draw=border, rounded corners=10pt, inner sep=8pt, text=text, align=left] at (0,-1.0)
  {$=2(8-9)=2(-1)=-2$};
}
\end{QAPair}

\begin{QAPair}{Question 1 (vi) --- MCQ}
\textcolor{gold}{\bfseries Question:} Reduced form of the expression
$\dfrac{x^2y^3-y^2x^3+x^2y^2z}{x-y-z}$ is\par
(a) $x^2y^2$ \quad (b) not possible \quad (c) $\dfrac{x^2y^2(x-y-z)}{y-x+z}$ \quad (d) $-x^2y^2$
\tcblower
\textcolor{green}{\bfseries Answer:} \textbf{(d) $-x^2y^2$}\par

\StepFig{1}{Factor and cancel the common factor $(x-y-z)$.}{%
  \node[draw=cyan, rounded corners=10pt, inner sep=8pt, text=text] at (0,1.1)
  {$x^2y^3-y^2x^3+x^2y^2z=x^2y^2(y-x+z)$};
  \node[draw=border, rounded corners=10pt, inner sep=8pt, text=text] at (0,0.05)
  {$y-x+z=-(x-y-z)$};
  \node[draw=cyan, rounded corners=10pt, inner sep=8pt, text=text] at (0,-1.0)
  {$\dfrac{x^2y^2\big(-(x-y-z)\big)}{x-y-z}=-x^2y^2$};
}
\end{QAPair}

\begin{QAPair}{Question 1 (vii) --- MCQ}
\textcolor{gold}{\bfseries Question:} If $y=2-\dfrac{1}{y}$, then the value of $y^2+\dfrac{1}{y^2}$ is\par
(a) 4 \quad (b) zero \quad (c) not possible \quad (d) 2
\tcblower
\textcolor{green}{\bfseries Answer:} \textbf{(d) 2}\par

\StepFig{1}{Solve for $y$ first, then evaluate.}{%
  \node[draw=border, rounded corners=10pt, inner sep=8pt, text=text] at (0,0.9)
  {$y=2-\dfrac{1}{y}\ \Rightarrow\ y^2=2y-1$};
  \node[draw=border, rounded corners=10pt, inner sep=8pt, text=text] at (0,-0.05)
  {$y^2-2y+1=0\ \Rightarrow\ (y-1)^2=0\Rightarrow y=1$};
  \node[draw=cyan, rounded corners=10pt, inner sep=8pt, text=text] at (0,-1.0)
  {$y^2+\dfrac{1}{y^2}=1+1=2$};
}
\end{QAPair}

\begin{QAPair}{Question 1 (viii) --- MCQ}
\textcolor{gold}{\bfseries Question:} Simplified form of
$\dfrac{(a+b)^2-(a-b)^2}{8ab}$ is\par
(a) $\dfrac{2(a^2+b^2)}{8ab}$ \quad (b) 2 \quad (c) $\dfrac{a^2+b^2}{ab}$ \quad (d) $\dfrac12$
\tcblower
\textcolor{green}{\bfseries Answer:} \textbf{(d) $\dfrac12$}\par

\StepFig{1}{Use expansion and subtract.}{%
  \node[draw=border, rounded corners=10pt, inner sep=8pt, text=text, align=left] at (0,0.8)
  {$(a+b)^2=a^2+2ab+b^2$};
  \node[draw=border, rounded corners=10pt, inner sep=8pt, text=text, align=left] at (0,-0.05)
  {$(a-b)^2=a^2-2ab+b^2$};
  \node[draw=cyan, rounded corners=10pt, inner sep=8pt, text=text, align=left] at (0,-0.9)
  {$\dfrac{4ab}{8ab}=\dfrac12$};
}
\end{QAPair}

\begin{QAPair}{Question 1 (ix) --- MCQ}
\textcolor{gold}{\bfseries Question:} Difference of the sum of $a$ and $b$ from the product of $a$ and $b$ is\par
(a) $ab-a-b$ \quad (b) $a+b-ab$ \quad (c) $2ab-b$ \quad (d) none
\tcblower
\textcolor{green}{\bfseries Answer:} \textbf{(a) $ab-a-b$}\par

\StepFig{1}{``Difference of the sum from the product'' means $ab-(a+b)$.}{%
  \node[draw=cyan, rounded corners=10pt, inner sep=10pt, text=text] at (0,0.6) {$ab-(a+b)=ab-a-b$};
  \node[note] at (0,-0.35) {Product minus sum};
}
\end{QAPair}

\begin{QAPair}{Question 1 (x) --- MCQ}
\textcolor{gold}{\bfseries Question:}
$\dfrac{x^3y^3+y^3z^3+z^3x^3}{x^3y^3z^3}=$\par
(a) $x^3+y^3+z^3$ \quad (b) $x^6+y^6+z^6$ \quad
(c) $\dfrac1{x^3}+\dfrac1{y^3}+\dfrac1{z^3}$ \quad
(d) $\dfrac1{x^6}+\dfrac1{y^6}+\dfrac1{z^6}$
\tcblower
\textcolor{green}{\bfseries Answer:} \textbf{(c) $\dfrac1{x^3}+\dfrac1{y^3}+\dfrac1{z^3}$}\par

\StepFig{1}{Split the fraction term-by-term.}{%
  \node[draw=border, rounded corners=10pt, inner sep=8pt, text=text] at (0,0.9)
  {$\dfrac{x^3y^3}{x^3y^3z^3}=\dfrac1{z^3}$};
  \node[draw=border, rounded corners=10pt, inner sep=8pt, text=text] at (0,0.05)
  {$\dfrac{y^3z^3}{x^3y^3z^3}=\dfrac1{x^3}$};
  \node[draw=border, rounded corners=10pt, inner sep=8pt, text=text] at (0,-0.8)
  {$\dfrac{z^3x^3}{x^3y^3z^3}=\dfrac1{y^3}$};
  \node[note] at (0,-1.55) {Sum $\Rightarrow \dfrac1{x^3}+\dfrac1{y^3}+\dfrac1{z^3}$};
}
\end{QAPair}

\begin{QAPair}{Question 1 (xi) --- MCQ}
\textcolor{gold}{\bfseries Question:} Leading coefficient in
$\dfrac{x^2}{2}-\dfrac18-\dfrac{x^4}{4}+\dfrac{x^3}{7}$ is\par
(a) $\dfrac12$ \quad (b) $\dfrac14$ \quad (c) $\dfrac17$ \quad (d) $-\dfrac14$
\tcblower
\textcolor{green}{\bfseries Answer:} \textbf{(d) $-\dfrac14$}\par

\StepFig{1}{Highest power is $x^4$, so leading coefficient is the coefficient of $x^4$.}{%
  \node[draw=cyan, rounded corners=10pt, inner sep=10pt, text=text] at (0,0.65)
  {$-\dfrac{x^4}{4}+\dfrac{x^3}{7}+\dfrac{x^2}{2}-\dfrac18$};
  \node[note] at (0,-0.2) {Highest degree term is $-\dfrac{x^4}{4}$};
  \node[draw=border, rounded corners=10pt, inner sep=10pt, text=text] at (0,-1.0)
  {Leading coefficient $=-\dfrac14$};
}
\end{QAPair}

\begin{QAPair}{Question 1 (xii) --- MCQ}
\textcolor{gold}{\bfseries Question:} Coefficients in the polynomial
$\sqrt{16}\,x^2y-\dfrac12y^3+\dfrac{22}{7}z$ are the elements of the set of:\par
(a) Integers \quad (b) Irrational numbers \quad (c) Odd numbers \quad (d) Rational numbers
\tcblower
\textcolor{green}{\bfseries Answer:} \textbf{(d) Rational numbers}\par

\StepFig{1}{Compute $\sqrt{16}=4$, then list coefficients.}{%
  \node[draw=border, rounded corners=10pt, inner sep=8pt, text=text] at (0,0.9)
  {$\sqrt{16}\,x^2y=4x^2y$};
  \node[draw=cyan, rounded corners=10pt, inner sep=8pt, text=text] at (0,-0.05)
  {Coefficients: $4,\ -\dfrac12,\ \dfrac{22}{7}$};
  \node[note] at (0,-1.05) {All are rational numbers};
}
\end{QAPair}

\begin{QAPair}{Question 1 (xiii) --- MCQ}
\textcolor{gold}{\bfseries Question:} The degree of the quotient in $(x-y)^3\div (x-y)^2$ will be\par
(a) 3 \quad (b) 2 \quad (c) 1 \quad (d) no
\tcblower
\textcolor{green}{\bfseries Answer:} \textbf{(c) 1}\par

\StepFig{1}{Use exponent rule: $(x-y)^3/(x-y)^2=(x-y)^{3-2}=x-y$ (degree 1).}{%
  \node[draw=cyan, rounded corners=10pt, inner sep=10pt, text=text] at (0,0.5)
  {$(x-y)^3\div (x-y)^2=(x-y)$};
  \node[note] at (0,-0.4) {Degree of $x-y$ is $1$};
}
\end{QAPair}

% ============================================================
% Q2
\begin{QAPair}{Question 2 --- Circumference of a circle ($r=12.5$ cm, take $\pi=\frac{22}{7}$)}
\textcolor{gold}{\bfseries Working:}\par
\Step{1} Use $C=2\pi r$.\par
\[
C=2\left(\frac{22}{7}\right)(12.5)=\frac{44\times 12.5}{7}=\frac{550}{7}
=78\frac{4}{7}\text{ cm}.
\]

\StepFig{2}{Diagram (radius shown).}{%
  \coordinate (O) at (0,0);
  \draw[strong] (O) circle (2.0);
  \draw[geom,->] (O) -- (2.0,0) node[lab, below right] {$r=12.5$ cm};
  \fill[pt] (O) circle(1.3pt) node[lab, left] {$O$};
  \node[note] at (0,-2.55) {$C=2\pi r$};
}

\tcblower
\textcolor{green}{\bfseries Answer:} $\boxed{\dfrac{550}{7}\text{ cm }=78\dfrac{4}{7}\text{ cm}}$
\end{QAPair}

% ============================================================
% Q3
\begin{QAPair}{Question 3 --- Surface area of a sphere ($S=4\times \frac{22}{7}r^2$)}
\textcolor{gold}{\bfseries (i) Radius $r=1.4$ inches}\par
\Step{1} $S=\dfrac{88}{7}r^2$ and $r^2=(1.4)^2=1.96=\dfrac{49}{25}$.\par
\[
S=\frac{88}{7}\cdot \frac{49}{25}=\frac{88\cdot 7}{25}=\frac{616}{25}=24.64\ \text{in}^2.
\]

\StepFig{2}{Sphere sketch with radius.}{%
  \coordinate (O) at (0,0);
  \draw[strong] (O) circle (2.0);
  \draw[helper] (-2.0,0) arc (180:360:2.0 and 0.6);
  \draw[geom] (-2.0,0) arc (180:0:2.0 and 0.6);
  \draw[geom,->] (O) -- (1.4,1.4) node[lab, above right] {$r$};
  \fill[pt] (O) circle(1.2pt) node[lab, left] {$O$};
}

\textcolor{gold}{\bfseries (ii) Surface $S=38\frac{1}{2}$ square feet}\par
\Step{1} $S=\dfrac{88}{7}r^2$, so
\[
r^2=\frac{38.5}{88/7}=\frac{77/2}{88/7}=\frac{77}{2}\cdot\frac{7}{88}
=\frac{49}{16}
\Rightarrow r=\frac{7}{4}=1.75\ \text{ft}.
\]

\StepFig{2}{Same idea: solve for $r$ from $S=4\pi r^2$.}{%
  \node[draw=cyan, rounded corners=12pt, inner sep=10pt, text=text] at (0,0.4)
  {$r=\sqrt{\dfrac{S}{4\pi}}$};
  \node[note] at (0,-0.55) {Here $\pi=\dfrac{22}{7}$};
}

\tcblower
\textcolor{green}{\bfseries Answers:}\par
(i) $\boxed{24.64\ \text{in}^2}$\qquad
(ii) $\boxed{r=1.75\ \text{ft}}$
\end{QAPair}

% ============================================================
% Q4
\begin{QAPair}{Question 4 --- Which sets represent sides of a right-angled triangle?}
\textcolor{gold}{\bfseries Condition:} The largest number is $c$ and must satisfy $c^2=a^2+b^2$.\par

\StepFig{1}{Right triangle model (label $a,b$ and hypotenuse $c$).}{%
  \coordinate (B) at (0,0);
  \coordinate (A) at (0,2.6);
  \coordinate (C) at (4.0,0);
  \draw[geom] (A)--(B)--(C)--cycle;
  \RightAngleMark{B}{0.22}
  \node[lab] at (-0.25,1.3) {$a$};
  \node[lab] at (2.0,-0.25) {$b$};
  \node[lab] at (2.3,1.5) {$c$};
}

\textcolor{gold}{\bfseries (i) $7,24,25$}\par
\[
7^2+24^2=49+576=625=25^2 \Rightarrow \text{Yes.}
\]

\StepFig{2}{Scaled sketch for $(7,24,25)$ (illustrative).}{%
  \coordinate (B) at (0,0);
  \coordinate (A) at (0,1.2);
  \coordinate (C) at (4.1,0);
  \draw[strong] (A)--(B)--(C)--cycle;
  \RightAngleMark{B}{0.18}
  \node[note] at (0.2,0.6) {$7$};
  \node[note] at (2.0,-0.35) {$24$};
  \node[note] at (2.2,0.9) {$25$};
}

\textcolor{gold}{\bfseries (ii) $1.6,6.3,6.5$}\par
\[
1.6^2+6.3^2=2.56+39.69=42.25=6.5^2 \Rightarrow \text{Yes.}
\]

\StepFig{3}{Scaled sketch for $(1.6,6.3,6.5)$ (illustrative).}{%
  \coordinate (B) at (0,0);
  \coordinate (A) at (0,0.65);
  \coordinate (C) at (3.0,0);
  \draw[strong] (A)--(B)--(C)--cycle;
  \RightAngleMark{B}{0.15}
  \node[note] at (0.25,0.3) {$1.6$};
  \node[note] at (1.5,-0.35) {$6.3$};
  \node[note] at (1.7,0.45) {$6.5$};
}

\textcolor{gold}{\bfseries (iii) $12,35,36$}\par
\[
12^2+35^2=144+1225=1369=37^2\neq 36^2 \Rightarrow \text{No.}
\]

\StepFig{4}{Mismatch shown: $\sqrt{12^2+35^2}=37\neq 36$.}{%
  \node[draw=border, rounded corners=12pt, inner sep=10pt, text=text] at (0,0.5)
  {$12^2+35^2=37^2\ \Rightarrow\ \text{not }36^2$};
  \node[note] at (0,-0.45) {So it is NOT a right triangle set.};
}

\textcolor{gold}{\bfseries (iv) $3,4,5$}\par
\[
3^2+4^2=9+16=25=5^2 \Rightarrow \text{Yes.}
\]

\StepFig{5}{Classic $3$-$4$-$5$ right triangle.}{%
  \coordinate (B) at (0,0);
  \coordinate (A) at (0,1.2);
  \coordinate (C) at (1.6,0);
  \draw[strong] (A)--(B)--(C)--cycle;
  \RightAngleMark{B}{0.16}
  \node[note] at (0.2,0.6) {$3$};
  \node[note] at (0.8,-0.35) {$4$};
  \node[note] at (0.9,0.35) {$5$};
}

\tcblower
\textcolor{green}{\bfseries Answer:} Sets \textbf{(i), (ii), and (iv)} represent right-angled triangles.
\end{QAPair}

% ============================================================
% Q5
\begin{QAPair}{Question 5 --- If $a=3$, $b=4$ and $c=1$, find the value of $\sqrt{2ab+4ac}+\sqrt{9b}+\dfrac{2abc}{3}$}
\textcolor{gold}{\bfseries Working:}\par
\Step{1} Compute each part.\par
\[
\sqrt{2ab+4ac}=\sqrt{2(3)(4)+4(3)(1)}=\sqrt{24+12}=\sqrt{36}=6
\]
\[
\sqrt{9b}=\sqrt{9\cdot 4}=\sqrt{36}=6,\qquad
\frac{2abc}{3}=\frac{2(3)(4)(1)}{3}=\frac{24}{3}=8
\]
\[
\Rightarrow\ 6+6+8=20
\]

\StepFig{2}{Substitution diagram (plug in values).}{%
  \node[draw=cyan, rounded corners=12pt, inner sep=10pt, text=text, align=center] at (0,0.6)
  {$a=3,\ b=4,\ c=1$};
  \node[draw=border, rounded corners=12pt, inner sep=10pt, text=text, align=center] at (0,-0.55)
  {$\sqrt{2ab+4ac}=6,\ \sqrt{9b}=6,\ \dfrac{2abc}{3}=8$};
}

\tcblower
\textcolor{green}{\bfseries Answer:} $\boxed{20}$
\end{QAPair}

% ============================================================
% Q6
\begin{QAPair}{Question 6 --- Subtract the sum from the given expression}
\textcolor{gold}{\bfseries Question:}\par
Subtract the sum of $\,2x^3-3x+4$ and $-3x^2+2x-7$ from
$\,4x^3-3x^2+x-6-\{2x^3-(x-6)\}$.

\textcolor{gold}{\bfseries Working:}\par
\Step{1} Find the sum:
\[
(2x^3-3x+4)+(-3x^2+2x-7)=2x^3-3x^2-x-3
\]
\Step{2} Simplify the big expression:
\[
\{2x^3-(x-6)\}=2x^3-x+6
\]
\[
4x^3-3x^2+x-6-(2x^3-x+6)=2x^3-3x^2+2x-12
\]
\Step{3} Subtract:
\[
(2x^3-3x^2+2x-12)-(2x^3-3x^2-x-3)=3x-9=3(x-3)
\]

\StepFig{4}{Flow diagram: (Big expression) $-$ (Sum).}{%
  \node[draw=border, rounded corners=12pt, inner sep=10pt, text=text, align=center] (B) at (0,1.1)
  {$B=2x^3-3x^2+2x-12$};
  \node[draw=border, rounded corners=12pt, inner sep=10pt, text=text, align=center] (A) at (0,0)
  {$A=2x^3-3x^2-x-3$};
  \draw[strong,->] (B) -- (A) node[midway, right, note] {$B-A$};
  \node[draw=cyan, rounded corners=12pt, inner sep=10pt, text=text] at (0,-1.1)
  {$3x-9=3(x-3)$};
}

\tcblower
\textcolor{green}{\bfseries Answer:} $\boxed{3x-9=3(x-3)}$
\end{QAPair}

% ============================================================
% Q7
\begin{QAPair}{Question 7 --- Division of algebraic expressions}
\textcolor{gold}{\bfseries Question:}\par
Divide the product of $(x-2)$, $(x+3)$ and $(2x-7)$ by the sum of $3(x^2-2x-2)$ and $5x-x^2-15$.

\textcolor{gold}{\bfseries Working:}\par
\Step{1} Denominator:
\[
3(x^2-2x-2)+(5x-x^2-15)=3x^2-6x-6+5x-x^2-15
\]
\[
=2x^2-x-21=(2x-7)(x+3)
\]
\Step{2} Now divide:
\[
\frac{(x-2)(x+3)(2x-7)}{(2x-7)(x+3)}=x-2
\]

\StepFig{3}{Cancellation picture: common factors cancel.}{%
  \node[draw=border, rounded corners=12pt, inner sep=10pt, text=text] (N) at (0,1.0)
  {Numerator: $(x-2)\,(x+3)\,(2x-7)$};
  \node[draw=border, rounded corners=12pt, inner sep=10pt, text=text] (D) at (0,0.0)
  {Denominator: $(x+3)\,(2x-7)$};
  \draw[strong] (-2.3,0.55) -- (2.3,0.55);
  \node[draw=cyan, rounded corners=12pt, inner sep=10pt, text=text] at (0,-1.05)
  {Result: $x-2$};
}

\tcblower
\textcolor{green}{\bfseries Answer:} $\boxed{x-2}$ \quad (with $x\neq -3,\ \frac72$)
\end{QAPair}

% ============================================================
% Q8
\begin{QAPair}{Question 8 --- Simplify}
\textcolor{gold}{\bfseries (i)} $\displaystyle \frac{x}{x+2}-\frac{5x+3}{x-2}+\frac12$\par

\textcolor{gold}{\bfseries Working:}\par
\Step{1} Take common denominator $2(x+2)(x-2)$:
\[
\frac{2x(x-2)-2(5x+3)(x+2)+(x^2-4)}{2(x+2)(x-2)}
\]
\Step{2} Simplify numerator:
\[
2x(x-2)=2x^2-4x,\quad (5x+3)(x+2)=5x^2+13x+6
\]
\[
\Rightarrow \text{numerator}= (2x^2-4x)-2(5x^2+13x+6)+(x^2-4)
\]
\[
= -7x^2-30x-16=-(7x+16)(x+2)
\]
\Step{3} Cancel $(x+2)$:
\[
\frac{-(7x+16)(x+2)}{2(x+2)(x-2)}=-\frac{7x+16}{2(x-2)}
\]

\StepFig{4}{Common denominator idea.}{%
  \node[draw=cyan, rounded corners=12pt, inner sep=10pt, text=text] at (0,0.65)
  {$\text{LCD}=2(x+2)(x-2)$};
  \node[note] at (0,-0.25) {Convert each term to the same denominator, then combine.};
}

\textcolor{gold}{\bfseries (ii)} $\displaystyle \frac{x}{x^2-y^2}\times\frac{x^2+2xy+y^2}{x+y}\div\frac{3x}{x-y}$\par

\textcolor{gold}{\bfseries Working:}\par
\Step{1} Factor:
\[
x^2-y^2=(x-y)(x+y),\qquad x^2+2xy+y^2=(x+y)^2
\]
\Step{2} Simplify step-by-step:
\[
\frac{x}{(x-y)(x+y)}\times\frac{(x+y)^2}{x+y}=\frac{x}{x-y}
\]
\[
\frac{x}{x-y}\div\frac{3x}{x-y}=\frac{x}{x-y}\times\frac{x-y}{3x}=\frac13
\]

\StepFig{3}{Factor-cancel picture.}{%
  \node[draw=border, rounded corners=12pt, inner sep=10pt, text=text] at (0,0.9)
  {$\frac{x}{(x-y)(x+y)}\times(x+y)\times\frac{x-y}{3x}$};
  \node[draw=cyan, rounded corners=12pt, inner sep=10pt, text=text] at (0,-0.2)
  {Cancel $(x+y)$, $(x-y)$, and $x$};
  \node[draw=border, rounded corners=12pt, inner sep=10pt, text=text] at (0,-1.2)
  {$=\dfrac13$};
}

\tcblower
\textcolor{green}{\bfseries Answers:}\par
(i) $\boxed{-\dfrac{7x+16}{2(x-2)}}$\quad ($x\neq \pm 2$)\par
(ii) $\boxed{\dfrac13}$\quad ($x\neq 0,\ x\neq \pm y$)
\end{QAPair}

% ============================================================
% Q9
\begin{QAPair}{Question 9 --- Solve}
\textcolor{gold}{\bfseries (i)} $\displaystyle \frac{12}{x^2-16}-\frac{24}{x-4}=3$\par

\textcolor{gold}{\bfseries Working:}\par
\Step{1} Factor $x^2-16=(x-4)(x+4)$ (so $x\neq \pm 4$).\par
\Step{2} Multiply both sides by $(x-4)(x+4)$:
\[
12-24(x+4)=3(x^2-16)
\]
\[
12-24x-96=3x^2-48\Rightarrow -24x-84=3x^2-48
\]
\[
3x^2+24x+36=0\Rightarrow x^2+8x+12=0
\]
\[
(x+2)(x+6)=0\Rightarrow x=-2,\,-6
\]

\StepFig{3}{Number line: forbidden points and solutions.}{%
  \draw[geom,->] (-4.8,0) -- (4.8,0);
  \foreach \t/\lab in {-4/$-4$,4/$4$,-2/$-2$,-3.2/$-6$}{
    \draw[geom] (\t,0.12) -- (\t,-0.12);
    \node[lab, below] at (\t,-0.2) {\lab};
  }
  \node[note] at (-4,0.65) {$\times$ not allowed};
  \node[note] at (4,0.65) {$\times$ not allowed};
  \fill[pt] (-2,0) circle(1.5pt);
  \fill[pt] (-3.2,0) circle(1.5pt);
  \node[note] at (-2,0.55) {$\checkmark$};
  \node[note] at (-3.2,0.55) {$\checkmark$};
}

\textcolor{gold}{\bfseries (ii)} $\displaystyle \frac{y}{2y-6}-\frac{3}{x^2-6x+9}=\frac{y-2}{3y-9}$\par
\textcolor{muted}{\small In the scan, the middle denominator is printed with $x$ while the rest uses $y$.\;
Usually it should be one variable. Below are both cases.}\par

\textcolor{gold}{\bfseries Case A (most likely): take it as } $\displaystyle \frac{y}{2y-6}-\frac{3}{y^2-6y+9}=\frac{y-2}{3y-9}$\par
\Step{1} Note $2y-6=2(y-3)$, $y^2-6y+9=(y-3)^2$, $3y-9=3(y-3)$, so $y\neq 3$.\par
Multiply by $6(y-3)^2$:
\[
3y(y-3)-18=2(y-2)(y-3)
\]
\[
3y^2-9y-18=2y^2-10y+12
\]
\[
y^2+y-30=0\Rightarrow (y+6)(y-5)=0
\Rightarrow y=-6,\ 5
\]

\StepFig{4}{Number line for Case A ($y\neq 3$).}{%
  \draw[geom,->] (-5.6,0) -- (5.8,0);
  \draw[geom] (0,0.12) -- (0,-0.12);
  \node[lab, below] at (0,-0.2) {$3$};
  \node[note] at (0,0.65) {$\times$ not allowed};

  \draw[geom] (-3.0,0.12) -- (-3.0,-0.12);
  \node[lab, below] at (-3.0,-0.2) {$-6$};
  \fill[pt] (-3.0,0) circle(1.5pt);

  \draw[geom] (2.2,0.12) -- (2.2,-0.12);
  \node[lab, below] at (2.2,-0.2) {$5$};
  \fill[pt] (2.2,0) circle(1.5pt);
}

\textcolor{gold}{\bfseries Case B (if $x$ is really intended):}\par
\[
\frac{y}{2(y-3)}-\frac{y-2}{3(y-3)}=\frac{3}{(x-3)^2}
\Rightarrow \frac{y+4}{6(y-3)}=\frac{3}{(x-3)^2}
\]
\[
(x-3)^2=\frac{18(y-3)}{y+4}\quad (y\neq 3,\ -4)
\Rightarrow x=3\pm \sqrt{\frac{18(y-3)}{y+4}}.
\]

\tcblower
\textcolor{green}{\bfseries Answers:}\par
(i) $\boxed{x=-2,\,-6}$\par
(ii) Case A: $\boxed{y=5,\,-6}$ (and $y\neq 3$)\par
\textcolor{muted}{\small If the book truly meant two variables (Case B), then $x=3\pm \sqrt{\frac{18(y-3)}{y+4}}$.}
\end{QAPair}

% ============================================================
% Q10
\begin{QAPair}{Question 10 --- Check denominators in a rational equation (Why?)}
\textcolor{gold}{\bfseries Question:} When checking a possible solution of a rational equation, is it necessary to check that the solution does not make any denominator equal $0$? Why or why not?

\tcblower
\textcolor{green}{\bfseries Answer:} \textbf{Yes, it is necessary.}\par
If a value makes any denominator $0$, then the expression is \textbf{undefined} (division by zero is not allowed). Also, when we solve rational equations we often \emph{multiply} both sides by denominators to remove fractions; that step can create \textbf{extraneous solutions}. So we must substitute the answer back and confirm no denominator becomes $0$.

\StepFig{1}{Visual: denominator $0$ means the fraction is not defined.}{%
  \node[draw=border, rounded corners=12pt, inner sep=10pt, text=text] at (0,0.8)
  {$\dfrac{P(x)}{Q(x)}$};
  \node[draw=cyan, rounded corners=12pt, inner sep=10pt, text=text] at (0,-0.1)
  {$Q(x)=0\ \Rightarrow\ \text{NOT defined}$};
  \draw[strong] (-1.4,-0.65) -- (1.4,0.65);
  \draw[strong] (-1.4,0.65) -- (1.4,-0.65);
}
\end{QAPair}

\end{document}
