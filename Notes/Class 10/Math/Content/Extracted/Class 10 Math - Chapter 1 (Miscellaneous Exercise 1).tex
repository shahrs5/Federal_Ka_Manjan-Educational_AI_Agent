% !TEX TS-program = pdflatex
\documentclass[11pt]{article}

% -------------------- Packages --------------------
\usepackage[a4paper,margin=1in]{geometry}
\usepackage{amsmath,amssymb}
\usepackage[T1]{fontenc}
\usepackage{lmodern}
\usepackage{xcolor}
\usepackage{tcolorbox}
\tcbuselibrary{skins,breakable}
\usepackage{enumitem}
\usepackage{hyperref}
\usepackage{tikz}
\usetikzlibrary{calc,patterns,angles,quotes,intersections}

\pagestyle{empty}

% -------------------- Dark Theme Colors --------------------
\definecolor{bg}{HTML}{000000}
\definecolor{pairbg}{HTML}{121212}
\definecolor{solbg}{HTML}{0A0A0A}
\definecolor{border}{HTML}{2A2A2A}
\definecolor{text}{HTML}{FFFFFF}
\definecolor{muted}{HTML}{C9CDD3}
\definecolor{gold}{HTML}{FFD700}
\definecolor{green}{HTML}{4ADE80}
\definecolor{cyan}{HTML}{38BDF8}

\pagecolor{bg}
\color{text}

\hypersetup{
  colorlinks=true,
  linkcolor=cyan,
  urlcolor=cyan
}

\setlength{\parindent}{0pt}
\setlength{\parskip}{10pt}

\setlist[itemize]{left=1.4em,itemsep=6pt,topsep=6pt}
\setlist[enumerate]{left=1.6em,itemsep=4pt,topsep=4pt}

% -------------------- tcolorbox Base --------------------
\tcbset{
  enhanced,
  breakable,
  arc=12pt,
  boxrule=0.8pt,
  left=16pt,right=16pt,top=12pt,bottom=12pt
}

\newtcolorbox{QAPair}[1]{%
  colback=pairbg,
  colbacklower=solbg,
  colframe=border,
  coltext=text,
  title=\textcolor{gold}{\bfseries #1},
  fonttitle=\bfseries,
  coltitle=text,
  segmentation style={draw=border, dashed, line width=0.6pt},
}

\newtcolorbox{QuickBox}{%
  colback=pairbg,
  colframe=cyan,
  coltext=text,
  fontupper=\color{text},
  borderline north={4pt}{0pt}{cyan},
  arc=14pt,
  boxrule=0.8pt
}

% Helper for step headings
\newcommand{\Step}[1]{\textcolor{muted}{\textbf{Step #1:}}}

% -------------------- TikZ Styles --------------------
\tikzset{
  geom/.style={draw=muted, line width=0.95pt},
  strong/.style={draw=cyan, line width=1.05pt},
  helper/.style={draw=muted, dashed, line width=0.75pt},
  arcH/.style={draw=muted, dashed, line width=0.75pt},
  pt/.style={circle, fill=cyan, inner sep=1.2pt},
  lab/.style={text=text, font=\small},
  ang/.style={draw=cyan, line width=0.9pt},
  note/.style={text=muted, font=\small}
}

% -------------------- Step + Diagram Macro --------------------
% Usage:
% \StepFig{1}{<text>}{<tikzpicture contents ONLY>}
\newcommand{\StepFig}[3]{%
  \Step{#1} #2\par\medskip
  \begin{center}
    \begin{tikzpicture}[scale=0.92]
      #3
    \end{tikzpicture}
  \end{center}
  \vspace{-2pt}
}

% Tiny right-angle mark macro (kept as-is; may be useful)
\newcommand{\RightAngleMark}[2]{%
  \draw[ang] ($(#1)+(#2,0)$) -- ($(#1)+(#2,#2)$) -- ($(#1)+(0,#2)$);
}

% -------------------- Argand Plane Helper --------------------
% Call inside tikzpicture:
% \ArgandAxes{<xmax>}{<ymax>}
\newcommand{\ArgandAxes}[2]{%
  \draw[helper] (-0.2,0) -- (#1,0) node[lab, right] {$\Re$};
  \draw[helper] (0,-0.2) -- (0,#2) node[lab, above] {$\Im$};
  \fill[pt] (0,0) circle(1.2pt) node[lab, below left] {$0$};
}

% ============================================================
\begin{document}

\begin{center}
{\LARGE\bfseries \textcolor{gold}{Miscellaneous Exercise 1 --- Solutions}}\\[-2pt]
\end{center}

\begin{QuickBox}
{\color{cyan}\bfseries Quick facts (Complex Numbers)}\par\medskip
\begin{itemize}
\item $i=\sqrt{-1}$, so $i^2=-1,\; i^3=-i,\; i^4=1$ (powers repeat every 4).
\item For $a>0$: $\sqrt{-a}=i\sqrt{a}$ (principal square root).
\item If $z=a+bi$, then $\overline{z}=a-bi$ and $z\overline{z}=|z|^2=a^2+b^2\ge 0$.
\item Modulus: $|a+bi|=\sqrt{a^2+b^2}$.
\item Rationalize: $\displaystyle \frac{1}{a+bi}=\frac{a-bi}{a^2+b^2}$.
\end{itemize}
\end{QuickBox}

% ============================================================
% Q1 (MCQs)
\begin{QAPair}{Question 1 (i) --- MCQ}
\textcolor{gold}{\bfseries Question:} $\sqrt{-1}$ is equal to:\;
(a) $1$\; (b) $-1$\; (c) $i$\; (d) $-i$
\tcblower
\textcolor{green}{\bfseries Answer:} \textbf{(c) $i$}\par
\[
\sqrt{-1}=i.
\]
\StepFig{1}{On the Argand plane, $i$ lies on the positive imaginary axis.}{%
  \ArgandAxes{3}{3}
  \fill[pt] (0,2) circle(1.5pt) node[lab, left] {$i$};
  \draw[strong] (0,0) -- (0,2);
  \node[note] at (1.6,1.6) {$i=0+1i$};
}
\end{QAPair}

\begin{QAPair}{Question 1 (ii) --- MCQ}
\textcolor{gold}{\bfseries Question:} If $x<0$, then $\sqrt{x}$ is:\;
(a) Real \; (b) Complex \; (c) Irrational \; (d) Rational
\tcblower
\textcolor{green}{\bfseries Answer:} \textbf{(b) Complex}\par
If $x=-a$ with $a>0$, then $\sqrt{x}=\sqrt{-a}=i\sqrt{a}$, which is imaginary (hence complex).
\end{QAPair}

\begin{QAPair}{Question 1 (iii) --- MCQ}
\textcolor{gold}{\bfseries Question:} Conjugate of $\sqrt{x}-i\sqrt{y}$ is:\;
(a) $\sqrt{x}+i\sqrt{y}$ \; (b) $x-iy$ \; (c) $x-y$ \; (d) $x+iy$
\tcblower
\textcolor{green}{\bfseries Answer:} \textbf{(a) $\sqrt{x}+i\sqrt{y}$}\par
\[
\overline{\bigl(\sqrt{x}-i\sqrt{y}\bigr)}=\sqrt{x}+i\sqrt{y}.
\]
\StepFig{1}{Conjugate reflects a point across the real axis.}{%
  \ArgandAxes{5}{4}
  \coordinate (Z) at (3,2);
  \coordinate (Zb) at (3,-2);
  \draw[strong] (0,0)--(Z);
  \draw[strong] (0,0)--(Zb);
  \fill[pt] (Z) circle(1.5pt) node[lab, right] {$\sqrt{x}+i\sqrt{y}$};
  \fill[pt] (Zb) circle(1.5pt) node[lab, right] {$\sqrt{x}-i\sqrt{y}$};
  \draw[helper] (Z) -- (Zb);
  \node[note] at (4.2,0.2) {Mirror about $\Re$-axis};
}
\end{QAPair}

\begin{QAPair}{Question 1 (iv) --- MCQ}
\textcolor{gold}{\bfseries Question:} If $z=x+iy$, then $z\overline{z}$ is:\;
(a) Imaginary \; (b) Complex \; (c) Non-negative number \; (d) Negative number
\tcblower
\textcolor{green}{\bfseries Answer:} \textbf{(c) Non-negative number}\par
\[
z\overline{z}=(x+iy)(x-iy)=x^2+y^2\ge 0.
\]
\StepFig{1}{Geometrically, $z\overline{z}=|z|^2$ is a squared distance $\ge 0$.}{%
  \ArgandAxes{5}{4}
  \coordinate (Z) at (3,2);
  \fill[pt] (Z) circle(1.5pt) node[lab, right] {$z$};
  \draw[strong] (0,0)--(Z);
  \draw[helper] (3,0)--(Z);
  \draw[helper] (0,2)--(Z);
  \node[note] at (2.0,0.35) {$x$};
  \node[note] at (0.35,1.4) {$y$};
  \node[note] at (2.2,2.2) {$|z|=\sqrt{x^2+y^2}$};
}
\end{QAPair}

\begin{QAPair}{Question 1 (v) --- MCQ}
\textcolor{gold}{\bfseries Question:} $\sqrt{-25}+\sqrt[3]{8}$ is equal to:\;
(a) $-5+\sqrt{8}$ \; (b) $2+5i$ \; (c) $-5+2i$ \; (d) $2\sqrt{2}+5i$
\tcblower
\textcolor{green}{\bfseries Answer:} \textbf{(b) $2+5i$}\par
\[
\sqrt{-25}=5i,\qquad \sqrt[3]{8}=2
\;\Rightarrow\;
\sqrt{-25}+\sqrt[3]{8}=2+5i.
\]
\end{QAPair}

\begin{QAPair}{Question 1 (vi) --- MCQ}
\textcolor{gold}{\bfseries Question:} $1+(-i)^9=$\;
(a) $1+i$ \; (b) $1+\sqrt{-1}$ \; (c) $1-i$ \; (d) $-i$
\tcblower
\textcolor{green}{\bfseries Answer:} \textbf{(c) $1-i$}\par
\[
(-i)^9 = (-i)^8(-i) = 1\cdot(-i)=-i
\quad\Rightarrow\quad
1+(-i)^9=1-i.
\]
\end{QAPair}

\begin{QAPair}{Question 1 (vii) --- MCQ}
\textcolor{gold}{\bfseries Question:} $\dfrac{2}{1-i}=$\;
(a) $\dfrac{1+i}{2}$ \; (b) $\dfrac{(1+i)^2}{2}$ \; (c) $1-i$ \; (d) $1+i$
\tcblower
\textcolor{green}{\bfseries Answer:} \textbf{(d) $1+i$}\par
\[
\frac{2}{1-i}=\frac{2(1+i)}{(1-i)(1+i)}=\frac{2(1+i)}{1+1}=1+i.
\]
\StepFig{1}{Rationalize by multiplying with the conjugate $(1+i)$.}{%
  \node[lab] at (0,0) {$\displaystyle \frac{2}{1-i}\times\frac{1+i}{1+i}$};
  \node[lab] at (0,-0.8) {$\displaystyle=\frac{2(1+i)}{(1-i)(1+i)}=\frac{2(1+i)}{2}=1+i$};
}
\end{QAPair}

\begin{QAPair}{Question 1 (viii) --- MCQ}
\textcolor{gold}{\bfseries Question:} $(-xi)^{19}=$\;
(a) $-x^{19}i$ \; (b) $x^{19}i$ \; (c) $-i^{19}$ \; (d) $-x^{19}$
\tcblower
\textcolor{green}{\bfseries Answer:} \textbf{(b) $x^{19}i$}\par
\[
(-xi)^{19}=(-1)^{19}x^{19}i^{19}=-x^{19}i^{19}.
\]
Since $i^{19}=i^{16}i^3=1\cdot(-i)=-i$, we get
\[
-x^{19}(-i)=x^{19}i.
\]
\end{QAPair}

\begin{QAPair}{Question 1 (ix) --- MCQ}
\textcolor{gold}{\bfseries Question:} If $z=3+4i$, then $|z|^2$ is:\;
(a) $5$ \; (b) $\sqrt{5}$ \; (c) $25$ \; (d) $16$
\tcblower
\textcolor{green}{\bfseries Answer:} \textbf{(c) $25$}\par
\[
|z|^2=3^2+4^2=9+16=25.
\]
\StepFig{1}{Point $3+4i$ makes a $3$--$4$--$5$ right triangle, so $|z|=5$.}{%
  \ArgandAxes{6}{6}
  \coordinate (Z) at (3,4);
  \fill[pt] (Z) circle(1.5pt) node[lab, right] {$3+4i$};
  \draw[strong] (0,0)--(Z);
  \draw[helper] (0,0)--(3,0)--(3,4);
  \node[note] at (1.5,-0.35) {$3$};
  \node[note] at (3.35,2.0) {$4$};
  \node[note] at (2.0,3.8) {$|z|=5$};
}
\end{QAPair}

\begin{QAPair}{Question 1 (x) --- MCQ}
\textcolor{gold}{\bfseries Question:} The solution of $x^2+4=0$ is:\;
(a) $2i$ \; (b) $-2i$ \; (c) $\pm2$ \; (d) $\pm2i$
\tcblower
\textcolor{green}{\bfseries Answer:} \textbf{(d) $\pm2i$}\par
\[
x^2=-4 \;\Rightarrow\; x=\pm\sqrt{-4}=\pm(2i)=\pm2i.
\]
\end{QAPair}

% ============================================================
% Q2
\begin{QAPair}{Question 2 --- Simplify}
\textcolor{gold}{\bfseries (a)} $(-2+4i)-(8-5i)$\par
\tcblower
\textcolor{green}{\bfseries Answer:}
\[
\begin{aligned}
\Step{1}\;&(-2+4i)-(8-5i)=(-2+4i)-8+5i\\
\Step{2}\;&=(-2-8)+(4i+5i)\\
\Step{3}\;&=-10+9i.
\end{aligned}
\]
\end{QAPair}

\begin{QAPair}{Question 2 --- Simplify (continued)}
\textcolor{gold}{\bfseries (b)} $(-3+4i)+(-7i+4)$\par
\tcblower
\textcolor{green}{\bfseries Answer:}
\[
\begin{aligned}
\Step{1}\;&(-3+4i)+(-7i+4)=(-3+4)+(4i-7i)\\
\Step{2}\;&=1-3i.
\end{aligned}
\]
\end{QAPair}

% ============================================================
% Q3
\begin{QAPair}{Question 3 --- Find the product}
\textcolor{gold}{\bfseries (a)} $(x+iy)(2+3i)$\par
\tcblower
\textcolor{green}{\bfseries Answer:}
\[
\begin{aligned}
\Step{1}\;&(x+iy)(2+3i)=x\cdot2+x\cdot3i+iy\cdot2+iy\cdot3i\\
\Step{2}\;&=2x+3xi+2iy+3i^2y\\
\Step{3}\;&=(2x-3y)+i(3x+2y).
\end{aligned}
\]
\end{QAPair}

\begin{QAPair}{Question 3 --- Find the product (continued)}
\textcolor{gold}{\bfseries (b)} $(-3+6i)(-6+3i)$\par
\tcblower
\textcolor{green}{\bfseries Answer:}
\[
\begin{aligned}
\Step{1}\;&(-3+6i)(-6+3i)=(-3)(-6)+(-3)(3i)+(6i)(-6)+(6i)(3i)\\
\Step{2}\;&=18-9i-36i+18i^2\\
\Step{3}\;&=18-45i-18\\
\Step{4}\;&=-45i.
\end{aligned}
\]
\end{QAPair}

% ============================================================
% Q4
\begin{QAPair}{Question 4 --- Write in the form $a+bi$, then find its conjugate}
\textcolor{gold}{\bfseries (a)} $3\sqrt{2}-\sqrt{-7}$\par
\tcblower
\textcolor{green}{\bfseries Answer:}
\[
\begin{aligned}
\Step{1}\;&\sqrt{-7}=i\sqrt{7}.\\
\Step{2}\;&3\sqrt{2}-\sqrt{-7}=3\sqrt{2}-i\sqrt{7}.
\end{aligned}
\]
\[
\text{Conjugate}:\quad \overline{(3\sqrt{2}-i\sqrt{7})}=3\sqrt{2}+i\sqrt{7}.
\]
\StepFig{3}{On Argand plane, conjugates have same real part and opposite imaginary part.}{%
  \ArgandAxes{6}{5}
  \coordinate (Z) at (4,2);
  \coordinate (Zb) at (4,-2);
  \fill[pt] (Z) circle(1.5pt) node[lab, right] {$3\sqrt2+i\sqrt7$};
  \fill[pt] (Zb) circle(1.5pt) node[lab, right] {$3\sqrt2-i\sqrt7$};
  \draw[helper] (Z)--(Zb);
}
\end{QAPair}

\begin{QAPair}{Question 4 --- (continued)}
\textcolor{gold}{\bfseries (b)} $\sqrt{-2}$\par
\tcblower
\textcolor{green}{\bfseries Answer:}
\[
\sqrt{-2}=i\sqrt{2}=0+i\sqrt2.
\]
\[
\text{Conjugate}:\quad \overline{i\sqrt2}=-i\sqrt2.
\]
\end{QAPair}

% ============================================================
% Q5
\begin{QAPair}{Question 5 --- Find $z\overline{z}$}
\textcolor{gold}{\bfseries (a)} $z=-\dfrac12+i$\par
\tcblower
\textcolor{green}{\bfseries Answer:}
\[
\overline{z}=-\frac12-i,\qquad
z\overline{z}=\left(-\frac12\right)^2+1^2=\frac14+1=\frac54.
\]
\StepFig{2}{Here $z\overline{z}=|z|^2$ (a non-negative number).}{%
  \ArgandAxes{4}{3}
  \coordinate (Z) at (-1.2,2.0);
  \draw[helper] (-2.8,0) -- (3.8,0);
  \draw[helper] (0,-2.2) -- (0,3.2);
  \fill[pt] (Z) circle(1.5pt) node[lab, left] {$z$};
  \draw[strong] (0,0)--(Z);
  \node[note] at (-1.6,1.25) {$|z|^2=\frac54$};
}
\end{QAPair}

\begin{QAPair}{Question 5 --- (continued)}
\textcolor{gold}{\bfseries (b)} $z=14-7i$\par
\tcblower
\textcolor{green}{\bfseries Answer:}
\[
z\overline{z}=14^2+(-7)^2=196+49=245.
\]
\end{QAPair}

% ============================================================
% Q6
\begin{QAPair}{Question 6 --- Simplify}
\textcolor{gold}{\bfseries (a)} $\dfrac{-3-i}{-3+i}$\par
\tcblower
\textcolor{green}{\bfseries Answer:}
\[
\begin{aligned}
\Step{1}\;&\frac{-3-i}{-3+i}\cdot\frac{-3-i}{-3-i}
=\frac{(-3-i)^2}{(-3+i)(-3-i)}\\
\Step{2}\;&=\frac{9+6i+i^2}{9- i^2}
=\frac{9+6i-1}{9+1}
=\frac{8+6i}{10}\\
\Step{3}\;&=\frac45+\frac35\,i.
\end{aligned}
\]
\end{QAPair}

\begin{QAPair}{Question 6 --- (continued)}
\textcolor{gold}{\bfseries (b)} $\dfrac{1+3i}{i\sqrt5}$\par
\tcblower
\textcolor{green}{\bfseries Answer:}
\[
\begin{aligned}
\Step{1}\;&\frac{1+3i}{i\sqrt5}\cdot\frac{-i}{-i}
=\frac{(1+3i)(-i)}{(i\sqrt5)(-i)}\\
\Step{2}\;&=\frac{-i-3i^2}{\sqrt5}
=\frac{-i+3}{\sqrt5}\\
\Step{3}\;&=\frac{3-i}{\sqrt5}.
\end{aligned}
\]
(Also acceptable: $\displaystyle \frac{3\sqrt5-i\sqrt5}{5}$.)
\end{QAPair}

% ============================================================
% Q7
\begin{QAPair}{Question 7 --- Factorize}
\textcolor{gold}{\bfseries (a)} $2x^2+18$\par
\tcblower
\textcolor{green}{\bfseries Answer:}
\[
2x^2+18=2(x^2+9).
\]
Over complex numbers:
\[
2(x^2+9)=2(x-3i)(x+3i).
\]
\end{QAPair}

\begin{QAPair}{Question 7 --- Factorize (continued)}
\textcolor{gold}{\bfseries (b)} $-x^2-25y^4$\par
\tcblower
\textcolor{green}{\bfseries Answer:}
\[
-x^2-25y^4=-(x^2+25y^4)=-(x^2+(5y^2)^2).
\]
Over complex numbers:
\[
-(x^2+(5y^2)^2)=-(x-5iy^2)(x+5iy^2).
\]
\end{QAPair}

% ============================================================
% Q8
\begin{QAPair}{Question 8 --- Solve}
\textcolor{gold}{\bfseries (a)} $3x^2+15=0$\par
\tcblower
\textcolor{green}{\bfseries Answer:}
\[
\begin{aligned}
\Step{1}\;&3x^2+15=0 \Rightarrow 3x^2=-15 \Rightarrow x^2=-5\\
\Step{2}\;&x=\pm\sqrt{-5}=\pm i\sqrt5.
\end{aligned}
\]
\end{QAPair}

\begin{QAPair}{Question 8 --- Solve (continued)}
\textcolor{gold}{\bfseries (b)} $6y^2+36=0$\par
\tcblower
\textcolor{green}{\bfseries Answer:}
\[
\begin{aligned}
\Step{1}\;&6y^2+36=0 \Rightarrow 6y^2=-36 \Rightarrow y^2=-6\\
\Step{2}\;&y=\pm\sqrt{-6}=\pm i\sqrt6.
\end{aligned}
\]
\end{QAPair}

\end{document}
