% !TEX TS-program = pdflatex
\documentclass[11pt]{article}

% -------------------- Packages --------------------
\usepackage[a4paper,margin=1in]{geometry}
\usepackage{amsmath,amssymb}
\usepackage[T1]{fontenc}
\usepackage{lmodern}
\usepackage{xcolor}
\usepackage{tcolorbox}
\tcbuselibrary{skins,breakable}
\usepackage{enumitem}
\usepackage{hyperref}
\usepackage{tikz}
\usepackage{pgfplots}
\pgfplotsset{compat=1.18}
\usetikzlibrary{arrows.meta}

\pagestyle{empty}

% -------------------- Dark Theme Colors --------------------
\definecolor{bg}{HTML}{000000}
\definecolor{pairbg}{HTML}{121212}
\definecolor{solbg}{HTML}{0A0A0A}
\definecolor{border}{HTML}{2A2A2A}
\definecolor{text}{HTML}{FFFFFF}
\definecolor{muted}{HTML}{C9CDD3}
\definecolor{gold}{HTML}{FFD700}
\definecolor{green}{HTML}{4ADE80}
\definecolor{cyan}{HTML}{38BDF8}

\pagecolor{bg}
\color{text}

\hypersetup{
  colorlinks=true,
  linkcolor=cyan,
  urlcolor=cyan
}

\setlength{\parindent}{0pt}
\setlength{\parskip}{10pt}

% Helps avoid overfull lines
\setlength{\emergencystretch}{2em}

\setlist[itemize]{left=1.4em,itemsep=6pt,topsep=6pt}
\setlist[enumerate]{left=1.6em,itemsep=4pt,topsep=4pt}

% -------------------- tcolorbox Base --------------------
\tcbset{
  enhanced,
  breakable,
  arc=12pt,
  boxrule=0.8pt,
  left=16pt,right=16pt,top=12pt,bottom=12pt
}

\newtcolorbox{QAPair}[1]{%
  colback=pairbg,
  colbacklower=solbg,
  colframe=border,
  coltext=text,
  title=\textcolor{gold}{\bfseries #1},
  fonttitle=\bfseries,
  coltitle=text,
  segmentation style={draw=border, dashed, line width=0.6pt},
  before upper=\raggedright,
  before lower=\raggedright,
}

\newtcolorbox{QuickBox}{%
  colback=pairbg,
  colframe=cyan,
  coltext=text,
  fontupper=\color{text},
  borderline north={4pt}{0pt}{cyan},
  arc=14pt,
  boxrule=0.8pt
}

% Helper for step headings
\newcommand{\Step}[1]{\textcolor{muted}{\textbf{Step #1:}}}

% -------------------- PGFPlots style for dark theme (FAST) --------------------
\pgfplotsset{
  darkaxis/.style={
    axis lines=middle,
    axis line style={-Stealth, color=text},
    tick style={color=text},
    ticklabel style={color=text},
    label style={color=text},

    grid=major,
    minor tick num=0,
    grid style={color=border},

    samples=80,
    width=0.86\linewidth,
    height=4.8cm
  }
}

% ============================================================
\begin{document}

\begin{center}
{\LARGE\bfseries \textcolor{gold}{Exercise 6.3 --- Solutions}}\\[-2pt]
\end{center}

\begin{QuickBox}
{\color{cyan}\bfseries Quick notes (graphs \& functions)}\par\medskip
\begin{itemize}
\item \textbf{Vertical line test:} a graph is a \textbf{function} if every vertical line cuts it at \textbf{most once}.
\item \textbf{Linear function:} $y=mx+c$ (a straight line). If the graph is not a single straight line, it is \textbf{non-linear}.
\item \textbf{Intercepts:} $x$-intercept(s) $\Rightarrow y=0$; \; $y$-intercept $\Rightarrow x=0$.
\item \textbf{Parabola $ax^2+bx+c$:} opens \textbf{up} if $a>0$, \textbf{down} if $a<0$; vertex at $x=-\dfrac{b}{2a}$.
\end{itemize}
\end{QuickBox}

% ============================================================
% Q1
\begin{QAPair}{Question 1}
\textcolor{gold}{\bfseries Question:} Identify which graphs are \textbf{linear / non-linear}. Also decide which are \textbf{functions} using the vertical line test.\\
\tcblower
\textcolor{green}{\bfseries Answer:}

\[
\begin{array}{c|c|c}
\textbf{Graph} & \textbf{Linear / Non-linear} & \textbf{Function? (VLT)}\\ \hline
(i) & \text{Non-linear (parabola)} & \text{Yes}\\
(ii) & \text{Non-linear (ellipse)} & \text{No}\\
(iii) & \text{Non-linear (cubic-like)} & \text{Yes}\\
(iv) & \text{Linear (straight line)} & \text{Yes}\\
(v) & \text{Non-linear (sideways parabola)} & \text{No}\\
(vi) & \text{Non-linear ($|x|$-shape)} & \text{Yes}
\end{array}
\]

\textcolor{muted}{\textbf{Sketches (for reference):}}
\begin{center}
\begin{minipage}{0.48\linewidth}\centering
\textcolor{muted}{(i)}
\begin{tikzpicture}
\begin{axis}[darkaxis,xmin=-4,xmax=4,ymin=-4,ymax=4]
\addplot[color=cyan,very thick,domain=-3.5:3.5]{(x-1)^2-2};
\end{axis}
\end{tikzpicture}
\end{minipage}
\hfill
\begin{minipage}{0.48\linewidth}\centering
\textcolor{muted}{(ii)}
\begin{tikzpicture}
\begin{axis}[darkaxis,xmin=-4,xmax=4,ymin=-4,ymax=4]
\addplot[color=cyan,very thick,domain=0:360,samples=120]({3*cos(x)},{2*sin(x)});
\end{axis}
\end{tikzpicture}
\end{minipage}

\begin{minipage}{0.48\linewidth}\centering
\textcolor{muted}{(iii)}
\begin{tikzpicture}
\begin{axis}[darkaxis,xmin=-4,xmax=4,ymin=-4,ymax=4]
\addplot[color=cyan,very thick,domain=-3.2:3.2]{0.3*x^3};
\end{axis}
\end{tikzpicture}
\end{minipage}
\hfill
\begin{minipage}{0.48\linewidth}\centering
\textcolor{muted}{(iv)}
\begin{tikzpicture}
\begin{axis}[darkaxis,xmin=-4,xmax=4,ymin=-4,ymax=4]
\addplot[color=cyan,very thick,domain=-4:4]{x+1};
\end{axis}
\end{tikzpicture}
\end{minipage}

\begin{minipage}{0.48\linewidth}\centering
\textcolor{muted}{(v)}
\begin{tikzpicture}
\begin{axis}[darkaxis,xmin=-1,xmax=6,ymin=-4,ymax=4]
\addplot[color=cyan,very thick,domain=-2.2:2.2,samples=120]({x^2},{x});
\end{axis}
\end{tikzpicture}
\end{minipage}
\hfill
\begin{minipage}{0.48\linewidth}\centering
\textcolor{muted}{(vi)}
\begin{tikzpicture}
\begin{axis}[darkaxis,xmin=-4,xmax=4,ymin=-1,ymax=4]
\addplot[color=cyan,very thick,domain=-4:4]{abs(x)};
\end{axis}
\end{tikzpicture}
\end{minipage}
\end{center}

\end{QAPair}

% ============================================================
% Q2
\begin{QAPair}{Question 2}
\textcolor{gold}{\bfseries Question:} Sketch $y=3x+2$, $y=x$, $y=x^2$ and identify which are linear.\\
\tcblower
\textcolor{green}{\bfseries Answer:}

\Step{1} $y=3x+2$ is linear (straight line).\par
\Step{2} $y=x$ is linear (straight line).\par
\Step{3} $y=x^2$ is non-linear (parabola).\par

\begin{center}
\begin{minipage}{0.32\linewidth}\centering
\textcolor{muted}{(i) $y=3x+2$}
\begin{tikzpicture}
\begin{axis}[darkaxis,xmin=-3,xmax=3,ymin=-6,ymax=8]
\addplot[color=cyan,very thick,domain=-3:3]{3*x+2};
\end{axis}
\end{tikzpicture}
\end{minipage}
\hfill
\begin{minipage}{0.32\linewidth}\centering
\textcolor{muted}{(ii) $y=x$}
\begin{tikzpicture}
\begin{axis}[darkaxis,xmin=-3,xmax=3,ymin=-3,ymax=3]
\addplot[color=cyan,very thick,domain=-3:3]{x};
\end{axis}
\end{tikzpicture}
\end{minipage}
\hfill
\begin{minipage}{0.32\linewidth}\centering
\textcolor{muted}{(iii) $y=x^2$}
\begin{tikzpicture}
\begin{axis}[darkaxis,xmin=-3,xmax=3,ymin=-1,ymax=9]
\addplot[color=cyan,very thick,domain=-3:3]{x^2};
\end{axis}
\end{tikzpicture}
\end{minipage}
\end{center}
\end{QAPair}

% ============================================================
% Q3
\begin{QAPair}{Question 3}
\textcolor{gold}{\bfseries Question:} Sketch the graphs of: (i) $y=3^x$, (ii) $y=|x|$, (iii) $y=x^3$, (iv) $y=\sqrt{x+2}$, (v) $y=\dfrac{1}{x-1}$.\\
\tcblower
\textcolor{green}{\bfseries Answer:} \textcolor{muted}{(Sketches below.)}

\begin{center}
\begin{minipage}{0.48\linewidth}\centering
\textcolor{muted}{(i) $y=3^x$}
\begin{tikzpicture}
\begin{axis}[darkaxis,xmin=-3,xmax=3,ymin=0,ymax=30]
\addplot[color=cyan,very thick,domain=-3:3]{3^x};
\end{axis}
\end{tikzpicture}
\end{minipage}
\hfill
\begin{minipage}{0.48\linewidth}\centering
\textcolor{muted}{(ii) $y=|x|$}
\begin{tikzpicture}
\begin{axis}[darkaxis,xmin=-4,xmax=4,ymin=-1,ymax=4]
\addplot[color=cyan,very thick,domain=-4:4]{abs(x)};
\end{axis}
\end{tikzpicture}
\end{minipage}

\begin{minipage}{0.48\linewidth}\centering
\textcolor{muted}{(iii) $y=x^3$}
\begin{tikzpicture}
\begin{axis}[darkaxis,xmin=-3,xmax=3,ymin=-10,ymax=10]
\addplot[color=cyan,very thick,domain=-3:3]{x^3/3};
\end{axis}
\end{tikzpicture}
\end{minipage}
\hfill
\begin{minipage}{0.48\linewidth}\centering
\textcolor{muted}{(iv) $y=\sqrt{x+2}$}
\begin{tikzpicture}
\begin{axis}[darkaxis,xmin=-3,xmax=7,ymin=-1,ymax=4]
\addplot[color=cyan,very thick,domain=-2:7]{sqrt(x+2)};
\end{axis}
\end{tikzpicture}
\end{minipage}

\begin{minipage}{0.98\linewidth}\centering
\textcolor{muted}{(v) $y=\dfrac{1}{x-1}$}
\begin{tikzpicture}
\begin{axis}[darkaxis,xmin=-4,xmax=6,ymin=-6,ymax=6]
\addplot[color=cyan,very thick,domain=-4:0.85]{1/(x-1)};
\addplot[color=cyan,very thick,domain=1.15:6]{1/(x-1)};
\addplot[color=muted,dashed,very thick,domain=-4:6]{0};
\addplot[color=muted,dashed,very thick] coordinates {(1,-6) (1,6)};
\end{axis}
\end{tikzpicture}
\end{minipage}
\end{center}

\end{QAPair}

% ============================================================
% (Rest unchanged)
% ============================================================

% ============================================================
% Q4 (i)
\begin{QAPair}{Question 4 (i)}
\textcolor{gold}{\bfseries Question:} $f(x)=(x-1)^2$\\
\tcblower
\textcolor{green}{\bfseries Answer:}
\[
\begin{aligned}
\Step{1}\;& f(x)=(x-1)^2 \text{ has } a>0 \Rightarrow \text{opens upwards.}\\
\Step{2}\;& x\text{-intercept: } (x-1)^2=0 \Rightarrow x=1 \Rightarrow (1,0).\\
\Step{3}\;& y\text{-intercept: } f(0)=(0-1)^2=1 \Rightarrow (0,1).
\end{aligned}
\]

\begin{center}
\begin{tikzpicture}
\begin{axis}[darkaxis,xmin=-3,xmax=5,ymin=-1,ymax=9]
\addplot[color=cyan,very thick,domain=-3:5]{(x-1)^2};
\addplot[color=green,only marks,mark=*] coordinates {(1,0) (0,1)};
\end{axis}
\end{tikzpicture}
\end{center}
\end{QAPair}

% ============================================================
% Q4 (ii)
\begin{QAPair}{Question 4 (ii)}
\textcolor{gold}{\bfseries Question:} $f(x)=-x^2+2$\\
\tcblower
\textcolor{green}{\bfseries Answer:}
\[
\begin{aligned}
\Step{1}\;& f(x)=-x^2+2 \text{ has } a<0 \Rightarrow \text{opens downwards.}\\
\Step{2}\;& x\text{-intercepts: } -x^2+2=0 \Rightarrow x^2=2 \Rightarrow x=\pm\sqrt{2}.\\
&\Rightarrow (\sqrt2,0),\,(-\sqrt2,0).\\
\Step{3}\;& y\text{-intercept: } f(0)=2 \Rightarrow (0,2).
\end{aligned}
\]

\begin{center}
\begin{tikzpicture}
\begin{axis}[darkaxis,xmin=-3,xmax=3,ymin=-3,ymax=3]
\addplot[color=cyan,very thick,domain=-3:3]{-x^2+2};
\addplot[color=green,only marks,mark=*] coordinates {(0,2)};
\end{axis}
\end{tikzpicture}
\end{center}
\end{QAPair}

% ============================================================
% Q4 (iii)
\begin{QAPair}{Question 4 (iii)}
\textcolor{gold}{\bfseries Question:} $f(x)=3-(x+2)^2$\\
\tcblower
\textcolor{green}{\bfseries Answer:}
\[
\begin{aligned}
\Step{1}\;& f(x)=3-(x+2)^2=-(x+2)^2+3 \Rightarrow a<0 \Rightarrow \text{opens downwards.}\\
\Step{2}\;& x\text{-intercepts: } 3-(x+2)^2=0 \Rightarrow (x+2)^2=3\\
&\Rightarrow x=-2\pm\sqrt3 \Rightarrow (-2+\sqrt3,0),\,(-2-\sqrt3,0).\\
\Step{3}\;& y\text{-intercept: } f(0)=3-(2)^2=-1 \Rightarrow (0,-1).
\end{aligned}
\]

\begin{center}
\begin{tikzpicture}
\begin{axis}[darkaxis,xmin=-6,xmax=2,ymin=-3,ymax=4]
\addplot[color=cyan,very thick,domain=-6:2]{3-(x+2)^2};
\addplot[color=green,only marks,mark=*] coordinates {(0,-1)};
\end{axis}
\end{tikzpicture}
\end{center}
\end{QAPair}

% ============================================================
% Q4 (iv)
\begin{QAPair}{Question 4 (iv)}
\textcolor{gold}{\bfseries Question:} $f(x)=-x^2+2x+3$\\
\tcblower
\textcolor{green}{\bfseries Answer:}
\[
\begin{aligned}
\Step{1}\;& a=-1<0 \Rightarrow \text{parabola opens downwards.}\\
\Step{2}\;& x\text{-intercepts: } -x^2+2x+3=0\\
&\Rightarrow x^2-2x-3=0 \Rightarrow (x-3)(x+1)=0\\
&\Rightarrow x=3,-1 \Rightarrow (3,0),\,(-1,0).\\
\Step{3}\;& y\text{-intercept: } f(0)=3 \Rightarrow (0,3).\\
\Step{4}\;& \text{Vertex: } x=-\frac{b}{2a}=-\frac{2}{2(-1)}=1,\; f(1)=4 \Rightarrow (1,4).
\end{aligned}
\]

\begin{center}
\begin{tikzpicture}
\begin{axis}[darkaxis,xmin=-4,xmax=5,ymin=-3,ymax=6]
\addplot[color=cyan,very thick,domain=-4:5]{-x^2+2*x+3};
\addplot[color=green,only marks,mark=*] coordinates {(3,0) (-1,0) (0,3) (1,4)};
\end{axis}
\end{tikzpicture}
\end{center}
\end{QAPair}

% ============================================================
% Q5
\begin{QAPair}{Question 5}
\textcolor{gold}{\bfseries Question:} Find $h$ and $k$ if $y=hx^2+2x+k$ cuts the $x$-axis at $(-3,0)$ and the $y$-axis at $(0,2)$.\\
\tcblower
\textcolor{green}{\bfseries Answer:}
\[
\begin{aligned}
\Step{1}\;& \text{At } (0,2):\quad 2=h(0)^2+2(0)+k \Rightarrow k=2.\\
\Step{2}\;& \text{At } (-3,0):\quad 0=h(9)+2(-3)+2\\
&\Rightarrow 0=9h-6+2=9h-4 \Rightarrow h=\frac{4}{9}.
\end{aligned}
\]
\[
\boxed{h=\frac{4}{9},\quad k=2.}
\]
\end{QAPair}

% ============================================================
% Q6 (i)
\begin{QAPair}{Question 6 (i)}
\textcolor{gold}{\bfseries Question:} Solve graphically: $f(x)=x+2$ and $g(x)=x^2-3x+2$.\\
\tcblower
\textcolor{green}{\bfseries Answer:}
\[
\begin{aligned}
\Step{1}\;& x+2=x^2-3x+2 \Rightarrow x^2-4x=0\\
\Step{2}\;& x(x-4)=0 \Rightarrow x=0 \text{ or } x=4.\\
\Step{3}\;& y=f(0)=2,\quad y=f(4)=6.
\end{aligned}
\]
\[
\boxed{\text{Intersection points: } (0,2)\ \text{and}\ (4,6).}
\]

\begin{center}
\begin{tikzpicture}
\begin{axis}[darkaxis,xmin=-1,xmax=6,ymin=-3,ymax=9]
\addplot[color=cyan,very thick,domain=-1:6]{x+2};
\addplot[color=green,very thick,domain=-1:6]{x^2-3*x+2};
\addplot[color=gold,only marks,mark=*] coordinates {(0,2) (4,6)};
\end{axis}
\end{tikzpicture}
\end{center}
\end{QAPair}

% ============================================================
% Q6 (ii)
\begin{QAPair}{Question 6 (ii)}
\textcolor{gold}{\bfseries Question:} Solve graphically: $f(x)=2x+5$ and $g(x)=2x^2+1$.\\
\tcblower
\textcolor{green}{\bfseries Answer:}
\[
\begin{aligned}
\Step{1}\;& 2x+5=2x^2+1 \Rightarrow 2x^2-2x-4=0\\
\Step{2}\;& x^2-x-2=0 \Rightarrow (x-2)(x+1)=0\\
\Step{3}\;& x=2 \text{ or } x=-1.\\
\Step{4}\;& y=f(2)=9,\quad y=f(-1)=3.
\end{aligned}
\]
\[
\boxed{\text{Intersection points: } (-1,3)\ \text{and}\ (2,9).}
\]

\begin{center}
\begin{tikzpicture}
\begin{axis}[darkaxis,xmin=-4,xmax=4,ymin=-2,ymax=12]
\addplot[color=cyan,very thick,domain=-4:4]{2*x+5};
\addplot[color=green,very thick,domain=-4:4]{2*x^2+1};
\addplot[color=gold,only marks,mark=*] coordinates {(-1,3) (2,9)};
\end{axis}
\end{tikzpicture}
\end{center}
\end{QAPair}

% ============================================================
% Q7
\begin{QAPair}{Question 7}
\textcolor{gold}{\bfseries Question:} $D(x)=100-5x$ and $S(x)=x-200$. Find $x$ for which supply equals demand.\\
\tcblower
\textcolor{green}{\bfseries Answer:}
\[
\begin{aligned}
\Step{1}\;& 100-5x=x-200\\
\Step{2}\;& 300=6x \Rightarrow x=50.
\end{aligned}
\]
\[
\boxed{x=50 \text{ (the lines intersect at } (50,\,-150)\text{).}}
\]

\begin{center}
\begin{tikzpicture}
\begin{axis}[darkaxis,xmin=0,xmax=60,ymin=-260,ymax=120]
\addplot[color=cyan,very thick,domain=0:60]{100-5*x};
\addplot[color=green,very thick,domain=0:60]{x-200};
\addplot[color=gold,only marks,mark=*] coordinates {(50,-150)};
\end{axis}
\end{tikzpicture}
\end{center}
\end{QAPair}

% ============================================================
% Q8
\begin{QAPair}{Question 8}
\textcolor{gold}{\bfseries Question:} $P(t)=50{,}000(1.05)^t$. Draw its graph and find the population growth over 5 years.\\
\tcblower
\textcolor{green}{\bfseries Answer:}
\[
\begin{aligned}
\Step{1}\;& P(0)=50{,}000.\\
\Step{2}\;& P(5)=50{,}000(1.05)^5=50{,}000(1.2762815625)\approx 63{,}814.\\
\Step{3}\;& \text{Growth over 5 years }=P(5)-P(0)\approx 63{,}814-50{,}000=13{,}814.
\end{aligned}
\]
\[
\boxed{P(5)\approx 63{,}814,\quad \text{growth}\approx 13{,}814.}
\]

\begin{center}
\begin{tikzpicture}
\begin{axis}[darkaxis,xmin=0,xmax=5,ymin=48000,ymax=66000,xtick={0,1,2,3,4,5}]
\addplot[color=cyan,very thick,domain=0:5]{50000*(1.05^x)};
\addplot[color=gold,only marks,mark=*] coordinates {(0,50000) (5,63814.078)};
\end{axis}
\end{tikzpicture}
\end{center}
\end{QAPair}

% ============================================================
% Q9
\begin{QAPair}{Question 9}
\textcolor{gold}{\bfseries Question:} Draw gradient of $y=\dfrac{1}{2}x^2$ by drawing tangent at $x=2$.\\
\tcblower
\textcolor{green}{\bfseries Answer:}
\[
\begin{aligned}
\Step{1}\;& y=\frac12 x^2 \Rightarrow \frac{dy}{dx}=x.\\
\Step{2}\;& \text{At } x=2:\ \text{slope } m=2,\ \text{point } (2,\tfrac12\cdot 4)=(2,2).\\
\Step{3}\;& \text{Tangent: } y-2=2(x-2)\Rightarrow y=2x-2.
\end{aligned}
\]
\[
\boxed{\text{Gradient at } x=2 \text{ is } 2,\ \text{tangent line } y=2x-2.}
\]

\begin{center}
\begin{tikzpicture}
\begin{axis}[darkaxis,xmin=-2,xmax=4,ymin=-2,ymax=6]
\addplot[color=cyan,very thick,domain=-2:4]{0.5*x^2};
\addplot[color=green,very thick,domain=-2:4]{2*x-2};
\addplot[color=gold,only marks,mark=*] coordinates {(2,2)};
\end{axis}
\end{tikzpicture}
\end{center}
\end{QAPair}

% ============================================================
% Q10
\begin{QAPair}{Question 10}
\textcolor{gold}{\bfseries Question:} A logistics company charges a base fee of \$50 plus \$0.5 per mile. If shipment travels 250 miles, find total cost.\\
\tcblower
\textcolor{green}{\bfseries Answer:}
\[
\begin{aligned}
\Step{1}\;& \text{Cost} = 50 + 0.5(250)\\
\Step{2}\;& = 50 + 125 = 175.
\end{aligned}
\]
\[
\boxed{\text{Total cost}=\$175.}
\]
\end{QAPair}

\end{document}
