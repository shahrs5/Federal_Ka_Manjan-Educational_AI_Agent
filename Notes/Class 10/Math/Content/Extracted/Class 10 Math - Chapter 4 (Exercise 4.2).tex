% !TEX TS-program = pdflatex
\documentclass[11pt]{article}

% -------------------- Packages --------------------
\usepackage[a4paper,margin=1in]{geometry}
\usepackage{amsmath,amssymb}
\usepackage[T1]{fontenc}
\usepackage{lmodern}
\usepackage{xcolor}
\usepackage{tcolorbox}
\tcbuselibrary{skins,breakable}
\usepackage{enumitem}
\usepackage{hyperref}

\pagestyle{empty}

% -------------------- Dark Theme Colors --------------------
\definecolor{bg}{HTML}{000000}
\definecolor{pairbg}{HTML}{121212}
\definecolor{solbg}{HTML}{0A0A0A}
\definecolor{border}{HTML}{2A2A2A}
\definecolor{text}{HTML}{FFFFFF}
\definecolor{muted}{HTML}{C9CDD3}
\definecolor{gold}{HTML}{FFD700}
\definecolor{green}{HTML}{4ADE80}
\definecolor{cyan}{HTML}{38BDF8}

\pagecolor{bg}
\color{text}

\hypersetup{
  colorlinks=true,
  linkcolor=cyan,
  urlcolor=cyan
}

\setlength{\parindent}{0pt}
\setlength{\parskip}{10pt}

\setlist[itemize]{left=1.4em,itemsep=6pt,topsep=6pt}
\setlist[enumerate]{left=1.6em,itemsep=4pt,topsep=4pt}

% -------------------- tcolorbox Base --------------------
\tcbset{
  enhanced,
  breakable,
  arc=12pt,
  boxrule=0.8pt,
  left=16pt,right=16pt,top=12pt,bottom=12pt
}

\newtcolorbox{QAPair}[1]{%
  colback=pairbg,
  colbacklower=solbg,
  colframe=border,
  coltext=text,
  title=\textcolor{gold}{\bfseries #1},
  fonttitle=\bfseries,
  coltitle=text,
  segmentation style={draw=border, dashed, line width=0.6pt},
}

% Visible text inside this box
\newtcolorbox{QuickBox}{%
  colback=pairbg,
  colframe=cyan,
  coltext=text,
  fontupper=\color{text},
  borderline north={4pt}{0pt}{cyan},
  arc=14pt,
  boxrule=0.8pt
}

% Helper for step headings
\newcommand{\Step}[1]{\textcolor{muted}{\textbf{Step #1:}}}

% ============================================================
\begin{document}

\begin{center}
{\LARGE\bfseries \textcolor{gold}{Exercise 4.2 --- Solutions}}\\[-2pt]
\end{center}

\begin{QuickBox}
{\color{cyan}\bfseries Quick formulas (inequality graphing)}\par\medskip
\begin{itemize}
\item \textbf{Check an ordered pair:} Substitute $(x,y)$ into the inequality. If it becomes true, the pair is a solution.
\item \textbf{Graphing rule:} First graph the \emph{boundary line} by replacing the inequality with equality.
\item \textbf{Solid vs dashed:} Use \textbf{solid} line for $\le,\ge$; use \textbf{dashed} line for $<,>$.
\item \textbf{Shading:} Pick a \textbf{test point} (often $(0,0)$ if not on the boundary). If it satisfies the inequality, shade the side containing that point; otherwise shade the other side.
\end{itemize}
\end{QuickBox}

% ============================================================
% Part A: Tell whether the ordered pair is a solution
\begin{QAPair}{Question 1}
\textcolor{gold}{\bfseries Question:} $(0,0);\; x+y<-4$\\
\tcblower
\textcolor{green}{\bfseries Answer:}
\[
\begin{aligned}
\Step{1}\;& x+y=0+0=0.\\
\Step{2}\;& \text{Check: } 0<-4 \text{ (false).}\\
\Step{3}\;& \Rightarrow (0,0)\ \textbf{is not} \text{ a solution.}
\end{aligned}
\]
\end{QAPair}

\begin{QAPair}{Question 2}
\textcolor{gold}{\bfseries Question:} $(-2,-3);\; y-x>-2$\\
\tcblower
\textcolor{green}{\bfseries Answer:}
\[
\begin{aligned}
\Step{1}\;& y-x=-3-(-2)=-1.\\
\Step{2}\;& \text{Check: } -1>-2 \text{ (true).}\\
\Step{3}\;& \Rightarrow (-2,-3)\ \textbf{is} \text{ a solution.}
\end{aligned}
\]
\end{QAPair}

\begin{QAPair}{Question 3}
\textcolor{gold}{\bfseries Question:} $(5,2);\; 2x+3y\ge 14$\\
\tcblower
\textcolor{green}{\bfseries Answer:}
\[
\begin{aligned}
\Step{1}\;& 2x+3y=2(5)+3(2)=10+6=16.\\
\Step{2}\;& \text{Check: } 16\ge 14 \text{ (true).}\\
\Step{3}\;& \Rightarrow (5,2)\ \textbf{is} \text{ a solution.}
\end{aligned}
\]
\end{QAPair}

\begin{QAPair}{Question 4}
\textcolor{gold}{\bfseries Question:} $(-1,5);\; 4x-7y>28$\\
\tcblower
\textcolor{green}{\bfseries Answer:}
\[
\begin{aligned}
\Step{1}\;& 4x-7y=4(-1)-7(5)=-4-35=-39.\\
\Step{2}\;& \text{Check: } -39>28 \text{ (false).}\\
\Step{3}\;& \Rightarrow (-1,5)\ \textbf{is not} \text{ a solution.}
\end{aligned}
\]
\end{QAPair}

\begin{QAPair}{Question 5}
\textcolor{gold}{\bfseries Question:} $(-9,-7);\; y\le 8$\\
\tcblower
\textcolor{green}{\bfseries Answer:}
\[
\begin{aligned}
\Step{1}\;& y=-7.\\
\Step{2}\;& \text{Check: } -7\le 8 \text{ (true).}\\
\Step{3}\;& \Rightarrow (-9,-7)\ \textbf{is} \text{ a solution.}
\end{aligned}
\]
\end{QAPair}

\begin{QAPair}{Question 6}
\textcolor{gold}{\bfseries Question:} $(-4,0);\; x\ge -3$\\
\tcblower
\textcolor{green}{\bfseries Answer:}
\[
\begin{aligned}
\Step{1}\;& x=-4.\\
\Step{2}\;& \text{Check: } -4\ge -3 \text{ (false).}\\
\Step{3}\;& \Rightarrow (-4,0)\ \textbf{is not} \text{ a solution.}
\end{aligned}
\]
\end{QAPair}

% ============================================================
% Part B: Graph the inequality
\begin{QAPair}{Question 7}
\textcolor{gold}{\bfseries Question:} Graph $x-y<-3$\\
\tcblower
\textcolor{green}{\bfseries Answer:}
\[
\begin{aligned}
\Step{1}\;& x-y<-3 \;\Rightarrow\; -y<-3-x \;\Rightarrow\; y>x+3.\\
\Step{2}\;& \text{Boundary: } y=x+3\ \ (\textbf{dashed},\text{ since } >).\\
\Step{3}\;& \text{Plot points } (0,3),\,(-3,0).\\
\Step{4}\;& \text{Shade the region } \textbf{above} \text{ the line } y=x+3.
\end{aligned}
\]
\end{QAPair}

\begin{QAPair}{Question 8}
\textcolor{gold}{\bfseries Question:} Graph $3y-2x<12$\\
\tcblower
\textcolor{green}{\bfseries Answer:}
\[
\begin{aligned}
\Step{1}\;& 3y-2x<12 \;\Rightarrow\; 3y<2x+12 \;\Rightarrow\; y<\frac{2}{3}x+4.\\
\Step{2}\;& \text{Boundary: } y=\frac{2}{3}x+4\ \ (\textbf{dashed}).\\
\Step{3}\;& \text{Plot points } (0,4),\,(-6,0).\\
\Step{4}\;& \text{Shade the region } \textbf{below} \text{ the line.}
\end{aligned}
\]
\end{QAPair}

\begin{QAPair}{Question 9}
\textcolor{gold}{\bfseries Question:} Graph $x-y\ge 2$\\
\tcblower
\textcolor{green}{\bfseries Answer:}
\[
\begin{aligned}
\Step{1}\;& x-y\ge 2 \;\Rightarrow\; -y\ge 2-x \;\Rightarrow\; y\le x-2.\\
\Step{2}\;& \text{Boundary: } y=x-2\ \ (\textbf{solid},\text{ since } \le).\\
\Step{3}\;& \text{Plot points } (0,-2),\,(2,0).\\
\Step{4}\;& \text{Shade the region } \textbf{below} \text{ the line.}
\end{aligned}
\]
\end{QAPair}

\begin{QAPair}{Question 10}
\textcolor{gold}{\bfseries Question:} Graph $2x+y\ge 8$\\
\tcblower
\textcolor{green}{\bfseries Answer:}
\[
\begin{aligned}
\Step{1}\;& 2x+y\ge 8 \;\Rightarrow\; y\ge 8-2x.\\
\Step{2}\;& \text{Boundary: } y=8-2x\ \ (\textbf{solid}).\\
\Step{3}\;& \text{Plot points } (0,8),\,(4,0).\\
\Step{4}\;& \text{Shade the region } \textbf{above} \text{ the line.}
\end{aligned}
\]
\end{QAPair}

\begin{QAPair}{Question 11}
\textcolor{gold}{\bfseries Question:} Graph $x-y\le -11$\\
\tcblower
\textcolor{green}{\bfseries Answer:}
\[
\begin{aligned}
\Step{1}\;& x-y\le -11 \;\Rightarrow\; -y\le -11-x \;\Rightarrow\; y\ge x+11.\\
\Step{2}\;& \text{Boundary: } y=x+11\ \ (\textbf{solid}).\\
\Step{3}\;& \text{Plot points } (0,11),\,(-11,0).\\
\Step{4}\;& \text{Shade the region } \textbf{above} \text{ the line.}
\end{aligned}
\]
\end{QAPair}

\begin{QAPair}{Question 12}
\textcolor{gold}{\bfseries Question:} Graph $y<-5$\\
\tcblower
\textcolor{green}{\bfseries Answer:}
\[
\begin{aligned}
\Step{1}\;& \text{Boundary: } y=-5\ \ (\textbf{dashed}).\\
\Step{2}\;& \text{Shade the region } \textbf{below} \text{ the line } y=-5.
\end{aligned}
\]
\end{QAPair}

\begin{QAPair}{Question 13}
\textcolor{gold}{\bfseries Question:} Graph $x\ge 4$\\
\tcblower
\textcolor{green}{\bfseries Answer:}
\[
\begin{aligned}
\Step{1}\;& \text{Boundary: } x=4\ \ (\textbf{solid}).\\
\Step{2}\;& \text{Shade the region } \textbf{to the right} \text{ of the line } x=4.
\end{aligned}
\]
\end{QAPair}

\begin{QAPair}{Question 14}
\textcolor{gold}{\bfseries Question:} Graph $\dfrac12(x+2)+3y<8$\\
\tcblower
\textcolor{green}{\bfseries Answer:}
\[
\begin{aligned}
\Step{1}\;& \frac12(x+2)+3y<8 \;\Rightarrow\; \frac12x+1+3y<8\\
&\Rightarrow 3y<7-\frac12x \;\Rightarrow\; y<\frac{7}{3}-\frac{1}{6}x.\\
\Step{2}\;& \text{Boundary: } y=\frac{7}{3}-\frac{1}{6}x\ \ (\textbf{dashed}).\\
\Step{3}\;& \text{Plot points } (0,\tfrac{7}{3}),\,(14,0).\\
\Step{4}\;& \text{Shade the region } \textbf{below} \text{ the line.}
\end{aligned}
\]
\end{QAPair}

% ============================================================
% Q15
\begin{QAPair}{Question 15}
\textcolor{gold}{\bfseries Question:} Can we use $(0,0)$ as a test point when graphing $x+y>0$? Explain with reason.\\
\tcblower
\textcolor{green}{\bfseries Answer:}
\[
\begin{aligned}
\Step{1}\;& \text{The boundary line is } x+y=0.\\
\Step{2}\;& \text{At }(0,0):\; x+y=0 \;\Rightarrow\; (0,0)\ \text{lies on the boundary line.}\\
\Step{3}\;& \text{A test point must not be on the boundary, otherwise it won't tell which side satisfies }(>).\\
\Step{4}\;& \Rightarrow\ \textbf{No.} \text{ Use a different point, e.g. }(1,0)\text{ or }(0,1).
\end{aligned}
\]
\end{QAPair}

% ============================================================
% Part C: Write the verbal sentence as an inequality, then graph
\begin{QAPair}{Question 16}
\textcolor{gold}{\bfseries Question:} Three less than $x$ is greater than or equal to $y$.\\
\tcblower
\textcolor{green}{\bfseries Answer:}
\[
\begin{aligned}
\Step{1}\;& \text{Three less than }x \text{ is } x-3.\\
\Step{2}\;& x-3\ge y \;\;\Longleftrightarrow\;\; y\le x-3.\\
\Step{3}\;& \text{Graph: boundary } y=x-3\ (\textbf{solid});\ \text{shade } \textbf{below} \text{ the line.}
\end{aligned}
\]
\end{QAPair}

\begin{QAPair}{Question 17}
\textcolor{gold}{\bfseries Question:} The product of $-2$ and $y$ is less than or equal to the sum of $x$ and $6$.\\
\tcblower
\textcolor{green}{\bfseries Answer:}
\[
\begin{aligned}
\Step{1}\;& \text{Product of }-2\text{ and }y \text{ is } -2y,\ \text{sum of }x\text{ and }6 \text{ is } x+6.\\
\Step{2}\;& -2y\le x+6.\\
\Step{3}\;& \text{Solve for }y:\; y\ge -\frac12x-3 \quad (\text{divide by }-2 \text{ flips the sign}).\\
\Step{4}\;& \text{Graph: boundary } y=-\frac12x-3\ (\textbf{solid});\ \text{shade } \textbf{above} \text{ the line.}
\end{aligned}
\]
\end{QAPair}

\begin{QAPair}{Question 18}
\textcolor{gold}{\bfseries Question:} The sum of $x$ and the product of $4$ and $y$ is less than $-2$.\\
\tcblower
\textcolor{green}{\bfseries Answer:}
\[
\begin{aligned}
\Step{1}\;& \text{Sum }=x+4y,\ \text{so } x+4y<-2.\\
\Step{2}\;& \text{Solve for }y:\; 4y<-x-2 \;\Rightarrow\; y<-\frac14x-\frac12.\\
\Step{3}\;& \text{Graph: boundary } y=-\frac14x-\frac12\ (\textbf{dashed});\ \text{shade } \textbf{below} \text{ the line.}
\end{aligned}
\]
\end{QAPair}

\end{document}
